As described in Ch.\ref{chap:dataformat}, LArLite has its own data format.
Event though it is designed to be similar to that of LArSoft, it is not equivalent.
So how can we analyze LArSoft data file using LArLite?

There are two possibilities:
\begin{itemize}
\item Convert LArSoft data file into LArLite format
\item Implement a utility class to interface LArSoft data file
\end{itemize}
The first point is currently done by using a dedicated LArSoft analyzer module called {\ttfamily LiteScanner}. This is described in details in this section.

The second point does not mean we have to introduce {\ttfamily art} framework dependency in LArLite: that would screw up the whole purpose of LArLite and probably almost all ``strength'' listed in Ch.\ref{chap:introduction} will be gone. Although this is not yet available, extracting data from LArSoft files without {\ttfamily art} or LArSoft framework seems possible as there is a beautiful working example, {\ttfamily Argo}. It would be great if someone can implement such feature also in LArLite someday...

\subsection{LArSoft $\Rightarrow$ LArLite: \LiteScanner}
 \LiteScanner is a LArSoft module that is maintained under {\ttfamily uboonecode} repository (official MicroBooNE code repository)

\subsubsection{0. Prerequisite}
As mentioned above, it uses LArSoft. So you have to have LArSoft environment set up and running. In particular, you have to have a shell environment variable {\ttfamily \$MRB\_TOP} set and pointing to your local MRB repository. There is a good guidance available \cite{LArDevWiki}. For the moment, the author recommends to use the {\ttfamily v03\_04\_01 -q e6:prof} version of {\ttfamily LArSoft}.

\subsubsection{1. Checking Out and Bulding LArLite}
Once you set up LArSoft environment, you should have \git and \ROOT. Then we can checkout and build LArLite:
\begin{lstlisting}
    > cd $WHEREVER_YOU_WANT_TO_BUILD
    > git clone git@github.com:larlight/larlite LArLite
    > cd LArLite
    > git checkout trunk
    > source config/setup.sh
    > make -j4
\end{lstlisting}
Once the build finishes, you are good to go.

\subsubsection{1. Checking Out and Bulding \LiteScanner}
OK, we assume you have gone through prerequisite and you have {\ttfamily \$MRB\_TOP} shell environment variable set. 
Let us checkout {\ttfamily uboonecode}:
\begin{lstlisting}
    > cd $MRB_TOP/srcs
    > mrb g uboonecode
\end{lstlisting}
You find LiteMaker package here:
\begin{lstlisting}
    uboonecode/uboone/LiteMaker
\end{lstlisting}
Before start compiling {\ttfamily uboonecode}, you have to change
\begin{lstlisting}
    uboonecode/uboone/CMakeLists.txt
\end{lstlisting}
which, by default, excludes {\ttfamily LiteMaker} package from the build.
It is as simple as un-comment the following line
\begin{lstlisting}
    #add_subdirectory(LiteMake)
\end{lstlisting}
Once that is done, try:
\begin{lstlisting}
    > mrbsetenv
    > mrb i -j4
\end{lstlisting}

\subsubsection{2. Running \LiteScanner Build}
We go over how to configure \LiteScanner, but why not just running a fcl file first? If this fails, you the author should fix it w/o furhter wasting your time to learn how to configure \LiteScanner.
Go to \LiteScanner directory:
\begin{lstlisting}
    > cd $MRB_TOP/srcs/uboonecode/uboone/LiteMaker
\end{lstlisting}
Let's run a simple 2 muon event generation.
\begin{lstlisting}
    > lar -c job/larlite_maker.fcl -s /uboone/data/users/kterao/sample.root -n 5
\end{lstlisting}
If the above execution of {\ttfamily lar} goes well, then you should have generated {\ttfamily larlite.root}. Otherwise please contact the author! 

\subsubsection{3. Configuring \LiteScanner}
You have already run \LiteScanner in {\bf Testing \LiteScanner Build} section with a fcl file, {\ttfamily job/larlite\_maker.fcl}. In this fcl file, you can find (hopefully) self-descriptive parameters.

There are two types of parameter name strings (other than module label):
\begin{itemize}
    \item {\ttfamily store\_association} $\ldots$ enable/disable storing association information
    \item {\ttfamily larlite::data::kDATA\_TREE\_NAME} strings $\ldots$ specifies data product type to store from LArSoft file. You provide producer modules' label in LArSoft file as a vector of strings.
\end{itemize}
If the {\ttfamily store\_association} is set to {\ttfamily true}, then all associations among stored data products will be stored. Note that associations between stored data products and those not stored are not saved. The author thinks a little more flexibility is needed, so there will be an improved feature soon but not yet available.
If you do not want to store certain data product type, you can either comment out the corresponding string or provide an empty vector as an argument.

Here is a few points to remember:
\begin{itemize}
\item If you put an wrong producer name, you will get a corresponding LArLite data {\ttfamily TTree} filled with correct number of events. But each event will be empty.
\item If you put an empty string vector as producers' name, then you will not get any corresponding LArLite data stored (better this way to save disk space a bit).
\item Association is stored only if a corresponding data product is stored. For instance, if you want to store associations to {\ttfamily larlite::hit}, you must store this data product (for now).
\end{itemize}

Any question/comment/suggestion? As always feel free to contact the author.

