
\anaproc runs an event loop and is responsible for executing \anaunit(s). You can attach as many \anaunit(s) as you wish. Those \anaunit(s) are executed in the respective order. You find {\ttfamily ana\_processor} is the actualy \CPP class that takes this role.

Here is an analysis processor's task list in order:
\begin{itemize}
\item Start I/O using {\ttfamily storage\_manager}
\item Call {\ttfamily initialize} function of attached \anaunit(s)
\item Start event loop via {\ttfamily storage\_manager}
\item For each event, call {\ttfamily analyze} of \anaunit(s)
\item Call {\ttfamily finalize} of \anaunit(s) at the end of an event loop
\item Close I/O
\end{itemize}
Before the 1st bullet point, a user has to (1) configure \anaproc with file I/O information and (2) attach \anaunit(s).

\subsection{Example: Running \anaproc}
Probably seeing an example code is much easier. Using, again, {\ttfamily KazuAna} \anaunit, here is an example \CINT code:
\begin{lstlisting}
{
    gSystem->Load(``libLArLite_Analysis'');
    // Create ana_processor instance
    larlite::ana_processor my_proc;
    // Set I/O mode, like storage_manager.
    my_proc.set_io_mode(larlite::storage_manager::READ);
    // Set analysis output file: not data output file.
    my_proc.set_ana_output_file(``ana.out'');
    // Set TDirectory name under which TTree resides
    my_proc.set_input_rootdir(``scanner'');
    // Now add input file(s)
    my_proc.add_input_file(``hit_sample.root'');
    
    // Create analysis unit on heap
    larlite::KazuAna* k = new larlite::KazuAna;
    
    // Attach
    my_proc.add_process(k);

    // Now run event loop
    my_proc.run();
}
\end{lstlisting}
The I/O configuration is very similar to what you learned about {\ttfamily storage\_manager}. As you guessed, this is because \anaproc is using {\ttfamily storage\_manager} underneath.

After the I/O configuration, like described before, you attach \anaunit(s). You can add as many as you wish.

Finally, {\ttfamily ana\_processor::run()} starts a batch (uninterrupted) event processing. Another option is to use {\ttfamily ana\_processor::process\_event()} which process one event per function call. This can be useful when you want to do an interactive event processing.

\subsection{Filter Mode}
\label{sec:analysis:filtermode}
\anaproc has a feature called ``filtering mode''. This can be enabled by a function:
\begin{lstlisting}
    ana_processor::enable_filter(bool doit=true)
\end{lstlisting}
When enabled, attached \anaunit's {\ttfamily analyze()} function return value causes to abort the execution of the rest of \anaunit(s) in an event.
Let's say we attached multiple \anaunit(s) and enabled a filter mode.
\anaunit(s) are executed in the order we attach them to an \anaproc instance.
If filter mode is enabled and if any one of \anaunit returns false, \anaproc aborts the execution of the rest of \anaunit and move onto the next event.
It does not mean to filter out the subject event from an output file (which is a feature that can be easily implemented if desired... let the author know).
