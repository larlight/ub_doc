
Analysis processor runs an event loop and is responsible for executing analysis units. You can attach as many analysis units as you wish. Those analysis units are executed in the respective order. You find {\ttfamily ana\_processor} is the actualy \CPP class that takes this role.

Here is an analysis processor's task list in order:
\begin{itemize}
\item Start I/O using {\ttfamily storage\_manager}
\item Call {\ttfamily initialize} function of attached analysis units
\item Start event loop via {\ttfamily storage\_manager}
\item For each event, call {\ttfamily analyze} of analysis units
\item Call {\ttfamily finalize} of analysis units at the end of an event loop
\item Close I/O
\end{itemize}
Before the 1st bullet point, a user has to (1) configure {\ttfamily ana\_processor} with file I/O information and (2) attach analysis units.

\subsection{Example: Running Analysis Processor}
Probably seeing an example code is much easier. Using, again, {\ttfamily KazuAna} analysis unit, here is an example \CINT code:
\begin{lstlisting}
{
    gSystem->Load(``libAnalysis'');
    // Create ana_processor instance
    larlite::ana_processor my_proc;
    // Set I/O mode, like storage_manager.
    my_proc.set_io_mode(larlite::storage_manager::READ);
    // Set analysis output file: not data output file.
    my_proc.set_ana_output_file(``ana.out'');
    // Set TDirectory name under which TTree resides
    my_proc.set_input_rootdir(``scanner'');
    // Now add input file(s)
    my_proc.add_input_file(``hit_sample.root'');
    
    // Create analysis unit on heap
    larlite::KazuAna* k = new larlite::KazuAna;
    
    // Attach
    my_proc.add_process(k);

    // Now run event loop
    my_proc.run();
}
\end{lstlisting}
The I/O configuration is very similar to what you learned about {\ttfamily storage\_manager}. As you guessed, this is because {\ttfamily ana\_processor} is using {\ttfamily storage\_manager} underneath.

After the I/O configuration, like described before, you attach analysis unit(s). You can add as many as you wish.

Finally, {\ttfamily ana\_processor::run()} starts a batch (uninterrupted) event processing. Another option is to use {\ttfamily ana\_processor::process\_event()} which process one event per function call. This can be useful when you want to do an interactive event processing.

\subsection{Filter Mode}
\label{sec:analysis:filtermode}
{\ttfamily ana\_processor} has a feature called ``filtering mode''. This can be enabled by a function:
\begin{lstlisting}
    ana_processor::enable_filter(bool doit=true)
\end{lstlisting}
When enabled, attached analysis unit's {\ttfamily analyze()} function return value causes to abort the execution of the rest of analysis units in an event.
Let's say we attached multiple analysis units and enabled a filter mode.
Analysis units are executed in the order we attach them to an {\ttfamily ana\_processor} instance.
If filter mode is enabled and if any one of analysis unit returns false, {\ttfamily ana\_processor} aborts the execution of the rest of analysis unit and move onto the next event.
It does not mean to filter out the subject event from an output file (which is a feature that can be easily implemented if desired... let the author know).
