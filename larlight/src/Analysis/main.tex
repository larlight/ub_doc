
This chapter describes briefly about \Analysis package, the analysis framework of LArLite.
It comes in largely two pieces:
\begin{itemize}
\item \anaunit
\item \anaproc
\end{itemize}
The first bullet corresponds to LArSoft's ``module'' (i.e. {\ttfamily art::EDAnalyzer/EDProducer}).
The second bullet is an executor of \anaunit(s) by interfacing with {\ttfamily storage\_manager} (i.e. LArLite I/O interface class, see Sec.\ref{sec:io}). \\

Before going into details, do not misunderstand that you are constrained to do your analysis this way.
Like it was described in Sec.\ref{sec:io}, you have an interface to run your own event loop in whatever the way you wish.
Using the analysis framework is just a suggestion and not a requirement.
Or please share if you made something better ;)

\section{\anaunit}
\label{sec:ana_base}

An \anaunit is a \CPP class that inherits from {\ttfamily ana\_base} class. 
This base class has three important functions that would concern your \anaunit code development.
\begin{lstlisting}
    bool ana_base::initialize()
    bool ana_base::analyze(larlite::storage_manager* storage)
    bool ana_base::finalize()
\end{lstlisting}
As it is described in the next subsection, an \anaunit is held by analysis processor which runs an event loop.

\subsection{Flow of Function Calls}

\subsubsection{{\ttfamily initialize()}}
The attached \anaunit's {\ttfamily initialize} function is called only once at the beginning of an event loop. 
If the \anaunit returns {\ttfamily false} at that point, event loop will not be even started by analysis processor.

\subsubsection{{\ttfamily analyze(storage\_manager* storage)}}
Then for each event, analysis processor calls {\ttfamily analyze} function. Event-by-event analysis code should be implemented within this function. Access to an event's data information is done through {\ttfamily storage\_manager} pointer. The return value of {\ttfamily analyze} function has a meaning if one enables ``filter mode'' for running {\ttfamily ana\_processor}. See Sec.\ref{sec:analysis:filtermode} for details.

\subsubsection{{\ttfamily finalize()}}
At the end of event loop, {\ttfamily finalize} function is called. This is where you might want to do final wrap-up of your analysis.

\subsubsection{Storing Output}
This is sometimes confusing, but there are two kinds of outputs you can store:
\begin{itemize}
\item Data product
\item Analysis output
\end{itemize}

To store modified/new data product (i.e. classes defined in Sec.\ref{sec:dataproduct}), simply use provided {\ttfamily storage\_manager} pointer that is given as a function argument of {\ttfamily analyze}. 

To store your analysis output \ROOT objects, like your {\ttfamily TH1} histogram, {\ttfamily TGraph} and such, use {\ttfamily ana\_base::\_fout} attribute. As it is defined in {\ttfamily ana\_base}, all \anaunit should have it. This is a {\ttfamily TFile} pointer provided by analysis processor before calling {\ttfamily initialize} function. 

The analysis output file (i.e. {\ttfamily TFile} object to which {\ttfamily \_fout} points) is separately created by analysis processor if a user configured to create it. So your object will be stored in a separate output \ROOT file.

For example, let's say I have \anaunit called {\ttfamily KazuAna} and it has {\ttfamily TH1D} pointer called {\ttfamily h1} with real object on the heap. To save this object in an analysis output file, I can write the following {\ttfamily finalize} function.
\begin{lstlisting}
    bool KazuAna::finalize() {
        if(_fout) {
         _fout->cd();
         h1->Write();
        }
    }
\end{lstlisting}
Note that:
\begin{itemize}
\item Checking whether {\ttfamily \_fout} is valid or not will avoid code crash in case analysis processor is configured not to create output file
\item Since {\ttfamily \_fout} is shared among all \anaunit, no unit should not selfishly close the file.
\end{itemize}


\subsection{Why Unit Modulation?}
I hope we agree it is better to have multiple functions rather than one function with thousands of lines of code. 
The basic idea is same: modulation helps to maintaine code and also sharing/reusing the code.
For this reason, it is better to restrict one \anaunit does one thing.
It does not mean one \anaunit should create more than one histogram. 
But it means to have a single or a few words answer to the question, ``what does this unit do?''
Don't continue your ``a few words answer'' by saing ``it also does...''. That's cheating ;)

If this is not clear, here is a practical example. 
Let's say we want to reconstruct number of photo-electrons from PMT signal.
You are given a digitized waveform, and the goal is to reconstruct a pulse from waveform and you compute its area or height to estimate electric charge. Then you convert the charge into number of photo-electrons.
Unit modulation means, in this example, you write one module to reconstruct a pulse and a separate module to calculate number of photo-electrons from charge in a pulse.
So if someone asks what each module does, your answers are: ``pulse reconstruction'' and ``charge to photo-electron conversion''.

What we benefit from this? Imagine tomorrow the smarter you come up with a better reconstruction algorithm but you have to prove it is better. Instead of copy and paste the entire code, you can simply add another pulse reconstruction unit in your package. Then you can run your analysis by switching one unit to the other to compare the result. Your \anaunit to convert charge into photo-electrons is undisturbed. You do not have to do any ``copy and paste'' of code, which often introduces a bug in your program. This is the benefit you get.

Even possibly greater benefit: tomorrow you are doing TPC waveform analysis and you want to use a similar way of reconstructing a pulse, though there is no sense of ``photo-electron'' involved in this case. By doing unit modulation you can use exact same pulse reconstruction unit to perform the task.

















\section{\anaproc}
\label{sec:ana_processor}

Analysis processor runs an event loop and is responsible for executing analysis units. You can attach as many analysis units as you wish. Those analysis units are executed in the respective order. You find {\ttfamily ana\_processor} is the actualy \CPP class that takes this role.

Here is an analysis processor's task list in order:
\begin{itemize}
\item Start I/O using {\ttfamily storage\_manager}
\item Call {\ttfamily initialize} function of attached analysis units
\item Start event loop via {\ttfamily storage\_manager}
\item For each event, call {\ttfamily analyze} of analysis units
\item Call {\ttfamily finalize} of analysis units at the end of an event loop
\item Close I/O
\end{itemize}
Before the 1st bullet point, a user has to (1) configure {\ttfamily ana\_processor} with file I/O information and (2) attach analysis units.

\subsection{Example: Running Analysis Processor}
Probably seeing an example code is much easier. Using, again, {\ttfamily KazuAna} analysis unit, here is an example \CINT code:
\begin{lstlisting}
{
    gSystem->Load(``libAnalysis'');
    // Create ana_processor instance
    larlight::ana_processor my_proc;
    // Set I/O mode, like storage_manager.
    my_proc.set_io_mode(larlight::storage_manager::READ);
    // Set analysis output file: not data output file.
    my_proc.set_ana_output_file(``ana.out'');
    // Set TDirectory name under which TTree resides
    my_proc.set_input_rootdir(``scanner'');
    // Now add input file(s)
    my_proc.add_input_file(``hit_sample.root'');
    
    // Create analysis unit on heap
    larlight::KazuAna* k = new larlight::KazuAna;
    
    // Attach
    my_proc.add_process(k);

    // Now run event loop
    my_proc.run();
}
\end{lstlisting}
The I/O configuration is very similar to what you learned about {\ttfamily storage\_manager}. As you guessed, this is because {\ttfamily ana\_processor} is using {\ttfamily storage\_manager} underneath.

After the I/O configuration, like described before, you attach analysis unit(s). You can add as many as you wish.

Finally, {\ttfamily ana\_processor::run()} starts a batch (uninterrupted) event processing. Another option is to use {\ttfamily ana\_processor::process\_event()} which process one event per function call. This can be useful when you want to do an interactive event processing.




