
\subsection{Building \Core}

The repository contents are not discussed here, but we attempt build the framework blindly.
Make sure you have a list of prerequisite (see \ref{sec:prerequisite}) checked out.
Try:
\begin{lstlisting}
    > source config/setup.sh
\end{lstlisting}

If above commands give you a message telling you to type ``make'', then your set up is complete.
If that is not the case, ask for a help. The execution of this shell script sets necessary
environment variables to build and use LArLite. In particular, it might be handy to remember 
aliases set by this script (see Sec.\ref{sec:configure} for details).

Assuming you were succesful, try:
\begin{lstlisting}
    > make -j4
\end{lstlisting}

where ``-j4'' means to use 4 threads/cores to compile. 
Change the number to what is suitable for your case (you can ask more than available but that might slow down the process).

Look for a compiler error... No? Great! Yes? Sorry... please contact the author!

The build takes about 1 minute on the author's MacBook Pro (OSX 10.9.3) from scratch. 
It takes about same time on Ubuntu 12.04LTS running on the virtualbox. 
It takes, however, a few minutes on uboonegpvm nodes for the first time build.
This is likely related to the fact these machines are often under heavy use.

\subsubsection{Success Build?}

Once the build is finished, check:
\begin{lstlisting}
    > ls lib
\end{lstlisting}
You should find {\ttfamily libLArLite\_Base.so, libLArLite\_DataFormat.so, libLArLite\_Analysis.so, libLArLite\_LArUtil.so}. These are four building blocks of the framework and will be described in the later section of this document.  

\subsection{Building \UserDev}

I hope you will eventually write your own package and enjoy coding, and \UserDev is the place for you.
In LArLite repository, \UserDev only contains {\ttfamily bin} directory (you might see {\ttfamily GNUmakefile} but that is generated by {\ttfamily setup.sh} and not a part of repository).
This directory \UserDev is meant to hold a users' (or possibly multiple) sub-repository which could be, for instance, a \git repository.

For the sake of tutorial, let us use my example. Try:
\begin{lstlisting}
    > cd $LARLITE_USERDEVDIR
    > git clone https://github.com/drinkingkazu/larlite_example LArLiteExample
\end{lstlisting}
This should have created {\ttfamily LArLiteExample} sub-directory under your \UserDev.
This is what I mean by users' sub-repository.
The newly created {\ttfamily LArLiteExample} repository is, althogh it is under \UserDev, not recognized as
LArLite repository because of {\ttfamily .gitignore} rule placed in LArLite top directory.

How to create your own package is covered in Ch.\ref{chap:generate}.

There are two ways to build sub-repositories under \UserDev.
A simple method is simply type ``make'' in that sub-repository:
\begin{lstlisting}
    > cd LArLiteExample
    > make
\end{lstlisting}

You may not want to do this, however, if you have multiple sub-repositories (simply troublesome).
Instead, you can combine a compilation of chosen sub-repositories to the LArLite build.
Here's what you can try:
\begin{lstlisting}
    > cd $LARLITE_BASEDIR
    > export USER_MODULE=``LArLiteExample''
    > source config/setup.sh
    > make -j4
\end{lstlisting}
As you see, the packages to be compiled under \UserDev can be specified by {\ttfamily \$USER\_MODULE} shell environment variable.  
Sourcing {\ttfamily config/setup.sh} creates a new GNUmakefile to compile the specified packages in the respective order to keep the dependency straight if there is any.


\subsection{Generating Documentation}
\label{sec:gendoxygen}
If you have {\ttfamily doxygen} installed, you can generate your own web documentation that includes \CPP class/namespace index, just like that of \ROOT.
\begin{lstlisting}
    > cd $LARLITE_BASEDIR
    > make doxygen
\end{lstlisting}
You can open the top page using firefox:
\begin{lstlisting}
    > firefox  $LARLITE_BASEDIR/doc/dOxygenMyProject/html/index.html
\end{lstlisting}
or, on Mac OSX,
\begin{lstlisting}
    > open $LARLITE_BASEDIR/doc/dOxygenMyProject/html/index.html
\end{lstlisting}

