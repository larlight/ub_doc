
Here is a description of minimal set of tests to see if your build was successful or not. Try the following commands on \CINT or \PyROOT:
\begin{lstlisting}
    > root
    root[0] gSystem->Load(``libLArLite_Base'');
    0
    root[1] gSystem->Load(``libLArLite_DataFormat'');
    0
    root[2] gSystem->Load(``libLArLite_Analysis'');
    0
    root[3] gSystem->Load(``libLArLite_LArUtil'');
    0
    root[4] 
\end{lstlisting}
Each line above tells \CINT to load compiled shared object libraries. They can be physically found under {\ttfamily \$LARLITE\_BASEDIR/lib} directory. Return value of 0 means the library was successfully found and loaded.

Once libraries are validated, you can compile some very basic test routines in each package.
Each of \Base, \DataFormat, \Analysis, and \LArUtil has {\ttfamily bin} subdirectory. 
The {\ttfamily bin} directory holds test routine source code, README with descriptions, and 
its own GNUmakefile (i.e. you can type ``make'' there and compile {\ttfamily bin} directory). 
Following subsections describe briefly about available test routines.

\subsection{Testing \Base}
You can read Ch.\ref{chap:base} to learn about this package.
A simple routine available here is {\ttfamily test\_base}. This routine simply checks if \CPP
classes defined  in the package are available and functions as expected. 

It means a success if there is no segmentation fault upon running the executable.

\subsection{Testing \DataFormat}
You can read Ch.\ref{chap:dataformat} to learn about this package.
A simple routine available here include:
\begin{itemize}
\item {\ttfamily test\_simple\_io} ... This is a self-contained test routine. It creates a sample LArLite
\ROOT file with some events using I/O interface class ({\ttfamily storage\_manager}), then
it uses the same class to run and evnet loop to validate the written file. 
\item {\ttfamily simple\_write} ... This routine writes some events with fake track data product using 
I/O interface class defined in \DataFormat. Output test file can be used for {\ttfamily simple\_read}.
\item {\ttfamily simple\_read} ... This routine runs an event loop using the compiled I/O interface class.
In particular it attempts to read a track data product (and does nothing else). So you want to feed in
a LArLite \ROOT file with a track data product. You can use an output of {\ttfamily simple\_write} routine
to try out.                                       
\end{itemize}

\subsection{Testing \Analysis}
You can read Ch.\ref{chap:analysis} to learn about this package.
A simple routine available here is {\ttfamily test\_simple\_ana}. This routine takes an input
LArLite \ROOT file, and runs an event loop using the \Analysis framework. For an input file to test
the package, if you do not have one, you can use a file generated by {\ttfamily simple\_write} routine
introduced in the previous subsection.

\subsection{Testing \LArUtil}
You can read Ch.\ref{chap:larutil} to learn about this package. 
A simple routine available here include:
\begin{itemize}
\item {\ttfamily print\_constants} ... This is a simple routine to print some parameters from {\ttfamily Geometry}, {\ttfamily LArProperties}, and {\ttfamily DetectorProperties}, utility classes equivalent of those in LArSoft.
\end{itemize}

