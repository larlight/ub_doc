The data product classes are meant to store LArSoft output data, hence their design follows closely with what is in LArSoft.
If you look at a header file of data product class, you should find LArLight data product classes carry same or very similar attributes name to reduce confusion.
For adding further data product classes to store LArSoft data, it is recommended to keep this trend to reduce confusion.
Here is a list of LArLight data products (left) with corresponding LArSoft data product (right) as of now.
\begin{lstlisting}
    larlight::mctruth <==> simb::MCTruth
    larlight::mcnu <==> simb::MCNeutrino
    larlight::mctrack <==> simb::MCTrajectory
    larlight::mcpart <==> simb::MCParticle
    larlight::tpcfifo <==> raw::RawDigit
    larlight::pmtfifo <==> optdata::FIFOChannel
    larlight::wire <==> recob::Wire
    larlight::hit  <==> recob::Hit
    larlight::cluster <==> recob::Cluster
    larlight::spacepoint <==> recob::SpacePoint
    larlight::track <==> recob::Track
    larlight::shower <==> recob::Shower
    larlight::endpoint2d <==> recob::EndPoint2D
    larlight::vertex <==> recob::Vertex
    larlight::calorimetry <==> anab::Calorimetry
    larlight::trigger <==> (not yet existing in LArSoft)
    larlight::user_info ... LArLight-only custom data product
\end{lstlisting}
There is additional data product called {\ttfamily user\_info} which will be described later.

All data product class shares the base class {\ttfamily larlight::data\_base} which stores two variables:
\begin{itemize}
\item {\ttfamily larlight::DATA::DATA\_TYPE}
  \begin{itemize}
    \item[] This \enum specifies the type of data stored. For instance, {\ttfamily larlight::hit} can be assigned with {\ttfamily larlight::DATA::FFTHit}, {\ttfamily larlight::DATA::APAHit}, and so on. By accessing this enum value, a user knows what kind of ``hit'' is stored.
  \end{itemize}
\item {\ttfamily std::map<larlight::DATA::DATA\_TYPE,std::vector<std::vector<unsigned short> > >}
  \begin{itemize}
    \item[] This is used to specify the association. This is described later in detail.
  \end{itemize}
\end{itemize}

\subsection{What is stored per event?}
Despite a few exceptions (which I describe next), each data product class {\ttfamily T} has a corresponding collection class which inherits from {\ttfamily std::vector<T>}, named as {\ttfamily event\_T}. 
For instance, in {\ttfamily cluster.hh}, you find there is a class {\ttfamily larlight::cluster} and {\ttfamily larlight::event\_cluster} where the latter is a type {\ttfamily std::vector<larlight::cluster>}. What is stored per event is {\ttfamily std::vector<T>}, just like a collection of object is stored in LArSoft. This collection data product also inherits from {\ttfamily data\_base}, but it carries additional inforamtion:
\begin{itemize}
\item Event number
\item Run number
\item Sub-Run number
\end{itemize}

Exceptions to this event-wise collection storage scheme include {\ttfamily larlight::mcnu} and {\ttfamily larlight::mctrack}.
In LArSoft, {\ttfamily simb::MCNeutrino} is stored as a part of {\ttfamily simb::MCTruth}. Likewise, {\ttfamily simb::MCTrajectory} is stored as a part of {\ttfamily simb::MCParticle}. LArLight respects this scheme and stores {\ttfamily larlight::mcnu} in {\ttfamily larlight::mctruth} and {\ttfamily larlight::mctrack} in {\ttfamily larlight::mcpart}.

\subsubsection{A few remarks on MC data products}
In case you are unfamiliar with these MC information, though this is a side-track, {\ttfamily larlight::mcnu} stores neutrino specific information from neutrino generator. It is a part of {\ttfamily larlight::mctruth} because a generator (whether it is neutrino generator or not) stores {\ttfamily larlight::mctruth} in general. Similarly, {\ttfamily larlight::mctrack} is part of {\ttfamily larlight::mcpart} because a particle's trajectory points (that is what {\ttfamily larlight::mctrack} is) belong to a particle information which is {\ttfamily larlight::mcpart}.

\subsubsection{More useful info to store?}
If you have some information you think is very useful, let's store it :)
For instance, it was advised that knowing a part of {\ttfamily larlight::mctrack} that is inside the fiducial volume is very useful information.
So we added this information in {\ttfamily larlight::mctrack} for users so that a user can access particle trajectory points inside the fiducial volume quickly instead of looping over all trajectory points and inspect whether it is in the fiducial volume or not.

Your ideas can help others. So please share it!

\subsubsection{Adding new data product?}
By all means, go ahead. There are important pieces missing, for instance vertex!
That being said, however, a mistake can break DataFormat which is shared among all users.
So feel free to consult with the author.\\

Alternative solution is to ask the author to add it. 
Believe me, it's so simple and it won't take me > 10 minutes to add another on the list above.

\subsection{Custon data product for dynamic variable declaration ... {\ttfamily user\_info}}

Sometimes you want a custom data product and store information in the output data root file.
For instance:
\begin{itemize}
\item you want to store some parameter values but not worth making a new data product (say studying a parameter)
\item you want to store some parameter values for certain events/objects, but not all.
\end{itemize}

Then {\ttfamily user\_info} might be handy for you. This data product class contains several {\ttfamily std::map} which allows you to store following basic variable types:
\begin{lstlisting}
    Bool_t
    Int_t
    Double_t
    std::string
    std::vector<Bool_t>
    std::vector<Int_t>
    std::vector<Double_t>
    std::vector<std::string>
\end{lstlisting}
When you store these variables in {\ttfamily user\_info}, you store with a {\ttfamily std::string} key. 
Then you can access the stored value by this key. 
You can use the same key to store different type of variables: mixing is prevented.
If you reuse a key that is already used to store the same type of variable, you will overwrite the value.
There is a function to check if the key is already in use or not, so a user can avoid overwriting the value.\\

Using {\ttfamily std::map} is not great for speed when retrieving the variable. 
But this seems to be a matter of micro-seconds or less, depending on number of keys used.

\subsection{Association}

The author has no idea how association is stored in \ART. But that does not matter.

As you have seen in the previous section, all data product are stored in terms of {\ttfamily std::vector} collection.
So ultimately what you need to associate one object to (an)other are:
\begin{itemize}
\item A set of indexes that point to the right {\ttfamily std::vector} index.
\item Associated data type to know which collection {\ttfamily std::vector} to access.
\end{itemize}
To do this, {\ttfamily data\_base}, the base of all data product, has:
\begin{lstlisting}
    std::map<larlight::DATA::DATA_TYPE,std::vector<std::vector<unsigned short> > > 
\end{lstlisting}
My god, such a long specialization name... 

Then we wonder why we store a ``vector of vector'' instead just a ``vector''?
This is because \ART is capable to store more than one set of association (I think).
Even if \ART won't do that, having LArLight being capable to store multiple association of the same data type could be handy.
So that's why we have such thing in {\ttfamily data\_base}.\\

Retrieving association is as simple as calling the following function:
\begin{lstlisting}
    data_base::association(larlight::DATA::DATA_TYPE, size_t i=0)
\end{lstlisting}
You provide the association type in the 1st argument.
The 2nd argument is to specify which association of the provided type to access.
Because we have only one set of association in many cases (for example, one cluster has only one set of hits associated), almost always the 2nd argument is 0 (i.e. 0th element of vector of associations).
For this reason, the default value of the 2nd argument is 0, and most users do not need to provide 2nd argument.
