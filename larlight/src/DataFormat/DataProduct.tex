The data product classes are meant to store LArSoft output data, hence their design follows closely with what is in LArSoft.
If you look at a header file of data product class, you should find LArLite data product classes carry same or very similar attributes name to reduce confusion.
For adding further data product classes to store LArSoft data, it is recommended to keep this trend to reduce confusion.

\subsection{Classification of ``Data'' Type}
There are three types of data:
\begin{itemize}
\item event-wise data ... {\ttfamily enum larlite::DataType\_t}
\item run-wise data ... {\ttfamily enum larlite::RunDataType\_t}
\item subrun-wise data ... {\ttfamily enum larlite::SubRunDataType\_t}
\end{itemize}
Each \enum listed above contains further ``data type'' that belongs to each classification above.
As you might guess, above classifications correspond to the unit data:
there can be more than one event in a sub-run, and there can be more than one subrun in one run.
An event-wise data is stored at each event boundary, and run/subrun-wise data are stored at
each run/subrun boundary.

\subsection{Data Products}

Here is a list of LArLite data products (left) with corresponding LArSoft data product (right) as of now.
\begin{lstlisting}
    larlite::gtruth <==> simb::GTruth
    larlite::mcflux <==> simb::MCFlux
    larlite::mctruth <==> simb::MCTruth
    larlite::mcnu <==> simb::MCNeutrino
    larlite::mctrajectory <==> simb::MCTrajectory
    larlite::mcpart <==> simb::MCParticle
    larlite::mcshower <==> sim::MCTruth
    larlite::simch <==> sim::SimChannel
    larlite::rawdigit <==> raw::RawDigit
    larlite::wire <==> recob::Wire
    larlite::hit  <==> recob::Hit
    larlite::seed  <==> recob::Seed
    larlite::cluster <==> recob::Cluster
    larlite::spacepoint <==> recob::SpacePoint
    larlite::track <==> recob::Track
    larlite::shower <==> recob::Shower
    larlite::endpoint2d <==> recob::EndPoint2D
    larlite::vertex <==> recob::Vertex
    larlite::calorimetry <==> anab::Calorimetry
    larlite::partid <==> anab::ParticleID
    larlite::pfpart <==> recob::PFParticle
    larlite::pcaxis <==> recob::PCAxis
    larlite::potsummary <==> sumdata::POTSummary
    larlite::user_info ... LArLite-only custom data product
    larlite::event_ass ... LArLite-only custom data product
\end{lstlisting}
Additional data products, {\ttfamily user\_info} and {\ttfamily event\_ass}, are described later.

All data product class shares the base class {\ttfamily larlite::data\_base} which stores only one variable: {\ttfamily larlite::data::DataType\_t}. This \enum value specifies the type of data stored. For instance, {\ttfamily larlite::hit} data product stores a type {\ttfamily larlite::data::kHit}.

\subsection{What is stored per event?}
Despite a few exceptions (which I describe next), each data product class {\ttfamily T} has a corresponding event-wise collection class
named as {\ttfamily event\_T}. 
For instance, in {\ttfamily cluster.h}, you find there is a class {\ttfamily larlite::cluster} and {\ttfamily larlite::event\_cluster} where the latter is a type {\ttfamily std::vector<larlite::cluster>}. What is stored per event is {\ttfamily std::vector<T>}, just like a collection of object is stored in LArSoft. This collection data product also inherits from {\ttfamily event\_base} which carries additional inforamtion:
\begin{itemize}
\item {\ttfamily unsigned int} Event number
\item {\ttfamily unsigned int} Run number
\item {\ttfamily unsigned int} Sub-Run number
\item {\ttfamily product\_id} Data product ID 
\end{itemize}
Just like everything else, {\ttfamily event\_base} inherits from {\ttfamily data\_base} and hence knows its own data product type. Together with the producer's name string, this forms a unique data product identification instance called {\ttfamily product\_id}. For instance, type {\ttfamily data::kHit} made by ``gaushit'' forms a specific data product ID. {\ttfamily product\_id} is a class defined under:
\begin{lstlisting}
  core/DataFormat/larlite_dataformat_utils.h
\end{lstlisting}
and is simply {\ttfamily std::pair<data::DataType\_t,std::string>} to hold such data product identity.

It should be noted {\it most of} {\ttfamily larlite::event\_T} inherits from {\ttfamily std::vector<T>} with an exception of {\ttfamily event\_ass}, the association data product. In fact there is no such data product called {\ttfamily ass} (of course, it's inappropriate, right?). More details about this data product is described later in Sec.\ref{chap:dataformat:association}.

\subsubsection{Exception to {\ttfamily event\_T}}
Exceptions to this event-wise collection storage scheme include {\ttfamily potsummary}, {\ttfamily mcnu}, and {\ttfamily mctrajectory}.
A reason for {\ttfamily potsummary} is simple: this is a sub-run data type and not an event-wise data type.
In LArSoft, {\ttfamily simb::MCNeutrino} is stored as a part of {\ttfamily simb::MCTruth}. Likewise, {\ttfamily simb::MCTrajectory} is stored as a part of {\ttfamily simb::MCParticle}. LArLite respects this scheme and stores {\ttfamily mcnu} in {\ttfamily mctruth} and {\ttfamily mctrajectory} in {\ttfamily mcpart}.

In case you are unfamiliar with these MC information, though this is a side-track, {\ttfamily mcnu} stores neutrino specific information from neutrino generator. It is a part of {\ttfamily mctruth} because it is created by a generator in general. Similarly, {\ttfamily mctrajectory} is part of {\ttfamily mcpart} because a particle's trajectory points (that is what {\ttfamily mctrajectory} is) belong to a particle information which is {\ttfamily mcpart}.

\subsubsection{More useful info to store?}
If you have some information you think is very useful, let's store it :)
Your ideas can help others. So please share it!

\subsubsection{Adding new data product?}
By all means, go ahead. 
That being said, however, a mistake can break DataFormat which is shared among all users.
So feel free to consult with the author.\\

Alternative solution is to ask the author to add it. 
Believe me, it's so simple and it won't take me $>$10 minutes to add another on the list above.

\subsection{Dynamic Data Container ... {\ttfamily user\_info}}

Sometimes you want a custom data product and store information in the output data root file.
For instance:
\begin{itemize}
\item you want to store some parameter values but not worth making a new data product (say studying a parameter)
\item you want to store some parameter values for certain events/objects, but not all.
\end{itemize}

Then {\ttfamily user\_info} might be handy for you. This data product class contains several {\ttfamily std::map} which allows you to store following basic variable types:
\begin{lstlisting}
    Bool_t
    Int_t
    Double_t
    std::string
    std::vector<Bool_t>
    std::vector<Int_t>
    std::vector<Double_t>
    std::vector<std::string>
\end{lstlisting}
When you store these variables in {\ttfamily user\_info}, you store with a {\ttfamily std::string} key. 
Then you can access the stored value by this key. 
You can use the same key to store different type of variables: mixing is prevented.
If you reuse a key that is already used to store the same type of variable, you will overwrite the value.
There is a function to check if the key is already in use or not, so a user can avoid overwriting the value.\\

Using {\ttfamily std::map} is not great for speed when retrieving the variable. 
But this seems to be a matter of micro-seconds or less, depending on number of keys used.

\subsection{Association}
\label{chap:dataformat:association}
The author has little idea about how exactly association is stored in \ART. But that does not matter.

As you have seen in the previous section, all event-wise data products are stored in terms of {\ttfamily std::vector} collection.
So ultimately what you need to associate one object (A) to another (B) are:
\begin{enumerate}
\item {\ttfamily product\_id} for A and B
\item A set of indexes (A) that point to the right {\ttfamily std::vector} index (B).
\end{enumerate}
When referring an association of A to B, information such as 2 above can be represented as a two-dimensional array of integers.
In LArLite, 2 is represented by {\ttfamily AssSet\_t} defined as:
\begin{lstlisting}
typedef std::vector<std::vector<unsigned int> > AssSet_t;
\end{lstlisting}
The association data product, {\ttfamily event\_ass}, stores {\ttfamily AssSet\_t} for a unique combination
of {\ttfamily product\_id} pair. Note that there is no limitation on how many of such association to be stored in
this data product. Also, two separate instance of {\ttfamily event\_ass} may store a different association information
for an identical {\ttfamily product\_id} pair, although that rarely happens and the author has never seen in a real
practice.

\subsubsection{Accessing a Stored Association}

Retrieving association is as simple as calling the following function:
\begin{lstlisting}
    event_base::association(const& product_id A, const& product_id B)
\end{lstlisting}
You provide a pair of product ID, A and B respectively, and you receive an association of A=>B.

For instance, here's a code snipset to loop over associated hits for clusters to calculate charge sum per cluster and fill a histogram.
\begin{lstlisting}
  auto clusters = storage->get_data<event_cluster>(``fuzzycluster''); // Get clusters
  auto hits     = storage->get_data<event_hit>(``gaushit''); // Get hits
  auto ass_data = storage->get_data<event_ass>(``fuzzycluster''); // Get association

  // Get association cluster => hit
  auto const& cluster_to_hit_ass = ass_data->association(clusters->id(),hits->id());

  // Loop over associated groups of hits
  double cluster_q=0;
  for(auto const& hit_indices : cluster_to_hit_ass)
    cluster_q = 0;
    for(auto const& hit_index : hit_indices)
    cluster_q += (*ev_hit)[hit_index].Charge();
    h->Fill(cluster_q);
  }
\end{lstlisting}
where {\ttfamily storage} is a {\ttfamily storage\_manager} class instance, and {\ttfamily h} is a {\ttfamily TH1D} histogram instance.

It might be also helpful to remember a simple command to dump association information:
\begin{lstlisting}
    ev_cluster.list_association()
\end{lstlisting}
will output stored association information to the text output stream. \\

\subsubsection{Handy Utilities to {\it Find} Associations}
{\ttfamily event\_ass::association} function described above requires a user to provide two product IDs.
Sometimes (actually quite often) you want to ``search'' an association by providing one product ID and a type of the other.
For instance, if I am analyzing a cluster data product made by ``fuzzycluster'', I may not necessarily know the producer
name of an associated hit data product. In that case, all I know for sure is ``I want an associated {\it hit}.''

{\ttfamily event\_ass} implements a few functions to allow you a search with such partial product ID pair information.
Those are:
\begin{lstlisting}
  AssID_t event_ass::find_one_assid(const T& a, const T& b);
  AssID_t event_ass::find_unique_assid(const T& a, const T& b);
  std::vector<AssID_t> event_ass::find_all_assid(const T& a, const T& b);
\end{lstlisting}
These are templated functions, and one of two input argument has to be {\ttfamily product\_id} while the other
can be also {\ttfamily product\_id} or simply {\ttfamily data::DataType\_t}, a data product type.
The return, {\ttfamily AssID\_t} is simply an unsigned integer that identifies a specific product ID pair and the corresponding
association information that can bre retrieved by the following functions:
\begin{lstlisting}
  /// returns a pair of product ID A and B
  std::pair<product_id,product_id> event_ass::association_keys(AssID_t);
  /// returns AssSet_t for a specified association info
  AssSet_t event_ass::association(AssID_t); 
\end{lstlisting}

Note that a function {\ttfamily find\_one\_assid} returns {\it any} one of associations that matches with input argument information.
That means, if there are more than one type of association that matches your request, you do not have a control on which one to be
returned (although that is, again, very rare and the author has never seen such use of association data product). If you wish to
ensure the returned association corresponds to a unique pair of product IDs, you may want to use {\ttfamily find\_unique\_assid}
although this method takes more time to ensure the uniqueness of returned association information. Finally, if you wish to obtain
{\it all} matched association information, you can use {\ttfamily find\_all\_assid}. But again, so far always one association
data product stores (possibly multiple) unique pairs of product IDs.

More practical usecase of association data product usage is described in the later analysis section, and the author suggests to
read there before you write a code to retrieve/use association information.

Please contact the author to ask more questions or suggestions to expand this section.
