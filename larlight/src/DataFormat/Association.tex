This section describes some practical usage of an association information that is initially discussed
in Sec.\ref{chap:dataformat:association}. Note that not everyone needs to know/learn about association
data product. If you don't have a plan to use it yet, the author recommends you to skip this section
and come back when you need to learn it. It's cumbersome. Trust him.

OK fine, you really need it. Let's take a simple example case of accessing a {\ttfamily cluster} data
product produced by ``fuzzycluster''. Say we want to access associated {\ttfamily hit} data product,
which happened to be produced by ``gaushit'' with an association information
The following is an example code to do such thing:
\begin{lstlisting}
  // get cluster
  auto clusters = storage->get_data<event_cluster>(``fuzzycluster'');
  // get hit
  auto hits     = storage->get_data<event_hit>(``gaushit'');
  // get association
  auto ass_data = storage->get_data<event_ass>(``fuzzycluster'');

  // retrieve cluster=>hit association info
  auto const& ass_info = ass_info->association(clusters->id(), hits->id()); 
\end{lstlisting}
There are two typical complaints to above lines of code: 1) too many letters to type, and 2) a user
has to know a producer name of {\ttfamily hit} data product. The latter is particularly cumbersome
because the code does not work if a user provided an wrong hit producer name. So the bottom line
is this is terrible.

We clearly want an ability to ask: ``here's my cluster, give me associated hits.''
In other words, by a user specifying an associated data product type (i.e. {\ttfamily hit}), we
want the utility to go look for an associated information. You might remember a similar functionality
existed in {\ttfamily event\_ass} data product, namely {\ttfamily event\_ass::find\_one\_assid}.
Yes, you could do this:
\begin{lstlisting}
  // get cluster
  auto clusters = storage->get_data<event_cluster>(``fuzzycluster'');
  // get association
  auto ass_data = storage->get_data<event_ass>(``fuzzycluster''); 

  // Search association I want ... clusters=>hit
  auto ass_id = ass_data->find_one_assid(clusters->id(),data::kHit);
  // Get association I want
  auto const& ass_info = ass_data->association(ass_id);
  // Get hit data product
  auto hit_product_id = ass_data->association_keys(ass_id).second;
  auto hits = storage->get_data<event_hit>(ass_id.second);
\end{lstlisting}
Great! This approach ended up 0) more letters to type, and also 1) longer time to process as
the approach involves a ``search'' for a pair of product IDs that matches my request.

Please do not throw away your laptop (or something to the author's office) just yet.
Let's first approach to reduce number of letters to type. For this, we introduce a function
{\ttfamily storage\_manager::find\_one\_ass}.
\begin{lstlisting}
  // get cluster
  auto clusters = storage->get_data<event_cluster>(``fuzzycluster''); 

  // Retrieve hit data product & association at the same time
  event_hits* hits = nullptr;
  auto const& ass_info = storage->find_one_ass(clusters->id(), hits);
\end{lstlisting}

Now the number of letters you have to type is dramatically reduced. But let's be careful
and make sure we understand what we are doing. First of all, the idea is still the same:
you provide key information for a utility to search an association that corresponds to
a specific pair of product IDs, A and B. The product ID of A is given by the first
argument, which is {\ttfamily clusters->id()}. The {\it type} of product ID B is identified
as {\ttfamily hit} because you provided {\ttfamily event\_hit} type pointer in the second
argument. The return of the function is a matched association's {\ttfamily AssSet\_t}, and
when this return is valid the pointer {\ttfamily hits} is pointed to the associated data
product. Prior to this function call, {\ttfamily hits} variable must be a null pointer.

So this all sounds good. Note that this function call is very similar to
{\ttfamily event\_ass::find\_one\_assid}, and indeed {\ttfamily storage\_manager} is
internally using this function to go look for a specific product ID pair.
So how does {\ttfamily storage\_manager} knows {\it which} {\ttfamily event\_ass} data
product to look for requested product ID pair? The answer is {\it it does not know},
and hence perform a linear search in {\it all} {\ttfamily event\_ass} data products that
exist in the input file. This means you potentially read lots of {\ttfamily event\_ass}
data products from a file, and hence causing a huge I/O overhead.

The above mentioned problem can be avoided if you can provide a ``producer name'' of
a specific {\ttfamily event\_ass} data product, which is something you usually know.
In our example, an association from {\ttfamily cluster} to {\ttfamily hit} is created
by the same producer as the {\ttfamily cluster}. In such case, you can specify which
{\ttfamily event\_ass} to search for an association you requested:
\begin{lstlisting}
  // get cluster
  auto clusters = storage->get_data<event_cluster>(``fuzzycluster''); 

  // Retrieve hit data product & association at the same time
  event_hits* hits = nullptr;
  auto const& ass_info = storage->find_one_ass( clusters->id(),
                                                hits,
                                                clusters->id().second );
\end{lstlisting}
As you can see, now the third argument is provided in the {\ttfamily find\_one\_ass} function
which is a string specifying {\ttfamily event\_ass} producer name (which is {\ttfamily cluster}
producer name in this case). This last example runs significantly faster than the previous one.

Finally, we should know the actual timing optimization scale we are talking about here.
A {\ttfamily cluster} to {\ttfamily hit} association is one of the heaviest simply because
there are many objects (i.e. hits) to make an association. A typical time to load this association
in a busy events (like cosmics) is a few hundreds of micro-seconds. If you are running a
computation that might take more than that per event, you might find such timing improvement
rather negligible. Just always know whether it makes sense to perform an optimization or not.

I hope this section described a practical usecase of association data products well enough.
If you think there can be more utility methods to be added, please let the author know!
Also if you need more examples, again just ping the author!



