
LArLight uses \ROOT file as storage because the author is not smart enough to write his own streamer.
In a LArLight \ROOT file, you find {\ttfamily TTree} per stored data type (i.e. {\ttfamily larlight::DATA::DATATYPE}).
A standard approach of using {\ttfamily TTree::SetBranchAddress} works very easily, but LArLight also has dedicated I/O interface to make our life easier (at least that was the author's intention).\\

If you did not find the I/O functionality you want in here, please suggest and help us to improve our interface ;)

\subsection{LArLight \ROOT file structure}
In LArLight \ROOT file, TTrees are created per stored data type. 
You find, for instance, ``ffthit\_tree'' if  {\ttfamily larlight::DATA::FFTHit} data type is stored.

\subsubsection{Accessing ``by hand''}
Here's some simple commands you can try to access in \CINT (or \PyROOT) session:
\begin{lstlisting}
    > root
    root[0] gSystem->Load(``libDataFormat'');
    root[1] TFile* f = TFile::Open(``sample.root'',``READ'');
    root[2] TTree* t = (TTree*)(f->Get(``ffthit\_tree''));
    root[3] t->Show(0);
\end{lstlisting}
Similarly you can try {\ttfamily TTree::Scan} or {\ttfamily TTree::Draw} function.
Note that \ROOT supports class attribute access upon \CINT dictionary loading.
Since we do have \CINT dictionary, you can try accessing functions like {\ttfamily larlight::hit::Charge()} in the string argument you provide to those {\ttfamily TTree} functions.\\

Of course, you can also invoke {\ttfamily TTree::SetBranchAddress} to retrieve data product class pointer.
\begin{lstlisting}
    > root
    root[0] gSystem->Load(``libDataFormat'');
    root[1] TFile* f = TFile::Open(``sample.root'',``READ'');
    root[2] TTree* t = (TTree*)(f->Get(``ffthit_tree''));
    root[3] larlight::event_hit* my_hit_v = new larlight::event_hit;
    root[4] t->SetBranchAddress(``ffthit_branch'',my_hit_v);
    root[5] t->GetEntry(0);
    root[6] my_hit_v->size();
\end{lstlisting}
But this is tedious. Let's use I/O interface class.

\subsection{Framework I/O Interface: {\ttfamily storage\_manager}}

{\ttfamily storage\_manager} is the LArLight's \CPP interface class for \ROOT file I/O.
The work flow is the following:
\begin{itemize}
\item Configure (tell input and/or output file information)
\item Open
\item Loop over events
\item Close
\end{itemize}

We go over some examples in the followings. I assume we have an example file, ``hit\_sample.root''
which contains {\ttfamily larlight::DATA::FFTHit} type data product.

\subsubsection{Simple Example: Opening an input file}
Here's a \CINT script to configure/use {\ttfamily storage\_manager} to open a \ROOT file.
\begin{lstlisting}
{
    gSystem->Load(``libDataFormat'');
    larlight::storage_manager mgr;
    mgr.set_io_mode(mgr.READ);                // Option: READ, WRITE, BOTH
    mgr.add_in_filename(``hit_sample.root''); // Specify input file name
    mgr.set_in_rootdir(``'');                 // Specify TDirectory name if there exists
    if(!mgr.open()) {
      std::cerr << ``Failed to open ROOT file. Aborting.'' << std::endl;
      return 1;
    }
    std::cout << Form(``Retrieved \%d events!'',mgr.get_entries()) << std::endl;
    mgr.close();
    return 0;
}
\end{lstlisting}


\subsubsection{Simple Example: Event loop (Read Only)}
This is an example to run an event loop.
\begin{lstlisting}
{
    gSystem->Load(``libDataFormat'');
    larlight::storage_manager mgr;
    mgr.set_io_mode(mgr.READ);                // Option: READ, WRITE, BOTH
    mgr.add_in_filename(``hit_sample.root''); // Specify input file name
    mgr.set_in_rootdir(``'');                 // Specify TDirectory name if there exists
    if(!mgr.open()) {
      std::cerr << ``Failed to open ROOT file. Aborting.'' << std::endl;
      return 1;
    }
    std::cout << Form(``Retrieved \%d events!'',mgr.get_entries()) << std::endl;
    while(mgr.next_event()) {
      const event_hit* my_hit_v = (const event_hit*)(mgr.get_data(larlight::DATA::FFTHit));
      if(my_hit_v) {
        std::cout << Form(``Found event \%d!'',my_hit_v.event_id()) << std::endl;
      }
    }
    std::cout << ``Closing a file...'' << std::endl;
    mgr.close();
    return 0;
}
\end{lstlisting}

\subsubsection{Simple Example: Event loop (Read \& Write)}
This is an example to run an event loop.
\begin{lstlisting}
{
    gSystem->Load(``libDataFormat'');
    larlight::storage_manager mgr;
    mgr.set_io_mode(mgr.READ);                // Option: READ, WRITE, BOTH
    mgr.add_in_filename(``hit_sample.root''); // Specify input file name
    mgr.set_in_rootdir(``'');                 // Specify TDirectory name if there exists
    mgr.set_out_filename(``out_sample.root'');// Specify output file name
    if(!mgr.open()) {
      std::cerr << ``Failed to open ROOT file. Aborting.'' << std::endl;
      return 1;
    }
    std::cout << Form(``Retrieved \%d events!'',mgr.get_entries()) << std::endl;
    while(mgr.next_event()) {
      const event_hit* my_hit_v = (const event_hit*)(mgr.get_data(larlight::DATA::FFTHit));
      if(my_hit_v) {
        std::cout << Form(``Copying event \%d!'',my_hit_v.event_id()) << std::endl;
      }
    }
    std::cout << ``Closing a file...'' << std::endl;
    mgr.close();
    return 0;
}
\end{lstlisting}
Note that this script makes an exact copy of input file to the output.
If you want to modify it, you can use non-const {\ttfamily event\_hit} pointer in the script above.

\subsubsection{Advanced Example: Optimizing I/O by limiting TTree to read}
Sometimes your LArLight file may contain much more information than you needed.
The file size can be huge if you have stored {\ttfamily RawDigit, Wire, MCTrajectory} and such.
By default, {\ttfamily storage\_manager} reads in all LArLight type {\ttfamily TTree}, and
this can slow down your event loop.

If you don't need it, let's cancel reading in. There is a switch you can enable TTree-wise
read:
\begin{lstlisting}
    storage_manager::set_data_to_read(larlight::DATA::DATA_TYPE type, bool enable);
\end{lstlisting}
The first argument specifies the {\ttfamily TTree} type, and second argument specifies enable (or disable) reading.
Call this function before calling {\ttfamily storage\_manager::open()} function.




