If you never ever want to use an interpreter language like \python and \CINT, then 
you should really skip it.

Still here? Well, sure, still you don't have to use \python and skip this chapter ;) 
But the author strongly recommends, at least, to learn how to use it.
Let us flip the question: why do we use \CINT?
Probably only because \CINT was the only interpreter language (= a language that is ``easy'' to use) 
and \ROOT classes are available in \CINT. But now \ROOT classes are also available in \python, 
and here's why you might want to consider this option.
\begin{itemize}
\item \python is one of the most popular languages. Knowing it, you can even get a job.
\item IMHO it is a better option than \CINT because it is
      \begin{itemize}
      \item far more robust
      \item purely object oriented
      \item much better documented (ask Prof. Google)
      \end{itemize}
\item You can use compiled libraries in \python through \PyROOT. It means your execution speed can be as fast as compiled executable while flexibility and simplicity as an interpreter language remains same.
\end{itemize}
Below we go through a few sections to see how one can use LArLight in \python.

\section{\python 101}
\label{sec:python101}
UNDER CONSTRUCTION

%(*) Mention about interpreter
%




\section{\PyROOT: Using \ROOT Classes in \python}
\label{sec:pyroot}
\PyROOT is a \python interface to \ROOT classes. Basically it makes \ROOT classes
available in your \python module, just like \ROOT \CINT knows about \ROOT classes.

\subsubsection{Obtain \PyROOT}
It is recommended to use \python 2.7 or later (see Sec.\ref{sec:prerequisite}). 
You can refer to Sec.\ref{sec:pyroothelp} To install \PyROOT. Here we assume you have
\PyROOT installed. Check your installation in your \python session:
\begin{lstlisting}
    $> python
    >>> import ROOT
    >>>
\end{lstlisting}
If you see an error with above commands, then you don't have \PyROOT properly set up.

\subsection{Creating and Filling a Histogram in \PyROOT}
Like mentioned above, with \PyROOT, you have an access to \ROOT classes in \python.
Here is an example usage:
\begin{lstlisting}
    $> python
    >>> from ROOT import TH1D
    >>> h=TH1D(`h',`My Histogram; X-axis; Y-axis',100,0,10)
    >>> h.Fill(5)
    >>> h.Draw()
\end{lstlisting}
You should see {\ttfamily TCanvas} popped up with your histogram drawn on the pad.
This is equivalent to the following \CINT commands:
\begin{lstlisting}
    $> root
    root[0] TH1D* h = new TH1D(``h'',``My Histogram; X-axis; Y-axis'',100,0,10)
    root[1] h->Fill(5)
    root[2] h->Draw()
\end{lstlisting}
Let's take a break, back up and review basic facts here...
\begin{itemize}

    \item {\bf \ROOT classes can be imported into \python}
      \begin{itemize}
        \item As advertised. If you want to import {\it ALL} \ROOT classes, you can try
          ``from ROOT import *'' though that will take a little bit longer time to import
          a whole \ROOT classes (takes roughly 1 second).
      \end{itemize}

    \item {\bf \python does not need type declaration}
      \begin{itemize}
        \item When creating a histogram, ``{\ttfamily h}'', in \python, we did not specify it is
          a type of TH1D object. Like \CPP {\ttfamily auto}, \python figures out the type of 
          ``{\ttfamily h}'' from left-hand-side of ``\ttfamily{=}'' operator (i.e. it knows it's TH1D).
      \end{itemize}

    \item {\bf Everything in \python is a pointer (objects are created on {\ttfamily HEAP})}

    \item {\bf \python object attributes are accessed by ``.''}
      \begin{itemize}
        \item Basically you can replace ``$\rightarrow$'' and ``::'' in \CPP with ``.''
      \end{itemize}

\end{itemize}
Especially the fact that we do not need any type declaration is handy. This reduces lots of mistakes
that can otherwise happen in \CINT where one can cast a pointer to wrong type.

\subsection{{\ttfamily C}-array, Reference, and {\ttfamily STL} Container in \PyROOT}
Including myself many people encounter an issue with the usage of {\ttfamily C}-array in \PyROOT.
For instance this happens when your \CPP function takes a {\ttfamily double} pointer which is
treated as an array in the function. Someone who struggled enough with \CPP probably come up and tell you: 
don't write such function. This is because there's no way of knowing the size of provided array inside the 
function, often causing invalid memory access. So it's not a good implementation for public functions. 

Enough comments from me and THAT \CPP guy/gal: here's how you can use {\ttfamily C}-array in \python.
We take an example of constructing \ROOT object {\ttfamily TGraph}.
\begin{lstlisting}
    > python
    >>> from ROOT import *
    >>> from array import array
    >>> x_points = array('d',[0,1,2])
    >>> y_points = array('d',[1,2,3])
    >>> g=TGraph(len(x_points), x_points, y_points)
    >>> g.SetMarkerStyle(22)
    >>> g.SetMarkerSize(2)
    >>> g.Draw(`AP')
\end{lstlisting}
As you can see, we used {\ttfamily array} object as if it is a {\ttfamily C}-array.
The instantiation of {\ttfamily array} takes 2 arguments: 1st is the value type where `d' specifies
a double-precision type while the 2nd argument specifies the length of array and values to be initialized to.
The 2nd argument can be a list which is what I did. In above example, {\ttfamily x\_points} is initialized
to an array of length 3 (as my input list has length 3) with elements initialized to 0, 1, and 2 in respective
order (as those are my input list values). For {\ttfamily y\_points}, I initialized values to 1, 2, and 3.
You should see three points from the {\ttfamily TGraph} drawn on canbas, showing (0,1), (1,2), and (2,3).

Now, what about reference?
You can simply provide a \python object just like you provide \CPP object when using pass-by-reference.
But you might have a problem when the reference is a simple data type such as {\ttfamily double} or {\ttfamily int}.
This is because \ROOT has a wrapper on those simple types. You can instead use \ROOT provided types.
For instance, say I have a function called {\ttfamily FillDouble(double \&value)}. You can do:
\begin{lstlisting}
    >>> import ROOT
    >>> my_value = ROOT.Double(0)
    >>> FillDouble(my_value)
    >>> print my_value
\end{lstlisting}

Finally, if you love using {\ttfamily STL} containers (I do), they are available through \PyROOT.
Here's how you can do this:
\begin{lstlisting}
    > python
    >>> import ROOT
    >>> my_int_vector=ROOT.std.vector(int)()
    >>> my_int_map=ROOT.std.map(int,int)()
\end{lstlisting}
The equivalent of \CPP's {\ttfamily std::vector<int>}, which is a specialization of template {\ttfamily std::vector} class
with {\ttfamily double} type, is {\ttfamily ROOT.std.vector(int)} in the code above. Then I call a constructor, {\ttfamily ()},
to create an object.

If you have more questions abouve these items, contact the author and he is happy to expand this section.
If you are into more cool statistical/mathematical tools, there are extension modules available in \python called 
{\ttfamily numpy} and {\ttfamily scipy}. They are used in science research and private sector, and has a very
robust and fast implementation of computing-intensive routines, probably better than \ROOT.

\subsection{Creating an Array of Objects}
A typical interpreter language like \python has a very useful array module that can hold a set
of various objects. Here is an example \python script to make a list of histograms.
\begin{lstlisting}
    > python
    >>> from ROOT import *
    >>> myHistoArray=[]
    >>> for x in xrange(10):
    >>>     h=TH1D(`h%d' % x, `Histo %d' % x, 100,0,10)
    >>>     myHistoArray.append(h)
\end{lstlisting}
Well, this can be done similarly in \CINT using {\ttfamily std::vector<TH1D*>}. So what is
great about \python? Remember, \python list does not restrict a type of object to be held.
This is because \python is completely object oriented language. See following:
\begin{lstlisting}
    > python
    >>> from ROOT import *
    >>> myDataCollection = []

    >>> h=TH1D(`h',`Histogram',100,0,10)
    >>> g=TGraph(10)
    >>> v=TVector3(0,0,0)

    >>> myDataCollection.append(h)
    >>> myDataCollection.append(g)
    >>> myDataCollection.append(v)

\end{lstlisting}
There is no need of preparing separate container per data type, nor preparing a dedicated
data holder {\ttfamily class} or {\ttfamily struct}. Some \ROOT collection classes can 
also do a similar thing. But recall \python knows each object's type without specifying it.
This makes it very easy to access the list element. In above script, by
{\ttfamily myDataCollection[0]}, \python knows this is TH1D while \CINT requires a correct
type casting.\\

Finally, remember \python is completely object oriented. So even a {\ttfamily class} is an
object. You can make a list of {\ttfamily class} as shown below:
\begin{lstlisting}
    > python
    >>> from ROOT import *
    >>> myClassList = [TH1D, TGraph, TVector3]

    >>> myVector = myClassList[2](0,0,0)
    >>> myGraph  = myClassList[1](10)
    >>> myHisto  = myClassList[0]('h','Histo',100,0,10)
\end{lstlisting}
It looks strange, doesn't it? But this feature of object orientation allows very abstract
way of writing a code, and becomes handy when writing a driver script (i.e. a code that
executes different programs and functions).\\

... MORE DOCUMENTATION ONGOING ...




\section{LArLight Classes}
\label{sec:larlightwithpython}
Like you can access \ROOT classes in \python, LArLight provides a support of
\CINT dictionary generation, which allows you to access LArLight classes from
\python in the very similar way. Let's try to instantiate {\ttfamily storage\_manager} object.
\begin{lstlisting}
    import ROOT
    from ROOT import larlight
    my_storage = storage_manager()
\end{lstlisting}
Easy, right? This section describes why this can be useful for us.

\subsection{Fast Execution}
One downside of using \python is that it can make your code slow if you do a computationally
intensive task, such as running loop like {\ttfamily for}/{\ttfamily while}. Usually we
compile our \CPP code to (1) debug mistakes and (2) optimize execution speed.

But if you have compiled your \CPP code in LArLight, and import your compiled code to use it
in \python, the execution speed of your code remains same as the compiled code. A good example
is an event loop. Recall {\ttfamily ana\_processor::run()} runs a batch event loop. Because
it is a compiled code, whether calling this function in \python through \PyROOT or writing
a \CPP compiled executable to call this function, you get the same speed. 

\PyROOT brings a very handy merging of compiled \CPP libraries with handy scripting language, 
\python. This is fruitful and hence supported in LArLight as much as possible.

\subsection{Easy (possibly dirty) Data Access}
DOCUMENTATION ONGOING


