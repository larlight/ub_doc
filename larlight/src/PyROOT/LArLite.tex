Like you can access \ROOT classes in \python, LArLite provides a support of
\CINT dictionary generation, which allows you to access LArLite classes from
\python in the very similar way. Let's try to instantiate {\ttfamily storage\_manager} object.
\begin{lstlisting}
    import ROOT
    from ROOT import larlite
    my_storage = larlite.storage_manager()
\end{lstlisting}
Easy, right? This section describes why this can be useful for us.

\subsection{Fast Execution}
One downside of using \python is that it can make your code slow if you do a computationally
intensive task, such as running loop like {\ttfamily for}/{\ttfamily while}. Usually we
compile our \CPP code to (1) debug mistakes and (2) optimize execution speed.

But if you have compiled your \CPP code in LArLite, and import your compiled code to use it
in \python, the execution speed of your code remains same as the compiled code. A good example
is an event loop. Recall {\ttfamily ana\_processor::run()} runs a batch event loop. Because
it is a compiled code, whether calling this function in \python through \PyROOT or writing
a \CPP compiled executable to call this function, you get the same speed. 

\PyROOT brings a very handy merging of compiled \CPP libraries with handy scripting language, 
\python. This is fruitful and hence supported in LArLite as much as possible.

\subsection{Easy (possibly dirty) Data Access}
DOCUMENTATION ONGOING
