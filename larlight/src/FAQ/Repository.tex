List of repository related FAQs.

\begin{itemize}

  \item[]{\bf I do not need to commit any code. Can I start w/o having \git account?}
    \begin{itemize}
      \item You can. Try:
        \begin{lstlisting}
          git clone http://github.com/larlight/LArLight LArLight
        \end{lstlisting}
        Keep in mind that, however, github account is free and very easy to make. Also you are motivated to commit your code: your code won't break others' installation as long as you work in your area (see Ch.\ref{chap:generate}).
    \end{itemize}

  \item[]{\bf Can I commit my code?}
    \begin{itemize}
      \item Yes, of course! Here's gentlemen \& ladies' agreement:
        \begin{itemize}
          \item Do not commit a data file (binary, text, \ROOT files or any format that contains ``data'').
          \item Do not commit object/library files (files with {\ttfamily *.o} or {\ttfamily *.so} extension).
          \item If you alter someone else's package, make sure (s)he is aware. Else, create and work in a separate \git branch than the main (shared) one.
        \end{itemize}
    \end{itemize}

  \item[] {\bf Does my commit break LArLight?}
    \begin{itemize}
      \item If you are committing changes/additions to your own package (see Ch.\ref{chap:generate}), then you won't break LArLight. By default, only minimal components are compiled: \Base, \DataFormat, and \Analysis. Your package is not even included in the default compilation set (hence cannot break compilation chain)! 
    \end{itemize}

  \item[] {\bf Does my commit break someone else's code?}
    \begin{itemize}
      \item This is possible if someone else is using your code. But hey, if this person wants to keep a snapshot of your code in a history, (s)he can use older version from \git repository. Keep developing your code ;)
    \end{itemize}

  \item[] {\bf I want to start a new \CPP framework. Can I take LArLight into parts?}
    \begin{itemize}
      \item Go ahead! In fact the author has a few levels of framework templates. Feel free to contact him if you think he can help you.
    \end{itemize}

\end{itemize}
