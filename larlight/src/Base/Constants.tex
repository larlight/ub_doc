
Constants are spread over several header files. 
The author is not sure if that was a good idea, but he thought that might appear more organized.
It is not because making more header files look framework cool and complicated. Definitely not.
A list of header files is shown below:
\begin{itemize}
\item[] {\ttfamily FrameworkConstants.hh} ... constants used to configure this framework
\item[] {\ttfamily DataFormatConstants.hh} ... constants related to data products in this framework
\item[] {\ttfamily MCConstants.hh} ... MC constants copied from LArSoft
\item[] {\ttfamily GeoConstants.hh} ... geo constants copied from LArSoft
\item[] {\ttfamily FEMConstants.hh} ... FEM electronics related constants
\item[] {\ttfamily DecoderConstants.hh} ... Used in separate package (NevisDecoder), you can ignore
\end{itemize}
Following subsections describe what's defined in each of them.

\subsection{FrameworkConstants.hh}
This framework comes with some message service class ({\ttfamily Message}).
This message service is not sophisticated at all but described in the later section.
It works based on message ``levels'', which is specified by {\ttfamily enum MSG::Level}.
This message level and color-coding scheme of each message can be found in this header file.

\subsection{DataFormatConstants.hh}
There are three kinds of important parameters defined in this file.

\subsubsection{Numeric Limit: {\ttfamily const XXX INVALID\_XXX}}
There are list of numeric limits that can be used to label some ``undetermined'' variable.
I follow a strategy used in LArSoft's MC default variables and use maximum possible value
for each variable type. That being said, sometimes LArSoft uses ``999'' or this sort to
mark ``undetermined''. So watch out: that is likely not corrected in your LArLight file
if you copied data contents from LArSoft.

\subsubsection{Data Type: {\ttfamily \enum DATA::DATA\_TYPE}}
Probably most important \enum. This defines a set of data types in LArLight.
You should find, for each type, a corresponding LArSoft data type.
This is used to retrieve data using I/O interface class (described later).

If you added a new data product class, you should add it in this \enum.
You can add a new element anywhere, but the gentlemen's rule is to add it
in the order of basic information => higher level information.
For instance, this order: {\ttfamily Wire, Hit, Cluster, Shower}.

\subsubsection{Data Name: {\ttfamily std::string DATA\_TREE\_NAME[]}}
This is an array of {\ttfamily std::string} to define a corresponding name for
each {\ttfamily DATA::DATA\_TYPE}. The length of the array is, therefore,
{\ttfamily DATA::DATA\_TYPE\_MAX}. This is used to name {\ttfamily TTree}
stored in LArLight \ROOT file and such, described in later sections.

\subsection{MCConstants.hh}
This defines MC constants used in LArSoft data products.
To store same information, LArLight needs these constants, too.
Both constants name and values are exact copy from LArSoft to avoid a confusion.
Copied constants include:
\begin{itemize}
\item[] {\ttfamily Origin\_t} ... specifies particle generator type
\item[] {\ttfamily curr\_type} ... specifies neutrino interaction current type
\item[] {\ttfamily int\_type\_} ... specifies neutrino interaction categories
\end{itemize}
Unfortunately you cannot blame the author for inconsistent naming scheme. 
If you insist, please change in LArSoft! Then we may propagate a change to here.

\subsection{GeoConstants.hh}
This defines two \enum that belong to {\ttfamily geo} in LArSoft that is stored in the data product.
The first is {\ttfamily geo::View\_t} that specifies wire plane IDs.
The second is {\ttfamily geo::SigType\_t} that specifies signal type, induction or collection.

\subsection{FEMConstants.hh}
Some FEM specific constants and \enum are defined in this file.
This is originally written for {\ttfamily NevisDecoder} package and now merged with LArLight.
It has a correspondency to LArSoft's constants in {\ttfamily optdata} namespace, defined under OpticalTypes.h.

All users care, I believe, is {\ttfamily FEM::DISCRIMINATOR} which defines which discriminator type
is fired in PMT-FEM. If you are dealing with TPC-FEM, you can ignore this \enum.





