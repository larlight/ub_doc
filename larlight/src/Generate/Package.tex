Here, I assume we are working under {\ttfamily UserDev/MyRepo}. 
In fact, since it's tedious, I will just refer to {\ttfamily MyRepo}.

\subsection{What is a ``Package''?}
``Package'' is a directory under a repository that is compiled and generates one {\bf shred object library}
(i.e. a file with {\ttfamily .so} extension). This library is loaded at a run-time or linked via compiler as one byte code file.
This gives you a sense of what to include in one package: having too many classes (and especially unrelated classes) in one package
means an extra overhead cost for a run-time loading or linking at compilation. In other words, you probably do not want make one 
library per class, but also not one library for all classes. A package should contain a group of \CPP classes/functions which you
think it makes sense to put together.

\subsection{Making a Simple \CPP Package}

Like advertised many times, LArLite was originally a \CPP project play ground for summer students. 
The author thinks it's very important to have an empty \CPP package generator that comes with build system, 
so that a user can focus on writing the algorithm instead of figuring out how to compile and such.

So here it is: there is a \python script to generate an empty \CPP package:
\begin{lstlisting}
    $LARLITE_BASEDIR/bin/gen_package
\end{lstlisting}
This script takes one input argument which is used to name a ``package'', a directory to be created under {\ttfamily MyRepo}.
This script {\bf needs to be executed somewhere under {\ttfamily MyRepo}} (otherwise you'll see an error message).
One can run this script via alias:
\begin{lstlisting}
    > llgen_package MyProject
\end{lstlisting}
which will create a directory {\ttfamily MyRepo/MyProject} that include minimal set of source codes to compile an empty \CPP class. 
You can have your name choice in place of ``MyProject''. After running the command, try:
\begin{lstlisting}
    > cd MyProject
    > make -j4
\end{lstlisting}
This compiles your code and makes {\ttfamily libMyRepo\_MyProject.so} under {\ttfamily lib} directory.
What you compiled is a \CPP class called ``sample'' defined in {\ttfamily sample.h}. 


\subsection{Using in Interpreter}

So how can you ``use'' this \CPP class? You can write a binary executable code, or try out in \CINT or \PyROOT. Here is an example:
\begin{lstlisting}
    > root
    root[0] sample k;
\end{lstlisting}
Above lines work (i.e. your class instance is created) because \CINT is informed about your class. 

\subsection{``Hello World'' Development}
\label{sec:helloworld}
Let's try a simple modification to your class. Here's an example ``hello world'' program using \CPP class. Open {\ttfamily sample.h} and add a {\ttfamily void} function as shown below:
\begin{lstlisting}
class sample{
public:
  /// Default constructor
  sample(){};
  /// Default destructor
  virtual ~sample(){};
  /// Hello world!
  void HelloWorld() { std::cout << ``Hello World!'' << std::endl; }
};
\end{lstlisting}
Save, close and compile (i.e. type ``make''). Now, try the following in \CINT or \PyROOT:
\begin{lstlisting}
    > root
    root[0] sample k;
    root[1] k.HelloWorld();
    Hello World!
    root[2]
\end{lstlisting}

Whenever you want to start a new project with an empty class, you can come back to this example and create your new \CPP project!



