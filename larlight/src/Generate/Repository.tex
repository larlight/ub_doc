
LArLite supports user code development under \UserDev.
More precisely, it is assumed to be under any path that is set to the value of shell environment variable {\ttfamily \$LARLITE\_USERDEVDIR}. 
For the sake of simplicity we stick with \UserDev in this section.

\subsection{Note About \UserDev}
Important note first:
\begin{itemize}
  \item {\ttfamily UserDev/GNUmakefile}, if it exists, is generated by setup.sh and it does not belong to LArLite repository (ignore it).
  \item Some sub-directories belong to LArLite (see below). Some of these depend on LArLite and some don't. It is useful to note them briefly here.
    \begin{itemize}
        \item {\ttfamily BasicTool} contains useful packages like {\ttfamily GeoAlgo} and has no dependency outside
        \item {\ttfamily SelectionTool} contains useful packages like {\ttfamily ERTool} and depends on {\ttfamily BasicTool}
        \item {\ttfamily RecoTool} contains shower reconstruction code and depends on {\ttfamily core}
        \item {\ttfamily LArLiteApp} contains LArLite application from {\ttfamily GeoAlgo} and {\ttfamily ERTool} 
    \end{itemize}
  \item You can create new directories under \UserDev and that won't bother other users. We'll do this below. 
  This is all done through {\ttfamily .gitignore} rule placed under the top directory. 
\end{itemize}

These features of \UserDev are there so that you feel more free to make a mess (sorry, I meant, to develop code) there!

\subsection{Creating Your Sub-Repository in \UserDev}
\label{sec:devrepo:makenew}
Obviously I cannot just say ``do whatever under \UserDev'' and leave: I would love to support a very easy way to develop code (or make a mess) under \UserDev. I will just show you how to do things here:
\begin{lstlisting}
     > llgen_repository MyRepo
\end{lstlisting}
where the execution command is an alias explained in Sec.\ref{sec:configure}.
This should create a new directory called {\ttfamily MyRepo} under \UserDev. 
That is your new repository. As said, this does not affect LArLite repository. 
If you don't want to keep {\ttfamily MyRepo}, simply ``{\ttfamily rm -r}'' it.

Another thing to note here: your repository will contain code and you will compile them, making a compiled shared object library. 
Your repository name will be used to name your library file (if you follow LArLite code generation scripts described in the followings). 
So pick a name that suits for your purpose. You do not want to pick a generic name that might coincide with some other libraries on your machine.

\subsection{What's in MyRepo?}

Your new repository comes with a GNUmakefile (which does nothing for now) and {\ttfamily doc} 
directory with a doxygen script but empty otherwise. Under this space, you can create your ``packages'' as a 
set of \CPP code to be compiled into a library. We will discuss about more later.

\subsection{Creating Your Sub-Repository in {\ttfamily github}}
In the previous section, we created a new repository {\ttfamily MyRepo} under \UserDev. But that's just a directory on your laptop, and you may want to keep it as your code repository using either {\ttfamily svn} or {\ttfamily github} (or anything else you would like to use). Here, I briefly mention how you can do this using {\ttfamily github} because it's very easy given that you already have a {\ttfamily github} account.

First of all, go to {\ttfamily github.com} and create your own repository: go to your {\ttfamily github} account web page, and click on ``+'' symbol that is toward right-top of the web page. Choose ``New repository''. Then enter the repository name you are about to create. Ideally you may want to choose a somewhat unique name as described in Sec.\ref{sec:devrepo:makenew}. You can always remove your repository if you don't like it.

Then checkout your empty repository. As an example, I use my empty repo called {\ttfamily EmptyRepo}.
\begin{lstlisting}
    > cd $LARLITE_USERDEVDIR
    > git clone git@github.com:drinkingkazu/EmptyRepo EmptyRepo
\end{lstlisting}
Now simply run:
\begin{lstlisting}
    > python bin/gen_new_repository EmptyRepo
    > cd EmptyRepo
    > git add .
    > git commit -m ``new repository''
    > gitpush -u origin master
\end{lstlisting}
and you are done!
From the next time, if you want to install LArLite from scratch, simply checkout your repository under UserDev with the same name.

As said many times, of course, your repository is independent of LArLite repository.



