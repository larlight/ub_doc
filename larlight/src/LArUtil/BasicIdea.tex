All of \LArUtil classes are defined under {\ttfamily larutil} namespace, which is different from the rest of core packages. The author is not sure why it was done so. If you think they should be moved into {\ttfamily larlite} namespace, tell him to do so.

There are two features that all utility classes in \LArUtil shared. These are (1) singleton design pattern, and (2) use \ROOT {\ttfamily TTree} to instantiate data members. These are described briefly in the following subsections.

\subsection{Basic Usage}
Like they are in LArSoft, all of \Geometry, \LArProperties, and \DetectorProperties are implemented as singleton class. That means you cannot access these class instances by invoking their constructor. Instead, all of them have {\ttfamily GetME()} function which returns the constant pointer to its own instance.
\begin{lstlisting}
    > root
    root[0] gSystem->Load(``libLArLite_LArUtil'')
    root[1] larutil::Geometry::GetME()
\end{lstlisting}
or in \PyROOT
\begin{lstlisting}
    > python
    >>> from ROOT import *
    >>> gSystem.Load(``libLArLite_LArUtil'');
    >>> larutil.Geometry.GetME()
\end{lstlisting}
You might say ``I worry about overhead of calling {\ttfamily GetME()} function each time I need it.'' First, this won't take even a micro-second. If your code takes more than 10 micro-seconds to run per this function call, you can discard such worry! Secondly, feel free to store {\ttfamily larutil::Geometry} const pointer so that you don't have to call {\ttfamily GetME()}. But never delete it (i.e. respect it should be const)! 

As you see above, \LArUtil classes are supported in \CINT and \PyROOT just like other LArLite classes. Accessing the geometry or detector property information is, therefore, very starightforward. When you forget how many TPC channels exist in the detector, just try:
\begin{lstlisting}
    > root
    root[0] gSystem->Load(``libLArLite_LArUtil'')
    root[1] larutil::Geometry::GetME()->Nchannels()
    (const unsigned int) 8254
\end{lstlisting}
Or maybe getting the position of PMT channel 1:
\begin{lstlisting}
    > root
    root[0] gSystem->Load(``libLArLite_LArUtil'')
    root[1] Double_t xyz[3]={0.}
    root[2] larutil::Geometry::GetME()->GetOpChannelPosition(1,xyz)
    root[3] std::cout<<xyz[0]<<`` : ``<<xyz[1]<<`` : ``<<xyz[2]<<std::endl;
    -5.755 : 59.0965 : 938.5
    root[4] 
\end{lstlisting}
Or maybe getting the view type of TPC channel 5000
\begin{lstlisting}
    > root
    root[0] gSystem->Load(``libLArLite_LArUtil'')
    root[1] Double_t xyz[3]={0.}
    root[2] larutil::Geometry::GetME()->View(5000)
    (const enum larlite::geo::View_t)2
\end{lstlisting}

\subsection{Support for Other Experiments}
By default all utility classes are set for MicroBooNE experiment. Though the design of {\ttfamily Geometry} is simplified compared to LArSoft, it can support experiments as long as it has only 1 cryostat and 1 TPC. Please make a request and the author can help to support experiments not yet supported. 

Currently another experiment supported is ArgoNeuT. To reconfigure all utilities for ArgoNeuT, you can do:
\begin{lstlisting}
    > root
    root[0] gSystem->Load(``libLArLite_LArUtil'')
    root[1] larutil::LArUtilManager::Reconfigure(larlite::geo::kArgoNeuT)
\end{lstlisting}
The class {\ttfamily LArUtilManager} has a single static function that takes experiment type, which is an {\ttfamily enum} defined under {\ttfamily core/Base/GeoConstants.hh}, and reconfigure all utility classes. If the input experiment type is same as what is already set, then it does nothing. You can call above function at the beginning of your script or {\ttfamily main} function, and the rest of code execution uses the specified experiment type.

\subsection{Custom Geometry}
As described above, utility classes under \LArUtil have their member variables instantiated through \ROOT {\ttfamily TTree}. Accordingly their attribute variables depend on values stored in {\ttfamily TTree}. These {\ttfamily TTree}s are created from LArSoft module and is updated whenever there is a change on the LArSoft side. 

That being said, you can create your own version of TTrees using routines under {\ttfamily LArUtil/bin} directory. There, you find {\ttfamily gen\_tree\_XXX.cc}. Simply type {\ttfamily make} in this directory, and you find a complied executable that can generate a data {\ttfamily TTree} for a utility class {\ttfamily XXX}. The author understands that this way of changing the data member values is certainly cumbersome. If anyone wants a text-file interface, please ask him and he will be happy to implement.

If you have your custom made {\ttfamily TTree} to instantiate the data member, you can feed it in the following manner:
\begin{lstlisting}
    > root
    root[0] gSystem->Load(``libLArLite_LArUtil'')
    root[1] larutil::Geometry::GetME()->SetTreeName(``tree_name'')
    root[2] larutil::Geometry::GetME()->SetFileName(``tree.root'')
    root[3] larutil::Geometry::GetME()->LoadData()
\end{lstlisting}
where we assumed the {\ttfamily TTree} name ``tree\_name'' stored in a \ROOT file ``tree.root''. If you want to re-load data after instantiating the class object, you can also use {\ttfamily SetFileName()} and {\ttfamily SetTreeName()} functions, and then invoke {\ttfamily LoadData()}. This sequence of commands force re-loading of data members from specified input file/tree name.
