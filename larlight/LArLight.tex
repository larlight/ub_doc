\documentclass[a4paper,12pt]{report}
\usepackage{lgrind}
\pagestyle{plain}
\usepackage{amsmath,amsfonts,amssymb,amsthm}
\usepackage{listings}
\usepackage{cancel}
\usepackage{appendix}
\usepackage{fancyhdr}   
\usepackage{url}
\usepackage{path}
\usepackage[dvipdfm]{graphicx}
\usepackage{color}
\usepackage{array}
\usepackage{authblk}
\usepackage[margin=2cm, head=2cm, foot=1cm]{geometry}
\usepackage{verbatim}
\setlength{\topmargin}{-2.5cm}
\setlength{\textwidth}{17cm}
\setlength{\oddsidemargin}{-0.5cm}
\setlength{\evensidemargin}{-0.5cm}
\setlength{\textheight}{22.5cm}
\setlength{\footskip}{1cm}
\renewcommand{\topfraction}{0.9}
\renewcommand{\bottomfraction}{0.8}
\setcounter{topnumber}{2}
\setcounter{bottomnumber}{2}
\setcounter{totalnumber}{4}
\usepackage{hyperref}
\usepackage{pdfpages}
\lstset{columns=fullflexible,basicstyle=\ttfamily}


\renewcommand{\floatpagefraction}{0.9}%

\newcommand{\ROOT}{{\ttfamily ROOT}}
\newcommand{\CINT}{{\ttfamily CINT}}
\newcommand{\CPP}{{\ttfamily C++}}
\newcommand{\gpp}{{\ttfamily g++}}
\newcommand{\clang}{{\ttfamily clang}}
\newcommand{\python}{{\ttfamily python}}
\newcommand{\PyROOT}{{\ttfamily PyROOT}}
\newcommand{\larlight}{{\ttfamily LArLight}}
\newcommand{\larsoft}{{\ttfamily LArSoft}}
\newcommand{\git}{{\ttfamily git}}
\newcommand{\doxygen}{{\ttfamily doxygen}}
\newcommand{\HTML}{{\ttfamily HTML}}
\newcommand{\ART}{{\ttfamily ART}}
\newcommand{\enum}{{\ttfamily enum}}

\newcommand{\cmtool}{{\ttfamily CMTool}}
\newcommand{\cmerge}{{\ttfamily CMergeManager}}
\newcommand{\cmatch}{{\ttfamily CMatchManager}}
\newcommand{\cru}{{\ttfamily ClusterRecoUtil}}
\newcommand{\cpan}{{\ttfamily CPAN}}


\newcommand{\cbalgo}{{\ttfamily CBoolAlgoBase}}
\newcommand{\cmalgo}{{\ttfamily CMAlgo}}
\newcommand{\cbkeeper}{{\ttfamily CBookKeeper}}
\newcommand{\pxutil}{{\ttfamily PxUtil}}
\newcommand{\pxhit}{{\ttfamily PxHit}}
\newcommand{\pxpoint}{{\ttfamily PxPoint}}
\newcommand{\pxline}{{\ttfamily PxLine}}
\newcommand{\geoutil}{{\ttfamily GeometryUtilities}}
\newcommand{\crue}{{\ttfamily CRUException}}


\begin{document}

\title{LArLight \\ \vspace{0.1in} Simple LArSoft Data Analysis Tool}
\date{March 12th, 2014}
\author{Kazuhiro Terao, kazuhiro@nevis.columbia.edu}
\maketitle

\begin{abstract}
LArLight is a simple \CPP + \ROOT analysis framework for LArSoft data. The goal of the framework is to provide (1) a generic \CPP \& \ROOT code development environment with build support on various platform, (2) an access to {\it ala} LArSoft data and utiltiy classes outside LArSoft framework, and (3) analysis framework for data processing. It fits somewhere between LArSoft and {\ttfamily AnalysisTree}. It is suitable for those who want to access details of LArSoft data and develop analysis/reconstruction code quickly. One of LArLight design principles is to be flexible and comfortable to developers. The framework builds on multiple platforms (Linux and OSX Darwin) and supports various ways to execute the compiled code (compiled executable or \CINT/\PyROOT interpreter scripts). It comes with a simple package generator that include a build system. It can be also used just as a \CPP play ground for students. Please feel free to use it for any purpose. The author is happy to be contacted for questions/suggestions anytime.
\end{abstract}

% Table of Contents
\tableofcontents

\newpage
% How-to on setting things up
\chapter{Installation}
\label{chap:installation}

Despite its extreme simplicity let us call {\cmtool} a framework, or FMWK. 
In this chapter we discuss the building blocks of {\cmtool}.

\section{External Dependency}
\label{sec:fmwk:external}

The {\cmtool} development effort is originated from EM shower reconstruction for MicroBooNE experiment, and it is an extension of existing tool set. The following is the brief list of packages or {\CPP} classes on which {\cmtool} depends.
\begin{itemize}
\item[] {\cru}
    \begin{itemize}
    \item larreco/RecoAlg/ClusterRecoUtil in {\larsoft}
    \item UserDev/ClusterRecoUtil in {\larlight}
    \end{itemize}
\item[] {\geoutil}
    \begin{itemize}
    \item lardata/Utilities in {\larsoft}
    \item core/LArUtil in {\larlight}
    \end{itemize}
\item[] {\ttfamily PxUtil}
    \begin{itemize}
    \item lardata/Utilities in {\larsoft}
    \item core/LArUtil in {\larlight}
    \end{itemize}
\end{itemize}
These items are (or should be if not yet) described in a separate document.
The rest of this section briefly covers what they are without going into a technical detail.

\subsection{{\ttfamily ClusterRecoUtil}}
\label{sec:fmwk:external:cru}
Reconstruction parameters for EM shower 2D cluster involves a lot more variety of parameters than those held in a 2D cluster data product defined by a framework ({\larsoft} or {\larlight}). {\cru} is a package that defines code for both calculation of and storage for those parameters. In particular, {\cmtool} uses following {\CPP} classes from {\cru}:
\begin{itemize}
\item {\ttfamily ClusterParamsAlg ... (\cpan)}
  \begin{itemize}
    \item {\cpan} is designed as both parameter computation algorithm and storage class, and is used along this design principle in {\cmtool}. In particular, {\cpan} is created one per input 2D cluster.
  \end{itemize}
\item {\ttfamily cluster\_params}
  \begin{itemize}
    \item {\ttfamily cluster\_params} is a utility storage class for important parameters computed by \cpan. A const reference can be retrieved from {\cpan} instance through {\ttfamily GetParams()} function call. Note, however, one should keep {\cpan} instance rather than {\ttfamily cluster\_params} as a cluster parameter data set (see {\cpan} description above).
  \end{itemize}
\item {\ttfamily CRUException}
  \begin{itemize}
    \item This is a very simple exception class derived from {\ttfamily std::exception}. It is used instead of {\ttfamily cet::exception} to avoid framework dependency. Only additional feature to the base class is that the argument string to the constructor is sent to standard output streem by overriden {\ttfamily std::exception::what()} function.
  \end{itemize}
\end{itemize}

For details of all classes listed above, you should see Ref.\cite{ClusterRecoUtil}.

\subsection{{\ttfamily PxUtil}}
\label{sec:fmwk:external:pxutil}
Both {\cru} and {\cmtool} are designed to be independent of an analysis framework of user's choice (in particular {\larlight} and {\larsoft}). 
As mentioned in Sec.\ref{sec:fmwk:external:cru}, output of {\cpan} (i.e. cluster parameters) is stored with {\ttfamily cluster\_params} instance. 
{\pxutil}, on the other hand, is used to define input data struct to {\cpan} (and hence {\cmtool} and {\cru} in general). {\pxutil} defines three data objects:
\begin{itemize}
  \item {\ttfamily PxPoint}
    \begin{itemize}
    \item represents 2D point and contains wire [cm], time [cm], and plane ID (unsigned char)
    \end{itemize}
  \item {\ttfamily PxHit}
    \begin{itemize}
    \item inhertis from {\ttfamily PxPoint} with an additional attribute to hold charge [ADC], representing 2D hit
    \end{itemize}
  \item {\ttfamily PxLine}
    \begin{itemize}
    \item represents 2D line and contains start and end coordinates, (wire [cm], time [cm]), with a plane ID (unsigned char)
    \end{itemize}
\end{itemize}

It is a FMWK interface module's respnsibility to translate input data product into these format. In particular, this module must convert a vector of FMWK hit data product (= cluster) into a vector of {\pxhit}. This is discussed in FMWK interface chapter (see Ch.\ref{chap:fmwk_interface}).

\subsection{{\ttfamily GeometryUtilities}}
\label{sec:fmwk:external:geoutil}
{\CPP} class {\geoutil} is a utility class that helps to retrieve detector geometry related information such as converting 3D point coordinate into a projected 2D coordinate on a given plane. 
{\cmtool} uses {\geoutil} to interpret geometry related variables in full extent. 
For development of any non-existing function that treats detector geometry information, it is encouraged to develop in {\geoutil} (and certainly not in {\cmtool}). 
Please see Ref.\cite{GeometryUtilities} for more detailed documentation.


\section{FMWK Bulding Blocks}
\label{sec:fmwk:base}
In this section we discuss two core buliding block classes: (1) {\cbalgo} and (2) {\cbkeeper}. Though we discuss these basic {\CPP} classes first, it might be easier to understand these classes after you read about {\cmerge}, described in Sec.{\ref{sec:fmwk:cmerge}}, because there we discuss how {\cmerge} works with practical usage of {\cbalgo} and {\cbkeeper}.

\subsection{Book Keeping: {\cbkeeper}}
\label{sec:fmwk:base:cbk}
{\cbkeeper} inherits from {\ttfamily std::vector<unsigned short>}, and hence it is a vector of integers by itself.
The length of vector can be specified using a constructor argument.

This class is used to track how input clusters are merged/matched into an output. 
Here we focus on merging of input clusters for the simplicity sake.
The design assumes one to perform a merging attempt of 2 clusters at a time.
When one has many clusters to make such attempt on all possible pairs of clusters, it could be troublesome to perform book-keeping for which clusters are merged together.
{\cbkeeper} is designed such that a user can just log two to-be-merged clusters' ID at each attempt, and it holds a summary log of all merged cluster pairs.


Here are technical details.
The index of vector corresponds to the initial cluster ID numbers.
If you have 8 clusters to be subject for merging, {\cbkeeper} instance should be initialized as length 8.
Initially the values stored in the vector is same as their index numbers as shown in the top of Fig.\ref{sec:fmwk:base:cbk}.
One can re-set the state of {\cbkeeper} instance by calling the following function.
\begin{lstlisting}
        void Reset(unsigned short n=0)
\end{lstlisting}

\begin{figure}\begin{center}
\includegraphics[width=14cm]{./src/Pictures/CBookKeeperExample.pdf}
\caption{Example diagram showing how {\cbkeeper} keeps track of input and output cluster IDs. It shows {\cbkeeper} vector index (input cluster ID) and stored values (output cluster ID) for 8 input clusters (top), and how the stored values change as merging takes place. The middle shows the result of merging input cluster 1 and 2, and the bottom is for further merging 2 and 5.}
\end{center}\end{figure}
When one finds 2 clusters to be merged together, (s)he can call {\cbkeeper} attribute function:
\begin{lstlisting}
        void CBookKeeper::Merge(unsigned short, unsigned short)
\end{lstlisting}
with corresponding clusters' IDs (i.e. index numbers). 
Figure \ref{sec:fmwk:base:cbk} shows how the process works for calling {\ttfamily CBookKeeper::Merge()} function twice with arguments (1,2) and (2,5).
As a result, unique output cluster IDs are reduced to 6 because 3 clusters are combined together.

You can see {\cbkeeper} instance allows to retrieve which output cluster (stored value) an input cluster (vector index) ended up with in an intuitive way: simply use {\ttfamily std::vector::at()} method.
However, to create an output cluster, it is probably easier to ask the other way around: given output cluster ID, which input cluster(s) does it contain?
{\cbkeeper} has a simple function to answer this question:
\begin{lstlisting}
        std::vector<std::vector<unsigned short> > GetResult() const 
\end{lstlisting}
which returns a vector with length same as number of output (i.e. merged) clusters.
The value of this vector is {\ttfamily std::vector<unsigned short>} to contain possibly multiple input cluster indexes.
For instance, the return from {\cbkeeper} shown in the bottom of Fig.\ref{sec:fmwk:base:cbk} would be a vector of length 6, and the index1 contains a vector of input cluster ID 1, 2, and 5 as they have been merged into output cluster ID = 1.
Equivalent result can be obtained also by using
\begin{lstlisting}
        void PassResult(std::vector<std::vector<unsigned short> >&)
\end{lstlisting}
if that suits one's usage. 

If you are interested in checking which input cluster(s) included in a particular output (merged) cluster, you can use:
\begin{lstlisting}
        std::vector<unsigned short> GetMergedSet(unsigned short)
\end{lstlisting}
which takes an output cluster index as an argument and returns a vector that contains a list of input clusters that consist the output.

\begin{figure}[ht]\begin{center}
\includegraphics[width=8.3cm]{./src/Pictures/CBookKeeperProhibitExample2.pdf}
\includegraphics[width=8.3cm]{./src/Pictures/CBookKeeperProhibitExample1.pdf}
\caption{Example diagrams showing how {\cbkeeper} reacts to conflict {\ttfamily Merge()} and {\ttfamily ProhibitMerge()} function calls on the input cluster IDs. The left diagram shows {\crue} thrown due to calling {\ttfamily Merge()} on a cluster pair that is registered as merge-prohibit pair. The right diagram shows a result from calling {\ttfamily ProhibitMerge()} on a cluster pair that is already merged.}
\label{sec:fmwk:base:cbk_prohibit}
\end{center}\end{figure}

Finally, {\cmerge} can also prohibit merging of specific two cluster pair(s), and this is incoorporated by {\cbkeeper} as well.
This is, again, to support user to log merging in a mindless manner.
A user may specify specific 2 cluster pair should not be merged. 
Then (s)he can use the following function to register these 2 clusters for merge-prohibit list.
\begin{lstlisting}
        void ProhibitMerge(unsigned short, unsigned short)
\end{lstlisting}
The above function takes input cluster IDs (i.e. index numbers). 
Once 2 clusters are registered as merge-prohibit pair, attempting a function call {\ttfamily Merge()} will result in {\crue} thrown.
This is illustrated in the left of Fig.\ref{sec:fmwk:base:cbk_prohibit}.
Another possible case that results in {\crue} thrown would be calling {\ttfamily ProhibitMerge()} function on two input cluster IDs that are already merged. This is shown in the right hand side of Fig.\ref{sec:fmwk:base:cbk_prohibit}.

To avoid {\crue}, one should register all cluster pairs to prohibit-merge list before trying any merge attempt.
Upon attempting to call {\ttfamily Merge()}, then, one can use the function:
\begin{lstlisting}
        bool MergeAllowed(unsigned short, unsigned short)
\end{lstlisting}
to check if a cluster pair is registered to merge-prohibit list.
Calling this function prior to evaluation of merging (i.e. running some computation algorithm to decide whether to merge or not) may save some computation time and help the overall speed of the program (this is implemented in {\cmerge}).

\subsection{Algorithm Base: {\ttfamily CBoolAlgoBase}}
\label{sec:fmwk:base:cbalgo}
{\cbalgo} is a simple {\CPP} template class and also a base class for an algorithm used for cluster merging and matching by {\cmerge} and {\cmatch} application classes. In this and later sections, we use the term {\cmalgo} to refer to an algorithm class derived from {\cbalgo}.

\begin{figure}[ht]\begin{center}
\includegraphics[width=13cm]{./src/Pictures/CMergeStream.pdf}
\caption{The data processing stream of {\cmerge} framework. The purple allow shows the ordered process flow. It requires an analysis framework interface to convert framework dependent cluster data produdct into a vector of {\pxhit} and hand over to {\cmerge} application. Internally {\cmerge} calls various members of {\cbalgo} instances: one for merging and another for prohibiting. There is a configuration in {\cmerge} to perform iterative approach, which repeats internal loop over possible cluster pairs until no more newly merged cluster is found in the output. The merged cluster groups (i.e. output) can be obtained through {\cbkeeper} instance by an analysis framework interface to form an appropriate output data product.}
\label{sec:fmwk:base:cmerge_stream}
\end{center}\end{figure}

The idea is very simple: given two clusters, it is expected to return a boolean for decision making from the following function.
\begin{lstlisting}
        virtual bool Bool(const cluster::ClusterParamsAlg &, const cluster::ClusterParamsAlg &)
\end{lstlisting}
As you can see, ``clusters'' are handed over in terms of {\cpan} instance.
In {\cmerge}, the cluster merging application class, for instance, this is used for making merge decision as well as prohibit-merge decision (i.e. {\cmerge} holds two {\cbalgo} inherit class instance, one for merging and another for prohibiting).
An algorithm developpers should write his/her own {\CPP} algorithm class by inheriting from {\cbalgo}.
In the derived class, the {\ttfamily Bool()} function implementation should be over-riden. 
There is a script to generate a template (i.e. empty) {\cbalgo} inherit class and everyone is encouraged to use it for starting a new algorithm class. These are:
\begin{itemize}
\item For {\larsoft}
\begin{itemize}
        \item[]{\ttfamily> python \$MRB\_TOP/srcs/larreco/RecoAlg/CMTool/mac/gen\_cmalgo.py NAME}
\end{itemize}
\item For {\larlight}
\begin{itemize}
        \item[]{\ttfamily> python \$MAKE\_TOP\_DIR/UserDev/ClusterStudy/bin/gen\_cmalgo NAME}
\end{itemize}
\end{itemize}
where {\ttfamily NAME} should be replaced by the new algorithm's name. 
Running above command results in creating {\ttfamily CMAlgoNAME.h} and {\ttfamily CMAlgoNAME.cxx} in the appropriate directory.
Then you can proceed to build following the framework's guidance.

As mentioned, {\cbalgo} is operated by {\cmerge} and {\cmatch} applications.
For a given event, there is an internal loop of calling {\ttfamily Bool()} on all possible cluster pairs.
To provide flexibility in algorithm design, those applications also call other functions of {\cbalgo} in its processing.

The following is a list of such {\cbalgo} functions.
\begin{itemize}

\item[]{\ttfamily virtual void EventBegin(const std::vector<cluster::ClusterParamsAlg>\&)}
  \begin{itemize}
    \item Called at the beginning of an event, prior to looping over cluster pairs. The provided function argument is a vector of {\cpan} for all clusters subject for merging. This is where computation that requires the whole, initial cluster sets should be done.
  \end{itemize}

\item[]{\ttfamily virtual void IterationBegin(const std::vector<cluster::ClusterParamsAlg>\&)}
  \begin{itemize}
    \item Called at the beginning of an each loop over cluster pairs. Note {\cmerge} and {\cmatch} can have multiple iteration of looping over possible cluster pairs if configured to do so. For instance {\cmerge} can take iterate over multiple loops till no more to-be-merged  cluster pair is found (see Sec.\ref{sec:fmwk:cmerge} and Sec.\ref{sec:fmwk:cmatch} for details). The argument of this function is an output cluster sets, in terms of {\cpan} instance, from the previous loop. In case of the 1st loop, the provided cluster sets is identical to those provided in the argument of {\ttfamily EventBegin()} function.
  \end{itemize}

\item[]{\ttfamily virtual void IterationEnd()}
  \begin{itemize}
    \item Called at the end of each merge or match attempt loop over all cluster pairs, to clean up or summarize information.
  \end{itemize}

\item[]{\ttfamily virtual void EventEnd()}
  \begin{itemize}
    \item Called at the end of each event, after all iterations over cluster pairs are done.
  \end{itemize}

\end{itemize}

Further, there are two functions that are not yet implemented in {\cmerge} nor {\cmatch} but ideas exist.
Your input on how these should be implemented is welcome and very much appreciated.
\begin{itemize}

\item[]{\ttfamily virtual void Reset()}
  \begin{itemize}
    \item Meant to clear some member variables. But one can do that within {\ttfamily IterationEnd()} and/or {\ttfamily EventEnd()}, so it is not clear where this should be called. Possibly at the end of the whole event processing.
  \end{itemize}

\item[]{\ttfamily virtual void Report()}
  \begin{itemize}
    \item Meant to support some standardized {\ttfamily std::cout} and {\ttfamily std::cerr} report from an algorithm instance. {\cmerge} and {\cmatch} have ability to control such text output for different levels, and there is a hope this {\ttfamily Report()} can be also included there.
  \end{itemize}

\end{itemize}

\paragraph{Tips for writing {\cmalgo}}
First of all, you are encouraged to use {\ttfamily gen\_cmalgo} script as described earlier.
Another thing to note is to reduce a dependency on analysis framework.
This makes it easier to port your code between different frameworks.



\section{{\cmerge}: Cluster Merging Tool}
\label{sec:fmwk:cmerge}

{\cmerge} is a {\CPP} class that performs merging of 2D clusters, and is an application of {\cmalgo} algorithm classes that are inherit from {\cbalgo}, the algorithm base class. In a few sentences, {\cmerge} perform merging by looping over all possible 2D cluster pairs among the given set of clusters. For each pair, {\ttfamily Bool()} function of both merging and prohibit {\cmalgo} are called to inspect possible 2D cluster merging. It uses {\cbkeeper} to keep track of which pair of clusters are merged. For details of {\cbalgo} and {\cbkeeper}, see Sec.\ref{sec:fmwk:base:cbalgo} and Sec.\ref{sec:fmwk:base:cbk} respectively. The overall process stream diagram is shown in Fig.\ref{sec:fmwk:base:cmerge_stream} which we keep referring in this section.

\subsection{Why {\cmerge}?}
Why using {\cmerge}? Why not just writing a program that does everything from scratch?
Be my guest and write your code! That might inspire a better design or strategy toward merging. 
That being said, there are a few reasons to use {\cmerge}.

\paragraph{Task Outside Merging Decision:}
{\cmerge} The core part of merging is to compare 2 clusters and make a decision to merge or not. Outside this process, however, there is a non-negligible overburden of running such algorithm such as (1) keeping track of which cluster pairs are merged and (2) optimizing the loop process of cluster pair comparison. {\cmerge} does those tedious jobs while out-sourcing the decision making to an independent {\CPP} algorithm class that inherit from {\cbalgo}. 

\paragraph{Easy Development via Modulated Design:}
Because {\cmerge} outsource an independent algorithm class to perform merging of a cluster pair, it supports modulated code design. Accordingly each {\cmalgo} class becomes a pure algorithm class. This makes a code development very easy because a new algorithm can be written without re-writing the I/O or book-keeping part. Also it makes life easy to compare different algorithms: you can simple change which algorithm to be attached to {\cmerge} at run-time, and this avoids re-compilation or re-writing of code.

\paragraph{X-Framework Implementation}
{\cmerge} does not use data product type specific to an analysis framework (for instance {\larlight} or {\larsoft}), so it can be ported easily among different frameworks. That means {\cmalgo} developed by one person in one analysis framework can be shared among other frameworks (assuming an algorithm developer follows x-framework design pattern).\\

\subsection{How Does It Work?}
\label{sec:fmwk:cmerge:how}
The core part of decision making is simple: provided 2 clusters, decide whether they should be merged or not.
This is done through {\cmalgo::Bool()} function call:
\begin{lstlisting}
  virtual bool Bool( const cluster::ClusterParamsAlg &, const cluster::ClusterParamsAlg &)
\end{lstlisting}
If the return is {\ttfamily true}, two clusters are merged. Else not.
This is ilustrated in Fig.\ref{sec:fmwk:cmerge:merge_example}.

\begin{figure}[ht]\begin{center}
\includegraphics[width=13cm]{./src/Pictures/MergeExample.pdf}
\caption{Example of 2D clusters in a plane before and after merging attempt. Each oval represents a cluster with cluster 2D axis shown in a black solid arrow. The direction of an arrow represents the direction of a cluster development. Each cluster is labeled with an input cluster ID number while the oval color represents distinct output cluster group. Left: initial state with 4 input clusters. Right: example output after running a merge loop over cluster pairs. }
\label{sec:fmwk:cmerge:merge_example}
\end{center}\end{figure}
{\cmerge} takes care of creating a loop over all possible pairs of 2D clusters and calling the above function of {\cmalgo} for merging.
In the process, book-keeping of which pairs to be merged is recorded using {\cbkeeper}.

\begin{figure}[ht]\begin{center}
\includegraphics[width=13cm]{./src/Pictures/MergeProhibitExample.pdf}
\caption{Example of how prohibit algorithm works. Each oval and solid arrow represents a 2D cluster with its direction while the color represents distinct output (merged) cluster. Input cluster index (ID) is shown in the label with the cluster 2D angle. Right-top shows an output of running {\cmalgo} that merge clusters with angle difference within 40 degrees. Right-bottom shows an output of the same {\cmalgo} for merging but with another {\cmalgo} that prohibits merging of clusters with angle difference larger than 40 degrees.}
\label{sec:fmwk:cmerge:merge_prohibit_example}
\end{center}\end{figure}
Let us consider a case of two merged pairs that shares the same input cluster ID. As described in Sec.\ref{sec:fmwk:base:cbk}, {\cbkeeper} combines two pairs of 2D clusters in such case (see Fig.\ref{sec:fmwk:base:cbk}). This could be problematic sometimes. For instance, consider an example {\cmalgo} that combines two clusters if their start points are very close and the angle between two clusters are within 40 degrees. Further assume we have 3 input clusters that share the start point but have different angles at 20, 40, and -20 degrees. For convention these have input cluster ID 0, 1, and 2 respectively. This example algorithm would conclude pairs (0,1) and (0,2) but not for (1,2). But because merged pairs share cluster ID 0, it ends up all 3 clusters being merged. This is shown in the right-top ilustration of Fig.\ref{sec:fmwk:cmerge:merge_prohibit_example}.

This could be a problem, or may not be: this can be only answered by the algorithm author.
The way to avoid this in {\cmerge} is to implement another {\cmalgo} as {\it merge-prohibit}, or simple {\it prohibit} algorithm.
When prohibit algorithm is provided, {\cmerge} first run over all cluster pairs and execute {\ttfamily Bool()} function to find all cluster pairs that are prohibited to merge.
Then {\cmerge} run over the same set of cluster pairs except for those are prohibited to be merged.
The right-bottom ilustration of Fig.\ref{sec:fmwk:cmerge:merge_prohibit_example} shows how prohibit algorithm affects the result.

The prohibit rule is also inherited to merged cluster pairs. For instance, consider input clusters 0, 1 and 2. The pair (1,2) is prohibited to merge, and (0,1) is found to be merged. Then the merged cluster, (0,1) cannot be merged with 2. That means even if a merge algorithm finds (0,2) to be merged, it cannot be because the prohibit rule is inherited from cluster 1 when 0 and 1 aremerged. This feature brings up a question: what determines 0 to be merged with 1 rather than 2? If the merge algorithm is executed on the pair (0,2) first, then the result would have been different. So what determines a different outcome is an order of cluster pairs in the loop of merging inspection. There are different ordering options available in {\cmerge}. Currently available options are defined as {\enum} type in {\cmerge} class, {\ttfamily CMergePriority\_t} which include following values.
\begin{itemize}
\item[] {\ttfamily kIndex}
  \begin{itemize}
    \item From the smaller to larger input cluster ID, basically following input cluster set ordering
  \end{itemize}

\item[] {\ttfamily kPolyArea}
  \begin{itemize}
    \item From the clusters with smaller polygon area to the larger
  \end{itemize}

\item[] {\ttfamily kChargeSum}
  \begin{itemize}
    \item From the clusters with smaller charge sum to the larrger
  \end{itemize}

\item[] {\ttfamily kNHits}
  \begin{itemize}
    \item From the clusters with smaller number of hits to the larger
  \end{itemize}

\end{itemize}
The ordering parameters including all listed above are taken from {\ttfamily cluster\_params} member attributes.
If more useful parameter is found, one should make a request to make it available. 
It is very simple to implement in {\cmerge}.

In summary, {\cmerge} uses one {\cmalgo} instance for merging and another for merge-prohibit. A possible ambiguity of which cluster pairs to be merged in the presence of prohibited pairs is resolved by defining a priority for cluster pairs in the merging attempt. 

\subsection{Process Flow for Cluster Merging}
\label{sec:fmwk:cmerge:process_flow}
An overall process flow diagram is already shown in Fig.\ref{sec:fmwk:base:cmerge_stream}, so here we describe slightly more details in texts by breaking down into 5 blocks.
\begin{enumerate}
\item Event Preparation
  \begin{itemize}
    \item[] {\cmerge} receives an input cluster in terms of a set of {\pxhit} (see Sec.\ref{sec:fmwk:cmerge:input}). Then it creates {\cpan} instance for each cluster, and calls {\cmalgo::EventBegin()} for merging and prohibit algorithms.
  \end{itemize}

\item Iterative Merge Operation
  \begin{itemize}
    \item[] A loop over cluster pairs for prohibit (if provided) and merge {\cmalgo} can be configured to repeat until no newly merged cluster is found. This is shown in Fig.\ref{sec:fmwk:cmerge:iterative_stream}. If a user does not configure to choose iterative approach, iteration will not happen even if there is a newly merged cluster in the output.
  \end{itemize}

\item Prohibit {\cmalgo} Loop Over Cluster Pairs
  \begin{itemize}
    \item[] See Fig.\ref{sec:fmwk:cmerge:iterative_stream}. Only occurs if prohibit algorithm is provided by a user (optional). Records which cluster pair to be prohibited for merging using {\cbkeeper}.
  \end{itemize}

\item Prohibit {\cmalgo} Loop Over Cluster Pairs
  \begin{itemize}
    \item[] See Fig.\ref{sec:fmwk:cmerge:iterative_stream}. The algorithm's {\ttfamily Bool()} is not executed for a subject pair of clusters if (a) they are prohibited for merging or (b) already inspected in the previous process and not merged (i.e. result should not change). This is to save computation time.
  \end{itemize}

\item Merge Clusters
  \begin{itemize}
    \item[] Based on the outcome of merge/prohibit {\cmalgo} algorithms, actual merging is performed at this step. It creates a new set of {\cpan} for merged clusters. For clusters that are not merged, {\cpan} instance is copied from input to save {\cpan} computation time.
  \end{itemize}

\end{enumerate}

\begin{figure}[ht]\begin{center}
\includegraphics[width=13cm]{./src/Pictures/IterativeMergeApproach.pdf}
\caption{Illustration of how iterative merge approach is processed. Internally a loop over cluster pairs for prohibit and merge algorithm are repeated if (a) user configured to use an iterative approach and (b) newly merged cluster is found.}
\label{sec:fmwk:cmerge:iterative_stream}
\end{center}\end{figure}


\subsection{Setting an Input 2D Clusters}
\label{sec:fmwk:cmerge:input}
Input to {\cmerge} is a set of clusters where each cluster is represented as a vector of hits, or more precisely {\ttfamily std::vector<util::PxHit>}. This input can be set by the following function:
\begin{lstlisting}
  void SetClusters(const std::vector<std::vector<util::PxHit> > &)
\end{lstlisting}
where the argument is a vector of 2D clusters. When this function is called, {\cmerge} internally calls
\begin{lstlisting}
  void Reset()
\end{lstlisting}
which initialize previously stored input clusters, output clusters and {\cbkeeper} instance.
Then it creates {\cpan} instance per input 2D cluster using a set of {\pxhit}.
This completes setting (or initiating) the input 2D cluster sets to be used for internal processing.

\subsection{Setting Merge/Prohibit {\cmalgo} Instances}
{\cmalgo} instance is necessary for merging and optional for prohibit. Call the function,
\begin{lstlisting}
  void AddMergeAlg(CBoolAlgoBase* algo)
\end{lstlisting}
to set {\cmalgo} for merging, and 
\begin{lstlisting}
  void AddSeparateAlgo(CBoolAlgoBase* algo)
\end{lstlisting}
to set {\cmalgo} for merge-prohibit.

\subsection{Setting Run Configuration}
There are only 3 run options to configure for {\cmerge}.

\paragraph{Verbosity:} 
There are 5 verbosity levels a user can configure as listed below:
\begin{itemize}
\item[] {\ttfamily kNone}
  \begin{itemize}
    \item No report from {\cmerge}
  \end{itemize}

\item[] {\ttfamily kPerEvent}
  \begin{itemize}
    \item Report made at the end of even processing
  \end{itemize}

\item[] {\ttfamily kPerIteration}
  \begin{itemize}
    \item Report made per merge iteration (see Sec.\ref{sec:fmwk:cmerge:process_flow}). This could be quite verbose.      
  \end{itemize}

\item[] {\ttfamily kPerMerging}
  \begin{itemize}
    \item Report made per {\cmalgo::Bool()} execution (see Sec.\ref{sec:fmwk:cmerge:process_flow}). This is even more verbose than {\ttfamily kPerIteration}.
  \end{itemize}

\end{itemize}
These are {\enum} type, {\ttfamily CMergeMSGLevel\_t}, defined under {\cmerge} class. 
The verbosity level can be set by the function:
\begin{lstlisting}
  void DebugMode(CMergeMSGLevel_t level)
\end{lstlisting}

\paragraph{Iterative Approach:}
As described in Sec.\ref{sec:fmwk:cmerge:process_flow}, {\cmerge} can be configured to attempt iterative approach for merging. This can be enabled by the following function call.
\begin{lstlisting}
  void MergeTillConverge(bool)
\end{lstlisting}
Providing {\ttfamily true} as an argument will enable this option. Disabled if {\ttfamily false} is provided.

\paragraph{Merge Priority:}
As described in Sec.\ref{sec:fmwk:cmerge:how}, ordering of cluster pairs for merging inspection matters to the result.
Options are also described in the same section. This can be set by the following function.
\begin{lstlisting}
  void SetMergePriority(CMergePriority_t level)
\end{lstlisting}

\subsection{Execute Merging}
To execute merging, simply call
\begin{lstlisting}
  void Process()
\end{lstlisting}

\subsection{Obtaining Result}
There are two kinds of results that may be useful for a user. First is a resulting (i.e. merged) set of clusters in terms of {\cpan}. This is useful for further processing as various cluster parameters are already computed as a part of {\cpan}. You can execute
\begin{lstlisting}
  const std::vector<cluster::ClusterParamsAlg>& GetClusters() const
\end{lstlisting}
to obtain such result set.

Second is a mapping of input cluster IDs (i.e. index) to the output (merged clusters). 
As described in Sec.\ref{sec:fmwk:base:cbk}, this can be easily obtained through {\cbkeeper} instance.
You can obtain the {\cbkeeper} instance held by {\cmerge} through the following function.
\begin{lstlisting}
  const CBookKeeper& GetBookKeeper() const
\end{lstlisting}


\section{{\cmatch}: Cluster Matching Tool}
\label{sec:fmwk:cmatch}
Tool not introduced yet in {\larsoft}. Will be documented when this is done.




% Overall introduction
\chapter{Introdcution}
\label{chap:introduction}

Despite its extreme simplicity let us call {\cmtool} a framework, or FMWK. 
In this chapter we discuss the building blocks of {\cmtool}.

\section{External Dependency}
\label{sec:fmwk:external}

The {\cmtool} development effort is originated from EM shower reconstruction for MicroBooNE experiment, and it is an extension of existing tool set. The following is the brief list of packages or {\CPP} classes on which {\cmtool} depends.
\begin{itemize}
\item[] {\cru}
    \begin{itemize}
    \item larreco/RecoAlg/ClusterRecoUtil in {\larsoft}
    \item UserDev/ClusterRecoUtil in {\larlight}
    \end{itemize}
\item[] {\geoutil}
    \begin{itemize}
    \item lardata/Utilities in {\larsoft}
    \item core/LArUtil in {\larlight}
    \end{itemize}
\item[] {\ttfamily PxUtil}
    \begin{itemize}
    \item lardata/Utilities in {\larsoft}
    \item core/LArUtil in {\larlight}
    \end{itemize}
\end{itemize}
These items are (or should be if not yet) described in a separate document.
The rest of this section briefly covers what they are without going into a technical detail.

\subsection{{\ttfamily ClusterRecoUtil}}
\label{sec:fmwk:external:cru}
Reconstruction parameters for EM shower 2D cluster involves a lot more variety of parameters than those held in a 2D cluster data product defined by a framework ({\larsoft} or {\larlight}). {\cru} is a package that defines code for both calculation of and storage for those parameters. In particular, {\cmtool} uses following {\CPP} classes from {\cru}:
\begin{itemize}
\item {\ttfamily ClusterParamsAlg ... (\cpan)}
  \begin{itemize}
    \item {\cpan} is designed as both parameter computation algorithm and storage class, and is used along this design principle in {\cmtool}. In particular, {\cpan} is created one per input 2D cluster.
  \end{itemize}
\item {\ttfamily cluster\_params}
  \begin{itemize}
    \item {\ttfamily cluster\_params} is a utility storage class for important parameters computed by \cpan. A const reference can be retrieved from {\cpan} instance through {\ttfamily GetParams()} function call. Note, however, one should keep {\cpan} instance rather than {\ttfamily cluster\_params} as a cluster parameter data set (see {\cpan} description above).
  \end{itemize}
\item {\ttfamily CRUException}
  \begin{itemize}
    \item This is a very simple exception class derived from {\ttfamily std::exception}. It is used instead of {\ttfamily cet::exception} to avoid framework dependency. Only additional feature to the base class is that the argument string to the constructor is sent to standard output streem by overriden {\ttfamily std::exception::what()} function.
  \end{itemize}
\end{itemize}

For details of all classes listed above, you should see Ref.\cite{ClusterRecoUtil}.

\subsection{{\ttfamily PxUtil}}
\label{sec:fmwk:external:pxutil}
Both {\cru} and {\cmtool} are designed to be independent of an analysis framework of user's choice (in particular {\larlight} and {\larsoft}). 
As mentioned in Sec.\ref{sec:fmwk:external:cru}, output of {\cpan} (i.e. cluster parameters) is stored with {\ttfamily cluster\_params} instance. 
{\pxutil}, on the other hand, is used to define input data struct to {\cpan} (and hence {\cmtool} and {\cru} in general). {\pxutil} defines three data objects:
\begin{itemize}
  \item {\ttfamily PxPoint}
    \begin{itemize}
    \item represents 2D point and contains wire [cm], time [cm], and plane ID (unsigned char)
    \end{itemize}
  \item {\ttfamily PxHit}
    \begin{itemize}
    \item inhertis from {\ttfamily PxPoint} with an additional attribute to hold charge [ADC], representing 2D hit
    \end{itemize}
  \item {\ttfamily PxLine}
    \begin{itemize}
    \item represents 2D line and contains start and end coordinates, (wire [cm], time [cm]), with a plane ID (unsigned char)
    \end{itemize}
\end{itemize}

It is a FMWK interface module's respnsibility to translate input data product into these format. In particular, this module must convert a vector of FMWK hit data product (= cluster) into a vector of {\pxhit}. This is discussed in FMWK interface chapter (see Ch.\ref{chap:fmwk_interface}).

\subsection{{\ttfamily GeometryUtilities}}
\label{sec:fmwk:external:geoutil}
{\CPP} class {\geoutil} is a utility class that helps to retrieve detector geometry related information such as converting 3D point coordinate into a projected 2D coordinate on a given plane. 
{\cmtool} uses {\geoutil} to interpret geometry related variables in full extent. 
For development of any non-existing function that treats detector geometry information, it is encouraged to develop in {\geoutil} (and certainly not in {\cmtool}). 
Please see Ref.\cite{GeometryUtilities} for more detailed documentation.


\section{FMWK Bulding Blocks}
\label{sec:fmwk:base}
In this section we discuss two core buliding block classes: (1) {\cbalgo} and (2) {\cbkeeper}. Though we discuss these basic {\CPP} classes first, it might be easier to understand these classes after you read about {\cmerge}, described in Sec.{\ref{sec:fmwk:cmerge}}, because there we discuss how {\cmerge} works with practical usage of {\cbalgo} and {\cbkeeper}.

\subsection{Book Keeping: {\cbkeeper}}
\label{sec:fmwk:base:cbk}
{\cbkeeper} inherits from {\ttfamily std::vector<unsigned short>}, and hence it is a vector of integers by itself.
The length of vector can be specified using a constructor argument.

This class is used to track how input clusters are merged/matched into an output. 
Here we focus on merging of input clusters for the simplicity sake.
The design assumes one to perform a merging attempt of 2 clusters at a time.
When one has many clusters to make such attempt on all possible pairs of clusters, it could be troublesome to perform book-keeping for which clusters are merged together.
{\cbkeeper} is designed such that a user can just log two to-be-merged clusters' ID at each attempt, and it holds a summary log of all merged cluster pairs.


Here are technical details.
The index of vector corresponds to the initial cluster ID numbers.
If you have 8 clusters to be subject for merging, {\cbkeeper} instance should be initialized as length 8.
Initially the values stored in the vector is same as their index numbers as shown in the top of Fig.\ref{sec:fmwk:base:cbk}.
One can re-set the state of {\cbkeeper} instance by calling the following function.
\begin{lstlisting}
        void Reset(unsigned short n=0)
\end{lstlisting}

\begin{figure}\begin{center}
\includegraphics[width=14cm]{./src/Pictures/CBookKeeperExample.pdf}
\caption{Example diagram showing how {\cbkeeper} keeps track of input and output cluster IDs. It shows {\cbkeeper} vector index (input cluster ID) and stored values (output cluster ID) for 8 input clusters (top), and how the stored values change as merging takes place. The middle shows the result of merging input cluster 1 and 2, and the bottom is for further merging 2 and 5.}
\end{center}\end{figure}
When one finds 2 clusters to be merged together, (s)he can call {\cbkeeper} attribute function:
\begin{lstlisting}
        void CBookKeeper::Merge(unsigned short, unsigned short)
\end{lstlisting}
with corresponding clusters' IDs (i.e. index numbers). 
Figure \ref{sec:fmwk:base:cbk} shows how the process works for calling {\ttfamily CBookKeeper::Merge()} function twice with arguments (1,2) and (2,5).
As a result, unique output cluster IDs are reduced to 6 because 3 clusters are combined together.

You can see {\cbkeeper} instance allows to retrieve which output cluster (stored value) an input cluster (vector index) ended up with in an intuitive way: simply use {\ttfamily std::vector::at()} method.
However, to create an output cluster, it is probably easier to ask the other way around: given output cluster ID, which input cluster(s) does it contain?
{\cbkeeper} has a simple function to answer this question:
\begin{lstlisting}
        std::vector<std::vector<unsigned short> > GetResult() const 
\end{lstlisting}
which returns a vector with length same as number of output (i.e. merged) clusters.
The value of this vector is {\ttfamily std::vector<unsigned short>} to contain possibly multiple input cluster indexes.
For instance, the return from {\cbkeeper} shown in the bottom of Fig.\ref{sec:fmwk:base:cbk} would be a vector of length 6, and the index1 contains a vector of input cluster ID 1, 2, and 5 as they have been merged into output cluster ID = 1.
Equivalent result can be obtained also by using
\begin{lstlisting}
        void PassResult(std::vector<std::vector<unsigned short> >&)
\end{lstlisting}
if that suits one's usage. 

If you are interested in checking which input cluster(s) included in a particular output (merged) cluster, you can use:
\begin{lstlisting}
        std::vector<unsigned short> GetMergedSet(unsigned short)
\end{lstlisting}
which takes an output cluster index as an argument and returns a vector that contains a list of input clusters that consist the output.

\begin{figure}[ht]\begin{center}
\includegraphics[width=8.3cm]{./src/Pictures/CBookKeeperProhibitExample2.pdf}
\includegraphics[width=8.3cm]{./src/Pictures/CBookKeeperProhibitExample1.pdf}
\caption{Example diagrams showing how {\cbkeeper} reacts to conflict {\ttfamily Merge()} and {\ttfamily ProhibitMerge()} function calls on the input cluster IDs. The left diagram shows {\crue} thrown due to calling {\ttfamily Merge()} on a cluster pair that is registered as merge-prohibit pair. The right diagram shows a result from calling {\ttfamily ProhibitMerge()} on a cluster pair that is already merged.}
\label{sec:fmwk:base:cbk_prohibit}
\end{center}\end{figure}

Finally, {\cmerge} can also prohibit merging of specific two cluster pair(s), and this is incoorporated by {\cbkeeper} as well.
This is, again, to support user to log merging in a mindless manner.
A user may specify specific 2 cluster pair should not be merged. 
Then (s)he can use the following function to register these 2 clusters for merge-prohibit list.
\begin{lstlisting}
        void ProhibitMerge(unsigned short, unsigned short)
\end{lstlisting}
The above function takes input cluster IDs (i.e. index numbers). 
Once 2 clusters are registered as merge-prohibit pair, attempting a function call {\ttfamily Merge()} will result in {\crue} thrown.
This is illustrated in the left of Fig.\ref{sec:fmwk:base:cbk_prohibit}.
Another possible case that results in {\crue} thrown would be calling {\ttfamily ProhibitMerge()} function on two input cluster IDs that are already merged. This is shown in the right hand side of Fig.\ref{sec:fmwk:base:cbk_prohibit}.

To avoid {\crue}, one should register all cluster pairs to prohibit-merge list before trying any merge attempt.
Upon attempting to call {\ttfamily Merge()}, then, one can use the function:
\begin{lstlisting}
        bool MergeAllowed(unsigned short, unsigned short)
\end{lstlisting}
to check if a cluster pair is registered to merge-prohibit list.
Calling this function prior to evaluation of merging (i.e. running some computation algorithm to decide whether to merge or not) may save some computation time and help the overall speed of the program (this is implemented in {\cmerge}).

\subsection{Algorithm Base: {\ttfamily CBoolAlgoBase}}
\label{sec:fmwk:base:cbalgo}
{\cbalgo} is a simple {\CPP} template class and also a base class for an algorithm used for cluster merging and matching by {\cmerge} and {\cmatch} application classes. In this and later sections, we use the term {\cmalgo} to refer to an algorithm class derived from {\cbalgo}.

\begin{figure}[ht]\begin{center}
\includegraphics[width=13cm]{./src/Pictures/CMergeStream.pdf}
\caption{The data processing stream of {\cmerge} framework. The purple allow shows the ordered process flow. It requires an analysis framework interface to convert framework dependent cluster data produdct into a vector of {\pxhit} and hand over to {\cmerge} application. Internally {\cmerge} calls various members of {\cbalgo} instances: one for merging and another for prohibiting. There is a configuration in {\cmerge} to perform iterative approach, which repeats internal loop over possible cluster pairs until no more newly merged cluster is found in the output. The merged cluster groups (i.e. output) can be obtained through {\cbkeeper} instance by an analysis framework interface to form an appropriate output data product.}
\label{sec:fmwk:base:cmerge_stream}
\end{center}\end{figure}

The idea is very simple: given two clusters, it is expected to return a boolean for decision making from the following function.
\begin{lstlisting}
        virtual bool Bool(const cluster::ClusterParamsAlg &, const cluster::ClusterParamsAlg &)
\end{lstlisting}
As you can see, ``clusters'' are handed over in terms of {\cpan} instance.
In {\cmerge}, the cluster merging application class, for instance, this is used for making merge decision as well as prohibit-merge decision (i.e. {\cmerge} holds two {\cbalgo} inherit class instance, one for merging and another for prohibiting).
An algorithm developpers should write his/her own {\CPP} algorithm class by inheriting from {\cbalgo}.
In the derived class, the {\ttfamily Bool()} function implementation should be over-riden. 
There is a script to generate a template (i.e. empty) {\cbalgo} inherit class and everyone is encouraged to use it for starting a new algorithm class. These are:
\begin{itemize}
\item For {\larsoft}
\begin{itemize}
        \item[]{\ttfamily> python \$MRB\_TOP/srcs/larreco/RecoAlg/CMTool/mac/gen\_cmalgo.py NAME}
\end{itemize}
\item For {\larlight}
\begin{itemize}
        \item[]{\ttfamily> python \$MAKE\_TOP\_DIR/UserDev/ClusterStudy/bin/gen\_cmalgo NAME}
\end{itemize}
\end{itemize}
where {\ttfamily NAME} should be replaced by the new algorithm's name. 
Running above command results in creating {\ttfamily CMAlgoNAME.h} and {\ttfamily CMAlgoNAME.cxx} in the appropriate directory.
Then you can proceed to build following the framework's guidance.

As mentioned, {\cbalgo} is operated by {\cmerge} and {\cmatch} applications.
For a given event, there is an internal loop of calling {\ttfamily Bool()} on all possible cluster pairs.
To provide flexibility in algorithm design, those applications also call other functions of {\cbalgo} in its processing.

The following is a list of such {\cbalgo} functions.
\begin{itemize}

\item[]{\ttfamily virtual void EventBegin(const std::vector<cluster::ClusterParamsAlg>\&)}
  \begin{itemize}
    \item Called at the beginning of an event, prior to looping over cluster pairs. The provided function argument is a vector of {\cpan} for all clusters subject for merging. This is where computation that requires the whole, initial cluster sets should be done.
  \end{itemize}

\item[]{\ttfamily virtual void IterationBegin(const std::vector<cluster::ClusterParamsAlg>\&)}
  \begin{itemize}
    \item Called at the beginning of an each loop over cluster pairs. Note {\cmerge} and {\cmatch} can have multiple iteration of looping over possible cluster pairs if configured to do so. For instance {\cmerge} can take iterate over multiple loops till no more to-be-merged  cluster pair is found (see Sec.\ref{sec:fmwk:cmerge} and Sec.\ref{sec:fmwk:cmatch} for details). The argument of this function is an output cluster sets, in terms of {\cpan} instance, from the previous loop. In case of the 1st loop, the provided cluster sets is identical to those provided in the argument of {\ttfamily EventBegin()} function.
  \end{itemize}

\item[]{\ttfamily virtual void IterationEnd()}
  \begin{itemize}
    \item Called at the end of each merge or match attempt loop over all cluster pairs, to clean up or summarize information.
  \end{itemize}

\item[]{\ttfamily virtual void EventEnd()}
  \begin{itemize}
    \item Called at the end of each event, after all iterations over cluster pairs are done.
  \end{itemize}

\end{itemize}

Further, there are two functions that are not yet implemented in {\cmerge} nor {\cmatch} but ideas exist.
Your input on how these should be implemented is welcome and very much appreciated.
\begin{itemize}

\item[]{\ttfamily virtual void Reset()}
  \begin{itemize}
    \item Meant to clear some member variables. But one can do that within {\ttfamily IterationEnd()} and/or {\ttfamily EventEnd()}, so it is not clear where this should be called. Possibly at the end of the whole event processing.
  \end{itemize}

\item[]{\ttfamily virtual void Report()}
  \begin{itemize}
    \item Meant to support some standardized {\ttfamily std::cout} and {\ttfamily std::cerr} report from an algorithm instance. {\cmerge} and {\cmatch} have ability to control such text output for different levels, and there is a hope this {\ttfamily Report()} can be also included there.
  \end{itemize}

\end{itemize}

\paragraph{Tips for writing {\cmalgo}}
First of all, you are encouraged to use {\ttfamily gen\_cmalgo} script as described earlier.
Another thing to note is to reduce a dependency on analysis framework.
This makes it easier to port your code between different frameworks.



\section{{\cmerge}: Cluster Merging Tool}
\label{sec:fmwk:cmerge}

{\cmerge} is a {\CPP} class that performs merging of 2D clusters, and is an application of {\cmalgo} algorithm classes that are inherit from {\cbalgo}, the algorithm base class. In a few sentences, {\cmerge} perform merging by looping over all possible 2D cluster pairs among the given set of clusters. For each pair, {\ttfamily Bool()} function of both merging and prohibit {\cmalgo} are called to inspect possible 2D cluster merging. It uses {\cbkeeper} to keep track of which pair of clusters are merged. For details of {\cbalgo} and {\cbkeeper}, see Sec.\ref{sec:fmwk:base:cbalgo} and Sec.\ref{sec:fmwk:base:cbk} respectively. The overall process stream diagram is shown in Fig.\ref{sec:fmwk:base:cmerge_stream} which we keep referring in this section.

\subsection{Why {\cmerge}?}
Why using {\cmerge}? Why not just writing a program that does everything from scratch?
Be my guest and write your code! That might inspire a better design or strategy toward merging. 
That being said, there are a few reasons to use {\cmerge}.

\paragraph{Task Outside Merging Decision:}
{\cmerge} The core part of merging is to compare 2 clusters and make a decision to merge or not. Outside this process, however, there is a non-negligible overburden of running such algorithm such as (1) keeping track of which cluster pairs are merged and (2) optimizing the loop process of cluster pair comparison. {\cmerge} does those tedious jobs while out-sourcing the decision making to an independent {\CPP} algorithm class that inherit from {\cbalgo}. 

\paragraph{Easy Development via Modulated Design:}
Because {\cmerge} outsource an independent algorithm class to perform merging of a cluster pair, it supports modulated code design. Accordingly each {\cmalgo} class becomes a pure algorithm class. This makes a code development very easy because a new algorithm can be written without re-writing the I/O or book-keeping part. Also it makes life easy to compare different algorithms: you can simple change which algorithm to be attached to {\cmerge} at run-time, and this avoids re-compilation or re-writing of code.

\paragraph{X-Framework Implementation}
{\cmerge} does not use data product type specific to an analysis framework (for instance {\larlight} or {\larsoft}), so it can be ported easily among different frameworks. That means {\cmalgo} developed by one person in one analysis framework can be shared among other frameworks (assuming an algorithm developer follows x-framework design pattern).\\

\subsection{How Does It Work?}
\label{sec:fmwk:cmerge:how}
The core part of decision making is simple: provided 2 clusters, decide whether they should be merged or not.
This is done through {\cmalgo::Bool()} function call:
\begin{lstlisting}
  virtual bool Bool( const cluster::ClusterParamsAlg &, const cluster::ClusterParamsAlg &)
\end{lstlisting}
If the return is {\ttfamily true}, two clusters are merged. Else not.
This is ilustrated in Fig.\ref{sec:fmwk:cmerge:merge_example}.

\begin{figure}[ht]\begin{center}
\includegraphics[width=13cm]{./src/Pictures/MergeExample.pdf}
\caption{Example of 2D clusters in a plane before and after merging attempt. Each oval represents a cluster with cluster 2D axis shown in a black solid arrow. The direction of an arrow represents the direction of a cluster development. Each cluster is labeled with an input cluster ID number while the oval color represents distinct output cluster group. Left: initial state with 4 input clusters. Right: example output after running a merge loop over cluster pairs. }
\label{sec:fmwk:cmerge:merge_example}
\end{center}\end{figure}
{\cmerge} takes care of creating a loop over all possible pairs of 2D clusters and calling the above function of {\cmalgo} for merging.
In the process, book-keeping of which pairs to be merged is recorded using {\cbkeeper}.

\begin{figure}[ht]\begin{center}
\includegraphics[width=13cm]{./src/Pictures/MergeProhibitExample.pdf}
\caption{Example of how prohibit algorithm works. Each oval and solid arrow represents a 2D cluster with its direction while the color represents distinct output (merged) cluster. Input cluster index (ID) is shown in the label with the cluster 2D angle. Right-top shows an output of running {\cmalgo} that merge clusters with angle difference within 40 degrees. Right-bottom shows an output of the same {\cmalgo} for merging but with another {\cmalgo} that prohibits merging of clusters with angle difference larger than 40 degrees.}
\label{sec:fmwk:cmerge:merge_prohibit_example}
\end{center}\end{figure}
Let us consider a case of two merged pairs that shares the same input cluster ID. As described in Sec.\ref{sec:fmwk:base:cbk}, {\cbkeeper} combines two pairs of 2D clusters in such case (see Fig.\ref{sec:fmwk:base:cbk}). This could be problematic sometimes. For instance, consider an example {\cmalgo} that combines two clusters if their start points are very close and the angle between two clusters are within 40 degrees. Further assume we have 3 input clusters that share the start point but have different angles at 20, 40, and -20 degrees. For convention these have input cluster ID 0, 1, and 2 respectively. This example algorithm would conclude pairs (0,1) and (0,2) but not for (1,2). But because merged pairs share cluster ID 0, it ends up all 3 clusters being merged. This is shown in the right-top ilustration of Fig.\ref{sec:fmwk:cmerge:merge_prohibit_example}.

This could be a problem, or may not be: this can be only answered by the algorithm author.
The way to avoid this in {\cmerge} is to implement another {\cmalgo} as {\it merge-prohibit}, or simple {\it prohibit} algorithm.
When prohibit algorithm is provided, {\cmerge} first run over all cluster pairs and execute {\ttfamily Bool()} function to find all cluster pairs that are prohibited to merge.
Then {\cmerge} run over the same set of cluster pairs except for those are prohibited to be merged.
The right-bottom ilustration of Fig.\ref{sec:fmwk:cmerge:merge_prohibit_example} shows how prohibit algorithm affects the result.

The prohibit rule is also inherited to merged cluster pairs. For instance, consider input clusters 0, 1 and 2. The pair (1,2) is prohibited to merge, and (0,1) is found to be merged. Then the merged cluster, (0,1) cannot be merged with 2. That means even if a merge algorithm finds (0,2) to be merged, it cannot be because the prohibit rule is inherited from cluster 1 when 0 and 1 aremerged. This feature brings up a question: what determines 0 to be merged with 1 rather than 2? If the merge algorithm is executed on the pair (0,2) first, then the result would have been different. So what determines a different outcome is an order of cluster pairs in the loop of merging inspection. There are different ordering options available in {\cmerge}. Currently available options are defined as {\enum} type in {\cmerge} class, {\ttfamily CMergePriority\_t} which include following values.
\begin{itemize}
\item[] {\ttfamily kIndex}
  \begin{itemize}
    \item From the smaller to larger input cluster ID, basically following input cluster set ordering
  \end{itemize}

\item[] {\ttfamily kPolyArea}
  \begin{itemize}
    \item From the clusters with smaller polygon area to the larger
  \end{itemize}

\item[] {\ttfamily kChargeSum}
  \begin{itemize}
    \item From the clusters with smaller charge sum to the larrger
  \end{itemize}

\item[] {\ttfamily kNHits}
  \begin{itemize}
    \item From the clusters with smaller number of hits to the larger
  \end{itemize}

\end{itemize}
The ordering parameters including all listed above are taken from {\ttfamily cluster\_params} member attributes.
If more useful parameter is found, one should make a request to make it available. 
It is very simple to implement in {\cmerge}.

In summary, {\cmerge} uses one {\cmalgo} instance for merging and another for merge-prohibit. A possible ambiguity of which cluster pairs to be merged in the presence of prohibited pairs is resolved by defining a priority for cluster pairs in the merging attempt. 

\subsection{Process Flow for Cluster Merging}
\label{sec:fmwk:cmerge:process_flow}
An overall process flow diagram is already shown in Fig.\ref{sec:fmwk:base:cmerge_stream}, so here we describe slightly more details in texts by breaking down into 5 blocks.
\begin{enumerate}
\item Event Preparation
  \begin{itemize}
    \item[] {\cmerge} receives an input cluster in terms of a set of {\pxhit} (see Sec.\ref{sec:fmwk:cmerge:input}). Then it creates {\cpan} instance for each cluster, and calls {\cmalgo::EventBegin()} for merging and prohibit algorithms.
  \end{itemize}

\item Iterative Merge Operation
  \begin{itemize}
    \item[] A loop over cluster pairs for prohibit (if provided) and merge {\cmalgo} can be configured to repeat until no newly merged cluster is found. This is shown in Fig.\ref{sec:fmwk:cmerge:iterative_stream}. If a user does not configure to choose iterative approach, iteration will not happen even if there is a newly merged cluster in the output.
  \end{itemize}

\item Prohibit {\cmalgo} Loop Over Cluster Pairs
  \begin{itemize}
    \item[] See Fig.\ref{sec:fmwk:cmerge:iterative_stream}. Only occurs if prohibit algorithm is provided by a user (optional). Records which cluster pair to be prohibited for merging using {\cbkeeper}.
  \end{itemize}

\item Prohibit {\cmalgo} Loop Over Cluster Pairs
  \begin{itemize}
    \item[] See Fig.\ref{sec:fmwk:cmerge:iterative_stream}. The algorithm's {\ttfamily Bool()} is not executed for a subject pair of clusters if (a) they are prohibited for merging or (b) already inspected in the previous process and not merged (i.e. result should not change). This is to save computation time.
  \end{itemize}

\item Merge Clusters
  \begin{itemize}
    \item[] Based on the outcome of merge/prohibit {\cmalgo} algorithms, actual merging is performed at this step. It creates a new set of {\cpan} for merged clusters. For clusters that are not merged, {\cpan} instance is copied from input to save {\cpan} computation time.
  \end{itemize}

\end{enumerate}

\begin{figure}[ht]\begin{center}
\includegraphics[width=13cm]{./src/Pictures/IterativeMergeApproach.pdf}
\caption{Illustration of how iterative merge approach is processed. Internally a loop over cluster pairs for prohibit and merge algorithm are repeated if (a) user configured to use an iterative approach and (b) newly merged cluster is found.}
\label{sec:fmwk:cmerge:iterative_stream}
\end{center}\end{figure}


\subsection{Setting an Input 2D Clusters}
\label{sec:fmwk:cmerge:input}
Input to {\cmerge} is a set of clusters where each cluster is represented as a vector of hits, or more precisely {\ttfamily std::vector<util::PxHit>}. This input can be set by the following function:
\begin{lstlisting}
  void SetClusters(const std::vector<std::vector<util::PxHit> > &)
\end{lstlisting}
where the argument is a vector of 2D clusters. When this function is called, {\cmerge} internally calls
\begin{lstlisting}
  void Reset()
\end{lstlisting}
which initialize previously stored input clusters, output clusters and {\cbkeeper} instance.
Then it creates {\cpan} instance per input 2D cluster using a set of {\pxhit}.
This completes setting (or initiating) the input 2D cluster sets to be used for internal processing.

\subsection{Setting Merge/Prohibit {\cmalgo} Instances}
{\cmalgo} instance is necessary for merging and optional for prohibit. Call the function,
\begin{lstlisting}
  void AddMergeAlg(CBoolAlgoBase* algo)
\end{lstlisting}
to set {\cmalgo} for merging, and 
\begin{lstlisting}
  void AddSeparateAlgo(CBoolAlgoBase* algo)
\end{lstlisting}
to set {\cmalgo} for merge-prohibit.

\subsection{Setting Run Configuration}
There are only 3 run options to configure for {\cmerge}.

\paragraph{Verbosity:} 
There are 5 verbosity levels a user can configure as listed below:
\begin{itemize}
\item[] {\ttfamily kNone}
  \begin{itemize}
    \item No report from {\cmerge}
  \end{itemize}

\item[] {\ttfamily kPerEvent}
  \begin{itemize}
    \item Report made at the end of even processing
  \end{itemize}

\item[] {\ttfamily kPerIteration}
  \begin{itemize}
    \item Report made per merge iteration (see Sec.\ref{sec:fmwk:cmerge:process_flow}). This could be quite verbose.      
  \end{itemize}

\item[] {\ttfamily kPerMerging}
  \begin{itemize}
    \item Report made per {\cmalgo::Bool()} execution (see Sec.\ref{sec:fmwk:cmerge:process_flow}). This is even more verbose than {\ttfamily kPerIteration}.
  \end{itemize}

\end{itemize}
These are {\enum} type, {\ttfamily CMergeMSGLevel\_t}, defined under {\cmerge} class. 
The verbosity level can be set by the function:
\begin{lstlisting}
  void DebugMode(CMergeMSGLevel_t level)
\end{lstlisting}

\paragraph{Iterative Approach:}
As described in Sec.\ref{sec:fmwk:cmerge:process_flow}, {\cmerge} can be configured to attempt iterative approach for merging. This can be enabled by the following function call.
\begin{lstlisting}
  void MergeTillConverge(bool)
\end{lstlisting}
Providing {\ttfamily true} as an argument will enable this option. Disabled if {\ttfamily false} is provided.

\paragraph{Merge Priority:}
As described in Sec.\ref{sec:fmwk:cmerge:how}, ordering of cluster pairs for merging inspection matters to the result.
Options are also described in the same section. This can be set by the following function.
\begin{lstlisting}
  void SetMergePriority(CMergePriority_t level)
\end{lstlisting}

\subsection{Execute Merging}
To execute merging, simply call
\begin{lstlisting}
  void Process()
\end{lstlisting}

\subsection{Obtaining Result}
There are two kinds of results that may be useful for a user. First is a resulting (i.e. merged) set of clusters in terms of {\cpan}. This is useful for further processing as various cluster parameters are already computed as a part of {\cpan}. You can execute
\begin{lstlisting}
  const std::vector<cluster::ClusterParamsAlg>& GetClusters() const
\end{lstlisting}
to obtain such result set.

Second is a mapping of input cluster IDs (i.e. index) to the output (merged clusters). 
As described in Sec.\ref{sec:fmwk:base:cbk}, this can be easily obtained through {\cbkeeper} instance.
You can obtain the {\cbkeeper} instance held by {\cmerge} through the following function.
\begin{lstlisting}
  const CBookKeeper& GetBookKeeper() const
\end{lstlisting}


\section{{\cmatch}: Cluster Matching Tool}
\label{sec:fmwk:cmatch}
Tool not introduced yet in {\larsoft}. Will be documented when this is done.



 
% Base package description
\chapter{Core Package: Base}
\label{chap:base}

Despite its extreme simplicity let us call {\cmtool} a framework, or FMWK. 
In this chapter we discuss the building blocks of {\cmtool}.

\section{External Dependency}
\label{sec:fmwk:external}

The {\cmtool} development effort is originated from EM shower reconstruction for MicroBooNE experiment, and it is an extension of existing tool set. The following is the brief list of packages or {\CPP} classes on which {\cmtool} depends.
\begin{itemize}
\item[] {\cru}
    \begin{itemize}
    \item larreco/RecoAlg/ClusterRecoUtil in {\larsoft}
    \item UserDev/ClusterRecoUtil in {\larlight}
    \end{itemize}
\item[] {\geoutil}
    \begin{itemize}
    \item lardata/Utilities in {\larsoft}
    \item core/LArUtil in {\larlight}
    \end{itemize}
\item[] {\ttfamily PxUtil}
    \begin{itemize}
    \item lardata/Utilities in {\larsoft}
    \item core/LArUtil in {\larlight}
    \end{itemize}
\end{itemize}
These items are (or should be if not yet) described in a separate document.
The rest of this section briefly covers what they are without going into a technical detail.

\subsection{{\ttfamily ClusterRecoUtil}}
\label{sec:fmwk:external:cru}
Reconstruction parameters for EM shower 2D cluster involves a lot more variety of parameters than those held in a 2D cluster data product defined by a framework ({\larsoft} or {\larlight}). {\cru} is a package that defines code for both calculation of and storage for those parameters. In particular, {\cmtool} uses following {\CPP} classes from {\cru}:
\begin{itemize}
\item {\ttfamily ClusterParamsAlg ... (\cpan)}
  \begin{itemize}
    \item {\cpan} is designed as both parameter computation algorithm and storage class, and is used along this design principle in {\cmtool}. In particular, {\cpan} is created one per input 2D cluster.
  \end{itemize}
\item {\ttfamily cluster\_params}
  \begin{itemize}
    \item {\ttfamily cluster\_params} is a utility storage class for important parameters computed by \cpan. A const reference can be retrieved from {\cpan} instance through {\ttfamily GetParams()} function call. Note, however, one should keep {\cpan} instance rather than {\ttfamily cluster\_params} as a cluster parameter data set (see {\cpan} description above).
  \end{itemize}
\item {\ttfamily CRUException}
  \begin{itemize}
    \item This is a very simple exception class derived from {\ttfamily std::exception}. It is used instead of {\ttfamily cet::exception} to avoid framework dependency. Only additional feature to the base class is that the argument string to the constructor is sent to standard output streem by overriden {\ttfamily std::exception::what()} function.
  \end{itemize}
\end{itemize}

For details of all classes listed above, you should see Ref.\cite{ClusterRecoUtil}.

\subsection{{\ttfamily PxUtil}}
\label{sec:fmwk:external:pxutil}
Both {\cru} and {\cmtool} are designed to be independent of an analysis framework of user's choice (in particular {\larlight} and {\larsoft}). 
As mentioned in Sec.\ref{sec:fmwk:external:cru}, output of {\cpan} (i.e. cluster parameters) is stored with {\ttfamily cluster\_params} instance. 
{\pxutil}, on the other hand, is used to define input data struct to {\cpan} (and hence {\cmtool} and {\cru} in general). {\pxutil} defines three data objects:
\begin{itemize}
  \item {\ttfamily PxPoint}
    \begin{itemize}
    \item represents 2D point and contains wire [cm], time [cm], and plane ID (unsigned char)
    \end{itemize}
  \item {\ttfamily PxHit}
    \begin{itemize}
    \item inhertis from {\ttfamily PxPoint} with an additional attribute to hold charge [ADC], representing 2D hit
    \end{itemize}
  \item {\ttfamily PxLine}
    \begin{itemize}
    \item represents 2D line and contains start and end coordinates, (wire [cm], time [cm]), with a plane ID (unsigned char)
    \end{itemize}
\end{itemize}

It is a FMWK interface module's respnsibility to translate input data product into these format. In particular, this module must convert a vector of FMWK hit data product (= cluster) into a vector of {\pxhit}. This is discussed in FMWK interface chapter (see Ch.\ref{chap:fmwk_interface}).

\subsection{{\ttfamily GeometryUtilities}}
\label{sec:fmwk:external:geoutil}
{\CPP} class {\geoutil} is a utility class that helps to retrieve detector geometry related information such as converting 3D point coordinate into a projected 2D coordinate on a given plane. 
{\cmtool} uses {\geoutil} to interpret geometry related variables in full extent. 
For development of any non-existing function that treats detector geometry information, it is encouraged to develop in {\geoutil} (and certainly not in {\cmtool}). 
Please see Ref.\cite{GeometryUtilities} for more detailed documentation.


\section{FMWK Bulding Blocks}
\label{sec:fmwk:base}
In this section we discuss two core buliding block classes: (1) {\cbalgo} and (2) {\cbkeeper}. Though we discuss these basic {\CPP} classes first, it might be easier to understand these classes after you read about {\cmerge}, described in Sec.{\ref{sec:fmwk:cmerge}}, because there we discuss how {\cmerge} works with practical usage of {\cbalgo} and {\cbkeeper}.

\subsection{Book Keeping: {\cbkeeper}}
\label{sec:fmwk:base:cbk}
{\cbkeeper} inherits from {\ttfamily std::vector<unsigned short>}, and hence it is a vector of integers by itself.
The length of vector can be specified using a constructor argument.

This class is used to track how input clusters are merged/matched into an output. 
Here we focus on merging of input clusters for the simplicity sake.
The design assumes one to perform a merging attempt of 2 clusters at a time.
When one has many clusters to make such attempt on all possible pairs of clusters, it could be troublesome to perform book-keeping for which clusters are merged together.
{\cbkeeper} is designed such that a user can just log two to-be-merged clusters' ID at each attempt, and it holds a summary log of all merged cluster pairs.


Here are technical details.
The index of vector corresponds to the initial cluster ID numbers.
If you have 8 clusters to be subject for merging, {\cbkeeper} instance should be initialized as length 8.
Initially the values stored in the vector is same as their index numbers as shown in the top of Fig.\ref{sec:fmwk:base:cbk}.
One can re-set the state of {\cbkeeper} instance by calling the following function.
\begin{lstlisting}
        void Reset(unsigned short n=0)
\end{lstlisting}

\begin{figure}\begin{center}
\includegraphics[width=14cm]{./src/Pictures/CBookKeeperExample.pdf}
\caption{Example diagram showing how {\cbkeeper} keeps track of input and output cluster IDs. It shows {\cbkeeper} vector index (input cluster ID) and stored values (output cluster ID) for 8 input clusters (top), and how the stored values change as merging takes place. The middle shows the result of merging input cluster 1 and 2, and the bottom is for further merging 2 and 5.}
\end{center}\end{figure}
When one finds 2 clusters to be merged together, (s)he can call {\cbkeeper} attribute function:
\begin{lstlisting}
        void CBookKeeper::Merge(unsigned short, unsigned short)
\end{lstlisting}
with corresponding clusters' IDs (i.e. index numbers). 
Figure \ref{sec:fmwk:base:cbk} shows how the process works for calling {\ttfamily CBookKeeper::Merge()} function twice with arguments (1,2) and (2,5).
As a result, unique output cluster IDs are reduced to 6 because 3 clusters are combined together.

You can see {\cbkeeper} instance allows to retrieve which output cluster (stored value) an input cluster (vector index) ended up with in an intuitive way: simply use {\ttfamily std::vector::at()} method.
However, to create an output cluster, it is probably easier to ask the other way around: given output cluster ID, which input cluster(s) does it contain?
{\cbkeeper} has a simple function to answer this question:
\begin{lstlisting}
        std::vector<std::vector<unsigned short> > GetResult() const 
\end{lstlisting}
which returns a vector with length same as number of output (i.e. merged) clusters.
The value of this vector is {\ttfamily std::vector<unsigned short>} to contain possibly multiple input cluster indexes.
For instance, the return from {\cbkeeper} shown in the bottom of Fig.\ref{sec:fmwk:base:cbk} would be a vector of length 6, and the index1 contains a vector of input cluster ID 1, 2, and 5 as they have been merged into output cluster ID = 1.
Equivalent result can be obtained also by using
\begin{lstlisting}
        void PassResult(std::vector<std::vector<unsigned short> >&)
\end{lstlisting}
if that suits one's usage. 

If you are interested in checking which input cluster(s) included in a particular output (merged) cluster, you can use:
\begin{lstlisting}
        std::vector<unsigned short> GetMergedSet(unsigned short)
\end{lstlisting}
which takes an output cluster index as an argument and returns a vector that contains a list of input clusters that consist the output.

\begin{figure}[ht]\begin{center}
\includegraphics[width=8.3cm]{./src/Pictures/CBookKeeperProhibitExample2.pdf}
\includegraphics[width=8.3cm]{./src/Pictures/CBookKeeperProhibitExample1.pdf}
\caption{Example diagrams showing how {\cbkeeper} reacts to conflict {\ttfamily Merge()} and {\ttfamily ProhibitMerge()} function calls on the input cluster IDs. The left diagram shows {\crue} thrown due to calling {\ttfamily Merge()} on a cluster pair that is registered as merge-prohibit pair. The right diagram shows a result from calling {\ttfamily ProhibitMerge()} on a cluster pair that is already merged.}
\label{sec:fmwk:base:cbk_prohibit}
\end{center}\end{figure}

Finally, {\cmerge} can also prohibit merging of specific two cluster pair(s), and this is incoorporated by {\cbkeeper} as well.
This is, again, to support user to log merging in a mindless manner.
A user may specify specific 2 cluster pair should not be merged. 
Then (s)he can use the following function to register these 2 clusters for merge-prohibit list.
\begin{lstlisting}
        void ProhibitMerge(unsigned short, unsigned short)
\end{lstlisting}
The above function takes input cluster IDs (i.e. index numbers). 
Once 2 clusters are registered as merge-prohibit pair, attempting a function call {\ttfamily Merge()} will result in {\crue} thrown.
This is illustrated in the left of Fig.\ref{sec:fmwk:base:cbk_prohibit}.
Another possible case that results in {\crue} thrown would be calling {\ttfamily ProhibitMerge()} function on two input cluster IDs that are already merged. This is shown in the right hand side of Fig.\ref{sec:fmwk:base:cbk_prohibit}.

To avoid {\crue}, one should register all cluster pairs to prohibit-merge list before trying any merge attempt.
Upon attempting to call {\ttfamily Merge()}, then, one can use the function:
\begin{lstlisting}
        bool MergeAllowed(unsigned short, unsigned short)
\end{lstlisting}
to check if a cluster pair is registered to merge-prohibit list.
Calling this function prior to evaluation of merging (i.e. running some computation algorithm to decide whether to merge or not) may save some computation time and help the overall speed of the program (this is implemented in {\cmerge}).

\subsection{Algorithm Base: {\ttfamily CBoolAlgoBase}}
\label{sec:fmwk:base:cbalgo}
{\cbalgo} is a simple {\CPP} template class and also a base class for an algorithm used for cluster merging and matching by {\cmerge} and {\cmatch} application classes. In this and later sections, we use the term {\cmalgo} to refer to an algorithm class derived from {\cbalgo}.

\begin{figure}[ht]\begin{center}
\includegraphics[width=13cm]{./src/Pictures/CMergeStream.pdf}
\caption{The data processing stream of {\cmerge} framework. The purple allow shows the ordered process flow. It requires an analysis framework interface to convert framework dependent cluster data produdct into a vector of {\pxhit} and hand over to {\cmerge} application. Internally {\cmerge} calls various members of {\cbalgo} instances: one for merging and another for prohibiting. There is a configuration in {\cmerge} to perform iterative approach, which repeats internal loop over possible cluster pairs until no more newly merged cluster is found in the output. The merged cluster groups (i.e. output) can be obtained through {\cbkeeper} instance by an analysis framework interface to form an appropriate output data product.}
\label{sec:fmwk:base:cmerge_stream}
\end{center}\end{figure}

The idea is very simple: given two clusters, it is expected to return a boolean for decision making from the following function.
\begin{lstlisting}
        virtual bool Bool(const cluster::ClusterParamsAlg &, const cluster::ClusterParamsAlg &)
\end{lstlisting}
As you can see, ``clusters'' are handed over in terms of {\cpan} instance.
In {\cmerge}, the cluster merging application class, for instance, this is used for making merge decision as well as prohibit-merge decision (i.e. {\cmerge} holds two {\cbalgo} inherit class instance, one for merging and another for prohibiting).
An algorithm developpers should write his/her own {\CPP} algorithm class by inheriting from {\cbalgo}.
In the derived class, the {\ttfamily Bool()} function implementation should be over-riden. 
There is a script to generate a template (i.e. empty) {\cbalgo} inherit class and everyone is encouraged to use it for starting a new algorithm class. These are:
\begin{itemize}
\item For {\larsoft}
\begin{itemize}
        \item[]{\ttfamily> python \$MRB\_TOP/srcs/larreco/RecoAlg/CMTool/mac/gen\_cmalgo.py NAME}
\end{itemize}
\item For {\larlight}
\begin{itemize}
        \item[]{\ttfamily> python \$MAKE\_TOP\_DIR/UserDev/ClusterStudy/bin/gen\_cmalgo NAME}
\end{itemize}
\end{itemize}
where {\ttfamily NAME} should be replaced by the new algorithm's name. 
Running above command results in creating {\ttfamily CMAlgoNAME.h} and {\ttfamily CMAlgoNAME.cxx} in the appropriate directory.
Then you can proceed to build following the framework's guidance.

As mentioned, {\cbalgo} is operated by {\cmerge} and {\cmatch} applications.
For a given event, there is an internal loop of calling {\ttfamily Bool()} on all possible cluster pairs.
To provide flexibility in algorithm design, those applications also call other functions of {\cbalgo} in its processing.

The following is a list of such {\cbalgo} functions.
\begin{itemize}

\item[]{\ttfamily virtual void EventBegin(const std::vector<cluster::ClusterParamsAlg>\&)}
  \begin{itemize}
    \item Called at the beginning of an event, prior to looping over cluster pairs. The provided function argument is a vector of {\cpan} for all clusters subject for merging. This is where computation that requires the whole, initial cluster sets should be done.
  \end{itemize}

\item[]{\ttfamily virtual void IterationBegin(const std::vector<cluster::ClusterParamsAlg>\&)}
  \begin{itemize}
    \item Called at the beginning of an each loop over cluster pairs. Note {\cmerge} and {\cmatch} can have multiple iteration of looping over possible cluster pairs if configured to do so. For instance {\cmerge} can take iterate over multiple loops till no more to-be-merged  cluster pair is found (see Sec.\ref{sec:fmwk:cmerge} and Sec.\ref{sec:fmwk:cmatch} for details). The argument of this function is an output cluster sets, in terms of {\cpan} instance, from the previous loop. In case of the 1st loop, the provided cluster sets is identical to those provided in the argument of {\ttfamily EventBegin()} function.
  \end{itemize}

\item[]{\ttfamily virtual void IterationEnd()}
  \begin{itemize}
    \item Called at the end of each merge or match attempt loop over all cluster pairs, to clean up or summarize information.
  \end{itemize}

\item[]{\ttfamily virtual void EventEnd()}
  \begin{itemize}
    \item Called at the end of each event, after all iterations over cluster pairs are done.
  \end{itemize}

\end{itemize}

Further, there are two functions that are not yet implemented in {\cmerge} nor {\cmatch} but ideas exist.
Your input on how these should be implemented is welcome and very much appreciated.
\begin{itemize}

\item[]{\ttfamily virtual void Reset()}
  \begin{itemize}
    \item Meant to clear some member variables. But one can do that within {\ttfamily IterationEnd()} and/or {\ttfamily EventEnd()}, so it is not clear where this should be called. Possibly at the end of the whole event processing.
  \end{itemize}

\item[]{\ttfamily virtual void Report()}
  \begin{itemize}
    \item Meant to support some standardized {\ttfamily std::cout} and {\ttfamily std::cerr} report from an algorithm instance. {\cmerge} and {\cmatch} have ability to control such text output for different levels, and there is a hope this {\ttfamily Report()} can be also included there.
  \end{itemize}

\end{itemize}

\paragraph{Tips for writing {\cmalgo}}
First of all, you are encouraged to use {\ttfamily gen\_cmalgo} script as described earlier.
Another thing to note is to reduce a dependency on analysis framework.
This makes it easier to port your code between different frameworks.



\section{{\cmerge}: Cluster Merging Tool}
\label{sec:fmwk:cmerge}

{\cmerge} is a {\CPP} class that performs merging of 2D clusters, and is an application of {\cmalgo} algorithm classes that are inherit from {\cbalgo}, the algorithm base class. In a few sentences, {\cmerge} perform merging by looping over all possible 2D cluster pairs among the given set of clusters. For each pair, {\ttfamily Bool()} function of both merging and prohibit {\cmalgo} are called to inspect possible 2D cluster merging. It uses {\cbkeeper} to keep track of which pair of clusters are merged. For details of {\cbalgo} and {\cbkeeper}, see Sec.\ref{sec:fmwk:base:cbalgo} and Sec.\ref{sec:fmwk:base:cbk} respectively. The overall process stream diagram is shown in Fig.\ref{sec:fmwk:base:cmerge_stream} which we keep referring in this section.

\subsection{Why {\cmerge}?}
Why using {\cmerge}? Why not just writing a program that does everything from scratch?
Be my guest and write your code! That might inspire a better design or strategy toward merging. 
That being said, there are a few reasons to use {\cmerge}.

\paragraph{Task Outside Merging Decision:}
{\cmerge} The core part of merging is to compare 2 clusters and make a decision to merge or not. Outside this process, however, there is a non-negligible overburden of running such algorithm such as (1) keeping track of which cluster pairs are merged and (2) optimizing the loop process of cluster pair comparison. {\cmerge} does those tedious jobs while out-sourcing the decision making to an independent {\CPP} algorithm class that inherit from {\cbalgo}. 

\paragraph{Easy Development via Modulated Design:}
Because {\cmerge} outsource an independent algorithm class to perform merging of a cluster pair, it supports modulated code design. Accordingly each {\cmalgo} class becomes a pure algorithm class. This makes a code development very easy because a new algorithm can be written without re-writing the I/O or book-keeping part. Also it makes life easy to compare different algorithms: you can simple change which algorithm to be attached to {\cmerge} at run-time, and this avoids re-compilation or re-writing of code.

\paragraph{X-Framework Implementation}
{\cmerge} does not use data product type specific to an analysis framework (for instance {\larlight} or {\larsoft}), so it can be ported easily among different frameworks. That means {\cmalgo} developed by one person in one analysis framework can be shared among other frameworks (assuming an algorithm developer follows x-framework design pattern).\\

\subsection{How Does It Work?}
\label{sec:fmwk:cmerge:how}
The core part of decision making is simple: provided 2 clusters, decide whether they should be merged or not.
This is done through {\cmalgo::Bool()} function call:
\begin{lstlisting}
  virtual bool Bool( const cluster::ClusterParamsAlg &, const cluster::ClusterParamsAlg &)
\end{lstlisting}
If the return is {\ttfamily true}, two clusters are merged. Else not.
This is ilustrated in Fig.\ref{sec:fmwk:cmerge:merge_example}.

\begin{figure}[ht]\begin{center}
\includegraphics[width=13cm]{./src/Pictures/MergeExample.pdf}
\caption{Example of 2D clusters in a plane before and after merging attempt. Each oval represents a cluster with cluster 2D axis shown in a black solid arrow. The direction of an arrow represents the direction of a cluster development. Each cluster is labeled with an input cluster ID number while the oval color represents distinct output cluster group. Left: initial state with 4 input clusters. Right: example output after running a merge loop over cluster pairs. }
\label{sec:fmwk:cmerge:merge_example}
\end{center}\end{figure}
{\cmerge} takes care of creating a loop over all possible pairs of 2D clusters and calling the above function of {\cmalgo} for merging.
In the process, book-keeping of which pairs to be merged is recorded using {\cbkeeper}.

\begin{figure}[ht]\begin{center}
\includegraphics[width=13cm]{./src/Pictures/MergeProhibitExample.pdf}
\caption{Example of how prohibit algorithm works. Each oval and solid arrow represents a 2D cluster with its direction while the color represents distinct output (merged) cluster. Input cluster index (ID) is shown in the label with the cluster 2D angle. Right-top shows an output of running {\cmalgo} that merge clusters with angle difference within 40 degrees. Right-bottom shows an output of the same {\cmalgo} for merging but with another {\cmalgo} that prohibits merging of clusters with angle difference larger than 40 degrees.}
\label{sec:fmwk:cmerge:merge_prohibit_example}
\end{center}\end{figure}
Let us consider a case of two merged pairs that shares the same input cluster ID. As described in Sec.\ref{sec:fmwk:base:cbk}, {\cbkeeper} combines two pairs of 2D clusters in such case (see Fig.\ref{sec:fmwk:base:cbk}). This could be problematic sometimes. For instance, consider an example {\cmalgo} that combines two clusters if their start points are very close and the angle between two clusters are within 40 degrees. Further assume we have 3 input clusters that share the start point but have different angles at 20, 40, and -20 degrees. For convention these have input cluster ID 0, 1, and 2 respectively. This example algorithm would conclude pairs (0,1) and (0,2) but not for (1,2). But because merged pairs share cluster ID 0, it ends up all 3 clusters being merged. This is shown in the right-top ilustration of Fig.\ref{sec:fmwk:cmerge:merge_prohibit_example}.

This could be a problem, or may not be: this can be only answered by the algorithm author.
The way to avoid this in {\cmerge} is to implement another {\cmalgo} as {\it merge-prohibit}, or simple {\it prohibit} algorithm.
When prohibit algorithm is provided, {\cmerge} first run over all cluster pairs and execute {\ttfamily Bool()} function to find all cluster pairs that are prohibited to merge.
Then {\cmerge} run over the same set of cluster pairs except for those are prohibited to be merged.
The right-bottom ilustration of Fig.\ref{sec:fmwk:cmerge:merge_prohibit_example} shows how prohibit algorithm affects the result.

The prohibit rule is also inherited to merged cluster pairs. For instance, consider input clusters 0, 1 and 2. The pair (1,2) is prohibited to merge, and (0,1) is found to be merged. Then the merged cluster, (0,1) cannot be merged with 2. That means even if a merge algorithm finds (0,2) to be merged, it cannot be because the prohibit rule is inherited from cluster 1 when 0 and 1 aremerged. This feature brings up a question: what determines 0 to be merged with 1 rather than 2? If the merge algorithm is executed on the pair (0,2) first, then the result would have been different. So what determines a different outcome is an order of cluster pairs in the loop of merging inspection. There are different ordering options available in {\cmerge}. Currently available options are defined as {\enum} type in {\cmerge} class, {\ttfamily CMergePriority\_t} which include following values.
\begin{itemize}
\item[] {\ttfamily kIndex}
  \begin{itemize}
    \item From the smaller to larger input cluster ID, basically following input cluster set ordering
  \end{itemize}

\item[] {\ttfamily kPolyArea}
  \begin{itemize}
    \item From the clusters with smaller polygon area to the larger
  \end{itemize}

\item[] {\ttfamily kChargeSum}
  \begin{itemize}
    \item From the clusters with smaller charge sum to the larrger
  \end{itemize}

\item[] {\ttfamily kNHits}
  \begin{itemize}
    \item From the clusters with smaller number of hits to the larger
  \end{itemize}

\end{itemize}
The ordering parameters including all listed above are taken from {\ttfamily cluster\_params} member attributes.
If more useful parameter is found, one should make a request to make it available. 
It is very simple to implement in {\cmerge}.

In summary, {\cmerge} uses one {\cmalgo} instance for merging and another for merge-prohibit. A possible ambiguity of which cluster pairs to be merged in the presence of prohibited pairs is resolved by defining a priority for cluster pairs in the merging attempt. 

\subsection{Process Flow for Cluster Merging}
\label{sec:fmwk:cmerge:process_flow}
An overall process flow diagram is already shown in Fig.\ref{sec:fmwk:base:cmerge_stream}, so here we describe slightly more details in texts by breaking down into 5 blocks.
\begin{enumerate}
\item Event Preparation
  \begin{itemize}
    \item[] {\cmerge} receives an input cluster in terms of a set of {\pxhit} (see Sec.\ref{sec:fmwk:cmerge:input}). Then it creates {\cpan} instance for each cluster, and calls {\cmalgo::EventBegin()} for merging and prohibit algorithms.
  \end{itemize}

\item Iterative Merge Operation
  \begin{itemize}
    \item[] A loop over cluster pairs for prohibit (if provided) and merge {\cmalgo} can be configured to repeat until no newly merged cluster is found. This is shown in Fig.\ref{sec:fmwk:cmerge:iterative_stream}. If a user does not configure to choose iterative approach, iteration will not happen even if there is a newly merged cluster in the output.
  \end{itemize}

\item Prohibit {\cmalgo} Loop Over Cluster Pairs
  \begin{itemize}
    \item[] See Fig.\ref{sec:fmwk:cmerge:iterative_stream}. Only occurs if prohibit algorithm is provided by a user (optional). Records which cluster pair to be prohibited for merging using {\cbkeeper}.
  \end{itemize}

\item Prohibit {\cmalgo} Loop Over Cluster Pairs
  \begin{itemize}
    \item[] See Fig.\ref{sec:fmwk:cmerge:iterative_stream}. The algorithm's {\ttfamily Bool()} is not executed for a subject pair of clusters if (a) they are prohibited for merging or (b) already inspected in the previous process and not merged (i.e. result should not change). This is to save computation time.
  \end{itemize}

\item Merge Clusters
  \begin{itemize}
    \item[] Based on the outcome of merge/prohibit {\cmalgo} algorithms, actual merging is performed at this step. It creates a new set of {\cpan} for merged clusters. For clusters that are not merged, {\cpan} instance is copied from input to save {\cpan} computation time.
  \end{itemize}

\end{enumerate}

\begin{figure}[ht]\begin{center}
\includegraphics[width=13cm]{./src/Pictures/IterativeMergeApproach.pdf}
\caption{Illustration of how iterative merge approach is processed. Internally a loop over cluster pairs for prohibit and merge algorithm are repeated if (a) user configured to use an iterative approach and (b) newly merged cluster is found.}
\label{sec:fmwk:cmerge:iterative_stream}
\end{center}\end{figure}


\subsection{Setting an Input 2D Clusters}
\label{sec:fmwk:cmerge:input}
Input to {\cmerge} is a set of clusters where each cluster is represented as a vector of hits, or more precisely {\ttfamily std::vector<util::PxHit>}. This input can be set by the following function:
\begin{lstlisting}
  void SetClusters(const std::vector<std::vector<util::PxHit> > &)
\end{lstlisting}
where the argument is a vector of 2D clusters. When this function is called, {\cmerge} internally calls
\begin{lstlisting}
  void Reset()
\end{lstlisting}
which initialize previously stored input clusters, output clusters and {\cbkeeper} instance.
Then it creates {\cpan} instance per input 2D cluster using a set of {\pxhit}.
This completes setting (or initiating) the input 2D cluster sets to be used for internal processing.

\subsection{Setting Merge/Prohibit {\cmalgo} Instances}
{\cmalgo} instance is necessary for merging and optional for prohibit. Call the function,
\begin{lstlisting}
  void AddMergeAlg(CBoolAlgoBase* algo)
\end{lstlisting}
to set {\cmalgo} for merging, and 
\begin{lstlisting}
  void AddSeparateAlgo(CBoolAlgoBase* algo)
\end{lstlisting}
to set {\cmalgo} for merge-prohibit.

\subsection{Setting Run Configuration}
There are only 3 run options to configure for {\cmerge}.

\paragraph{Verbosity:} 
There are 5 verbosity levels a user can configure as listed below:
\begin{itemize}
\item[] {\ttfamily kNone}
  \begin{itemize}
    \item No report from {\cmerge}
  \end{itemize}

\item[] {\ttfamily kPerEvent}
  \begin{itemize}
    \item Report made at the end of even processing
  \end{itemize}

\item[] {\ttfamily kPerIteration}
  \begin{itemize}
    \item Report made per merge iteration (see Sec.\ref{sec:fmwk:cmerge:process_flow}). This could be quite verbose.      
  \end{itemize}

\item[] {\ttfamily kPerMerging}
  \begin{itemize}
    \item Report made per {\cmalgo::Bool()} execution (see Sec.\ref{sec:fmwk:cmerge:process_flow}). This is even more verbose than {\ttfamily kPerIteration}.
  \end{itemize}

\end{itemize}
These are {\enum} type, {\ttfamily CMergeMSGLevel\_t}, defined under {\cmerge} class. 
The verbosity level can be set by the function:
\begin{lstlisting}
  void DebugMode(CMergeMSGLevel_t level)
\end{lstlisting}

\paragraph{Iterative Approach:}
As described in Sec.\ref{sec:fmwk:cmerge:process_flow}, {\cmerge} can be configured to attempt iterative approach for merging. This can be enabled by the following function call.
\begin{lstlisting}
  void MergeTillConverge(bool)
\end{lstlisting}
Providing {\ttfamily true} as an argument will enable this option. Disabled if {\ttfamily false} is provided.

\paragraph{Merge Priority:}
As described in Sec.\ref{sec:fmwk:cmerge:how}, ordering of cluster pairs for merging inspection matters to the result.
Options are also described in the same section. This can be set by the following function.
\begin{lstlisting}
  void SetMergePriority(CMergePriority_t level)
\end{lstlisting}

\subsection{Execute Merging}
To execute merging, simply call
\begin{lstlisting}
  void Process()
\end{lstlisting}

\subsection{Obtaining Result}
There are two kinds of results that may be useful for a user. First is a resulting (i.e. merged) set of clusters in terms of {\cpan}. This is useful for further processing as various cluster parameters are already computed as a part of {\cpan}. You can execute
\begin{lstlisting}
  const std::vector<cluster::ClusterParamsAlg>& GetClusters() const
\end{lstlisting}
to obtain such result set.

Second is a mapping of input cluster IDs (i.e. index) to the output (merged clusters). 
As described in Sec.\ref{sec:fmwk:base:cbk}, this can be easily obtained through {\cbkeeper} instance.
You can obtain the {\cbkeeper} instance held by {\cmerge} through the following function.
\begin{lstlisting}
  const CBookKeeper& GetBookKeeper() const
\end{lstlisting}


\section{{\cmatch}: Cluster Matching Tool}
\label{sec:fmwk:cmatch}
Tool not introduced yet in {\larsoft}. Will be documented when this is done.




% DataFormat package description
\chapter{Core Package: DataFormat}
\label{chap:dataformat}

Despite its extreme simplicity let us call {\cmtool} a framework, or FMWK. 
In this chapter we discuss the building blocks of {\cmtool}.

\section{External Dependency}
\label{sec:fmwk:external}

The {\cmtool} development effort is originated from EM shower reconstruction for MicroBooNE experiment, and it is an extension of existing tool set. The following is the brief list of packages or {\CPP} classes on which {\cmtool} depends.
\begin{itemize}
\item[] {\cru}
    \begin{itemize}
    \item larreco/RecoAlg/ClusterRecoUtil in {\larsoft}
    \item UserDev/ClusterRecoUtil in {\larlight}
    \end{itemize}
\item[] {\geoutil}
    \begin{itemize}
    \item lardata/Utilities in {\larsoft}
    \item core/LArUtil in {\larlight}
    \end{itemize}
\item[] {\ttfamily PxUtil}
    \begin{itemize}
    \item lardata/Utilities in {\larsoft}
    \item core/LArUtil in {\larlight}
    \end{itemize}
\end{itemize}
These items are (or should be if not yet) described in a separate document.
The rest of this section briefly covers what they are without going into a technical detail.

\subsection{{\ttfamily ClusterRecoUtil}}
\label{sec:fmwk:external:cru}
Reconstruction parameters for EM shower 2D cluster involves a lot more variety of parameters than those held in a 2D cluster data product defined by a framework ({\larsoft} or {\larlight}). {\cru} is a package that defines code for both calculation of and storage for those parameters. In particular, {\cmtool} uses following {\CPP} classes from {\cru}:
\begin{itemize}
\item {\ttfamily ClusterParamsAlg ... (\cpan)}
  \begin{itemize}
    \item {\cpan} is designed as both parameter computation algorithm and storage class, and is used along this design principle in {\cmtool}. In particular, {\cpan} is created one per input 2D cluster.
  \end{itemize}
\item {\ttfamily cluster\_params}
  \begin{itemize}
    \item {\ttfamily cluster\_params} is a utility storage class for important parameters computed by \cpan. A const reference can be retrieved from {\cpan} instance through {\ttfamily GetParams()} function call. Note, however, one should keep {\cpan} instance rather than {\ttfamily cluster\_params} as a cluster parameter data set (see {\cpan} description above).
  \end{itemize}
\item {\ttfamily CRUException}
  \begin{itemize}
    \item This is a very simple exception class derived from {\ttfamily std::exception}. It is used instead of {\ttfamily cet::exception} to avoid framework dependency. Only additional feature to the base class is that the argument string to the constructor is sent to standard output streem by overriden {\ttfamily std::exception::what()} function.
  \end{itemize}
\end{itemize}

For details of all classes listed above, you should see Ref.\cite{ClusterRecoUtil}.

\subsection{{\ttfamily PxUtil}}
\label{sec:fmwk:external:pxutil}
Both {\cru} and {\cmtool} are designed to be independent of an analysis framework of user's choice (in particular {\larlight} and {\larsoft}). 
As mentioned in Sec.\ref{sec:fmwk:external:cru}, output of {\cpan} (i.e. cluster parameters) is stored with {\ttfamily cluster\_params} instance. 
{\pxutil}, on the other hand, is used to define input data struct to {\cpan} (and hence {\cmtool} and {\cru} in general). {\pxutil} defines three data objects:
\begin{itemize}
  \item {\ttfamily PxPoint}
    \begin{itemize}
    \item represents 2D point and contains wire [cm], time [cm], and plane ID (unsigned char)
    \end{itemize}
  \item {\ttfamily PxHit}
    \begin{itemize}
    \item inhertis from {\ttfamily PxPoint} with an additional attribute to hold charge [ADC], representing 2D hit
    \end{itemize}
  \item {\ttfamily PxLine}
    \begin{itemize}
    \item represents 2D line and contains start and end coordinates, (wire [cm], time [cm]), with a plane ID (unsigned char)
    \end{itemize}
\end{itemize}

It is a FMWK interface module's respnsibility to translate input data product into these format. In particular, this module must convert a vector of FMWK hit data product (= cluster) into a vector of {\pxhit}. This is discussed in FMWK interface chapter (see Ch.\ref{chap:fmwk_interface}).

\subsection{{\ttfamily GeometryUtilities}}
\label{sec:fmwk:external:geoutil}
{\CPP} class {\geoutil} is a utility class that helps to retrieve detector geometry related information such as converting 3D point coordinate into a projected 2D coordinate on a given plane. 
{\cmtool} uses {\geoutil} to interpret geometry related variables in full extent. 
For development of any non-existing function that treats detector geometry information, it is encouraged to develop in {\geoutil} (and certainly not in {\cmtool}). 
Please see Ref.\cite{GeometryUtilities} for more detailed documentation.


\section{FMWK Bulding Blocks}
\label{sec:fmwk:base}
In this section we discuss two core buliding block classes: (1) {\cbalgo} and (2) {\cbkeeper}. Though we discuss these basic {\CPP} classes first, it might be easier to understand these classes after you read about {\cmerge}, described in Sec.{\ref{sec:fmwk:cmerge}}, because there we discuss how {\cmerge} works with practical usage of {\cbalgo} and {\cbkeeper}.

\subsection{Book Keeping: {\cbkeeper}}
\label{sec:fmwk:base:cbk}
{\cbkeeper} inherits from {\ttfamily std::vector<unsigned short>}, and hence it is a vector of integers by itself.
The length of vector can be specified using a constructor argument.

This class is used to track how input clusters are merged/matched into an output. 
Here we focus on merging of input clusters for the simplicity sake.
The design assumes one to perform a merging attempt of 2 clusters at a time.
When one has many clusters to make such attempt on all possible pairs of clusters, it could be troublesome to perform book-keeping for which clusters are merged together.
{\cbkeeper} is designed such that a user can just log two to-be-merged clusters' ID at each attempt, and it holds a summary log of all merged cluster pairs.


Here are technical details.
The index of vector corresponds to the initial cluster ID numbers.
If you have 8 clusters to be subject for merging, {\cbkeeper} instance should be initialized as length 8.
Initially the values stored in the vector is same as their index numbers as shown in the top of Fig.\ref{sec:fmwk:base:cbk}.
One can re-set the state of {\cbkeeper} instance by calling the following function.
\begin{lstlisting}
        void Reset(unsigned short n=0)
\end{lstlisting}

\begin{figure}\begin{center}
\includegraphics[width=14cm]{./src/Pictures/CBookKeeperExample.pdf}
\caption{Example diagram showing how {\cbkeeper} keeps track of input and output cluster IDs. It shows {\cbkeeper} vector index (input cluster ID) and stored values (output cluster ID) for 8 input clusters (top), and how the stored values change as merging takes place. The middle shows the result of merging input cluster 1 and 2, and the bottom is for further merging 2 and 5.}
\end{center}\end{figure}
When one finds 2 clusters to be merged together, (s)he can call {\cbkeeper} attribute function:
\begin{lstlisting}
        void CBookKeeper::Merge(unsigned short, unsigned short)
\end{lstlisting}
with corresponding clusters' IDs (i.e. index numbers). 
Figure \ref{sec:fmwk:base:cbk} shows how the process works for calling {\ttfamily CBookKeeper::Merge()} function twice with arguments (1,2) and (2,5).
As a result, unique output cluster IDs are reduced to 6 because 3 clusters are combined together.

You can see {\cbkeeper} instance allows to retrieve which output cluster (stored value) an input cluster (vector index) ended up with in an intuitive way: simply use {\ttfamily std::vector::at()} method.
However, to create an output cluster, it is probably easier to ask the other way around: given output cluster ID, which input cluster(s) does it contain?
{\cbkeeper} has a simple function to answer this question:
\begin{lstlisting}
        std::vector<std::vector<unsigned short> > GetResult() const 
\end{lstlisting}
which returns a vector with length same as number of output (i.e. merged) clusters.
The value of this vector is {\ttfamily std::vector<unsigned short>} to contain possibly multiple input cluster indexes.
For instance, the return from {\cbkeeper} shown in the bottom of Fig.\ref{sec:fmwk:base:cbk} would be a vector of length 6, and the index1 contains a vector of input cluster ID 1, 2, and 5 as they have been merged into output cluster ID = 1.
Equivalent result can be obtained also by using
\begin{lstlisting}
        void PassResult(std::vector<std::vector<unsigned short> >&)
\end{lstlisting}
if that suits one's usage. 

If you are interested in checking which input cluster(s) included in a particular output (merged) cluster, you can use:
\begin{lstlisting}
        std::vector<unsigned short> GetMergedSet(unsigned short)
\end{lstlisting}
which takes an output cluster index as an argument and returns a vector that contains a list of input clusters that consist the output.

\begin{figure}[ht]\begin{center}
\includegraphics[width=8.3cm]{./src/Pictures/CBookKeeperProhibitExample2.pdf}
\includegraphics[width=8.3cm]{./src/Pictures/CBookKeeperProhibitExample1.pdf}
\caption{Example diagrams showing how {\cbkeeper} reacts to conflict {\ttfamily Merge()} and {\ttfamily ProhibitMerge()} function calls on the input cluster IDs. The left diagram shows {\crue} thrown due to calling {\ttfamily Merge()} on a cluster pair that is registered as merge-prohibit pair. The right diagram shows a result from calling {\ttfamily ProhibitMerge()} on a cluster pair that is already merged.}
\label{sec:fmwk:base:cbk_prohibit}
\end{center}\end{figure}

Finally, {\cmerge} can also prohibit merging of specific two cluster pair(s), and this is incoorporated by {\cbkeeper} as well.
This is, again, to support user to log merging in a mindless manner.
A user may specify specific 2 cluster pair should not be merged. 
Then (s)he can use the following function to register these 2 clusters for merge-prohibit list.
\begin{lstlisting}
        void ProhibitMerge(unsigned short, unsigned short)
\end{lstlisting}
The above function takes input cluster IDs (i.e. index numbers). 
Once 2 clusters are registered as merge-prohibit pair, attempting a function call {\ttfamily Merge()} will result in {\crue} thrown.
This is illustrated in the left of Fig.\ref{sec:fmwk:base:cbk_prohibit}.
Another possible case that results in {\crue} thrown would be calling {\ttfamily ProhibitMerge()} function on two input cluster IDs that are already merged. This is shown in the right hand side of Fig.\ref{sec:fmwk:base:cbk_prohibit}.

To avoid {\crue}, one should register all cluster pairs to prohibit-merge list before trying any merge attempt.
Upon attempting to call {\ttfamily Merge()}, then, one can use the function:
\begin{lstlisting}
        bool MergeAllowed(unsigned short, unsigned short)
\end{lstlisting}
to check if a cluster pair is registered to merge-prohibit list.
Calling this function prior to evaluation of merging (i.e. running some computation algorithm to decide whether to merge or not) may save some computation time and help the overall speed of the program (this is implemented in {\cmerge}).

\subsection{Algorithm Base: {\ttfamily CBoolAlgoBase}}
\label{sec:fmwk:base:cbalgo}
{\cbalgo} is a simple {\CPP} template class and also a base class for an algorithm used for cluster merging and matching by {\cmerge} and {\cmatch} application classes. In this and later sections, we use the term {\cmalgo} to refer to an algorithm class derived from {\cbalgo}.

\begin{figure}[ht]\begin{center}
\includegraphics[width=13cm]{./src/Pictures/CMergeStream.pdf}
\caption{The data processing stream of {\cmerge} framework. The purple allow shows the ordered process flow. It requires an analysis framework interface to convert framework dependent cluster data produdct into a vector of {\pxhit} and hand over to {\cmerge} application. Internally {\cmerge} calls various members of {\cbalgo} instances: one for merging and another for prohibiting. There is a configuration in {\cmerge} to perform iterative approach, which repeats internal loop over possible cluster pairs until no more newly merged cluster is found in the output. The merged cluster groups (i.e. output) can be obtained through {\cbkeeper} instance by an analysis framework interface to form an appropriate output data product.}
\label{sec:fmwk:base:cmerge_stream}
\end{center}\end{figure}

The idea is very simple: given two clusters, it is expected to return a boolean for decision making from the following function.
\begin{lstlisting}
        virtual bool Bool(const cluster::ClusterParamsAlg &, const cluster::ClusterParamsAlg &)
\end{lstlisting}
As you can see, ``clusters'' are handed over in terms of {\cpan} instance.
In {\cmerge}, the cluster merging application class, for instance, this is used for making merge decision as well as prohibit-merge decision (i.e. {\cmerge} holds two {\cbalgo} inherit class instance, one for merging and another for prohibiting).
An algorithm developpers should write his/her own {\CPP} algorithm class by inheriting from {\cbalgo}.
In the derived class, the {\ttfamily Bool()} function implementation should be over-riden. 
There is a script to generate a template (i.e. empty) {\cbalgo} inherit class and everyone is encouraged to use it for starting a new algorithm class. These are:
\begin{itemize}
\item For {\larsoft}
\begin{itemize}
        \item[]{\ttfamily> python \$MRB\_TOP/srcs/larreco/RecoAlg/CMTool/mac/gen\_cmalgo.py NAME}
\end{itemize}
\item For {\larlight}
\begin{itemize}
        \item[]{\ttfamily> python \$MAKE\_TOP\_DIR/UserDev/ClusterStudy/bin/gen\_cmalgo NAME}
\end{itemize}
\end{itemize}
where {\ttfamily NAME} should be replaced by the new algorithm's name. 
Running above command results in creating {\ttfamily CMAlgoNAME.h} and {\ttfamily CMAlgoNAME.cxx} in the appropriate directory.
Then you can proceed to build following the framework's guidance.

As mentioned, {\cbalgo} is operated by {\cmerge} and {\cmatch} applications.
For a given event, there is an internal loop of calling {\ttfamily Bool()} on all possible cluster pairs.
To provide flexibility in algorithm design, those applications also call other functions of {\cbalgo} in its processing.

The following is a list of such {\cbalgo} functions.
\begin{itemize}

\item[]{\ttfamily virtual void EventBegin(const std::vector<cluster::ClusterParamsAlg>\&)}
  \begin{itemize}
    \item Called at the beginning of an event, prior to looping over cluster pairs. The provided function argument is a vector of {\cpan} for all clusters subject for merging. This is where computation that requires the whole, initial cluster sets should be done.
  \end{itemize}

\item[]{\ttfamily virtual void IterationBegin(const std::vector<cluster::ClusterParamsAlg>\&)}
  \begin{itemize}
    \item Called at the beginning of an each loop over cluster pairs. Note {\cmerge} and {\cmatch} can have multiple iteration of looping over possible cluster pairs if configured to do so. For instance {\cmerge} can take iterate over multiple loops till no more to-be-merged  cluster pair is found (see Sec.\ref{sec:fmwk:cmerge} and Sec.\ref{sec:fmwk:cmatch} for details). The argument of this function is an output cluster sets, in terms of {\cpan} instance, from the previous loop. In case of the 1st loop, the provided cluster sets is identical to those provided in the argument of {\ttfamily EventBegin()} function.
  \end{itemize}

\item[]{\ttfamily virtual void IterationEnd()}
  \begin{itemize}
    \item Called at the end of each merge or match attempt loop over all cluster pairs, to clean up or summarize information.
  \end{itemize}

\item[]{\ttfamily virtual void EventEnd()}
  \begin{itemize}
    \item Called at the end of each event, after all iterations over cluster pairs are done.
  \end{itemize}

\end{itemize}

Further, there are two functions that are not yet implemented in {\cmerge} nor {\cmatch} but ideas exist.
Your input on how these should be implemented is welcome and very much appreciated.
\begin{itemize}

\item[]{\ttfamily virtual void Reset()}
  \begin{itemize}
    \item Meant to clear some member variables. But one can do that within {\ttfamily IterationEnd()} and/or {\ttfamily EventEnd()}, so it is not clear where this should be called. Possibly at the end of the whole event processing.
  \end{itemize}

\item[]{\ttfamily virtual void Report()}
  \begin{itemize}
    \item Meant to support some standardized {\ttfamily std::cout} and {\ttfamily std::cerr} report from an algorithm instance. {\cmerge} and {\cmatch} have ability to control such text output for different levels, and there is a hope this {\ttfamily Report()} can be also included there.
  \end{itemize}

\end{itemize}

\paragraph{Tips for writing {\cmalgo}}
First of all, you are encouraged to use {\ttfamily gen\_cmalgo} script as described earlier.
Another thing to note is to reduce a dependency on analysis framework.
This makes it easier to port your code between different frameworks.



\section{{\cmerge}: Cluster Merging Tool}
\label{sec:fmwk:cmerge}

{\cmerge} is a {\CPP} class that performs merging of 2D clusters, and is an application of {\cmalgo} algorithm classes that are inherit from {\cbalgo}, the algorithm base class. In a few sentences, {\cmerge} perform merging by looping over all possible 2D cluster pairs among the given set of clusters. For each pair, {\ttfamily Bool()} function of both merging and prohibit {\cmalgo} are called to inspect possible 2D cluster merging. It uses {\cbkeeper} to keep track of which pair of clusters are merged. For details of {\cbalgo} and {\cbkeeper}, see Sec.\ref{sec:fmwk:base:cbalgo} and Sec.\ref{sec:fmwk:base:cbk} respectively. The overall process stream diagram is shown in Fig.\ref{sec:fmwk:base:cmerge_stream} which we keep referring in this section.

\subsection{Why {\cmerge}?}
Why using {\cmerge}? Why not just writing a program that does everything from scratch?
Be my guest and write your code! That might inspire a better design or strategy toward merging. 
That being said, there are a few reasons to use {\cmerge}.

\paragraph{Task Outside Merging Decision:}
{\cmerge} The core part of merging is to compare 2 clusters and make a decision to merge or not. Outside this process, however, there is a non-negligible overburden of running such algorithm such as (1) keeping track of which cluster pairs are merged and (2) optimizing the loop process of cluster pair comparison. {\cmerge} does those tedious jobs while out-sourcing the decision making to an independent {\CPP} algorithm class that inherit from {\cbalgo}. 

\paragraph{Easy Development via Modulated Design:}
Because {\cmerge} outsource an independent algorithm class to perform merging of a cluster pair, it supports modulated code design. Accordingly each {\cmalgo} class becomes a pure algorithm class. This makes a code development very easy because a new algorithm can be written without re-writing the I/O or book-keeping part. Also it makes life easy to compare different algorithms: you can simple change which algorithm to be attached to {\cmerge} at run-time, and this avoids re-compilation or re-writing of code.

\paragraph{X-Framework Implementation}
{\cmerge} does not use data product type specific to an analysis framework (for instance {\larlight} or {\larsoft}), so it can be ported easily among different frameworks. That means {\cmalgo} developed by one person in one analysis framework can be shared among other frameworks (assuming an algorithm developer follows x-framework design pattern).\\

\subsection{How Does It Work?}
\label{sec:fmwk:cmerge:how}
The core part of decision making is simple: provided 2 clusters, decide whether they should be merged or not.
This is done through {\cmalgo::Bool()} function call:
\begin{lstlisting}
  virtual bool Bool( const cluster::ClusterParamsAlg &, const cluster::ClusterParamsAlg &)
\end{lstlisting}
If the return is {\ttfamily true}, two clusters are merged. Else not.
This is ilustrated in Fig.\ref{sec:fmwk:cmerge:merge_example}.

\begin{figure}[ht]\begin{center}
\includegraphics[width=13cm]{./src/Pictures/MergeExample.pdf}
\caption{Example of 2D clusters in a plane before and after merging attempt. Each oval represents a cluster with cluster 2D axis shown in a black solid arrow. The direction of an arrow represents the direction of a cluster development. Each cluster is labeled with an input cluster ID number while the oval color represents distinct output cluster group. Left: initial state with 4 input clusters. Right: example output after running a merge loop over cluster pairs. }
\label{sec:fmwk:cmerge:merge_example}
\end{center}\end{figure}
{\cmerge} takes care of creating a loop over all possible pairs of 2D clusters and calling the above function of {\cmalgo} for merging.
In the process, book-keeping of which pairs to be merged is recorded using {\cbkeeper}.

\begin{figure}[ht]\begin{center}
\includegraphics[width=13cm]{./src/Pictures/MergeProhibitExample.pdf}
\caption{Example of how prohibit algorithm works. Each oval and solid arrow represents a 2D cluster with its direction while the color represents distinct output (merged) cluster. Input cluster index (ID) is shown in the label with the cluster 2D angle. Right-top shows an output of running {\cmalgo} that merge clusters with angle difference within 40 degrees. Right-bottom shows an output of the same {\cmalgo} for merging but with another {\cmalgo} that prohibits merging of clusters with angle difference larger than 40 degrees.}
\label{sec:fmwk:cmerge:merge_prohibit_example}
\end{center}\end{figure}
Let us consider a case of two merged pairs that shares the same input cluster ID. As described in Sec.\ref{sec:fmwk:base:cbk}, {\cbkeeper} combines two pairs of 2D clusters in such case (see Fig.\ref{sec:fmwk:base:cbk}). This could be problematic sometimes. For instance, consider an example {\cmalgo} that combines two clusters if their start points are very close and the angle between two clusters are within 40 degrees. Further assume we have 3 input clusters that share the start point but have different angles at 20, 40, and -20 degrees. For convention these have input cluster ID 0, 1, and 2 respectively. This example algorithm would conclude pairs (0,1) and (0,2) but not for (1,2). But because merged pairs share cluster ID 0, it ends up all 3 clusters being merged. This is shown in the right-top ilustration of Fig.\ref{sec:fmwk:cmerge:merge_prohibit_example}.

This could be a problem, or may not be: this can be only answered by the algorithm author.
The way to avoid this in {\cmerge} is to implement another {\cmalgo} as {\it merge-prohibit}, or simple {\it prohibit} algorithm.
When prohibit algorithm is provided, {\cmerge} first run over all cluster pairs and execute {\ttfamily Bool()} function to find all cluster pairs that are prohibited to merge.
Then {\cmerge} run over the same set of cluster pairs except for those are prohibited to be merged.
The right-bottom ilustration of Fig.\ref{sec:fmwk:cmerge:merge_prohibit_example} shows how prohibit algorithm affects the result.

The prohibit rule is also inherited to merged cluster pairs. For instance, consider input clusters 0, 1 and 2. The pair (1,2) is prohibited to merge, and (0,1) is found to be merged. Then the merged cluster, (0,1) cannot be merged with 2. That means even if a merge algorithm finds (0,2) to be merged, it cannot be because the prohibit rule is inherited from cluster 1 when 0 and 1 aremerged. This feature brings up a question: what determines 0 to be merged with 1 rather than 2? If the merge algorithm is executed on the pair (0,2) first, then the result would have been different. So what determines a different outcome is an order of cluster pairs in the loop of merging inspection. There are different ordering options available in {\cmerge}. Currently available options are defined as {\enum} type in {\cmerge} class, {\ttfamily CMergePriority\_t} which include following values.
\begin{itemize}
\item[] {\ttfamily kIndex}
  \begin{itemize}
    \item From the smaller to larger input cluster ID, basically following input cluster set ordering
  \end{itemize}

\item[] {\ttfamily kPolyArea}
  \begin{itemize}
    \item From the clusters with smaller polygon area to the larger
  \end{itemize}

\item[] {\ttfamily kChargeSum}
  \begin{itemize}
    \item From the clusters with smaller charge sum to the larrger
  \end{itemize}

\item[] {\ttfamily kNHits}
  \begin{itemize}
    \item From the clusters with smaller number of hits to the larger
  \end{itemize}

\end{itemize}
The ordering parameters including all listed above are taken from {\ttfamily cluster\_params} member attributes.
If more useful parameter is found, one should make a request to make it available. 
It is very simple to implement in {\cmerge}.

In summary, {\cmerge} uses one {\cmalgo} instance for merging and another for merge-prohibit. A possible ambiguity of which cluster pairs to be merged in the presence of prohibited pairs is resolved by defining a priority for cluster pairs in the merging attempt. 

\subsection{Process Flow for Cluster Merging}
\label{sec:fmwk:cmerge:process_flow}
An overall process flow diagram is already shown in Fig.\ref{sec:fmwk:base:cmerge_stream}, so here we describe slightly more details in texts by breaking down into 5 blocks.
\begin{enumerate}
\item Event Preparation
  \begin{itemize}
    \item[] {\cmerge} receives an input cluster in terms of a set of {\pxhit} (see Sec.\ref{sec:fmwk:cmerge:input}). Then it creates {\cpan} instance for each cluster, and calls {\cmalgo::EventBegin()} for merging and prohibit algorithms.
  \end{itemize}

\item Iterative Merge Operation
  \begin{itemize}
    \item[] A loop over cluster pairs for prohibit (if provided) and merge {\cmalgo} can be configured to repeat until no newly merged cluster is found. This is shown in Fig.\ref{sec:fmwk:cmerge:iterative_stream}. If a user does not configure to choose iterative approach, iteration will not happen even if there is a newly merged cluster in the output.
  \end{itemize}

\item Prohibit {\cmalgo} Loop Over Cluster Pairs
  \begin{itemize}
    \item[] See Fig.\ref{sec:fmwk:cmerge:iterative_stream}. Only occurs if prohibit algorithm is provided by a user (optional). Records which cluster pair to be prohibited for merging using {\cbkeeper}.
  \end{itemize}

\item Prohibit {\cmalgo} Loop Over Cluster Pairs
  \begin{itemize}
    \item[] See Fig.\ref{sec:fmwk:cmerge:iterative_stream}. The algorithm's {\ttfamily Bool()} is not executed for a subject pair of clusters if (a) they are prohibited for merging or (b) already inspected in the previous process and not merged (i.e. result should not change). This is to save computation time.
  \end{itemize}

\item Merge Clusters
  \begin{itemize}
    \item[] Based on the outcome of merge/prohibit {\cmalgo} algorithms, actual merging is performed at this step. It creates a new set of {\cpan} for merged clusters. For clusters that are not merged, {\cpan} instance is copied from input to save {\cpan} computation time.
  \end{itemize}

\end{enumerate}

\begin{figure}[ht]\begin{center}
\includegraphics[width=13cm]{./src/Pictures/IterativeMergeApproach.pdf}
\caption{Illustration of how iterative merge approach is processed. Internally a loop over cluster pairs for prohibit and merge algorithm are repeated if (a) user configured to use an iterative approach and (b) newly merged cluster is found.}
\label{sec:fmwk:cmerge:iterative_stream}
\end{center}\end{figure}


\subsection{Setting an Input 2D Clusters}
\label{sec:fmwk:cmerge:input}
Input to {\cmerge} is a set of clusters where each cluster is represented as a vector of hits, or more precisely {\ttfamily std::vector<util::PxHit>}. This input can be set by the following function:
\begin{lstlisting}
  void SetClusters(const std::vector<std::vector<util::PxHit> > &)
\end{lstlisting}
where the argument is a vector of 2D clusters. When this function is called, {\cmerge} internally calls
\begin{lstlisting}
  void Reset()
\end{lstlisting}
which initialize previously stored input clusters, output clusters and {\cbkeeper} instance.
Then it creates {\cpan} instance per input 2D cluster using a set of {\pxhit}.
This completes setting (or initiating) the input 2D cluster sets to be used for internal processing.

\subsection{Setting Merge/Prohibit {\cmalgo} Instances}
{\cmalgo} instance is necessary for merging and optional for prohibit. Call the function,
\begin{lstlisting}
  void AddMergeAlg(CBoolAlgoBase* algo)
\end{lstlisting}
to set {\cmalgo} for merging, and 
\begin{lstlisting}
  void AddSeparateAlgo(CBoolAlgoBase* algo)
\end{lstlisting}
to set {\cmalgo} for merge-prohibit.

\subsection{Setting Run Configuration}
There are only 3 run options to configure for {\cmerge}.

\paragraph{Verbosity:} 
There are 5 verbosity levels a user can configure as listed below:
\begin{itemize}
\item[] {\ttfamily kNone}
  \begin{itemize}
    \item No report from {\cmerge}
  \end{itemize}

\item[] {\ttfamily kPerEvent}
  \begin{itemize}
    \item Report made at the end of even processing
  \end{itemize}

\item[] {\ttfamily kPerIteration}
  \begin{itemize}
    \item Report made per merge iteration (see Sec.\ref{sec:fmwk:cmerge:process_flow}). This could be quite verbose.      
  \end{itemize}

\item[] {\ttfamily kPerMerging}
  \begin{itemize}
    \item Report made per {\cmalgo::Bool()} execution (see Sec.\ref{sec:fmwk:cmerge:process_flow}). This is even more verbose than {\ttfamily kPerIteration}.
  \end{itemize}

\end{itemize}
These are {\enum} type, {\ttfamily CMergeMSGLevel\_t}, defined under {\cmerge} class. 
The verbosity level can be set by the function:
\begin{lstlisting}
  void DebugMode(CMergeMSGLevel_t level)
\end{lstlisting}

\paragraph{Iterative Approach:}
As described in Sec.\ref{sec:fmwk:cmerge:process_flow}, {\cmerge} can be configured to attempt iterative approach for merging. This can be enabled by the following function call.
\begin{lstlisting}
  void MergeTillConverge(bool)
\end{lstlisting}
Providing {\ttfamily true} as an argument will enable this option. Disabled if {\ttfamily false} is provided.

\paragraph{Merge Priority:}
As described in Sec.\ref{sec:fmwk:cmerge:how}, ordering of cluster pairs for merging inspection matters to the result.
Options are also described in the same section. This can be set by the following function.
\begin{lstlisting}
  void SetMergePriority(CMergePriority_t level)
\end{lstlisting}

\subsection{Execute Merging}
To execute merging, simply call
\begin{lstlisting}
  void Process()
\end{lstlisting}

\subsection{Obtaining Result}
There are two kinds of results that may be useful for a user. First is a resulting (i.e. merged) set of clusters in terms of {\cpan}. This is useful for further processing as various cluster parameters are already computed as a part of {\cpan}. You can execute
\begin{lstlisting}
  const std::vector<cluster::ClusterParamsAlg>& GetClusters() const
\end{lstlisting}
to obtain such result set.

Second is a mapping of input cluster IDs (i.e. index) to the output (merged clusters). 
As described in Sec.\ref{sec:fmwk:base:cbk}, this can be easily obtained through {\cbkeeper} instance.
You can obtain the {\cbkeeper} instance held by {\cmerge} through the following function.
\begin{lstlisting}
  const CBookKeeper& GetBookKeeper() const
\end{lstlisting}


\section{{\cmatch}: Cluster Matching Tool}
\label{sec:fmwk:cmatch}
Tool not introduced yet in {\larsoft}. Will be documented when this is done.




% Analysis package description
\chapter{Core Package: Analysis}
\label{chap:analysis}

Despite its extreme simplicity let us call {\cmtool} a framework, or FMWK. 
In this chapter we discuss the building blocks of {\cmtool}.

\section{External Dependency}
\label{sec:fmwk:external}

The {\cmtool} development effort is originated from EM shower reconstruction for MicroBooNE experiment, and it is an extension of existing tool set. The following is the brief list of packages or {\CPP} classes on which {\cmtool} depends.
\begin{itemize}
\item[] {\cru}
    \begin{itemize}
    \item larreco/RecoAlg/ClusterRecoUtil in {\larsoft}
    \item UserDev/ClusterRecoUtil in {\larlight}
    \end{itemize}
\item[] {\geoutil}
    \begin{itemize}
    \item lardata/Utilities in {\larsoft}
    \item core/LArUtil in {\larlight}
    \end{itemize}
\item[] {\ttfamily PxUtil}
    \begin{itemize}
    \item lardata/Utilities in {\larsoft}
    \item core/LArUtil in {\larlight}
    \end{itemize}
\end{itemize}
These items are (or should be if not yet) described in a separate document.
The rest of this section briefly covers what they are without going into a technical detail.

\subsection{{\ttfamily ClusterRecoUtil}}
\label{sec:fmwk:external:cru}
Reconstruction parameters for EM shower 2D cluster involves a lot more variety of parameters than those held in a 2D cluster data product defined by a framework ({\larsoft} or {\larlight}). {\cru} is a package that defines code for both calculation of and storage for those parameters. In particular, {\cmtool} uses following {\CPP} classes from {\cru}:
\begin{itemize}
\item {\ttfamily ClusterParamsAlg ... (\cpan)}
  \begin{itemize}
    \item {\cpan} is designed as both parameter computation algorithm and storage class, and is used along this design principle in {\cmtool}. In particular, {\cpan} is created one per input 2D cluster.
  \end{itemize}
\item {\ttfamily cluster\_params}
  \begin{itemize}
    \item {\ttfamily cluster\_params} is a utility storage class for important parameters computed by \cpan. A const reference can be retrieved from {\cpan} instance through {\ttfamily GetParams()} function call. Note, however, one should keep {\cpan} instance rather than {\ttfamily cluster\_params} as a cluster parameter data set (see {\cpan} description above).
  \end{itemize}
\item {\ttfamily CRUException}
  \begin{itemize}
    \item This is a very simple exception class derived from {\ttfamily std::exception}. It is used instead of {\ttfamily cet::exception} to avoid framework dependency. Only additional feature to the base class is that the argument string to the constructor is sent to standard output streem by overriden {\ttfamily std::exception::what()} function.
  \end{itemize}
\end{itemize}

For details of all classes listed above, you should see Ref.\cite{ClusterRecoUtil}.

\subsection{{\ttfamily PxUtil}}
\label{sec:fmwk:external:pxutil}
Both {\cru} and {\cmtool} are designed to be independent of an analysis framework of user's choice (in particular {\larlight} and {\larsoft}). 
As mentioned in Sec.\ref{sec:fmwk:external:cru}, output of {\cpan} (i.e. cluster parameters) is stored with {\ttfamily cluster\_params} instance. 
{\pxutil}, on the other hand, is used to define input data struct to {\cpan} (and hence {\cmtool} and {\cru} in general). {\pxutil} defines three data objects:
\begin{itemize}
  \item {\ttfamily PxPoint}
    \begin{itemize}
    \item represents 2D point and contains wire [cm], time [cm], and plane ID (unsigned char)
    \end{itemize}
  \item {\ttfamily PxHit}
    \begin{itemize}
    \item inhertis from {\ttfamily PxPoint} with an additional attribute to hold charge [ADC], representing 2D hit
    \end{itemize}
  \item {\ttfamily PxLine}
    \begin{itemize}
    \item represents 2D line and contains start and end coordinates, (wire [cm], time [cm]), with a plane ID (unsigned char)
    \end{itemize}
\end{itemize}

It is a FMWK interface module's respnsibility to translate input data product into these format. In particular, this module must convert a vector of FMWK hit data product (= cluster) into a vector of {\pxhit}. This is discussed in FMWK interface chapter (see Ch.\ref{chap:fmwk_interface}).

\subsection{{\ttfamily GeometryUtilities}}
\label{sec:fmwk:external:geoutil}
{\CPP} class {\geoutil} is a utility class that helps to retrieve detector geometry related information such as converting 3D point coordinate into a projected 2D coordinate on a given plane. 
{\cmtool} uses {\geoutil} to interpret geometry related variables in full extent. 
For development of any non-existing function that treats detector geometry information, it is encouraged to develop in {\geoutil} (and certainly not in {\cmtool}). 
Please see Ref.\cite{GeometryUtilities} for more detailed documentation.


\section{FMWK Bulding Blocks}
\label{sec:fmwk:base}
In this section we discuss two core buliding block classes: (1) {\cbalgo} and (2) {\cbkeeper}. Though we discuss these basic {\CPP} classes first, it might be easier to understand these classes after you read about {\cmerge}, described in Sec.{\ref{sec:fmwk:cmerge}}, because there we discuss how {\cmerge} works with practical usage of {\cbalgo} and {\cbkeeper}.

\subsection{Book Keeping: {\cbkeeper}}
\label{sec:fmwk:base:cbk}
{\cbkeeper} inherits from {\ttfamily std::vector<unsigned short>}, and hence it is a vector of integers by itself.
The length of vector can be specified using a constructor argument.

This class is used to track how input clusters are merged/matched into an output. 
Here we focus on merging of input clusters for the simplicity sake.
The design assumes one to perform a merging attempt of 2 clusters at a time.
When one has many clusters to make such attempt on all possible pairs of clusters, it could be troublesome to perform book-keeping for which clusters are merged together.
{\cbkeeper} is designed such that a user can just log two to-be-merged clusters' ID at each attempt, and it holds a summary log of all merged cluster pairs.


Here are technical details.
The index of vector corresponds to the initial cluster ID numbers.
If you have 8 clusters to be subject for merging, {\cbkeeper} instance should be initialized as length 8.
Initially the values stored in the vector is same as their index numbers as shown in the top of Fig.\ref{sec:fmwk:base:cbk}.
One can re-set the state of {\cbkeeper} instance by calling the following function.
\begin{lstlisting}
        void Reset(unsigned short n=0)
\end{lstlisting}

\begin{figure}\begin{center}
\includegraphics[width=14cm]{./src/Pictures/CBookKeeperExample.pdf}
\caption{Example diagram showing how {\cbkeeper} keeps track of input and output cluster IDs. It shows {\cbkeeper} vector index (input cluster ID) and stored values (output cluster ID) for 8 input clusters (top), and how the stored values change as merging takes place. The middle shows the result of merging input cluster 1 and 2, and the bottom is for further merging 2 and 5.}
\end{center}\end{figure}
When one finds 2 clusters to be merged together, (s)he can call {\cbkeeper} attribute function:
\begin{lstlisting}
        void CBookKeeper::Merge(unsigned short, unsigned short)
\end{lstlisting}
with corresponding clusters' IDs (i.e. index numbers). 
Figure \ref{sec:fmwk:base:cbk} shows how the process works for calling {\ttfamily CBookKeeper::Merge()} function twice with arguments (1,2) and (2,5).
As a result, unique output cluster IDs are reduced to 6 because 3 clusters are combined together.

You can see {\cbkeeper} instance allows to retrieve which output cluster (stored value) an input cluster (vector index) ended up with in an intuitive way: simply use {\ttfamily std::vector::at()} method.
However, to create an output cluster, it is probably easier to ask the other way around: given output cluster ID, which input cluster(s) does it contain?
{\cbkeeper} has a simple function to answer this question:
\begin{lstlisting}
        std::vector<std::vector<unsigned short> > GetResult() const 
\end{lstlisting}
which returns a vector with length same as number of output (i.e. merged) clusters.
The value of this vector is {\ttfamily std::vector<unsigned short>} to contain possibly multiple input cluster indexes.
For instance, the return from {\cbkeeper} shown in the bottom of Fig.\ref{sec:fmwk:base:cbk} would be a vector of length 6, and the index1 contains a vector of input cluster ID 1, 2, and 5 as they have been merged into output cluster ID = 1.
Equivalent result can be obtained also by using
\begin{lstlisting}
        void PassResult(std::vector<std::vector<unsigned short> >&)
\end{lstlisting}
if that suits one's usage. 

If you are interested in checking which input cluster(s) included in a particular output (merged) cluster, you can use:
\begin{lstlisting}
        std::vector<unsigned short> GetMergedSet(unsigned short)
\end{lstlisting}
which takes an output cluster index as an argument and returns a vector that contains a list of input clusters that consist the output.

\begin{figure}[ht]\begin{center}
\includegraphics[width=8.3cm]{./src/Pictures/CBookKeeperProhibitExample2.pdf}
\includegraphics[width=8.3cm]{./src/Pictures/CBookKeeperProhibitExample1.pdf}
\caption{Example diagrams showing how {\cbkeeper} reacts to conflict {\ttfamily Merge()} and {\ttfamily ProhibitMerge()} function calls on the input cluster IDs. The left diagram shows {\crue} thrown due to calling {\ttfamily Merge()} on a cluster pair that is registered as merge-prohibit pair. The right diagram shows a result from calling {\ttfamily ProhibitMerge()} on a cluster pair that is already merged.}
\label{sec:fmwk:base:cbk_prohibit}
\end{center}\end{figure}

Finally, {\cmerge} can also prohibit merging of specific two cluster pair(s), and this is incoorporated by {\cbkeeper} as well.
This is, again, to support user to log merging in a mindless manner.
A user may specify specific 2 cluster pair should not be merged. 
Then (s)he can use the following function to register these 2 clusters for merge-prohibit list.
\begin{lstlisting}
        void ProhibitMerge(unsigned short, unsigned short)
\end{lstlisting}
The above function takes input cluster IDs (i.e. index numbers). 
Once 2 clusters are registered as merge-prohibit pair, attempting a function call {\ttfamily Merge()} will result in {\crue} thrown.
This is illustrated in the left of Fig.\ref{sec:fmwk:base:cbk_prohibit}.
Another possible case that results in {\crue} thrown would be calling {\ttfamily ProhibitMerge()} function on two input cluster IDs that are already merged. This is shown in the right hand side of Fig.\ref{sec:fmwk:base:cbk_prohibit}.

To avoid {\crue}, one should register all cluster pairs to prohibit-merge list before trying any merge attempt.
Upon attempting to call {\ttfamily Merge()}, then, one can use the function:
\begin{lstlisting}
        bool MergeAllowed(unsigned short, unsigned short)
\end{lstlisting}
to check if a cluster pair is registered to merge-prohibit list.
Calling this function prior to evaluation of merging (i.e. running some computation algorithm to decide whether to merge or not) may save some computation time and help the overall speed of the program (this is implemented in {\cmerge}).

\subsection{Algorithm Base: {\ttfamily CBoolAlgoBase}}
\label{sec:fmwk:base:cbalgo}
{\cbalgo} is a simple {\CPP} template class and also a base class for an algorithm used for cluster merging and matching by {\cmerge} and {\cmatch} application classes. In this and later sections, we use the term {\cmalgo} to refer to an algorithm class derived from {\cbalgo}.

\begin{figure}[ht]\begin{center}
\includegraphics[width=13cm]{./src/Pictures/CMergeStream.pdf}
\caption{The data processing stream of {\cmerge} framework. The purple allow shows the ordered process flow. It requires an analysis framework interface to convert framework dependent cluster data produdct into a vector of {\pxhit} and hand over to {\cmerge} application. Internally {\cmerge} calls various members of {\cbalgo} instances: one for merging and another for prohibiting. There is a configuration in {\cmerge} to perform iterative approach, which repeats internal loop over possible cluster pairs until no more newly merged cluster is found in the output. The merged cluster groups (i.e. output) can be obtained through {\cbkeeper} instance by an analysis framework interface to form an appropriate output data product.}
\label{sec:fmwk:base:cmerge_stream}
\end{center}\end{figure}

The idea is very simple: given two clusters, it is expected to return a boolean for decision making from the following function.
\begin{lstlisting}
        virtual bool Bool(const cluster::ClusterParamsAlg &, const cluster::ClusterParamsAlg &)
\end{lstlisting}
As you can see, ``clusters'' are handed over in terms of {\cpan} instance.
In {\cmerge}, the cluster merging application class, for instance, this is used for making merge decision as well as prohibit-merge decision (i.e. {\cmerge} holds two {\cbalgo} inherit class instance, one for merging and another for prohibiting).
An algorithm developpers should write his/her own {\CPP} algorithm class by inheriting from {\cbalgo}.
In the derived class, the {\ttfamily Bool()} function implementation should be over-riden. 
There is a script to generate a template (i.e. empty) {\cbalgo} inherit class and everyone is encouraged to use it for starting a new algorithm class. These are:
\begin{itemize}
\item For {\larsoft}
\begin{itemize}
        \item[]{\ttfamily> python \$MRB\_TOP/srcs/larreco/RecoAlg/CMTool/mac/gen\_cmalgo.py NAME}
\end{itemize}
\item For {\larlight}
\begin{itemize}
        \item[]{\ttfamily> python \$MAKE\_TOP\_DIR/UserDev/ClusterStudy/bin/gen\_cmalgo NAME}
\end{itemize}
\end{itemize}
where {\ttfamily NAME} should be replaced by the new algorithm's name. 
Running above command results in creating {\ttfamily CMAlgoNAME.h} and {\ttfamily CMAlgoNAME.cxx} in the appropriate directory.
Then you can proceed to build following the framework's guidance.

As mentioned, {\cbalgo} is operated by {\cmerge} and {\cmatch} applications.
For a given event, there is an internal loop of calling {\ttfamily Bool()} on all possible cluster pairs.
To provide flexibility in algorithm design, those applications also call other functions of {\cbalgo} in its processing.

The following is a list of such {\cbalgo} functions.
\begin{itemize}

\item[]{\ttfamily virtual void EventBegin(const std::vector<cluster::ClusterParamsAlg>\&)}
  \begin{itemize}
    \item Called at the beginning of an event, prior to looping over cluster pairs. The provided function argument is a vector of {\cpan} for all clusters subject for merging. This is where computation that requires the whole, initial cluster sets should be done.
  \end{itemize}

\item[]{\ttfamily virtual void IterationBegin(const std::vector<cluster::ClusterParamsAlg>\&)}
  \begin{itemize}
    \item Called at the beginning of an each loop over cluster pairs. Note {\cmerge} and {\cmatch} can have multiple iteration of looping over possible cluster pairs if configured to do so. For instance {\cmerge} can take iterate over multiple loops till no more to-be-merged  cluster pair is found (see Sec.\ref{sec:fmwk:cmerge} and Sec.\ref{sec:fmwk:cmatch} for details). The argument of this function is an output cluster sets, in terms of {\cpan} instance, from the previous loop. In case of the 1st loop, the provided cluster sets is identical to those provided in the argument of {\ttfamily EventBegin()} function.
  \end{itemize}

\item[]{\ttfamily virtual void IterationEnd()}
  \begin{itemize}
    \item Called at the end of each merge or match attempt loop over all cluster pairs, to clean up or summarize information.
  \end{itemize}

\item[]{\ttfamily virtual void EventEnd()}
  \begin{itemize}
    \item Called at the end of each event, after all iterations over cluster pairs are done.
  \end{itemize}

\end{itemize}

Further, there are two functions that are not yet implemented in {\cmerge} nor {\cmatch} but ideas exist.
Your input on how these should be implemented is welcome and very much appreciated.
\begin{itemize}

\item[]{\ttfamily virtual void Reset()}
  \begin{itemize}
    \item Meant to clear some member variables. But one can do that within {\ttfamily IterationEnd()} and/or {\ttfamily EventEnd()}, so it is not clear where this should be called. Possibly at the end of the whole event processing.
  \end{itemize}

\item[]{\ttfamily virtual void Report()}
  \begin{itemize}
    \item Meant to support some standardized {\ttfamily std::cout} and {\ttfamily std::cerr} report from an algorithm instance. {\cmerge} and {\cmatch} have ability to control such text output for different levels, and there is a hope this {\ttfamily Report()} can be also included there.
  \end{itemize}

\end{itemize}

\paragraph{Tips for writing {\cmalgo}}
First of all, you are encouraged to use {\ttfamily gen\_cmalgo} script as described earlier.
Another thing to note is to reduce a dependency on analysis framework.
This makes it easier to port your code between different frameworks.



\section{{\cmerge}: Cluster Merging Tool}
\label{sec:fmwk:cmerge}

{\cmerge} is a {\CPP} class that performs merging of 2D clusters, and is an application of {\cmalgo} algorithm classes that are inherit from {\cbalgo}, the algorithm base class. In a few sentences, {\cmerge} perform merging by looping over all possible 2D cluster pairs among the given set of clusters. For each pair, {\ttfamily Bool()} function of both merging and prohibit {\cmalgo} are called to inspect possible 2D cluster merging. It uses {\cbkeeper} to keep track of which pair of clusters are merged. For details of {\cbalgo} and {\cbkeeper}, see Sec.\ref{sec:fmwk:base:cbalgo} and Sec.\ref{sec:fmwk:base:cbk} respectively. The overall process stream diagram is shown in Fig.\ref{sec:fmwk:base:cmerge_stream} which we keep referring in this section.

\subsection{Why {\cmerge}?}
Why using {\cmerge}? Why not just writing a program that does everything from scratch?
Be my guest and write your code! That might inspire a better design or strategy toward merging. 
That being said, there are a few reasons to use {\cmerge}.

\paragraph{Task Outside Merging Decision:}
{\cmerge} The core part of merging is to compare 2 clusters and make a decision to merge or not. Outside this process, however, there is a non-negligible overburden of running such algorithm such as (1) keeping track of which cluster pairs are merged and (2) optimizing the loop process of cluster pair comparison. {\cmerge} does those tedious jobs while out-sourcing the decision making to an independent {\CPP} algorithm class that inherit from {\cbalgo}. 

\paragraph{Easy Development via Modulated Design:}
Because {\cmerge} outsource an independent algorithm class to perform merging of a cluster pair, it supports modulated code design. Accordingly each {\cmalgo} class becomes a pure algorithm class. This makes a code development very easy because a new algorithm can be written without re-writing the I/O or book-keeping part. Also it makes life easy to compare different algorithms: you can simple change which algorithm to be attached to {\cmerge} at run-time, and this avoids re-compilation or re-writing of code.

\paragraph{X-Framework Implementation}
{\cmerge} does not use data product type specific to an analysis framework (for instance {\larlight} or {\larsoft}), so it can be ported easily among different frameworks. That means {\cmalgo} developed by one person in one analysis framework can be shared among other frameworks (assuming an algorithm developer follows x-framework design pattern).\\

\subsection{How Does It Work?}
\label{sec:fmwk:cmerge:how}
The core part of decision making is simple: provided 2 clusters, decide whether they should be merged or not.
This is done through {\cmalgo::Bool()} function call:
\begin{lstlisting}
  virtual bool Bool( const cluster::ClusterParamsAlg &, const cluster::ClusterParamsAlg &)
\end{lstlisting}
If the return is {\ttfamily true}, two clusters are merged. Else not.
This is ilustrated in Fig.\ref{sec:fmwk:cmerge:merge_example}.

\begin{figure}[ht]\begin{center}
\includegraphics[width=13cm]{./src/Pictures/MergeExample.pdf}
\caption{Example of 2D clusters in a plane before and after merging attempt. Each oval represents a cluster with cluster 2D axis shown in a black solid arrow. The direction of an arrow represents the direction of a cluster development. Each cluster is labeled with an input cluster ID number while the oval color represents distinct output cluster group. Left: initial state with 4 input clusters. Right: example output after running a merge loop over cluster pairs. }
\label{sec:fmwk:cmerge:merge_example}
\end{center}\end{figure}
{\cmerge} takes care of creating a loop over all possible pairs of 2D clusters and calling the above function of {\cmalgo} for merging.
In the process, book-keeping of which pairs to be merged is recorded using {\cbkeeper}.

\begin{figure}[ht]\begin{center}
\includegraphics[width=13cm]{./src/Pictures/MergeProhibitExample.pdf}
\caption{Example of how prohibit algorithm works. Each oval and solid arrow represents a 2D cluster with its direction while the color represents distinct output (merged) cluster. Input cluster index (ID) is shown in the label with the cluster 2D angle. Right-top shows an output of running {\cmalgo} that merge clusters with angle difference within 40 degrees. Right-bottom shows an output of the same {\cmalgo} for merging but with another {\cmalgo} that prohibits merging of clusters with angle difference larger than 40 degrees.}
\label{sec:fmwk:cmerge:merge_prohibit_example}
\end{center}\end{figure}
Let us consider a case of two merged pairs that shares the same input cluster ID. As described in Sec.\ref{sec:fmwk:base:cbk}, {\cbkeeper} combines two pairs of 2D clusters in such case (see Fig.\ref{sec:fmwk:base:cbk}). This could be problematic sometimes. For instance, consider an example {\cmalgo} that combines two clusters if their start points are very close and the angle between two clusters are within 40 degrees. Further assume we have 3 input clusters that share the start point but have different angles at 20, 40, and -20 degrees. For convention these have input cluster ID 0, 1, and 2 respectively. This example algorithm would conclude pairs (0,1) and (0,2) but not for (1,2). But because merged pairs share cluster ID 0, it ends up all 3 clusters being merged. This is shown in the right-top ilustration of Fig.\ref{sec:fmwk:cmerge:merge_prohibit_example}.

This could be a problem, or may not be: this can be only answered by the algorithm author.
The way to avoid this in {\cmerge} is to implement another {\cmalgo} as {\it merge-prohibit}, or simple {\it prohibit} algorithm.
When prohibit algorithm is provided, {\cmerge} first run over all cluster pairs and execute {\ttfamily Bool()} function to find all cluster pairs that are prohibited to merge.
Then {\cmerge} run over the same set of cluster pairs except for those are prohibited to be merged.
The right-bottom ilustration of Fig.\ref{sec:fmwk:cmerge:merge_prohibit_example} shows how prohibit algorithm affects the result.

The prohibit rule is also inherited to merged cluster pairs. For instance, consider input clusters 0, 1 and 2. The pair (1,2) is prohibited to merge, and (0,1) is found to be merged. Then the merged cluster, (0,1) cannot be merged with 2. That means even if a merge algorithm finds (0,2) to be merged, it cannot be because the prohibit rule is inherited from cluster 1 when 0 and 1 aremerged. This feature brings up a question: what determines 0 to be merged with 1 rather than 2? If the merge algorithm is executed on the pair (0,2) first, then the result would have been different. So what determines a different outcome is an order of cluster pairs in the loop of merging inspection. There are different ordering options available in {\cmerge}. Currently available options are defined as {\enum} type in {\cmerge} class, {\ttfamily CMergePriority\_t} which include following values.
\begin{itemize}
\item[] {\ttfamily kIndex}
  \begin{itemize}
    \item From the smaller to larger input cluster ID, basically following input cluster set ordering
  \end{itemize}

\item[] {\ttfamily kPolyArea}
  \begin{itemize}
    \item From the clusters with smaller polygon area to the larger
  \end{itemize}

\item[] {\ttfamily kChargeSum}
  \begin{itemize}
    \item From the clusters with smaller charge sum to the larrger
  \end{itemize}

\item[] {\ttfamily kNHits}
  \begin{itemize}
    \item From the clusters with smaller number of hits to the larger
  \end{itemize}

\end{itemize}
The ordering parameters including all listed above are taken from {\ttfamily cluster\_params} member attributes.
If more useful parameter is found, one should make a request to make it available. 
It is very simple to implement in {\cmerge}.

In summary, {\cmerge} uses one {\cmalgo} instance for merging and another for merge-prohibit. A possible ambiguity of which cluster pairs to be merged in the presence of prohibited pairs is resolved by defining a priority for cluster pairs in the merging attempt. 

\subsection{Process Flow for Cluster Merging}
\label{sec:fmwk:cmerge:process_flow}
An overall process flow diagram is already shown in Fig.\ref{sec:fmwk:base:cmerge_stream}, so here we describe slightly more details in texts by breaking down into 5 blocks.
\begin{enumerate}
\item Event Preparation
  \begin{itemize}
    \item[] {\cmerge} receives an input cluster in terms of a set of {\pxhit} (see Sec.\ref{sec:fmwk:cmerge:input}). Then it creates {\cpan} instance for each cluster, and calls {\cmalgo::EventBegin()} for merging and prohibit algorithms.
  \end{itemize}

\item Iterative Merge Operation
  \begin{itemize}
    \item[] A loop over cluster pairs for prohibit (if provided) and merge {\cmalgo} can be configured to repeat until no newly merged cluster is found. This is shown in Fig.\ref{sec:fmwk:cmerge:iterative_stream}. If a user does not configure to choose iterative approach, iteration will not happen even if there is a newly merged cluster in the output.
  \end{itemize}

\item Prohibit {\cmalgo} Loop Over Cluster Pairs
  \begin{itemize}
    \item[] See Fig.\ref{sec:fmwk:cmerge:iterative_stream}. Only occurs if prohibit algorithm is provided by a user (optional). Records which cluster pair to be prohibited for merging using {\cbkeeper}.
  \end{itemize}

\item Prohibit {\cmalgo} Loop Over Cluster Pairs
  \begin{itemize}
    \item[] See Fig.\ref{sec:fmwk:cmerge:iterative_stream}. The algorithm's {\ttfamily Bool()} is not executed for a subject pair of clusters if (a) they are prohibited for merging or (b) already inspected in the previous process and not merged (i.e. result should not change). This is to save computation time.
  \end{itemize}

\item Merge Clusters
  \begin{itemize}
    \item[] Based on the outcome of merge/prohibit {\cmalgo} algorithms, actual merging is performed at this step. It creates a new set of {\cpan} for merged clusters. For clusters that are not merged, {\cpan} instance is copied from input to save {\cpan} computation time.
  \end{itemize}

\end{enumerate}

\begin{figure}[ht]\begin{center}
\includegraphics[width=13cm]{./src/Pictures/IterativeMergeApproach.pdf}
\caption{Illustration of how iterative merge approach is processed. Internally a loop over cluster pairs for prohibit and merge algorithm are repeated if (a) user configured to use an iterative approach and (b) newly merged cluster is found.}
\label{sec:fmwk:cmerge:iterative_stream}
\end{center}\end{figure}


\subsection{Setting an Input 2D Clusters}
\label{sec:fmwk:cmerge:input}
Input to {\cmerge} is a set of clusters where each cluster is represented as a vector of hits, or more precisely {\ttfamily std::vector<util::PxHit>}. This input can be set by the following function:
\begin{lstlisting}
  void SetClusters(const std::vector<std::vector<util::PxHit> > &)
\end{lstlisting}
where the argument is a vector of 2D clusters. When this function is called, {\cmerge} internally calls
\begin{lstlisting}
  void Reset()
\end{lstlisting}
which initialize previously stored input clusters, output clusters and {\cbkeeper} instance.
Then it creates {\cpan} instance per input 2D cluster using a set of {\pxhit}.
This completes setting (or initiating) the input 2D cluster sets to be used for internal processing.

\subsection{Setting Merge/Prohibit {\cmalgo} Instances}
{\cmalgo} instance is necessary for merging and optional for prohibit. Call the function,
\begin{lstlisting}
  void AddMergeAlg(CBoolAlgoBase* algo)
\end{lstlisting}
to set {\cmalgo} for merging, and 
\begin{lstlisting}
  void AddSeparateAlgo(CBoolAlgoBase* algo)
\end{lstlisting}
to set {\cmalgo} for merge-prohibit.

\subsection{Setting Run Configuration}
There are only 3 run options to configure for {\cmerge}.

\paragraph{Verbosity:} 
There are 5 verbosity levels a user can configure as listed below:
\begin{itemize}
\item[] {\ttfamily kNone}
  \begin{itemize}
    \item No report from {\cmerge}
  \end{itemize}

\item[] {\ttfamily kPerEvent}
  \begin{itemize}
    \item Report made at the end of even processing
  \end{itemize}

\item[] {\ttfamily kPerIteration}
  \begin{itemize}
    \item Report made per merge iteration (see Sec.\ref{sec:fmwk:cmerge:process_flow}). This could be quite verbose.      
  \end{itemize}

\item[] {\ttfamily kPerMerging}
  \begin{itemize}
    \item Report made per {\cmalgo::Bool()} execution (see Sec.\ref{sec:fmwk:cmerge:process_flow}). This is even more verbose than {\ttfamily kPerIteration}.
  \end{itemize}

\end{itemize}
These are {\enum} type, {\ttfamily CMergeMSGLevel\_t}, defined under {\cmerge} class. 
The verbosity level can be set by the function:
\begin{lstlisting}
  void DebugMode(CMergeMSGLevel_t level)
\end{lstlisting}

\paragraph{Iterative Approach:}
As described in Sec.\ref{sec:fmwk:cmerge:process_flow}, {\cmerge} can be configured to attempt iterative approach for merging. This can be enabled by the following function call.
\begin{lstlisting}
  void MergeTillConverge(bool)
\end{lstlisting}
Providing {\ttfamily true} as an argument will enable this option. Disabled if {\ttfamily false} is provided.

\paragraph{Merge Priority:}
As described in Sec.\ref{sec:fmwk:cmerge:how}, ordering of cluster pairs for merging inspection matters to the result.
Options are also described in the same section. This can be set by the following function.
\begin{lstlisting}
  void SetMergePriority(CMergePriority_t level)
\end{lstlisting}

\subsection{Execute Merging}
To execute merging, simply call
\begin{lstlisting}
  void Process()
\end{lstlisting}

\subsection{Obtaining Result}
There are two kinds of results that may be useful for a user. First is a resulting (i.e. merged) set of clusters in terms of {\cpan}. This is useful for further processing as various cluster parameters are already computed as a part of {\cpan}. You can execute
\begin{lstlisting}
  const std::vector<cluster::ClusterParamsAlg>& GetClusters() const
\end{lstlisting}
to obtain such result set.

Second is a mapping of input cluster IDs (i.e. index) to the output (merged clusters). 
As described in Sec.\ref{sec:fmwk:base:cbk}, this can be easily obtained through {\cbkeeper} instance.
You can obtain the {\cbkeeper} instance held by {\cmerge} through the following function.
\begin{lstlisting}
  const CBookKeeper& GetBookKeeper() const
\end{lstlisting}


\section{{\cmatch}: Cluster Matching Tool}
\label{sec:fmwk:cmatch}
Tool not introduced yet in {\larsoft}. Will be documented when this is done.




% LArUtil package description
\chapter{Core Package: LArUtil}
\label{chap:larutil}

Despite its extreme simplicity let us call {\cmtool} a framework, or FMWK. 
In this chapter we discuss the building blocks of {\cmtool}.

\section{External Dependency}
\label{sec:fmwk:external}

The {\cmtool} development effort is originated from EM shower reconstruction for MicroBooNE experiment, and it is an extension of existing tool set. The following is the brief list of packages or {\CPP} classes on which {\cmtool} depends.
\begin{itemize}
\item[] {\cru}
    \begin{itemize}
    \item larreco/RecoAlg/ClusterRecoUtil in {\larsoft}
    \item UserDev/ClusterRecoUtil in {\larlight}
    \end{itemize}
\item[] {\geoutil}
    \begin{itemize}
    \item lardata/Utilities in {\larsoft}
    \item core/LArUtil in {\larlight}
    \end{itemize}
\item[] {\ttfamily PxUtil}
    \begin{itemize}
    \item lardata/Utilities in {\larsoft}
    \item core/LArUtil in {\larlight}
    \end{itemize}
\end{itemize}
These items are (or should be if not yet) described in a separate document.
The rest of this section briefly covers what they are without going into a technical detail.

\subsection{{\ttfamily ClusterRecoUtil}}
\label{sec:fmwk:external:cru}
Reconstruction parameters for EM shower 2D cluster involves a lot more variety of parameters than those held in a 2D cluster data product defined by a framework ({\larsoft} or {\larlight}). {\cru} is a package that defines code for both calculation of and storage for those parameters. In particular, {\cmtool} uses following {\CPP} classes from {\cru}:
\begin{itemize}
\item {\ttfamily ClusterParamsAlg ... (\cpan)}
  \begin{itemize}
    \item {\cpan} is designed as both parameter computation algorithm and storage class, and is used along this design principle in {\cmtool}. In particular, {\cpan} is created one per input 2D cluster.
  \end{itemize}
\item {\ttfamily cluster\_params}
  \begin{itemize}
    \item {\ttfamily cluster\_params} is a utility storage class for important parameters computed by \cpan. A const reference can be retrieved from {\cpan} instance through {\ttfamily GetParams()} function call. Note, however, one should keep {\cpan} instance rather than {\ttfamily cluster\_params} as a cluster parameter data set (see {\cpan} description above).
  \end{itemize}
\item {\ttfamily CRUException}
  \begin{itemize}
    \item This is a very simple exception class derived from {\ttfamily std::exception}. It is used instead of {\ttfamily cet::exception} to avoid framework dependency. Only additional feature to the base class is that the argument string to the constructor is sent to standard output streem by overriden {\ttfamily std::exception::what()} function.
  \end{itemize}
\end{itemize}

For details of all classes listed above, you should see Ref.\cite{ClusterRecoUtil}.

\subsection{{\ttfamily PxUtil}}
\label{sec:fmwk:external:pxutil}
Both {\cru} and {\cmtool} are designed to be independent of an analysis framework of user's choice (in particular {\larlight} and {\larsoft}). 
As mentioned in Sec.\ref{sec:fmwk:external:cru}, output of {\cpan} (i.e. cluster parameters) is stored with {\ttfamily cluster\_params} instance. 
{\pxutil}, on the other hand, is used to define input data struct to {\cpan} (and hence {\cmtool} and {\cru} in general). {\pxutil} defines three data objects:
\begin{itemize}
  \item {\ttfamily PxPoint}
    \begin{itemize}
    \item represents 2D point and contains wire [cm], time [cm], and plane ID (unsigned char)
    \end{itemize}
  \item {\ttfamily PxHit}
    \begin{itemize}
    \item inhertis from {\ttfamily PxPoint} with an additional attribute to hold charge [ADC], representing 2D hit
    \end{itemize}
  \item {\ttfamily PxLine}
    \begin{itemize}
    \item represents 2D line and contains start and end coordinates, (wire [cm], time [cm]), with a plane ID (unsigned char)
    \end{itemize}
\end{itemize}

It is a FMWK interface module's respnsibility to translate input data product into these format. In particular, this module must convert a vector of FMWK hit data product (= cluster) into a vector of {\pxhit}. This is discussed in FMWK interface chapter (see Ch.\ref{chap:fmwk_interface}).

\subsection{{\ttfamily GeometryUtilities}}
\label{sec:fmwk:external:geoutil}
{\CPP} class {\geoutil} is a utility class that helps to retrieve detector geometry related information such as converting 3D point coordinate into a projected 2D coordinate on a given plane. 
{\cmtool} uses {\geoutil} to interpret geometry related variables in full extent. 
For development of any non-existing function that treats detector geometry information, it is encouraged to develop in {\geoutil} (and certainly not in {\cmtool}). 
Please see Ref.\cite{GeometryUtilities} for more detailed documentation.


\section{FMWK Bulding Blocks}
\label{sec:fmwk:base}
In this section we discuss two core buliding block classes: (1) {\cbalgo} and (2) {\cbkeeper}. Though we discuss these basic {\CPP} classes first, it might be easier to understand these classes after you read about {\cmerge}, described in Sec.{\ref{sec:fmwk:cmerge}}, because there we discuss how {\cmerge} works with practical usage of {\cbalgo} and {\cbkeeper}.

\subsection{Book Keeping: {\cbkeeper}}
\label{sec:fmwk:base:cbk}
{\cbkeeper} inherits from {\ttfamily std::vector<unsigned short>}, and hence it is a vector of integers by itself.
The length of vector can be specified using a constructor argument.

This class is used to track how input clusters are merged/matched into an output. 
Here we focus on merging of input clusters for the simplicity sake.
The design assumes one to perform a merging attempt of 2 clusters at a time.
When one has many clusters to make such attempt on all possible pairs of clusters, it could be troublesome to perform book-keeping for which clusters are merged together.
{\cbkeeper} is designed such that a user can just log two to-be-merged clusters' ID at each attempt, and it holds a summary log of all merged cluster pairs.


Here are technical details.
The index of vector corresponds to the initial cluster ID numbers.
If you have 8 clusters to be subject for merging, {\cbkeeper} instance should be initialized as length 8.
Initially the values stored in the vector is same as their index numbers as shown in the top of Fig.\ref{sec:fmwk:base:cbk}.
One can re-set the state of {\cbkeeper} instance by calling the following function.
\begin{lstlisting}
        void Reset(unsigned short n=0)
\end{lstlisting}

\begin{figure}\begin{center}
\includegraphics[width=14cm]{./src/Pictures/CBookKeeperExample.pdf}
\caption{Example diagram showing how {\cbkeeper} keeps track of input and output cluster IDs. It shows {\cbkeeper} vector index (input cluster ID) and stored values (output cluster ID) for 8 input clusters (top), and how the stored values change as merging takes place. The middle shows the result of merging input cluster 1 and 2, and the bottom is for further merging 2 and 5.}
\end{center}\end{figure}
When one finds 2 clusters to be merged together, (s)he can call {\cbkeeper} attribute function:
\begin{lstlisting}
        void CBookKeeper::Merge(unsigned short, unsigned short)
\end{lstlisting}
with corresponding clusters' IDs (i.e. index numbers). 
Figure \ref{sec:fmwk:base:cbk} shows how the process works for calling {\ttfamily CBookKeeper::Merge()} function twice with arguments (1,2) and (2,5).
As a result, unique output cluster IDs are reduced to 6 because 3 clusters are combined together.

You can see {\cbkeeper} instance allows to retrieve which output cluster (stored value) an input cluster (vector index) ended up with in an intuitive way: simply use {\ttfamily std::vector::at()} method.
However, to create an output cluster, it is probably easier to ask the other way around: given output cluster ID, which input cluster(s) does it contain?
{\cbkeeper} has a simple function to answer this question:
\begin{lstlisting}
        std::vector<std::vector<unsigned short> > GetResult() const 
\end{lstlisting}
which returns a vector with length same as number of output (i.e. merged) clusters.
The value of this vector is {\ttfamily std::vector<unsigned short>} to contain possibly multiple input cluster indexes.
For instance, the return from {\cbkeeper} shown in the bottom of Fig.\ref{sec:fmwk:base:cbk} would be a vector of length 6, and the index1 contains a vector of input cluster ID 1, 2, and 5 as they have been merged into output cluster ID = 1.
Equivalent result can be obtained also by using
\begin{lstlisting}
        void PassResult(std::vector<std::vector<unsigned short> >&)
\end{lstlisting}
if that suits one's usage. 

If you are interested in checking which input cluster(s) included in a particular output (merged) cluster, you can use:
\begin{lstlisting}
        std::vector<unsigned short> GetMergedSet(unsigned short)
\end{lstlisting}
which takes an output cluster index as an argument and returns a vector that contains a list of input clusters that consist the output.

\begin{figure}[ht]\begin{center}
\includegraphics[width=8.3cm]{./src/Pictures/CBookKeeperProhibitExample2.pdf}
\includegraphics[width=8.3cm]{./src/Pictures/CBookKeeperProhibitExample1.pdf}
\caption{Example diagrams showing how {\cbkeeper} reacts to conflict {\ttfamily Merge()} and {\ttfamily ProhibitMerge()} function calls on the input cluster IDs. The left diagram shows {\crue} thrown due to calling {\ttfamily Merge()} on a cluster pair that is registered as merge-prohibit pair. The right diagram shows a result from calling {\ttfamily ProhibitMerge()} on a cluster pair that is already merged.}
\label{sec:fmwk:base:cbk_prohibit}
\end{center}\end{figure}

Finally, {\cmerge} can also prohibit merging of specific two cluster pair(s), and this is incoorporated by {\cbkeeper} as well.
This is, again, to support user to log merging in a mindless manner.
A user may specify specific 2 cluster pair should not be merged. 
Then (s)he can use the following function to register these 2 clusters for merge-prohibit list.
\begin{lstlisting}
        void ProhibitMerge(unsigned short, unsigned short)
\end{lstlisting}
The above function takes input cluster IDs (i.e. index numbers). 
Once 2 clusters are registered as merge-prohibit pair, attempting a function call {\ttfamily Merge()} will result in {\crue} thrown.
This is illustrated in the left of Fig.\ref{sec:fmwk:base:cbk_prohibit}.
Another possible case that results in {\crue} thrown would be calling {\ttfamily ProhibitMerge()} function on two input cluster IDs that are already merged. This is shown in the right hand side of Fig.\ref{sec:fmwk:base:cbk_prohibit}.

To avoid {\crue}, one should register all cluster pairs to prohibit-merge list before trying any merge attempt.
Upon attempting to call {\ttfamily Merge()}, then, one can use the function:
\begin{lstlisting}
        bool MergeAllowed(unsigned short, unsigned short)
\end{lstlisting}
to check if a cluster pair is registered to merge-prohibit list.
Calling this function prior to evaluation of merging (i.e. running some computation algorithm to decide whether to merge or not) may save some computation time and help the overall speed of the program (this is implemented in {\cmerge}).

\subsection{Algorithm Base: {\ttfamily CBoolAlgoBase}}
\label{sec:fmwk:base:cbalgo}
{\cbalgo} is a simple {\CPP} template class and also a base class for an algorithm used for cluster merging and matching by {\cmerge} and {\cmatch} application classes. In this and later sections, we use the term {\cmalgo} to refer to an algorithm class derived from {\cbalgo}.

\begin{figure}[ht]\begin{center}
\includegraphics[width=13cm]{./src/Pictures/CMergeStream.pdf}
\caption{The data processing stream of {\cmerge} framework. The purple allow shows the ordered process flow. It requires an analysis framework interface to convert framework dependent cluster data produdct into a vector of {\pxhit} and hand over to {\cmerge} application. Internally {\cmerge} calls various members of {\cbalgo} instances: one for merging and another for prohibiting. There is a configuration in {\cmerge} to perform iterative approach, which repeats internal loop over possible cluster pairs until no more newly merged cluster is found in the output. The merged cluster groups (i.e. output) can be obtained through {\cbkeeper} instance by an analysis framework interface to form an appropriate output data product.}
\label{sec:fmwk:base:cmerge_stream}
\end{center}\end{figure}

The idea is very simple: given two clusters, it is expected to return a boolean for decision making from the following function.
\begin{lstlisting}
        virtual bool Bool(const cluster::ClusterParamsAlg &, const cluster::ClusterParamsAlg &)
\end{lstlisting}
As you can see, ``clusters'' are handed over in terms of {\cpan} instance.
In {\cmerge}, the cluster merging application class, for instance, this is used for making merge decision as well as prohibit-merge decision (i.e. {\cmerge} holds two {\cbalgo} inherit class instance, one for merging and another for prohibiting).
An algorithm developpers should write his/her own {\CPP} algorithm class by inheriting from {\cbalgo}.
In the derived class, the {\ttfamily Bool()} function implementation should be over-riden. 
There is a script to generate a template (i.e. empty) {\cbalgo} inherit class and everyone is encouraged to use it for starting a new algorithm class. These are:
\begin{itemize}
\item For {\larsoft}
\begin{itemize}
        \item[]{\ttfamily> python \$MRB\_TOP/srcs/larreco/RecoAlg/CMTool/mac/gen\_cmalgo.py NAME}
\end{itemize}
\item For {\larlight}
\begin{itemize}
        \item[]{\ttfamily> python \$MAKE\_TOP\_DIR/UserDev/ClusterStudy/bin/gen\_cmalgo NAME}
\end{itemize}
\end{itemize}
where {\ttfamily NAME} should be replaced by the new algorithm's name. 
Running above command results in creating {\ttfamily CMAlgoNAME.h} and {\ttfamily CMAlgoNAME.cxx} in the appropriate directory.
Then you can proceed to build following the framework's guidance.

As mentioned, {\cbalgo} is operated by {\cmerge} and {\cmatch} applications.
For a given event, there is an internal loop of calling {\ttfamily Bool()} on all possible cluster pairs.
To provide flexibility in algorithm design, those applications also call other functions of {\cbalgo} in its processing.

The following is a list of such {\cbalgo} functions.
\begin{itemize}

\item[]{\ttfamily virtual void EventBegin(const std::vector<cluster::ClusterParamsAlg>\&)}
  \begin{itemize}
    \item Called at the beginning of an event, prior to looping over cluster pairs. The provided function argument is a vector of {\cpan} for all clusters subject for merging. This is where computation that requires the whole, initial cluster sets should be done.
  \end{itemize}

\item[]{\ttfamily virtual void IterationBegin(const std::vector<cluster::ClusterParamsAlg>\&)}
  \begin{itemize}
    \item Called at the beginning of an each loop over cluster pairs. Note {\cmerge} and {\cmatch} can have multiple iteration of looping over possible cluster pairs if configured to do so. For instance {\cmerge} can take iterate over multiple loops till no more to-be-merged  cluster pair is found (see Sec.\ref{sec:fmwk:cmerge} and Sec.\ref{sec:fmwk:cmatch} for details). The argument of this function is an output cluster sets, in terms of {\cpan} instance, from the previous loop. In case of the 1st loop, the provided cluster sets is identical to those provided in the argument of {\ttfamily EventBegin()} function.
  \end{itemize}

\item[]{\ttfamily virtual void IterationEnd()}
  \begin{itemize}
    \item Called at the end of each merge or match attempt loop over all cluster pairs, to clean up or summarize information.
  \end{itemize}

\item[]{\ttfamily virtual void EventEnd()}
  \begin{itemize}
    \item Called at the end of each event, after all iterations over cluster pairs are done.
  \end{itemize}

\end{itemize}

Further, there are two functions that are not yet implemented in {\cmerge} nor {\cmatch} but ideas exist.
Your input on how these should be implemented is welcome and very much appreciated.
\begin{itemize}

\item[]{\ttfamily virtual void Reset()}
  \begin{itemize}
    \item Meant to clear some member variables. But one can do that within {\ttfamily IterationEnd()} and/or {\ttfamily EventEnd()}, so it is not clear where this should be called. Possibly at the end of the whole event processing.
  \end{itemize}

\item[]{\ttfamily virtual void Report()}
  \begin{itemize}
    \item Meant to support some standardized {\ttfamily std::cout} and {\ttfamily std::cerr} report from an algorithm instance. {\cmerge} and {\cmatch} have ability to control such text output for different levels, and there is a hope this {\ttfamily Report()} can be also included there.
  \end{itemize}

\end{itemize}

\paragraph{Tips for writing {\cmalgo}}
First of all, you are encouraged to use {\ttfamily gen\_cmalgo} script as described earlier.
Another thing to note is to reduce a dependency on analysis framework.
This makes it easier to port your code between different frameworks.



\section{{\cmerge}: Cluster Merging Tool}
\label{sec:fmwk:cmerge}

{\cmerge} is a {\CPP} class that performs merging of 2D clusters, and is an application of {\cmalgo} algorithm classes that are inherit from {\cbalgo}, the algorithm base class. In a few sentences, {\cmerge} perform merging by looping over all possible 2D cluster pairs among the given set of clusters. For each pair, {\ttfamily Bool()} function of both merging and prohibit {\cmalgo} are called to inspect possible 2D cluster merging. It uses {\cbkeeper} to keep track of which pair of clusters are merged. For details of {\cbalgo} and {\cbkeeper}, see Sec.\ref{sec:fmwk:base:cbalgo} and Sec.\ref{sec:fmwk:base:cbk} respectively. The overall process stream diagram is shown in Fig.\ref{sec:fmwk:base:cmerge_stream} which we keep referring in this section.

\subsection{Why {\cmerge}?}
Why using {\cmerge}? Why not just writing a program that does everything from scratch?
Be my guest and write your code! That might inspire a better design or strategy toward merging. 
That being said, there are a few reasons to use {\cmerge}.

\paragraph{Task Outside Merging Decision:}
{\cmerge} The core part of merging is to compare 2 clusters and make a decision to merge or not. Outside this process, however, there is a non-negligible overburden of running such algorithm such as (1) keeping track of which cluster pairs are merged and (2) optimizing the loop process of cluster pair comparison. {\cmerge} does those tedious jobs while out-sourcing the decision making to an independent {\CPP} algorithm class that inherit from {\cbalgo}. 

\paragraph{Easy Development via Modulated Design:}
Because {\cmerge} outsource an independent algorithm class to perform merging of a cluster pair, it supports modulated code design. Accordingly each {\cmalgo} class becomes a pure algorithm class. This makes a code development very easy because a new algorithm can be written without re-writing the I/O or book-keeping part. Also it makes life easy to compare different algorithms: you can simple change which algorithm to be attached to {\cmerge} at run-time, and this avoids re-compilation or re-writing of code.

\paragraph{X-Framework Implementation}
{\cmerge} does not use data product type specific to an analysis framework (for instance {\larlight} or {\larsoft}), so it can be ported easily among different frameworks. That means {\cmalgo} developed by one person in one analysis framework can be shared among other frameworks (assuming an algorithm developer follows x-framework design pattern).\\

\subsection{How Does It Work?}
\label{sec:fmwk:cmerge:how}
The core part of decision making is simple: provided 2 clusters, decide whether they should be merged or not.
This is done through {\cmalgo::Bool()} function call:
\begin{lstlisting}
  virtual bool Bool( const cluster::ClusterParamsAlg &, const cluster::ClusterParamsAlg &)
\end{lstlisting}
If the return is {\ttfamily true}, two clusters are merged. Else not.
This is ilustrated in Fig.\ref{sec:fmwk:cmerge:merge_example}.

\begin{figure}[ht]\begin{center}
\includegraphics[width=13cm]{./src/Pictures/MergeExample.pdf}
\caption{Example of 2D clusters in a plane before and after merging attempt. Each oval represents a cluster with cluster 2D axis shown in a black solid arrow. The direction of an arrow represents the direction of a cluster development. Each cluster is labeled with an input cluster ID number while the oval color represents distinct output cluster group. Left: initial state with 4 input clusters. Right: example output after running a merge loop over cluster pairs. }
\label{sec:fmwk:cmerge:merge_example}
\end{center}\end{figure}
{\cmerge} takes care of creating a loop over all possible pairs of 2D clusters and calling the above function of {\cmalgo} for merging.
In the process, book-keeping of which pairs to be merged is recorded using {\cbkeeper}.

\begin{figure}[ht]\begin{center}
\includegraphics[width=13cm]{./src/Pictures/MergeProhibitExample.pdf}
\caption{Example of how prohibit algorithm works. Each oval and solid arrow represents a 2D cluster with its direction while the color represents distinct output (merged) cluster. Input cluster index (ID) is shown in the label with the cluster 2D angle. Right-top shows an output of running {\cmalgo} that merge clusters with angle difference within 40 degrees. Right-bottom shows an output of the same {\cmalgo} for merging but with another {\cmalgo} that prohibits merging of clusters with angle difference larger than 40 degrees.}
\label{sec:fmwk:cmerge:merge_prohibit_example}
\end{center}\end{figure}
Let us consider a case of two merged pairs that shares the same input cluster ID. As described in Sec.\ref{sec:fmwk:base:cbk}, {\cbkeeper} combines two pairs of 2D clusters in such case (see Fig.\ref{sec:fmwk:base:cbk}). This could be problematic sometimes. For instance, consider an example {\cmalgo} that combines two clusters if their start points are very close and the angle between two clusters are within 40 degrees. Further assume we have 3 input clusters that share the start point but have different angles at 20, 40, and -20 degrees. For convention these have input cluster ID 0, 1, and 2 respectively. This example algorithm would conclude pairs (0,1) and (0,2) but not for (1,2). But because merged pairs share cluster ID 0, it ends up all 3 clusters being merged. This is shown in the right-top ilustration of Fig.\ref{sec:fmwk:cmerge:merge_prohibit_example}.

This could be a problem, or may not be: this can be only answered by the algorithm author.
The way to avoid this in {\cmerge} is to implement another {\cmalgo} as {\it merge-prohibit}, or simple {\it prohibit} algorithm.
When prohibit algorithm is provided, {\cmerge} first run over all cluster pairs and execute {\ttfamily Bool()} function to find all cluster pairs that are prohibited to merge.
Then {\cmerge} run over the same set of cluster pairs except for those are prohibited to be merged.
The right-bottom ilustration of Fig.\ref{sec:fmwk:cmerge:merge_prohibit_example} shows how prohibit algorithm affects the result.

The prohibit rule is also inherited to merged cluster pairs. For instance, consider input clusters 0, 1 and 2. The pair (1,2) is prohibited to merge, and (0,1) is found to be merged. Then the merged cluster, (0,1) cannot be merged with 2. That means even if a merge algorithm finds (0,2) to be merged, it cannot be because the prohibit rule is inherited from cluster 1 when 0 and 1 aremerged. This feature brings up a question: what determines 0 to be merged with 1 rather than 2? If the merge algorithm is executed on the pair (0,2) first, then the result would have been different. So what determines a different outcome is an order of cluster pairs in the loop of merging inspection. There are different ordering options available in {\cmerge}. Currently available options are defined as {\enum} type in {\cmerge} class, {\ttfamily CMergePriority\_t} which include following values.
\begin{itemize}
\item[] {\ttfamily kIndex}
  \begin{itemize}
    \item From the smaller to larger input cluster ID, basically following input cluster set ordering
  \end{itemize}

\item[] {\ttfamily kPolyArea}
  \begin{itemize}
    \item From the clusters with smaller polygon area to the larger
  \end{itemize}

\item[] {\ttfamily kChargeSum}
  \begin{itemize}
    \item From the clusters with smaller charge sum to the larrger
  \end{itemize}

\item[] {\ttfamily kNHits}
  \begin{itemize}
    \item From the clusters with smaller number of hits to the larger
  \end{itemize}

\end{itemize}
The ordering parameters including all listed above are taken from {\ttfamily cluster\_params} member attributes.
If more useful parameter is found, one should make a request to make it available. 
It is very simple to implement in {\cmerge}.

In summary, {\cmerge} uses one {\cmalgo} instance for merging and another for merge-prohibit. A possible ambiguity of which cluster pairs to be merged in the presence of prohibited pairs is resolved by defining a priority for cluster pairs in the merging attempt. 

\subsection{Process Flow for Cluster Merging}
\label{sec:fmwk:cmerge:process_flow}
An overall process flow diagram is already shown in Fig.\ref{sec:fmwk:base:cmerge_stream}, so here we describe slightly more details in texts by breaking down into 5 blocks.
\begin{enumerate}
\item Event Preparation
  \begin{itemize}
    \item[] {\cmerge} receives an input cluster in terms of a set of {\pxhit} (see Sec.\ref{sec:fmwk:cmerge:input}). Then it creates {\cpan} instance for each cluster, and calls {\cmalgo::EventBegin()} for merging and prohibit algorithms.
  \end{itemize}

\item Iterative Merge Operation
  \begin{itemize}
    \item[] A loop over cluster pairs for prohibit (if provided) and merge {\cmalgo} can be configured to repeat until no newly merged cluster is found. This is shown in Fig.\ref{sec:fmwk:cmerge:iterative_stream}. If a user does not configure to choose iterative approach, iteration will not happen even if there is a newly merged cluster in the output.
  \end{itemize}

\item Prohibit {\cmalgo} Loop Over Cluster Pairs
  \begin{itemize}
    \item[] See Fig.\ref{sec:fmwk:cmerge:iterative_stream}. Only occurs if prohibit algorithm is provided by a user (optional). Records which cluster pair to be prohibited for merging using {\cbkeeper}.
  \end{itemize}

\item Prohibit {\cmalgo} Loop Over Cluster Pairs
  \begin{itemize}
    \item[] See Fig.\ref{sec:fmwk:cmerge:iterative_stream}. The algorithm's {\ttfamily Bool()} is not executed for a subject pair of clusters if (a) they are prohibited for merging or (b) already inspected in the previous process and not merged (i.e. result should not change). This is to save computation time.
  \end{itemize}

\item Merge Clusters
  \begin{itemize}
    \item[] Based on the outcome of merge/prohibit {\cmalgo} algorithms, actual merging is performed at this step. It creates a new set of {\cpan} for merged clusters. For clusters that are not merged, {\cpan} instance is copied from input to save {\cpan} computation time.
  \end{itemize}

\end{enumerate}

\begin{figure}[ht]\begin{center}
\includegraphics[width=13cm]{./src/Pictures/IterativeMergeApproach.pdf}
\caption{Illustration of how iterative merge approach is processed. Internally a loop over cluster pairs for prohibit and merge algorithm are repeated if (a) user configured to use an iterative approach and (b) newly merged cluster is found.}
\label{sec:fmwk:cmerge:iterative_stream}
\end{center}\end{figure}


\subsection{Setting an Input 2D Clusters}
\label{sec:fmwk:cmerge:input}
Input to {\cmerge} is a set of clusters where each cluster is represented as a vector of hits, or more precisely {\ttfamily std::vector<util::PxHit>}. This input can be set by the following function:
\begin{lstlisting}
  void SetClusters(const std::vector<std::vector<util::PxHit> > &)
\end{lstlisting}
where the argument is a vector of 2D clusters. When this function is called, {\cmerge} internally calls
\begin{lstlisting}
  void Reset()
\end{lstlisting}
which initialize previously stored input clusters, output clusters and {\cbkeeper} instance.
Then it creates {\cpan} instance per input 2D cluster using a set of {\pxhit}.
This completes setting (or initiating) the input 2D cluster sets to be used for internal processing.

\subsection{Setting Merge/Prohibit {\cmalgo} Instances}
{\cmalgo} instance is necessary for merging and optional for prohibit. Call the function,
\begin{lstlisting}
  void AddMergeAlg(CBoolAlgoBase* algo)
\end{lstlisting}
to set {\cmalgo} for merging, and 
\begin{lstlisting}
  void AddSeparateAlgo(CBoolAlgoBase* algo)
\end{lstlisting}
to set {\cmalgo} for merge-prohibit.

\subsection{Setting Run Configuration}
There are only 3 run options to configure for {\cmerge}.

\paragraph{Verbosity:} 
There are 5 verbosity levels a user can configure as listed below:
\begin{itemize}
\item[] {\ttfamily kNone}
  \begin{itemize}
    \item No report from {\cmerge}
  \end{itemize}

\item[] {\ttfamily kPerEvent}
  \begin{itemize}
    \item Report made at the end of even processing
  \end{itemize}

\item[] {\ttfamily kPerIteration}
  \begin{itemize}
    \item Report made per merge iteration (see Sec.\ref{sec:fmwk:cmerge:process_flow}). This could be quite verbose.      
  \end{itemize}

\item[] {\ttfamily kPerMerging}
  \begin{itemize}
    \item Report made per {\cmalgo::Bool()} execution (see Sec.\ref{sec:fmwk:cmerge:process_flow}). This is even more verbose than {\ttfamily kPerIteration}.
  \end{itemize}

\end{itemize}
These are {\enum} type, {\ttfamily CMergeMSGLevel\_t}, defined under {\cmerge} class. 
The verbosity level can be set by the function:
\begin{lstlisting}
  void DebugMode(CMergeMSGLevel_t level)
\end{lstlisting}

\paragraph{Iterative Approach:}
As described in Sec.\ref{sec:fmwk:cmerge:process_flow}, {\cmerge} can be configured to attempt iterative approach for merging. This can be enabled by the following function call.
\begin{lstlisting}
  void MergeTillConverge(bool)
\end{lstlisting}
Providing {\ttfamily true} as an argument will enable this option. Disabled if {\ttfamily false} is provided.

\paragraph{Merge Priority:}
As described in Sec.\ref{sec:fmwk:cmerge:how}, ordering of cluster pairs for merging inspection matters to the result.
Options are also described in the same section. This can be set by the following function.
\begin{lstlisting}
  void SetMergePriority(CMergePriority_t level)
\end{lstlisting}

\subsection{Execute Merging}
To execute merging, simply call
\begin{lstlisting}
  void Process()
\end{lstlisting}

\subsection{Obtaining Result}
There are two kinds of results that may be useful for a user. First is a resulting (i.e. merged) set of clusters in terms of {\cpan}. This is useful for further processing as various cluster parameters are already computed as a part of {\cpan}. You can execute
\begin{lstlisting}
  const std::vector<cluster::ClusterParamsAlg>& GetClusters() const
\end{lstlisting}
to obtain such result set.

Second is a mapping of input cluster IDs (i.e. index) to the output (merged clusters). 
As described in Sec.\ref{sec:fmwk:base:cbk}, this can be easily obtained through {\cbkeeper} instance.
You can obtain the {\cbkeeper} instance held by {\cmerge} through the following function.
\begin{lstlisting}
  const CBookKeeper& GetBookKeeper() const
\end{lstlisting}


\section{{\cmatch}: Cluster Matching Tool}
\label{sec:fmwk:cmatch}
Tool not introduced yet in {\larsoft}. Will be documented when this is done.




% How-to's compilation
\chapter{Generate \& Build Your Package}
\label{chap:generate}

Despite its extreme simplicity let us call {\cmtool} a framework, or FMWK. 
In this chapter we discuss the building blocks of {\cmtool}.

\section{External Dependency}
\label{sec:fmwk:external}

The {\cmtool} development effort is originated from EM shower reconstruction for MicroBooNE experiment, and it is an extension of existing tool set. The following is the brief list of packages or {\CPP} classes on which {\cmtool} depends.
\begin{itemize}
\item[] {\cru}
    \begin{itemize}
    \item larreco/RecoAlg/ClusterRecoUtil in {\larsoft}
    \item UserDev/ClusterRecoUtil in {\larlight}
    \end{itemize}
\item[] {\geoutil}
    \begin{itemize}
    \item lardata/Utilities in {\larsoft}
    \item core/LArUtil in {\larlight}
    \end{itemize}
\item[] {\ttfamily PxUtil}
    \begin{itemize}
    \item lardata/Utilities in {\larsoft}
    \item core/LArUtil in {\larlight}
    \end{itemize}
\end{itemize}
These items are (or should be if not yet) described in a separate document.
The rest of this section briefly covers what they are without going into a technical detail.

\subsection{{\ttfamily ClusterRecoUtil}}
\label{sec:fmwk:external:cru}
Reconstruction parameters for EM shower 2D cluster involves a lot more variety of parameters than those held in a 2D cluster data product defined by a framework ({\larsoft} or {\larlight}). {\cru} is a package that defines code for both calculation of and storage for those parameters. In particular, {\cmtool} uses following {\CPP} classes from {\cru}:
\begin{itemize}
\item {\ttfamily ClusterParamsAlg ... (\cpan)}
  \begin{itemize}
    \item {\cpan} is designed as both parameter computation algorithm and storage class, and is used along this design principle in {\cmtool}. In particular, {\cpan} is created one per input 2D cluster.
  \end{itemize}
\item {\ttfamily cluster\_params}
  \begin{itemize}
    \item {\ttfamily cluster\_params} is a utility storage class for important parameters computed by \cpan. A const reference can be retrieved from {\cpan} instance through {\ttfamily GetParams()} function call. Note, however, one should keep {\cpan} instance rather than {\ttfamily cluster\_params} as a cluster parameter data set (see {\cpan} description above).
  \end{itemize}
\item {\ttfamily CRUException}
  \begin{itemize}
    \item This is a very simple exception class derived from {\ttfamily std::exception}. It is used instead of {\ttfamily cet::exception} to avoid framework dependency. Only additional feature to the base class is that the argument string to the constructor is sent to standard output streem by overriden {\ttfamily std::exception::what()} function.
  \end{itemize}
\end{itemize}

For details of all classes listed above, you should see Ref.\cite{ClusterRecoUtil}.

\subsection{{\ttfamily PxUtil}}
\label{sec:fmwk:external:pxutil}
Both {\cru} and {\cmtool} are designed to be independent of an analysis framework of user's choice (in particular {\larlight} and {\larsoft}). 
As mentioned in Sec.\ref{sec:fmwk:external:cru}, output of {\cpan} (i.e. cluster parameters) is stored with {\ttfamily cluster\_params} instance. 
{\pxutil}, on the other hand, is used to define input data struct to {\cpan} (and hence {\cmtool} and {\cru} in general). {\pxutil} defines three data objects:
\begin{itemize}
  \item {\ttfamily PxPoint}
    \begin{itemize}
    \item represents 2D point and contains wire [cm], time [cm], and plane ID (unsigned char)
    \end{itemize}
  \item {\ttfamily PxHit}
    \begin{itemize}
    \item inhertis from {\ttfamily PxPoint} with an additional attribute to hold charge [ADC], representing 2D hit
    \end{itemize}
  \item {\ttfamily PxLine}
    \begin{itemize}
    \item represents 2D line and contains start and end coordinates, (wire [cm], time [cm]), with a plane ID (unsigned char)
    \end{itemize}
\end{itemize}

It is a FMWK interface module's respnsibility to translate input data product into these format. In particular, this module must convert a vector of FMWK hit data product (= cluster) into a vector of {\pxhit}. This is discussed in FMWK interface chapter (see Ch.\ref{chap:fmwk_interface}).

\subsection{{\ttfamily GeometryUtilities}}
\label{sec:fmwk:external:geoutil}
{\CPP} class {\geoutil} is a utility class that helps to retrieve detector geometry related information such as converting 3D point coordinate into a projected 2D coordinate on a given plane. 
{\cmtool} uses {\geoutil} to interpret geometry related variables in full extent. 
For development of any non-existing function that treats detector geometry information, it is encouraged to develop in {\geoutil} (and certainly not in {\cmtool}). 
Please see Ref.\cite{GeometryUtilities} for more detailed documentation.


\section{FMWK Bulding Blocks}
\label{sec:fmwk:base}
In this section we discuss two core buliding block classes: (1) {\cbalgo} and (2) {\cbkeeper}. Though we discuss these basic {\CPP} classes first, it might be easier to understand these classes after you read about {\cmerge}, described in Sec.{\ref{sec:fmwk:cmerge}}, because there we discuss how {\cmerge} works with practical usage of {\cbalgo} and {\cbkeeper}.

\subsection{Book Keeping: {\cbkeeper}}
\label{sec:fmwk:base:cbk}
{\cbkeeper} inherits from {\ttfamily std::vector<unsigned short>}, and hence it is a vector of integers by itself.
The length of vector can be specified using a constructor argument.

This class is used to track how input clusters are merged/matched into an output. 
Here we focus on merging of input clusters for the simplicity sake.
The design assumes one to perform a merging attempt of 2 clusters at a time.
When one has many clusters to make such attempt on all possible pairs of clusters, it could be troublesome to perform book-keeping for which clusters are merged together.
{\cbkeeper} is designed such that a user can just log two to-be-merged clusters' ID at each attempt, and it holds a summary log of all merged cluster pairs.


Here are technical details.
The index of vector corresponds to the initial cluster ID numbers.
If you have 8 clusters to be subject for merging, {\cbkeeper} instance should be initialized as length 8.
Initially the values stored in the vector is same as their index numbers as shown in the top of Fig.\ref{sec:fmwk:base:cbk}.
One can re-set the state of {\cbkeeper} instance by calling the following function.
\begin{lstlisting}
        void Reset(unsigned short n=0)
\end{lstlisting}

\begin{figure}\begin{center}
\includegraphics[width=14cm]{./src/Pictures/CBookKeeperExample.pdf}
\caption{Example diagram showing how {\cbkeeper} keeps track of input and output cluster IDs. It shows {\cbkeeper} vector index (input cluster ID) and stored values (output cluster ID) for 8 input clusters (top), and how the stored values change as merging takes place. The middle shows the result of merging input cluster 1 and 2, and the bottom is for further merging 2 and 5.}
\end{center}\end{figure}
When one finds 2 clusters to be merged together, (s)he can call {\cbkeeper} attribute function:
\begin{lstlisting}
        void CBookKeeper::Merge(unsigned short, unsigned short)
\end{lstlisting}
with corresponding clusters' IDs (i.e. index numbers). 
Figure \ref{sec:fmwk:base:cbk} shows how the process works for calling {\ttfamily CBookKeeper::Merge()} function twice with arguments (1,2) and (2,5).
As a result, unique output cluster IDs are reduced to 6 because 3 clusters are combined together.

You can see {\cbkeeper} instance allows to retrieve which output cluster (stored value) an input cluster (vector index) ended up with in an intuitive way: simply use {\ttfamily std::vector::at()} method.
However, to create an output cluster, it is probably easier to ask the other way around: given output cluster ID, which input cluster(s) does it contain?
{\cbkeeper} has a simple function to answer this question:
\begin{lstlisting}
        std::vector<std::vector<unsigned short> > GetResult() const 
\end{lstlisting}
which returns a vector with length same as number of output (i.e. merged) clusters.
The value of this vector is {\ttfamily std::vector<unsigned short>} to contain possibly multiple input cluster indexes.
For instance, the return from {\cbkeeper} shown in the bottom of Fig.\ref{sec:fmwk:base:cbk} would be a vector of length 6, and the index1 contains a vector of input cluster ID 1, 2, and 5 as they have been merged into output cluster ID = 1.
Equivalent result can be obtained also by using
\begin{lstlisting}
        void PassResult(std::vector<std::vector<unsigned short> >&)
\end{lstlisting}
if that suits one's usage. 

If you are interested in checking which input cluster(s) included in a particular output (merged) cluster, you can use:
\begin{lstlisting}
        std::vector<unsigned short> GetMergedSet(unsigned short)
\end{lstlisting}
which takes an output cluster index as an argument and returns a vector that contains a list of input clusters that consist the output.

\begin{figure}[ht]\begin{center}
\includegraphics[width=8.3cm]{./src/Pictures/CBookKeeperProhibitExample2.pdf}
\includegraphics[width=8.3cm]{./src/Pictures/CBookKeeperProhibitExample1.pdf}
\caption{Example diagrams showing how {\cbkeeper} reacts to conflict {\ttfamily Merge()} and {\ttfamily ProhibitMerge()} function calls on the input cluster IDs. The left diagram shows {\crue} thrown due to calling {\ttfamily Merge()} on a cluster pair that is registered as merge-prohibit pair. The right diagram shows a result from calling {\ttfamily ProhibitMerge()} on a cluster pair that is already merged.}
\label{sec:fmwk:base:cbk_prohibit}
\end{center}\end{figure}

Finally, {\cmerge} can also prohibit merging of specific two cluster pair(s), and this is incoorporated by {\cbkeeper} as well.
This is, again, to support user to log merging in a mindless manner.
A user may specify specific 2 cluster pair should not be merged. 
Then (s)he can use the following function to register these 2 clusters for merge-prohibit list.
\begin{lstlisting}
        void ProhibitMerge(unsigned short, unsigned short)
\end{lstlisting}
The above function takes input cluster IDs (i.e. index numbers). 
Once 2 clusters are registered as merge-prohibit pair, attempting a function call {\ttfamily Merge()} will result in {\crue} thrown.
This is illustrated in the left of Fig.\ref{sec:fmwk:base:cbk_prohibit}.
Another possible case that results in {\crue} thrown would be calling {\ttfamily ProhibitMerge()} function on two input cluster IDs that are already merged. This is shown in the right hand side of Fig.\ref{sec:fmwk:base:cbk_prohibit}.

To avoid {\crue}, one should register all cluster pairs to prohibit-merge list before trying any merge attempt.
Upon attempting to call {\ttfamily Merge()}, then, one can use the function:
\begin{lstlisting}
        bool MergeAllowed(unsigned short, unsigned short)
\end{lstlisting}
to check if a cluster pair is registered to merge-prohibit list.
Calling this function prior to evaluation of merging (i.e. running some computation algorithm to decide whether to merge or not) may save some computation time and help the overall speed of the program (this is implemented in {\cmerge}).

\subsection{Algorithm Base: {\ttfamily CBoolAlgoBase}}
\label{sec:fmwk:base:cbalgo}
{\cbalgo} is a simple {\CPP} template class and also a base class for an algorithm used for cluster merging and matching by {\cmerge} and {\cmatch} application classes. In this and later sections, we use the term {\cmalgo} to refer to an algorithm class derived from {\cbalgo}.

\begin{figure}[ht]\begin{center}
\includegraphics[width=13cm]{./src/Pictures/CMergeStream.pdf}
\caption{The data processing stream of {\cmerge} framework. The purple allow shows the ordered process flow. It requires an analysis framework interface to convert framework dependent cluster data produdct into a vector of {\pxhit} and hand over to {\cmerge} application. Internally {\cmerge} calls various members of {\cbalgo} instances: one for merging and another for prohibiting. There is a configuration in {\cmerge} to perform iterative approach, which repeats internal loop over possible cluster pairs until no more newly merged cluster is found in the output. The merged cluster groups (i.e. output) can be obtained through {\cbkeeper} instance by an analysis framework interface to form an appropriate output data product.}
\label{sec:fmwk:base:cmerge_stream}
\end{center}\end{figure}

The idea is very simple: given two clusters, it is expected to return a boolean for decision making from the following function.
\begin{lstlisting}
        virtual bool Bool(const cluster::ClusterParamsAlg &, const cluster::ClusterParamsAlg &)
\end{lstlisting}
As you can see, ``clusters'' are handed over in terms of {\cpan} instance.
In {\cmerge}, the cluster merging application class, for instance, this is used for making merge decision as well as prohibit-merge decision (i.e. {\cmerge} holds two {\cbalgo} inherit class instance, one for merging and another for prohibiting).
An algorithm developpers should write his/her own {\CPP} algorithm class by inheriting from {\cbalgo}.
In the derived class, the {\ttfamily Bool()} function implementation should be over-riden. 
There is a script to generate a template (i.e. empty) {\cbalgo} inherit class and everyone is encouraged to use it for starting a new algorithm class. These are:
\begin{itemize}
\item For {\larsoft}
\begin{itemize}
        \item[]{\ttfamily> python \$MRB\_TOP/srcs/larreco/RecoAlg/CMTool/mac/gen\_cmalgo.py NAME}
\end{itemize}
\item For {\larlight}
\begin{itemize}
        \item[]{\ttfamily> python \$MAKE\_TOP\_DIR/UserDev/ClusterStudy/bin/gen\_cmalgo NAME}
\end{itemize}
\end{itemize}
where {\ttfamily NAME} should be replaced by the new algorithm's name. 
Running above command results in creating {\ttfamily CMAlgoNAME.h} and {\ttfamily CMAlgoNAME.cxx} in the appropriate directory.
Then you can proceed to build following the framework's guidance.

As mentioned, {\cbalgo} is operated by {\cmerge} and {\cmatch} applications.
For a given event, there is an internal loop of calling {\ttfamily Bool()} on all possible cluster pairs.
To provide flexibility in algorithm design, those applications also call other functions of {\cbalgo} in its processing.

The following is a list of such {\cbalgo} functions.
\begin{itemize}

\item[]{\ttfamily virtual void EventBegin(const std::vector<cluster::ClusterParamsAlg>\&)}
  \begin{itemize}
    \item Called at the beginning of an event, prior to looping over cluster pairs. The provided function argument is a vector of {\cpan} for all clusters subject for merging. This is where computation that requires the whole, initial cluster sets should be done.
  \end{itemize}

\item[]{\ttfamily virtual void IterationBegin(const std::vector<cluster::ClusterParamsAlg>\&)}
  \begin{itemize}
    \item Called at the beginning of an each loop over cluster pairs. Note {\cmerge} and {\cmatch} can have multiple iteration of looping over possible cluster pairs if configured to do so. For instance {\cmerge} can take iterate over multiple loops till no more to-be-merged  cluster pair is found (see Sec.\ref{sec:fmwk:cmerge} and Sec.\ref{sec:fmwk:cmatch} for details). The argument of this function is an output cluster sets, in terms of {\cpan} instance, from the previous loop. In case of the 1st loop, the provided cluster sets is identical to those provided in the argument of {\ttfamily EventBegin()} function.
  \end{itemize}

\item[]{\ttfamily virtual void IterationEnd()}
  \begin{itemize}
    \item Called at the end of each merge or match attempt loop over all cluster pairs, to clean up or summarize information.
  \end{itemize}

\item[]{\ttfamily virtual void EventEnd()}
  \begin{itemize}
    \item Called at the end of each event, after all iterations over cluster pairs are done.
  \end{itemize}

\end{itemize}

Further, there are two functions that are not yet implemented in {\cmerge} nor {\cmatch} but ideas exist.
Your input on how these should be implemented is welcome and very much appreciated.
\begin{itemize}

\item[]{\ttfamily virtual void Reset()}
  \begin{itemize}
    \item Meant to clear some member variables. But one can do that within {\ttfamily IterationEnd()} and/or {\ttfamily EventEnd()}, so it is not clear where this should be called. Possibly at the end of the whole event processing.
  \end{itemize}

\item[]{\ttfamily virtual void Report()}
  \begin{itemize}
    \item Meant to support some standardized {\ttfamily std::cout} and {\ttfamily std::cerr} report from an algorithm instance. {\cmerge} and {\cmatch} have ability to control such text output for different levels, and there is a hope this {\ttfamily Report()} can be also included there.
  \end{itemize}

\end{itemize}

\paragraph{Tips for writing {\cmalgo}}
First of all, you are encouraged to use {\ttfamily gen\_cmalgo} script as described earlier.
Another thing to note is to reduce a dependency on analysis framework.
This makes it easier to port your code between different frameworks.



\section{{\cmerge}: Cluster Merging Tool}
\label{sec:fmwk:cmerge}

{\cmerge} is a {\CPP} class that performs merging of 2D clusters, and is an application of {\cmalgo} algorithm classes that are inherit from {\cbalgo}, the algorithm base class. In a few sentences, {\cmerge} perform merging by looping over all possible 2D cluster pairs among the given set of clusters. For each pair, {\ttfamily Bool()} function of both merging and prohibit {\cmalgo} are called to inspect possible 2D cluster merging. It uses {\cbkeeper} to keep track of which pair of clusters are merged. For details of {\cbalgo} and {\cbkeeper}, see Sec.\ref{sec:fmwk:base:cbalgo} and Sec.\ref{sec:fmwk:base:cbk} respectively. The overall process stream diagram is shown in Fig.\ref{sec:fmwk:base:cmerge_stream} which we keep referring in this section.

\subsection{Why {\cmerge}?}
Why using {\cmerge}? Why not just writing a program that does everything from scratch?
Be my guest and write your code! That might inspire a better design or strategy toward merging. 
That being said, there are a few reasons to use {\cmerge}.

\paragraph{Task Outside Merging Decision:}
{\cmerge} The core part of merging is to compare 2 clusters and make a decision to merge or not. Outside this process, however, there is a non-negligible overburden of running such algorithm such as (1) keeping track of which cluster pairs are merged and (2) optimizing the loop process of cluster pair comparison. {\cmerge} does those tedious jobs while out-sourcing the decision making to an independent {\CPP} algorithm class that inherit from {\cbalgo}. 

\paragraph{Easy Development via Modulated Design:}
Because {\cmerge} outsource an independent algorithm class to perform merging of a cluster pair, it supports modulated code design. Accordingly each {\cmalgo} class becomes a pure algorithm class. This makes a code development very easy because a new algorithm can be written without re-writing the I/O or book-keeping part. Also it makes life easy to compare different algorithms: you can simple change which algorithm to be attached to {\cmerge} at run-time, and this avoids re-compilation or re-writing of code.

\paragraph{X-Framework Implementation}
{\cmerge} does not use data product type specific to an analysis framework (for instance {\larlight} or {\larsoft}), so it can be ported easily among different frameworks. That means {\cmalgo} developed by one person in one analysis framework can be shared among other frameworks (assuming an algorithm developer follows x-framework design pattern).\\

\subsection{How Does It Work?}
\label{sec:fmwk:cmerge:how}
The core part of decision making is simple: provided 2 clusters, decide whether they should be merged or not.
This is done through {\cmalgo::Bool()} function call:
\begin{lstlisting}
  virtual bool Bool( const cluster::ClusterParamsAlg &, const cluster::ClusterParamsAlg &)
\end{lstlisting}
If the return is {\ttfamily true}, two clusters are merged. Else not.
This is ilustrated in Fig.\ref{sec:fmwk:cmerge:merge_example}.

\begin{figure}[ht]\begin{center}
\includegraphics[width=13cm]{./src/Pictures/MergeExample.pdf}
\caption{Example of 2D clusters in a plane before and after merging attempt. Each oval represents a cluster with cluster 2D axis shown in a black solid arrow. The direction of an arrow represents the direction of a cluster development. Each cluster is labeled with an input cluster ID number while the oval color represents distinct output cluster group. Left: initial state with 4 input clusters. Right: example output after running a merge loop over cluster pairs. }
\label{sec:fmwk:cmerge:merge_example}
\end{center}\end{figure}
{\cmerge} takes care of creating a loop over all possible pairs of 2D clusters and calling the above function of {\cmalgo} for merging.
In the process, book-keeping of which pairs to be merged is recorded using {\cbkeeper}.

\begin{figure}[ht]\begin{center}
\includegraphics[width=13cm]{./src/Pictures/MergeProhibitExample.pdf}
\caption{Example of how prohibit algorithm works. Each oval and solid arrow represents a 2D cluster with its direction while the color represents distinct output (merged) cluster. Input cluster index (ID) is shown in the label with the cluster 2D angle. Right-top shows an output of running {\cmalgo} that merge clusters with angle difference within 40 degrees. Right-bottom shows an output of the same {\cmalgo} for merging but with another {\cmalgo} that prohibits merging of clusters with angle difference larger than 40 degrees.}
\label{sec:fmwk:cmerge:merge_prohibit_example}
\end{center}\end{figure}
Let us consider a case of two merged pairs that shares the same input cluster ID. As described in Sec.\ref{sec:fmwk:base:cbk}, {\cbkeeper} combines two pairs of 2D clusters in such case (see Fig.\ref{sec:fmwk:base:cbk}). This could be problematic sometimes. For instance, consider an example {\cmalgo} that combines two clusters if their start points are very close and the angle between two clusters are within 40 degrees. Further assume we have 3 input clusters that share the start point but have different angles at 20, 40, and -20 degrees. For convention these have input cluster ID 0, 1, and 2 respectively. This example algorithm would conclude pairs (0,1) and (0,2) but not for (1,2). But because merged pairs share cluster ID 0, it ends up all 3 clusters being merged. This is shown in the right-top ilustration of Fig.\ref{sec:fmwk:cmerge:merge_prohibit_example}.

This could be a problem, or may not be: this can be only answered by the algorithm author.
The way to avoid this in {\cmerge} is to implement another {\cmalgo} as {\it merge-prohibit}, or simple {\it prohibit} algorithm.
When prohibit algorithm is provided, {\cmerge} first run over all cluster pairs and execute {\ttfamily Bool()} function to find all cluster pairs that are prohibited to merge.
Then {\cmerge} run over the same set of cluster pairs except for those are prohibited to be merged.
The right-bottom ilustration of Fig.\ref{sec:fmwk:cmerge:merge_prohibit_example} shows how prohibit algorithm affects the result.

The prohibit rule is also inherited to merged cluster pairs. For instance, consider input clusters 0, 1 and 2. The pair (1,2) is prohibited to merge, and (0,1) is found to be merged. Then the merged cluster, (0,1) cannot be merged with 2. That means even if a merge algorithm finds (0,2) to be merged, it cannot be because the prohibit rule is inherited from cluster 1 when 0 and 1 aremerged. This feature brings up a question: what determines 0 to be merged with 1 rather than 2? If the merge algorithm is executed on the pair (0,2) first, then the result would have been different. So what determines a different outcome is an order of cluster pairs in the loop of merging inspection. There are different ordering options available in {\cmerge}. Currently available options are defined as {\enum} type in {\cmerge} class, {\ttfamily CMergePriority\_t} which include following values.
\begin{itemize}
\item[] {\ttfamily kIndex}
  \begin{itemize}
    \item From the smaller to larger input cluster ID, basically following input cluster set ordering
  \end{itemize}

\item[] {\ttfamily kPolyArea}
  \begin{itemize}
    \item From the clusters with smaller polygon area to the larger
  \end{itemize}

\item[] {\ttfamily kChargeSum}
  \begin{itemize}
    \item From the clusters with smaller charge sum to the larrger
  \end{itemize}

\item[] {\ttfamily kNHits}
  \begin{itemize}
    \item From the clusters with smaller number of hits to the larger
  \end{itemize}

\end{itemize}
The ordering parameters including all listed above are taken from {\ttfamily cluster\_params} member attributes.
If more useful parameter is found, one should make a request to make it available. 
It is very simple to implement in {\cmerge}.

In summary, {\cmerge} uses one {\cmalgo} instance for merging and another for merge-prohibit. A possible ambiguity of which cluster pairs to be merged in the presence of prohibited pairs is resolved by defining a priority for cluster pairs in the merging attempt. 

\subsection{Process Flow for Cluster Merging}
\label{sec:fmwk:cmerge:process_flow}
An overall process flow diagram is already shown in Fig.\ref{sec:fmwk:base:cmerge_stream}, so here we describe slightly more details in texts by breaking down into 5 blocks.
\begin{enumerate}
\item Event Preparation
  \begin{itemize}
    \item[] {\cmerge} receives an input cluster in terms of a set of {\pxhit} (see Sec.\ref{sec:fmwk:cmerge:input}). Then it creates {\cpan} instance for each cluster, and calls {\cmalgo::EventBegin()} for merging and prohibit algorithms.
  \end{itemize}

\item Iterative Merge Operation
  \begin{itemize}
    \item[] A loop over cluster pairs for prohibit (if provided) and merge {\cmalgo} can be configured to repeat until no newly merged cluster is found. This is shown in Fig.\ref{sec:fmwk:cmerge:iterative_stream}. If a user does not configure to choose iterative approach, iteration will not happen even if there is a newly merged cluster in the output.
  \end{itemize}

\item Prohibit {\cmalgo} Loop Over Cluster Pairs
  \begin{itemize}
    \item[] See Fig.\ref{sec:fmwk:cmerge:iterative_stream}. Only occurs if prohibit algorithm is provided by a user (optional). Records which cluster pair to be prohibited for merging using {\cbkeeper}.
  \end{itemize}

\item Prohibit {\cmalgo} Loop Over Cluster Pairs
  \begin{itemize}
    \item[] See Fig.\ref{sec:fmwk:cmerge:iterative_stream}. The algorithm's {\ttfamily Bool()} is not executed for a subject pair of clusters if (a) they are prohibited for merging or (b) already inspected in the previous process and not merged (i.e. result should not change). This is to save computation time.
  \end{itemize}

\item Merge Clusters
  \begin{itemize}
    \item[] Based on the outcome of merge/prohibit {\cmalgo} algorithms, actual merging is performed at this step. It creates a new set of {\cpan} for merged clusters. For clusters that are not merged, {\cpan} instance is copied from input to save {\cpan} computation time.
  \end{itemize}

\end{enumerate}

\begin{figure}[ht]\begin{center}
\includegraphics[width=13cm]{./src/Pictures/IterativeMergeApproach.pdf}
\caption{Illustration of how iterative merge approach is processed. Internally a loop over cluster pairs for prohibit and merge algorithm are repeated if (a) user configured to use an iterative approach and (b) newly merged cluster is found.}
\label{sec:fmwk:cmerge:iterative_stream}
\end{center}\end{figure}


\subsection{Setting an Input 2D Clusters}
\label{sec:fmwk:cmerge:input}
Input to {\cmerge} is a set of clusters where each cluster is represented as a vector of hits, or more precisely {\ttfamily std::vector<util::PxHit>}. This input can be set by the following function:
\begin{lstlisting}
  void SetClusters(const std::vector<std::vector<util::PxHit> > &)
\end{lstlisting}
where the argument is a vector of 2D clusters. When this function is called, {\cmerge} internally calls
\begin{lstlisting}
  void Reset()
\end{lstlisting}
which initialize previously stored input clusters, output clusters and {\cbkeeper} instance.
Then it creates {\cpan} instance per input 2D cluster using a set of {\pxhit}.
This completes setting (or initiating) the input 2D cluster sets to be used for internal processing.

\subsection{Setting Merge/Prohibit {\cmalgo} Instances}
{\cmalgo} instance is necessary for merging and optional for prohibit. Call the function,
\begin{lstlisting}
  void AddMergeAlg(CBoolAlgoBase* algo)
\end{lstlisting}
to set {\cmalgo} for merging, and 
\begin{lstlisting}
  void AddSeparateAlgo(CBoolAlgoBase* algo)
\end{lstlisting}
to set {\cmalgo} for merge-prohibit.

\subsection{Setting Run Configuration}
There are only 3 run options to configure for {\cmerge}.

\paragraph{Verbosity:} 
There are 5 verbosity levels a user can configure as listed below:
\begin{itemize}
\item[] {\ttfamily kNone}
  \begin{itemize}
    \item No report from {\cmerge}
  \end{itemize}

\item[] {\ttfamily kPerEvent}
  \begin{itemize}
    \item Report made at the end of even processing
  \end{itemize}

\item[] {\ttfamily kPerIteration}
  \begin{itemize}
    \item Report made per merge iteration (see Sec.\ref{sec:fmwk:cmerge:process_flow}). This could be quite verbose.      
  \end{itemize}

\item[] {\ttfamily kPerMerging}
  \begin{itemize}
    \item Report made per {\cmalgo::Bool()} execution (see Sec.\ref{sec:fmwk:cmerge:process_flow}). This is even more verbose than {\ttfamily kPerIteration}.
  \end{itemize}

\end{itemize}
These are {\enum} type, {\ttfamily CMergeMSGLevel\_t}, defined under {\cmerge} class. 
The verbosity level can be set by the function:
\begin{lstlisting}
  void DebugMode(CMergeMSGLevel_t level)
\end{lstlisting}

\paragraph{Iterative Approach:}
As described in Sec.\ref{sec:fmwk:cmerge:process_flow}, {\cmerge} can be configured to attempt iterative approach for merging. This can be enabled by the following function call.
\begin{lstlisting}
  void MergeTillConverge(bool)
\end{lstlisting}
Providing {\ttfamily true} as an argument will enable this option. Disabled if {\ttfamily false} is provided.

\paragraph{Merge Priority:}
As described in Sec.\ref{sec:fmwk:cmerge:how}, ordering of cluster pairs for merging inspection matters to the result.
Options are also described in the same section. This can be set by the following function.
\begin{lstlisting}
  void SetMergePriority(CMergePriority_t level)
\end{lstlisting}

\subsection{Execute Merging}
To execute merging, simply call
\begin{lstlisting}
  void Process()
\end{lstlisting}

\subsection{Obtaining Result}
There are two kinds of results that may be useful for a user. First is a resulting (i.e. merged) set of clusters in terms of {\cpan}. This is useful for further processing as various cluster parameters are already computed as a part of {\cpan}. You can execute
\begin{lstlisting}
  const std::vector<cluster::ClusterParamsAlg>& GetClusters() const
\end{lstlisting}
to obtain such result set.

Second is a mapping of input cluster IDs (i.e. index) to the output (merged clusters). 
As described in Sec.\ref{sec:fmwk:base:cbk}, this can be easily obtained through {\cbkeeper} instance.
You can obtain the {\cbkeeper} instance held by {\cmerge} through the following function.
\begin{lstlisting}
  const CBookKeeper& GetBookKeeper() const
\end{lstlisting}


\section{{\cmatch}: Cluster Matching Tool}
\label{sec:fmwk:cmatch}
Tool not introduced yet in {\larsoft}. Will be documented when this is done.




% python and LArLight
\chapter{Using LArLight in \python}
\label{chap:pyroot}

Despite its extreme simplicity let us call {\cmtool} a framework, or FMWK. 
In this chapter we discuss the building blocks of {\cmtool}.

\section{External Dependency}
\label{sec:fmwk:external}

The {\cmtool} development effort is originated from EM shower reconstruction for MicroBooNE experiment, and it is an extension of existing tool set. The following is the brief list of packages or {\CPP} classes on which {\cmtool} depends.
\begin{itemize}
\item[] {\cru}
    \begin{itemize}
    \item larreco/RecoAlg/ClusterRecoUtil in {\larsoft}
    \item UserDev/ClusterRecoUtil in {\larlight}
    \end{itemize}
\item[] {\geoutil}
    \begin{itemize}
    \item lardata/Utilities in {\larsoft}
    \item core/LArUtil in {\larlight}
    \end{itemize}
\item[] {\ttfamily PxUtil}
    \begin{itemize}
    \item lardata/Utilities in {\larsoft}
    \item core/LArUtil in {\larlight}
    \end{itemize}
\end{itemize}
These items are (or should be if not yet) described in a separate document.
The rest of this section briefly covers what they are without going into a technical detail.

\subsection{{\ttfamily ClusterRecoUtil}}
\label{sec:fmwk:external:cru}
Reconstruction parameters for EM shower 2D cluster involves a lot more variety of parameters than those held in a 2D cluster data product defined by a framework ({\larsoft} or {\larlight}). {\cru} is a package that defines code for both calculation of and storage for those parameters. In particular, {\cmtool} uses following {\CPP} classes from {\cru}:
\begin{itemize}
\item {\ttfamily ClusterParamsAlg ... (\cpan)}
  \begin{itemize}
    \item {\cpan} is designed as both parameter computation algorithm and storage class, and is used along this design principle in {\cmtool}. In particular, {\cpan} is created one per input 2D cluster.
  \end{itemize}
\item {\ttfamily cluster\_params}
  \begin{itemize}
    \item {\ttfamily cluster\_params} is a utility storage class for important parameters computed by \cpan. A const reference can be retrieved from {\cpan} instance through {\ttfamily GetParams()} function call. Note, however, one should keep {\cpan} instance rather than {\ttfamily cluster\_params} as a cluster parameter data set (see {\cpan} description above).
  \end{itemize}
\item {\ttfamily CRUException}
  \begin{itemize}
    \item This is a very simple exception class derived from {\ttfamily std::exception}. It is used instead of {\ttfamily cet::exception} to avoid framework dependency. Only additional feature to the base class is that the argument string to the constructor is sent to standard output streem by overriden {\ttfamily std::exception::what()} function.
  \end{itemize}
\end{itemize}

For details of all classes listed above, you should see Ref.\cite{ClusterRecoUtil}.

\subsection{{\ttfamily PxUtil}}
\label{sec:fmwk:external:pxutil}
Both {\cru} and {\cmtool} are designed to be independent of an analysis framework of user's choice (in particular {\larlight} and {\larsoft}). 
As mentioned in Sec.\ref{sec:fmwk:external:cru}, output of {\cpan} (i.e. cluster parameters) is stored with {\ttfamily cluster\_params} instance. 
{\pxutil}, on the other hand, is used to define input data struct to {\cpan} (and hence {\cmtool} and {\cru} in general). {\pxutil} defines three data objects:
\begin{itemize}
  \item {\ttfamily PxPoint}
    \begin{itemize}
    \item represents 2D point and contains wire [cm], time [cm], and plane ID (unsigned char)
    \end{itemize}
  \item {\ttfamily PxHit}
    \begin{itemize}
    \item inhertis from {\ttfamily PxPoint} with an additional attribute to hold charge [ADC], representing 2D hit
    \end{itemize}
  \item {\ttfamily PxLine}
    \begin{itemize}
    \item represents 2D line and contains start and end coordinates, (wire [cm], time [cm]), with a plane ID (unsigned char)
    \end{itemize}
\end{itemize}

It is a FMWK interface module's respnsibility to translate input data product into these format. In particular, this module must convert a vector of FMWK hit data product (= cluster) into a vector of {\pxhit}. This is discussed in FMWK interface chapter (see Ch.\ref{chap:fmwk_interface}).

\subsection{{\ttfamily GeometryUtilities}}
\label{sec:fmwk:external:geoutil}
{\CPP} class {\geoutil} is a utility class that helps to retrieve detector geometry related information such as converting 3D point coordinate into a projected 2D coordinate on a given plane. 
{\cmtool} uses {\geoutil} to interpret geometry related variables in full extent. 
For development of any non-existing function that treats detector geometry information, it is encouraged to develop in {\geoutil} (and certainly not in {\cmtool}). 
Please see Ref.\cite{GeometryUtilities} for more detailed documentation.


\section{FMWK Bulding Blocks}
\label{sec:fmwk:base}
In this section we discuss two core buliding block classes: (1) {\cbalgo} and (2) {\cbkeeper}. Though we discuss these basic {\CPP} classes first, it might be easier to understand these classes after you read about {\cmerge}, described in Sec.{\ref{sec:fmwk:cmerge}}, because there we discuss how {\cmerge} works with practical usage of {\cbalgo} and {\cbkeeper}.

\subsection{Book Keeping: {\cbkeeper}}
\label{sec:fmwk:base:cbk}
{\cbkeeper} inherits from {\ttfamily std::vector<unsigned short>}, and hence it is a vector of integers by itself.
The length of vector can be specified using a constructor argument.

This class is used to track how input clusters are merged/matched into an output. 
Here we focus on merging of input clusters for the simplicity sake.
The design assumes one to perform a merging attempt of 2 clusters at a time.
When one has many clusters to make such attempt on all possible pairs of clusters, it could be troublesome to perform book-keeping for which clusters are merged together.
{\cbkeeper} is designed such that a user can just log two to-be-merged clusters' ID at each attempt, and it holds a summary log of all merged cluster pairs.


Here are technical details.
The index of vector corresponds to the initial cluster ID numbers.
If you have 8 clusters to be subject for merging, {\cbkeeper} instance should be initialized as length 8.
Initially the values stored in the vector is same as their index numbers as shown in the top of Fig.\ref{sec:fmwk:base:cbk}.
One can re-set the state of {\cbkeeper} instance by calling the following function.
\begin{lstlisting}
        void Reset(unsigned short n=0)
\end{lstlisting}

\begin{figure}\begin{center}
\includegraphics[width=14cm]{./src/Pictures/CBookKeeperExample.pdf}
\caption{Example diagram showing how {\cbkeeper} keeps track of input and output cluster IDs. It shows {\cbkeeper} vector index (input cluster ID) and stored values (output cluster ID) for 8 input clusters (top), and how the stored values change as merging takes place. The middle shows the result of merging input cluster 1 and 2, and the bottom is for further merging 2 and 5.}
\end{center}\end{figure}
When one finds 2 clusters to be merged together, (s)he can call {\cbkeeper} attribute function:
\begin{lstlisting}
        void CBookKeeper::Merge(unsigned short, unsigned short)
\end{lstlisting}
with corresponding clusters' IDs (i.e. index numbers). 
Figure \ref{sec:fmwk:base:cbk} shows how the process works for calling {\ttfamily CBookKeeper::Merge()} function twice with arguments (1,2) and (2,5).
As a result, unique output cluster IDs are reduced to 6 because 3 clusters are combined together.

You can see {\cbkeeper} instance allows to retrieve which output cluster (stored value) an input cluster (vector index) ended up with in an intuitive way: simply use {\ttfamily std::vector::at()} method.
However, to create an output cluster, it is probably easier to ask the other way around: given output cluster ID, which input cluster(s) does it contain?
{\cbkeeper} has a simple function to answer this question:
\begin{lstlisting}
        std::vector<std::vector<unsigned short> > GetResult() const 
\end{lstlisting}
which returns a vector with length same as number of output (i.e. merged) clusters.
The value of this vector is {\ttfamily std::vector<unsigned short>} to contain possibly multiple input cluster indexes.
For instance, the return from {\cbkeeper} shown in the bottom of Fig.\ref{sec:fmwk:base:cbk} would be a vector of length 6, and the index1 contains a vector of input cluster ID 1, 2, and 5 as they have been merged into output cluster ID = 1.
Equivalent result can be obtained also by using
\begin{lstlisting}
        void PassResult(std::vector<std::vector<unsigned short> >&)
\end{lstlisting}
if that suits one's usage. 

If you are interested in checking which input cluster(s) included in a particular output (merged) cluster, you can use:
\begin{lstlisting}
        std::vector<unsigned short> GetMergedSet(unsigned short)
\end{lstlisting}
which takes an output cluster index as an argument and returns a vector that contains a list of input clusters that consist the output.

\begin{figure}[ht]\begin{center}
\includegraphics[width=8.3cm]{./src/Pictures/CBookKeeperProhibitExample2.pdf}
\includegraphics[width=8.3cm]{./src/Pictures/CBookKeeperProhibitExample1.pdf}
\caption{Example diagrams showing how {\cbkeeper} reacts to conflict {\ttfamily Merge()} and {\ttfamily ProhibitMerge()} function calls on the input cluster IDs. The left diagram shows {\crue} thrown due to calling {\ttfamily Merge()} on a cluster pair that is registered as merge-prohibit pair. The right diagram shows a result from calling {\ttfamily ProhibitMerge()} on a cluster pair that is already merged.}
\label{sec:fmwk:base:cbk_prohibit}
\end{center}\end{figure}

Finally, {\cmerge} can also prohibit merging of specific two cluster pair(s), and this is incoorporated by {\cbkeeper} as well.
This is, again, to support user to log merging in a mindless manner.
A user may specify specific 2 cluster pair should not be merged. 
Then (s)he can use the following function to register these 2 clusters for merge-prohibit list.
\begin{lstlisting}
        void ProhibitMerge(unsigned short, unsigned short)
\end{lstlisting}
The above function takes input cluster IDs (i.e. index numbers). 
Once 2 clusters are registered as merge-prohibit pair, attempting a function call {\ttfamily Merge()} will result in {\crue} thrown.
This is illustrated in the left of Fig.\ref{sec:fmwk:base:cbk_prohibit}.
Another possible case that results in {\crue} thrown would be calling {\ttfamily ProhibitMerge()} function on two input cluster IDs that are already merged. This is shown in the right hand side of Fig.\ref{sec:fmwk:base:cbk_prohibit}.

To avoid {\crue}, one should register all cluster pairs to prohibit-merge list before trying any merge attempt.
Upon attempting to call {\ttfamily Merge()}, then, one can use the function:
\begin{lstlisting}
        bool MergeAllowed(unsigned short, unsigned short)
\end{lstlisting}
to check if a cluster pair is registered to merge-prohibit list.
Calling this function prior to evaluation of merging (i.e. running some computation algorithm to decide whether to merge or not) may save some computation time and help the overall speed of the program (this is implemented in {\cmerge}).

\subsection{Algorithm Base: {\ttfamily CBoolAlgoBase}}
\label{sec:fmwk:base:cbalgo}
{\cbalgo} is a simple {\CPP} template class and also a base class for an algorithm used for cluster merging and matching by {\cmerge} and {\cmatch} application classes. In this and later sections, we use the term {\cmalgo} to refer to an algorithm class derived from {\cbalgo}.

\begin{figure}[ht]\begin{center}
\includegraphics[width=13cm]{./src/Pictures/CMergeStream.pdf}
\caption{The data processing stream of {\cmerge} framework. The purple allow shows the ordered process flow. It requires an analysis framework interface to convert framework dependent cluster data produdct into a vector of {\pxhit} and hand over to {\cmerge} application. Internally {\cmerge} calls various members of {\cbalgo} instances: one for merging and another for prohibiting. There is a configuration in {\cmerge} to perform iterative approach, which repeats internal loop over possible cluster pairs until no more newly merged cluster is found in the output. The merged cluster groups (i.e. output) can be obtained through {\cbkeeper} instance by an analysis framework interface to form an appropriate output data product.}
\label{sec:fmwk:base:cmerge_stream}
\end{center}\end{figure}

The idea is very simple: given two clusters, it is expected to return a boolean for decision making from the following function.
\begin{lstlisting}
        virtual bool Bool(const cluster::ClusterParamsAlg &, const cluster::ClusterParamsAlg &)
\end{lstlisting}
As you can see, ``clusters'' are handed over in terms of {\cpan} instance.
In {\cmerge}, the cluster merging application class, for instance, this is used for making merge decision as well as prohibit-merge decision (i.e. {\cmerge} holds two {\cbalgo} inherit class instance, one for merging and another for prohibiting).
An algorithm developpers should write his/her own {\CPP} algorithm class by inheriting from {\cbalgo}.
In the derived class, the {\ttfamily Bool()} function implementation should be over-riden. 
There is a script to generate a template (i.e. empty) {\cbalgo} inherit class and everyone is encouraged to use it for starting a new algorithm class. These are:
\begin{itemize}
\item For {\larsoft}
\begin{itemize}
        \item[]{\ttfamily> python \$MRB\_TOP/srcs/larreco/RecoAlg/CMTool/mac/gen\_cmalgo.py NAME}
\end{itemize}
\item For {\larlight}
\begin{itemize}
        \item[]{\ttfamily> python \$MAKE\_TOP\_DIR/UserDev/ClusterStudy/bin/gen\_cmalgo NAME}
\end{itemize}
\end{itemize}
where {\ttfamily NAME} should be replaced by the new algorithm's name. 
Running above command results in creating {\ttfamily CMAlgoNAME.h} and {\ttfamily CMAlgoNAME.cxx} in the appropriate directory.
Then you can proceed to build following the framework's guidance.

As mentioned, {\cbalgo} is operated by {\cmerge} and {\cmatch} applications.
For a given event, there is an internal loop of calling {\ttfamily Bool()} on all possible cluster pairs.
To provide flexibility in algorithm design, those applications also call other functions of {\cbalgo} in its processing.

The following is a list of such {\cbalgo} functions.
\begin{itemize}

\item[]{\ttfamily virtual void EventBegin(const std::vector<cluster::ClusterParamsAlg>\&)}
  \begin{itemize}
    \item Called at the beginning of an event, prior to looping over cluster pairs. The provided function argument is a vector of {\cpan} for all clusters subject for merging. This is where computation that requires the whole, initial cluster sets should be done.
  \end{itemize}

\item[]{\ttfamily virtual void IterationBegin(const std::vector<cluster::ClusterParamsAlg>\&)}
  \begin{itemize}
    \item Called at the beginning of an each loop over cluster pairs. Note {\cmerge} and {\cmatch} can have multiple iteration of looping over possible cluster pairs if configured to do so. For instance {\cmerge} can take iterate over multiple loops till no more to-be-merged  cluster pair is found (see Sec.\ref{sec:fmwk:cmerge} and Sec.\ref{sec:fmwk:cmatch} for details). The argument of this function is an output cluster sets, in terms of {\cpan} instance, from the previous loop. In case of the 1st loop, the provided cluster sets is identical to those provided in the argument of {\ttfamily EventBegin()} function.
  \end{itemize}

\item[]{\ttfamily virtual void IterationEnd()}
  \begin{itemize}
    \item Called at the end of each merge or match attempt loop over all cluster pairs, to clean up or summarize information.
  \end{itemize}

\item[]{\ttfamily virtual void EventEnd()}
  \begin{itemize}
    \item Called at the end of each event, after all iterations over cluster pairs are done.
  \end{itemize}

\end{itemize}

Further, there are two functions that are not yet implemented in {\cmerge} nor {\cmatch} but ideas exist.
Your input on how these should be implemented is welcome and very much appreciated.
\begin{itemize}

\item[]{\ttfamily virtual void Reset()}
  \begin{itemize}
    \item Meant to clear some member variables. But one can do that within {\ttfamily IterationEnd()} and/or {\ttfamily EventEnd()}, so it is not clear where this should be called. Possibly at the end of the whole event processing.
  \end{itemize}

\item[]{\ttfamily virtual void Report()}
  \begin{itemize}
    \item Meant to support some standardized {\ttfamily std::cout} and {\ttfamily std::cerr} report from an algorithm instance. {\cmerge} and {\cmatch} have ability to control such text output for different levels, and there is a hope this {\ttfamily Report()} can be also included there.
  \end{itemize}

\end{itemize}

\paragraph{Tips for writing {\cmalgo}}
First of all, you are encouraged to use {\ttfamily gen\_cmalgo} script as described earlier.
Another thing to note is to reduce a dependency on analysis framework.
This makes it easier to port your code between different frameworks.



\section{{\cmerge}: Cluster Merging Tool}
\label{sec:fmwk:cmerge}

{\cmerge} is a {\CPP} class that performs merging of 2D clusters, and is an application of {\cmalgo} algorithm classes that are inherit from {\cbalgo}, the algorithm base class. In a few sentences, {\cmerge} perform merging by looping over all possible 2D cluster pairs among the given set of clusters. For each pair, {\ttfamily Bool()} function of both merging and prohibit {\cmalgo} are called to inspect possible 2D cluster merging. It uses {\cbkeeper} to keep track of which pair of clusters are merged. For details of {\cbalgo} and {\cbkeeper}, see Sec.\ref{sec:fmwk:base:cbalgo} and Sec.\ref{sec:fmwk:base:cbk} respectively. The overall process stream diagram is shown in Fig.\ref{sec:fmwk:base:cmerge_stream} which we keep referring in this section.

\subsection{Why {\cmerge}?}
Why using {\cmerge}? Why not just writing a program that does everything from scratch?
Be my guest and write your code! That might inspire a better design or strategy toward merging. 
That being said, there are a few reasons to use {\cmerge}.

\paragraph{Task Outside Merging Decision:}
{\cmerge} The core part of merging is to compare 2 clusters and make a decision to merge or not. Outside this process, however, there is a non-negligible overburden of running such algorithm such as (1) keeping track of which cluster pairs are merged and (2) optimizing the loop process of cluster pair comparison. {\cmerge} does those tedious jobs while out-sourcing the decision making to an independent {\CPP} algorithm class that inherit from {\cbalgo}. 

\paragraph{Easy Development via Modulated Design:}
Because {\cmerge} outsource an independent algorithm class to perform merging of a cluster pair, it supports modulated code design. Accordingly each {\cmalgo} class becomes a pure algorithm class. This makes a code development very easy because a new algorithm can be written without re-writing the I/O or book-keeping part. Also it makes life easy to compare different algorithms: you can simple change which algorithm to be attached to {\cmerge} at run-time, and this avoids re-compilation or re-writing of code.

\paragraph{X-Framework Implementation}
{\cmerge} does not use data product type specific to an analysis framework (for instance {\larlight} or {\larsoft}), so it can be ported easily among different frameworks. That means {\cmalgo} developed by one person in one analysis framework can be shared among other frameworks (assuming an algorithm developer follows x-framework design pattern).\\

\subsection{How Does It Work?}
\label{sec:fmwk:cmerge:how}
The core part of decision making is simple: provided 2 clusters, decide whether they should be merged or not.
This is done through {\cmalgo::Bool()} function call:
\begin{lstlisting}
  virtual bool Bool( const cluster::ClusterParamsAlg &, const cluster::ClusterParamsAlg &)
\end{lstlisting}
If the return is {\ttfamily true}, two clusters are merged. Else not.
This is ilustrated in Fig.\ref{sec:fmwk:cmerge:merge_example}.

\begin{figure}[ht]\begin{center}
\includegraphics[width=13cm]{./src/Pictures/MergeExample.pdf}
\caption{Example of 2D clusters in a plane before and after merging attempt. Each oval represents a cluster with cluster 2D axis shown in a black solid arrow. The direction of an arrow represents the direction of a cluster development. Each cluster is labeled with an input cluster ID number while the oval color represents distinct output cluster group. Left: initial state with 4 input clusters. Right: example output after running a merge loop over cluster pairs. }
\label{sec:fmwk:cmerge:merge_example}
\end{center}\end{figure}
{\cmerge} takes care of creating a loop over all possible pairs of 2D clusters and calling the above function of {\cmalgo} for merging.
In the process, book-keeping of which pairs to be merged is recorded using {\cbkeeper}.

\begin{figure}[ht]\begin{center}
\includegraphics[width=13cm]{./src/Pictures/MergeProhibitExample.pdf}
\caption{Example of how prohibit algorithm works. Each oval and solid arrow represents a 2D cluster with its direction while the color represents distinct output (merged) cluster. Input cluster index (ID) is shown in the label with the cluster 2D angle. Right-top shows an output of running {\cmalgo} that merge clusters with angle difference within 40 degrees. Right-bottom shows an output of the same {\cmalgo} for merging but with another {\cmalgo} that prohibits merging of clusters with angle difference larger than 40 degrees.}
\label{sec:fmwk:cmerge:merge_prohibit_example}
\end{center}\end{figure}
Let us consider a case of two merged pairs that shares the same input cluster ID. As described in Sec.\ref{sec:fmwk:base:cbk}, {\cbkeeper} combines two pairs of 2D clusters in such case (see Fig.\ref{sec:fmwk:base:cbk}). This could be problematic sometimes. For instance, consider an example {\cmalgo} that combines two clusters if their start points are very close and the angle between two clusters are within 40 degrees. Further assume we have 3 input clusters that share the start point but have different angles at 20, 40, and -20 degrees. For convention these have input cluster ID 0, 1, and 2 respectively. This example algorithm would conclude pairs (0,1) and (0,2) but not for (1,2). But because merged pairs share cluster ID 0, it ends up all 3 clusters being merged. This is shown in the right-top ilustration of Fig.\ref{sec:fmwk:cmerge:merge_prohibit_example}.

This could be a problem, or may not be: this can be only answered by the algorithm author.
The way to avoid this in {\cmerge} is to implement another {\cmalgo} as {\it merge-prohibit}, or simple {\it prohibit} algorithm.
When prohibit algorithm is provided, {\cmerge} first run over all cluster pairs and execute {\ttfamily Bool()} function to find all cluster pairs that are prohibited to merge.
Then {\cmerge} run over the same set of cluster pairs except for those are prohibited to be merged.
The right-bottom ilustration of Fig.\ref{sec:fmwk:cmerge:merge_prohibit_example} shows how prohibit algorithm affects the result.

The prohibit rule is also inherited to merged cluster pairs. For instance, consider input clusters 0, 1 and 2. The pair (1,2) is prohibited to merge, and (0,1) is found to be merged. Then the merged cluster, (0,1) cannot be merged with 2. That means even if a merge algorithm finds (0,2) to be merged, it cannot be because the prohibit rule is inherited from cluster 1 when 0 and 1 aremerged. This feature brings up a question: what determines 0 to be merged with 1 rather than 2? If the merge algorithm is executed on the pair (0,2) first, then the result would have been different. So what determines a different outcome is an order of cluster pairs in the loop of merging inspection. There are different ordering options available in {\cmerge}. Currently available options are defined as {\enum} type in {\cmerge} class, {\ttfamily CMergePriority\_t} which include following values.
\begin{itemize}
\item[] {\ttfamily kIndex}
  \begin{itemize}
    \item From the smaller to larger input cluster ID, basically following input cluster set ordering
  \end{itemize}

\item[] {\ttfamily kPolyArea}
  \begin{itemize}
    \item From the clusters with smaller polygon area to the larger
  \end{itemize}

\item[] {\ttfamily kChargeSum}
  \begin{itemize}
    \item From the clusters with smaller charge sum to the larrger
  \end{itemize}

\item[] {\ttfamily kNHits}
  \begin{itemize}
    \item From the clusters with smaller number of hits to the larger
  \end{itemize}

\end{itemize}
The ordering parameters including all listed above are taken from {\ttfamily cluster\_params} member attributes.
If more useful parameter is found, one should make a request to make it available. 
It is very simple to implement in {\cmerge}.

In summary, {\cmerge} uses one {\cmalgo} instance for merging and another for merge-prohibit. A possible ambiguity of which cluster pairs to be merged in the presence of prohibited pairs is resolved by defining a priority for cluster pairs in the merging attempt. 

\subsection{Process Flow for Cluster Merging}
\label{sec:fmwk:cmerge:process_flow}
An overall process flow diagram is already shown in Fig.\ref{sec:fmwk:base:cmerge_stream}, so here we describe slightly more details in texts by breaking down into 5 blocks.
\begin{enumerate}
\item Event Preparation
  \begin{itemize}
    \item[] {\cmerge} receives an input cluster in terms of a set of {\pxhit} (see Sec.\ref{sec:fmwk:cmerge:input}). Then it creates {\cpan} instance for each cluster, and calls {\cmalgo::EventBegin()} for merging and prohibit algorithms.
  \end{itemize}

\item Iterative Merge Operation
  \begin{itemize}
    \item[] A loop over cluster pairs for prohibit (if provided) and merge {\cmalgo} can be configured to repeat until no newly merged cluster is found. This is shown in Fig.\ref{sec:fmwk:cmerge:iterative_stream}. If a user does not configure to choose iterative approach, iteration will not happen even if there is a newly merged cluster in the output.
  \end{itemize}

\item Prohibit {\cmalgo} Loop Over Cluster Pairs
  \begin{itemize}
    \item[] See Fig.\ref{sec:fmwk:cmerge:iterative_stream}. Only occurs if prohibit algorithm is provided by a user (optional). Records which cluster pair to be prohibited for merging using {\cbkeeper}.
  \end{itemize}

\item Prohibit {\cmalgo} Loop Over Cluster Pairs
  \begin{itemize}
    \item[] See Fig.\ref{sec:fmwk:cmerge:iterative_stream}. The algorithm's {\ttfamily Bool()} is not executed for a subject pair of clusters if (a) they are prohibited for merging or (b) already inspected in the previous process and not merged (i.e. result should not change). This is to save computation time.
  \end{itemize}

\item Merge Clusters
  \begin{itemize}
    \item[] Based on the outcome of merge/prohibit {\cmalgo} algorithms, actual merging is performed at this step. It creates a new set of {\cpan} for merged clusters. For clusters that are not merged, {\cpan} instance is copied from input to save {\cpan} computation time.
  \end{itemize}

\end{enumerate}

\begin{figure}[ht]\begin{center}
\includegraphics[width=13cm]{./src/Pictures/IterativeMergeApproach.pdf}
\caption{Illustration of how iterative merge approach is processed. Internally a loop over cluster pairs for prohibit and merge algorithm are repeated if (a) user configured to use an iterative approach and (b) newly merged cluster is found.}
\label{sec:fmwk:cmerge:iterative_stream}
\end{center}\end{figure}


\subsection{Setting an Input 2D Clusters}
\label{sec:fmwk:cmerge:input}
Input to {\cmerge} is a set of clusters where each cluster is represented as a vector of hits, or more precisely {\ttfamily std::vector<util::PxHit>}. This input can be set by the following function:
\begin{lstlisting}
  void SetClusters(const std::vector<std::vector<util::PxHit> > &)
\end{lstlisting}
where the argument is a vector of 2D clusters. When this function is called, {\cmerge} internally calls
\begin{lstlisting}
  void Reset()
\end{lstlisting}
which initialize previously stored input clusters, output clusters and {\cbkeeper} instance.
Then it creates {\cpan} instance per input 2D cluster using a set of {\pxhit}.
This completes setting (or initiating) the input 2D cluster sets to be used for internal processing.

\subsection{Setting Merge/Prohibit {\cmalgo} Instances}
{\cmalgo} instance is necessary for merging and optional for prohibit. Call the function,
\begin{lstlisting}
  void AddMergeAlg(CBoolAlgoBase* algo)
\end{lstlisting}
to set {\cmalgo} for merging, and 
\begin{lstlisting}
  void AddSeparateAlgo(CBoolAlgoBase* algo)
\end{lstlisting}
to set {\cmalgo} for merge-prohibit.

\subsection{Setting Run Configuration}
There are only 3 run options to configure for {\cmerge}.

\paragraph{Verbosity:} 
There are 5 verbosity levels a user can configure as listed below:
\begin{itemize}
\item[] {\ttfamily kNone}
  \begin{itemize}
    \item No report from {\cmerge}
  \end{itemize}

\item[] {\ttfamily kPerEvent}
  \begin{itemize}
    \item Report made at the end of even processing
  \end{itemize}

\item[] {\ttfamily kPerIteration}
  \begin{itemize}
    \item Report made per merge iteration (see Sec.\ref{sec:fmwk:cmerge:process_flow}). This could be quite verbose.      
  \end{itemize}

\item[] {\ttfamily kPerMerging}
  \begin{itemize}
    \item Report made per {\cmalgo::Bool()} execution (see Sec.\ref{sec:fmwk:cmerge:process_flow}). This is even more verbose than {\ttfamily kPerIteration}.
  \end{itemize}

\end{itemize}
These are {\enum} type, {\ttfamily CMergeMSGLevel\_t}, defined under {\cmerge} class. 
The verbosity level can be set by the function:
\begin{lstlisting}
  void DebugMode(CMergeMSGLevel_t level)
\end{lstlisting}

\paragraph{Iterative Approach:}
As described in Sec.\ref{sec:fmwk:cmerge:process_flow}, {\cmerge} can be configured to attempt iterative approach for merging. This can be enabled by the following function call.
\begin{lstlisting}
  void MergeTillConverge(bool)
\end{lstlisting}
Providing {\ttfamily true} as an argument will enable this option. Disabled if {\ttfamily false} is provided.

\paragraph{Merge Priority:}
As described in Sec.\ref{sec:fmwk:cmerge:how}, ordering of cluster pairs for merging inspection matters to the result.
Options are also described in the same section. This can be set by the following function.
\begin{lstlisting}
  void SetMergePriority(CMergePriority_t level)
\end{lstlisting}

\subsection{Execute Merging}
To execute merging, simply call
\begin{lstlisting}
  void Process()
\end{lstlisting}

\subsection{Obtaining Result}
There are two kinds of results that may be useful for a user. First is a resulting (i.e. merged) set of clusters in terms of {\cpan}. This is useful for further processing as various cluster parameters are already computed as a part of {\cpan}. You can execute
\begin{lstlisting}
  const std::vector<cluster::ClusterParamsAlg>& GetClusters() const
\end{lstlisting}
to obtain such result set.

Second is a mapping of input cluster IDs (i.e. index) to the output (merged clusters). 
As described in Sec.\ref{sec:fmwk:base:cbk}, this can be easily obtained through {\cbkeeper} instance.
You can obtain the {\cbkeeper} instance held by {\cmerge} through the following function.
\begin{lstlisting}
  const CBookKeeper& GetBookKeeper() const
\end{lstlisting}


\section{{\cmatch}: Cluster Matching Tool}
\label{sec:fmwk:cmatch}
Tool not introduced yet in {\larsoft}. Will be documented when this is done.




% LArSoft => LArLight converter
\chapter{Where is LArLight Data File?}
\label{chap:datafile}

Despite its extreme simplicity let us call {\cmtool} a framework, or FMWK. 
In this chapter we discuss the building blocks of {\cmtool}.

\section{External Dependency}
\label{sec:fmwk:external}

The {\cmtool} development effort is originated from EM shower reconstruction for MicroBooNE experiment, and it is an extension of existing tool set. The following is the brief list of packages or {\CPP} classes on which {\cmtool} depends.
\begin{itemize}
\item[] {\cru}
    \begin{itemize}
    \item larreco/RecoAlg/ClusterRecoUtil in {\larsoft}
    \item UserDev/ClusterRecoUtil in {\larlight}
    \end{itemize}
\item[] {\geoutil}
    \begin{itemize}
    \item lardata/Utilities in {\larsoft}
    \item core/LArUtil in {\larlight}
    \end{itemize}
\item[] {\ttfamily PxUtil}
    \begin{itemize}
    \item lardata/Utilities in {\larsoft}
    \item core/LArUtil in {\larlight}
    \end{itemize}
\end{itemize}
These items are (or should be if not yet) described in a separate document.
The rest of this section briefly covers what they are without going into a technical detail.

\subsection{{\ttfamily ClusterRecoUtil}}
\label{sec:fmwk:external:cru}
Reconstruction parameters for EM shower 2D cluster involves a lot more variety of parameters than those held in a 2D cluster data product defined by a framework ({\larsoft} or {\larlight}). {\cru} is a package that defines code for both calculation of and storage for those parameters. In particular, {\cmtool} uses following {\CPP} classes from {\cru}:
\begin{itemize}
\item {\ttfamily ClusterParamsAlg ... (\cpan)}
  \begin{itemize}
    \item {\cpan} is designed as both parameter computation algorithm and storage class, and is used along this design principle in {\cmtool}. In particular, {\cpan} is created one per input 2D cluster.
  \end{itemize}
\item {\ttfamily cluster\_params}
  \begin{itemize}
    \item {\ttfamily cluster\_params} is a utility storage class for important parameters computed by \cpan. A const reference can be retrieved from {\cpan} instance through {\ttfamily GetParams()} function call. Note, however, one should keep {\cpan} instance rather than {\ttfamily cluster\_params} as a cluster parameter data set (see {\cpan} description above).
  \end{itemize}
\item {\ttfamily CRUException}
  \begin{itemize}
    \item This is a very simple exception class derived from {\ttfamily std::exception}. It is used instead of {\ttfamily cet::exception} to avoid framework dependency. Only additional feature to the base class is that the argument string to the constructor is sent to standard output streem by overriden {\ttfamily std::exception::what()} function.
  \end{itemize}
\end{itemize}

For details of all classes listed above, you should see Ref.\cite{ClusterRecoUtil}.

\subsection{{\ttfamily PxUtil}}
\label{sec:fmwk:external:pxutil}
Both {\cru} and {\cmtool} are designed to be independent of an analysis framework of user's choice (in particular {\larlight} and {\larsoft}). 
As mentioned in Sec.\ref{sec:fmwk:external:cru}, output of {\cpan} (i.e. cluster parameters) is stored with {\ttfamily cluster\_params} instance. 
{\pxutil}, on the other hand, is used to define input data struct to {\cpan} (and hence {\cmtool} and {\cru} in general). {\pxutil} defines three data objects:
\begin{itemize}
  \item {\ttfamily PxPoint}
    \begin{itemize}
    \item represents 2D point and contains wire [cm], time [cm], and plane ID (unsigned char)
    \end{itemize}
  \item {\ttfamily PxHit}
    \begin{itemize}
    \item inhertis from {\ttfamily PxPoint} with an additional attribute to hold charge [ADC], representing 2D hit
    \end{itemize}
  \item {\ttfamily PxLine}
    \begin{itemize}
    \item represents 2D line and contains start and end coordinates, (wire [cm], time [cm]), with a plane ID (unsigned char)
    \end{itemize}
\end{itemize}

It is a FMWK interface module's respnsibility to translate input data product into these format. In particular, this module must convert a vector of FMWK hit data product (= cluster) into a vector of {\pxhit}. This is discussed in FMWK interface chapter (see Ch.\ref{chap:fmwk_interface}).

\subsection{{\ttfamily GeometryUtilities}}
\label{sec:fmwk:external:geoutil}
{\CPP} class {\geoutil} is a utility class that helps to retrieve detector geometry related information such as converting 3D point coordinate into a projected 2D coordinate on a given plane. 
{\cmtool} uses {\geoutil} to interpret geometry related variables in full extent. 
For development of any non-existing function that treats detector geometry information, it is encouraged to develop in {\geoutil} (and certainly not in {\cmtool}). 
Please see Ref.\cite{GeometryUtilities} for more detailed documentation.


\section{FMWK Bulding Blocks}
\label{sec:fmwk:base}
In this section we discuss two core buliding block classes: (1) {\cbalgo} and (2) {\cbkeeper}. Though we discuss these basic {\CPP} classes first, it might be easier to understand these classes after you read about {\cmerge}, described in Sec.{\ref{sec:fmwk:cmerge}}, because there we discuss how {\cmerge} works with practical usage of {\cbalgo} and {\cbkeeper}.

\subsection{Book Keeping: {\cbkeeper}}
\label{sec:fmwk:base:cbk}
{\cbkeeper} inherits from {\ttfamily std::vector<unsigned short>}, and hence it is a vector of integers by itself.
The length of vector can be specified using a constructor argument.

This class is used to track how input clusters are merged/matched into an output. 
Here we focus on merging of input clusters for the simplicity sake.
The design assumes one to perform a merging attempt of 2 clusters at a time.
When one has many clusters to make such attempt on all possible pairs of clusters, it could be troublesome to perform book-keeping for which clusters are merged together.
{\cbkeeper} is designed such that a user can just log two to-be-merged clusters' ID at each attempt, and it holds a summary log of all merged cluster pairs.


Here are technical details.
The index of vector corresponds to the initial cluster ID numbers.
If you have 8 clusters to be subject for merging, {\cbkeeper} instance should be initialized as length 8.
Initially the values stored in the vector is same as their index numbers as shown in the top of Fig.\ref{sec:fmwk:base:cbk}.
One can re-set the state of {\cbkeeper} instance by calling the following function.
\begin{lstlisting}
        void Reset(unsigned short n=0)
\end{lstlisting}

\begin{figure}\begin{center}
\includegraphics[width=14cm]{./src/Pictures/CBookKeeperExample.pdf}
\caption{Example diagram showing how {\cbkeeper} keeps track of input and output cluster IDs. It shows {\cbkeeper} vector index (input cluster ID) and stored values (output cluster ID) for 8 input clusters (top), and how the stored values change as merging takes place. The middle shows the result of merging input cluster 1 and 2, and the bottom is for further merging 2 and 5.}
\end{center}\end{figure}
When one finds 2 clusters to be merged together, (s)he can call {\cbkeeper} attribute function:
\begin{lstlisting}
        void CBookKeeper::Merge(unsigned short, unsigned short)
\end{lstlisting}
with corresponding clusters' IDs (i.e. index numbers). 
Figure \ref{sec:fmwk:base:cbk} shows how the process works for calling {\ttfamily CBookKeeper::Merge()} function twice with arguments (1,2) and (2,5).
As a result, unique output cluster IDs are reduced to 6 because 3 clusters are combined together.

You can see {\cbkeeper} instance allows to retrieve which output cluster (stored value) an input cluster (vector index) ended up with in an intuitive way: simply use {\ttfamily std::vector::at()} method.
However, to create an output cluster, it is probably easier to ask the other way around: given output cluster ID, which input cluster(s) does it contain?
{\cbkeeper} has a simple function to answer this question:
\begin{lstlisting}
        std::vector<std::vector<unsigned short> > GetResult() const 
\end{lstlisting}
which returns a vector with length same as number of output (i.e. merged) clusters.
The value of this vector is {\ttfamily std::vector<unsigned short>} to contain possibly multiple input cluster indexes.
For instance, the return from {\cbkeeper} shown in the bottom of Fig.\ref{sec:fmwk:base:cbk} would be a vector of length 6, and the index1 contains a vector of input cluster ID 1, 2, and 5 as they have been merged into output cluster ID = 1.
Equivalent result can be obtained also by using
\begin{lstlisting}
        void PassResult(std::vector<std::vector<unsigned short> >&)
\end{lstlisting}
if that suits one's usage. 

If you are interested in checking which input cluster(s) included in a particular output (merged) cluster, you can use:
\begin{lstlisting}
        std::vector<unsigned short> GetMergedSet(unsigned short)
\end{lstlisting}
which takes an output cluster index as an argument and returns a vector that contains a list of input clusters that consist the output.

\begin{figure}[ht]\begin{center}
\includegraphics[width=8.3cm]{./src/Pictures/CBookKeeperProhibitExample2.pdf}
\includegraphics[width=8.3cm]{./src/Pictures/CBookKeeperProhibitExample1.pdf}
\caption{Example diagrams showing how {\cbkeeper} reacts to conflict {\ttfamily Merge()} and {\ttfamily ProhibitMerge()} function calls on the input cluster IDs. The left diagram shows {\crue} thrown due to calling {\ttfamily Merge()} on a cluster pair that is registered as merge-prohibit pair. The right diagram shows a result from calling {\ttfamily ProhibitMerge()} on a cluster pair that is already merged.}
\label{sec:fmwk:base:cbk_prohibit}
\end{center}\end{figure}

Finally, {\cmerge} can also prohibit merging of specific two cluster pair(s), and this is incoorporated by {\cbkeeper} as well.
This is, again, to support user to log merging in a mindless manner.
A user may specify specific 2 cluster pair should not be merged. 
Then (s)he can use the following function to register these 2 clusters for merge-prohibit list.
\begin{lstlisting}
        void ProhibitMerge(unsigned short, unsigned short)
\end{lstlisting}
The above function takes input cluster IDs (i.e. index numbers). 
Once 2 clusters are registered as merge-prohibit pair, attempting a function call {\ttfamily Merge()} will result in {\crue} thrown.
This is illustrated in the left of Fig.\ref{sec:fmwk:base:cbk_prohibit}.
Another possible case that results in {\crue} thrown would be calling {\ttfamily ProhibitMerge()} function on two input cluster IDs that are already merged. This is shown in the right hand side of Fig.\ref{sec:fmwk:base:cbk_prohibit}.

To avoid {\crue}, one should register all cluster pairs to prohibit-merge list before trying any merge attempt.
Upon attempting to call {\ttfamily Merge()}, then, one can use the function:
\begin{lstlisting}
        bool MergeAllowed(unsigned short, unsigned short)
\end{lstlisting}
to check if a cluster pair is registered to merge-prohibit list.
Calling this function prior to evaluation of merging (i.e. running some computation algorithm to decide whether to merge or not) may save some computation time and help the overall speed of the program (this is implemented in {\cmerge}).

\subsection{Algorithm Base: {\ttfamily CBoolAlgoBase}}
\label{sec:fmwk:base:cbalgo}
{\cbalgo} is a simple {\CPP} template class and also a base class for an algorithm used for cluster merging and matching by {\cmerge} and {\cmatch} application classes. In this and later sections, we use the term {\cmalgo} to refer to an algorithm class derived from {\cbalgo}.

\begin{figure}[ht]\begin{center}
\includegraphics[width=13cm]{./src/Pictures/CMergeStream.pdf}
\caption{The data processing stream of {\cmerge} framework. The purple allow shows the ordered process flow. It requires an analysis framework interface to convert framework dependent cluster data produdct into a vector of {\pxhit} and hand over to {\cmerge} application. Internally {\cmerge} calls various members of {\cbalgo} instances: one for merging and another for prohibiting. There is a configuration in {\cmerge} to perform iterative approach, which repeats internal loop over possible cluster pairs until no more newly merged cluster is found in the output. The merged cluster groups (i.e. output) can be obtained through {\cbkeeper} instance by an analysis framework interface to form an appropriate output data product.}
\label{sec:fmwk:base:cmerge_stream}
\end{center}\end{figure}

The idea is very simple: given two clusters, it is expected to return a boolean for decision making from the following function.
\begin{lstlisting}
        virtual bool Bool(const cluster::ClusterParamsAlg &, const cluster::ClusterParamsAlg &)
\end{lstlisting}
As you can see, ``clusters'' are handed over in terms of {\cpan} instance.
In {\cmerge}, the cluster merging application class, for instance, this is used for making merge decision as well as prohibit-merge decision (i.e. {\cmerge} holds two {\cbalgo} inherit class instance, one for merging and another for prohibiting).
An algorithm developpers should write his/her own {\CPP} algorithm class by inheriting from {\cbalgo}.
In the derived class, the {\ttfamily Bool()} function implementation should be over-riden. 
There is a script to generate a template (i.e. empty) {\cbalgo} inherit class and everyone is encouraged to use it for starting a new algorithm class. These are:
\begin{itemize}
\item For {\larsoft}
\begin{itemize}
        \item[]{\ttfamily> python \$MRB\_TOP/srcs/larreco/RecoAlg/CMTool/mac/gen\_cmalgo.py NAME}
\end{itemize}
\item For {\larlight}
\begin{itemize}
        \item[]{\ttfamily> python \$MAKE\_TOP\_DIR/UserDev/ClusterStudy/bin/gen\_cmalgo NAME}
\end{itemize}
\end{itemize}
where {\ttfamily NAME} should be replaced by the new algorithm's name. 
Running above command results in creating {\ttfamily CMAlgoNAME.h} and {\ttfamily CMAlgoNAME.cxx} in the appropriate directory.
Then you can proceed to build following the framework's guidance.

As mentioned, {\cbalgo} is operated by {\cmerge} and {\cmatch} applications.
For a given event, there is an internal loop of calling {\ttfamily Bool()} on all possible cluster pairs.
To provide flexibility in algorithm design, those applications also call other functions of {\cbalgo} in its processing.

The following is a list of such {\cbalgo} functions.
\begin{itemize}

\item[]{\ttfamily virtual void EventBegin(const std::vector<cluster::ClusterParamsAlg>\&)}
  \begin{itemize}
    \item Called at the beginning of an event, prior to looping over cluster pairs. The provided function argument is a vector of {\cpan} for all clusters subject for merging. This is where computation that requires the whole, initial cluster sets should be done.
  \end{itemize}

\item[]{\ttfamily virtual void IterationBegin(const std::vector<cluster::ClusterParamsAlg>\&)}
  \begin{itemize}
    \item Called at the beginning of an each loop over cluster pairs. Note {\cmerge} and {\cmatch} can have multiple iteration of looping over possible cluster pairs if configured to do so. For instance {\cmerge} can take iterate over multiple loops till no more to-be-merged  cluster pair is found (see Sec.\ref{sec:fmwk:cmerge} and Sec.\ref{sec:fmwk:cmatch} for details). The argument of this function is an output cluster sets, in terms of {\cpan} instance, from the previous loop. In case of the 1st loop, the provided cluster sets is identical to those provided in the argument of {\ttfamily EventBegin()} function.
  \end{itemize}

\item[]{\ttfamily virtual void IterationEnd()}
  \begin{itemize}
    \item Called at the end of each merge or match attempt loop over all cluster pairs, to clean up or summarize information.
  \end{itemize}

\item[]{\ttfamily virtual void EventEnd()}
  \begin{itemize}
    \item Called at the end of each event, after all iterations over cluster pairs are done.
  \end{itemize}

\end{itemize}

Further, there are two functions that are not yet implemented in {\cmerge} nor {\cmatch} but ideas exist.
Your input on how these should be implemented is welcome and very much appreciated.
\begin{itemize}

\item[]{\ttfamily virtual void Reset()}
  \begin{itemize}
    \item Meant to clear some member variables. But one can do that within {\ttfamily IterationEnd()} and/or {\ttfamily EventEnd()}, so it is not clear where this should be called. Possibly at the end of the whole event processing.
  \end{itemize}

\item[]{\ttfamily virtual void Report()}
  \begin{itemize}
    \item Meant to support some standardized {\ttfamily std::cout} and {\ttfamily std::cerr} report from an algorithm instance. {\cmerge} and {\cmatch} have ability to control such text output for different levels, and there is a hope this {\ttfamily Report()} can be also included there.
  \end{itemize}

\end{itemize}

\paragraph{Tips for writing {\cmalgo}}
First of all, you are encouraged to use {\ttfamily gen\_cmalgo} script as described earlier.
Another thing to note is to reduce a dependency on analysis framework.
This makes it easier to port your code between different frameworks.



\section{{\cmerge}: Cluster Merging Tool}
\label{sec:fmwk:cmerge}

{\cmerge} is a {\CPP} class that performs merging of 2D clusters, and is an application of {\cmalgo} algorithm classes that are inherit from {\cbalgo}, the algorithm base class. In a few sentences, {\cmerge} perform merging by looping over all possible 2D cluster pairs among the given set of clusters. For each pair, {\ttfamily Bool()} function of both merging and prohibit {\cmalgo} are called to inspect possible 2D cluster merging. It uses {\cbkeeper} to keep track of which pair of clusters are merged. For details of {\cbalgo} and {\cbkeeper}, see Sec.\ref{sec:fmwk:base:cbalgo} and Sec.\ref{sec:fmwk:base:cbk} respectively. The overall process stream diagram is shown in Fig.\ref{sec:fmwk:base:cmerge_stream} which we keep referring in this section.

\subsection{Why {\cmerge}?}
Why using {\cmerge}? Why not just writing a program that does everything from scratch?
Be my guest and write your code! That might inspire a better design or strategy toward merging. 
That being said, there are a few reasons to use {\cmerge}.

\paragraph{Task Outside Merging Decision:}
{\cmerge} The core part of merging is to compare 2 clusters and make a decision to merge or not. Outside this process, however, there is a non-negligible overburden of running such algorithm such as (1) keeping track of which cluster pairs are merged and (2) optimizing the loop process of cluster pair comparison. {\cmerge} does those tedious jobs while out-sourcing the decision making to an independent {\CPP} algorithm class that inherit from {\cbalgo}. 

\paragraph{Easy Development via Modulated Design:}
Because {\cmerge} outsource an independent algorithm class to perform merging of a cluster pair, it supports modulated code design. Accordingly each {\cmalgo} class becomes a pure algorithm class. This makes a code development very easy because a new algorithm can be written without re-writing the I/O or book-keeping part. Also it makes life easy to compare different algorithms: you can simple change which algorithm to be attached to {\cmerge} at run-time, and this avoids re-compilation or re-writing of code.

\paragraph{X-Framework Implementation}
{\cmerge} does not use data product type specific to an analysis framework (for instance {\larlight} or {\larsoft}), so it can be ported easily among different frameworks. That means {\cmalgo} developed by one person in one analysis framework can be shared among other frameworks (assuming an algorithm developer follows x-framework design pattern).\\

\subsection{How Does It Work?}
\label{sec:fmwk:cmerge:how}
The core part of decision making is simple: provided 2 clusters, decide whether they should be merged or not.
This is done through {\cmalgo::Bool()} function call:
\begin{lstlisting}
  virtual bool Bool( const cluster::ClusterParamsAlg &, const cluster::ClusterParamsAlg &)
\end{lstlisting}
If the return is {\ttfamily true}, two clusters are merged. Else not.
This is ilustrated in Fig.\ref{sec:fmwk:cmerge:merge_example}.

\begin{figure}[ht]\begin{center}
\includegraphics[width=13cm]{./src/Pictures/MergeExample.pdf}
\caption{Example of 2D clusters in a plane before and after merging attempt. Each oval represents a cluster with cluster 2D axis shown in a black solid arrow. The direction of an arrow represents the direction of a cluster development. Each cluster is labeled with an input cluster ID number while the oval color represents distinct output cluster group. Left: initial state with 4 input clusters. Right: example output after running a merge loop over cluster pairs. }
\label{sec:fmwk:cmerge:merge_example}
\end{center}\end{figure}
{\cmerge} takes care of creating a loop over all possible pairs of 2D clusters and calling the above function of {\cmalgo} for merging.
In the process, book-keeping of which pairs to be merged is recorded using {\cbkeeper}.

\begin{figure}[ht]\begin{center}
\includegraphics[width=13cm]{./src/Pictures/MergeProhibitExample.pdf}
\caption{Example of how prohibit algorithm works. Each oval and solid arrow represents a 2D cluster with its direction while the color represents distinct output (merged) cluster. Input cluster index (ID) is shown in the label with the cluster 2D angle. Right-top shows an output of running {\cmalgo} that merge clusters with angle difference within 40 degrees. Right-bottom shows an output of the same {\cmalgo} for merging but with another {\cmalgo} that prohibits merging of clusters with angle difference larger than 40 degrees.}
\label{sec:fmwk:cmerge:merge_prohibit_example}
\end{center}\end{figure}
Let us consider a case of two merged pairs that shares the same input cluster ID. As described in Sec.\ref{sec:fmwk:base:cbk}, {\cbkeeper} combines two pairs of 2D clusters in such case (see Fig.\ref{sec:fmwk:base:cbk}). This could be problematic sometimes. For instance, consider an example {\cmalgo} that combines two clusters if their start points are very close and the angle between two clusters are within 40 degrees. Further assume we have 3 input clusters that share the start point but have different angles at 20, 40, and -20 degrees. For convention these have input cluster ID 0, 1, and 2 respectively. This example algorithm would conclude pairs (0,1) and (0,2) but not for (1,2). But because merged pairs share cluster ID 0, it ends up all 3 clusters being merged. This is shown in the right-top ilustration of Fig.\ref{sec:fmwk:cmerge:merge_prohibit_example}.

This could be a problem, or may not be: this can be only answered by the algorithm author.
The way to avoid this in {\cmerge} is to implement another {\cmalgo} as {\it merge-prohibit}, or simple {\it prohibit} algorithm.
When prohibit algorithm is provided, {\cmerge} first run over all cluster pairs and execute {\ttfamily Bool()} function to find all cluster pairs that are prohibited to merge.
Then {\cmerge} run over the same set of cluster pairs except for those are prohibited to be merged.
The right-bottom ilustration of Fig.\ref{sec:fmwk:cmerge:merge_prohibit_example} shows how prohibit algorithm affects the result.

The prohibit rule is also inherited to merged cluster pairs. For instance, consider input clusters 0, 1 and 2. The pair (1,2) is prohibited to merge, and (0,1) is found to be merged. Then the merged cluster, (0,1) cannot be merged with 2. That means even if a merge algorithm finds (0,2) to be merged, it cannot be because the prohibit rule is inherited from cluster 1 when 0 and 1 aremerged. This feature brings up a question: what determines 0 to be merged with 1 rather than 2? If the merge algorithm is executed on the pair (0,2) first, then the result would have been different. So what determines a different outcome is an order of cluster pairs in the loop of merging inspection. There are different ordering options available in {\cmerge}. Currently available options are defined as {\enum} type in {\cmerge} class, {\ttfamily CMergePriority\_t} which include following values.
\begin{itemize}
\item[] {\ttfamily kIndex}
  \begin{itemize}
    \item From the smaller to larger input cluster ID, basically following input cluster set ordering
  \end{itemize}

\item[] {\ttfamily kPolyArea}
  \begin{itemize}
    \item From the clusters with smaller polygon area to the larger
  \end{itemize}

\item[] {\ttfamily kChargeSum}
  \begin{itemize}
    \item From the clusters with smaller charge sum to the larrger
  \end{itemize}

\item[] {\ttfamily kNHits}
  \begin{itemize}
    \item From the clusters with smaller number of hits to the larger
  \end{itemize}

\end{itemize}
The ordering parameters including all listed above are taken from {\ttfamily cluster\_params} member attributes.
If more useful parameter is found, one should make a request to make it available. 
It is very simple to implement in {\cmerge}.

In summary, {\cmerge} uses one {\cmalgo} instance for merging and another for merge-prohibit. A possible ambiguity of which cluster pairs to be merged in the presence of prohibited pairs is resolved by defining a priority for cluster pairs in the merging attempt. 

\subsection{Process Flow for Cluster Merging}
\label{sec:fmwk:cmerge:process_flow}
An overall process flow diagram is already shown in Fig.\ref{sec:fmwk:base:cmerge_stream}, so here we describe slightly more details in texts by breaking down into 5 blocks.
\begin{enumerate}
\item Event Preparation
  \begin{itemize}
    \item[] {\cmerge} receives an input cluster in terms of a set of {\pxhit} (see Sec.\ref{sec:fmwk:cmerge:input}). Then it creates {\cpan} instance for each cluster, and calls {\cmalgo::EventBegin()} for merging and prohibit algorithms.
  \end{itemize}

\item Iterative Merge Operation
  \begin{itemize}
    \item[] A loop over cluster pairs for prohibit (if provided) and merge {\cmalgo} can be configured to repeat until no newly merged cluster is found. This is shown in Fig.\ref{sec:fmwk:cmerge:iterative_stream}. If a user does not configure to choose iterative approach, iteration will not happen even if there is a newly merged cluster in the output.
  \end{itemize}

\item Prohibit {\cmalgo} Loop Over Cluster Pairs
  \begin{itemize}
    \item[] See Fig.\ref{sec:fmwk:cmerge:iterative_stream}. Only occurs if prohibit algorithm is provided by a user (optional). Records which cluster pair to be prohibited for merging using {\cbkeeper}.
  \end{itemize}

\item Prohibit {\cmalgo} Loop Over Cluster Pairs
  \begin{itemize}
    \item[] See Fig.\ref{sec:fmwk:cmerge:iterative_stream}. The algorithm's {\ttfamily Bool()} is not executed for a subject pair of clusters if (a) they are prohibited for merging or (b) already inspected in the previous process and not merged (i.e. result should not change). This is to save computation time.
  \end{itemize}

\item Merge Clusters
  \begin{itemize}
    \item[] Based on the outcome of merge/prohibit {\cmalgo} algorithms, actual merging is performed at this step. It creates a new set of {\cpan} for merged clusters. For clusters that are not merged, {\cpan} instance is copied from input to save {\cpan} computation time.
  \end{itemize}

\end{enumerate}

\begin{figure}[ht]\begin{center}
\includegraphics[width=13cm]{./src/Pictures/IterativeMergeApproach.pdf}
\caption{Illustration of how iterative merge approach is processed. Internally a loop over cluster pairs for prohibit and merge algorithm are repeated if (a) user configured to use an iterative approach and (b) newly merged cluster is found.}
\label{sec:fmwk:cmerge:iterative_stream}
\end{center}\end{figure}


\subsection{Setting an Input 2D Clusters}
\label{sec:fmwk:cmerge:input}
Input to {\cmerge} is a set of clusters where each cluster is represented as a vector of hits, or more precisely {\ttfamily std::vector<util::PxHit>}. This input can be set by the following function:
\begin{lstlisting}
  void SetClusters(const std::vector<std::vector<util::PxHit> > &)
\end{lstlisting}
where the argument is a vector of 2D clusters. When this function is called, {\cmerge} internally calls
\begin{lstlisting}
  void Reset()
\end{lstlisting}
which initialize previously stored input clusters, output clusters and {\cbkeeper} instance.
Then it creates {\cpan} instance per input 2D cluster using a set of {\pxhit}.
This completes setting (or initiating) the input 2D cluster sets to be used for internal processing.

\subsection{Setting Merge/Prohibit {\cmalgo} Instances}
{\cmalgo} instance is necessary for merging and optional for prohibit. Call the function,
\begin{lstlisting}
  void AddMergeAlg(CBoolAlgoBase* algo)
\end{lstlisting}
to set {\cmalgo} for merging, and 
\begin{lstlisting}
  void AddSeparateAlgo(CBoolAlgoBase* algo)
\end{lstlisting}
to set {\cmalgo} for merge-prohibit.

\subsection{Setting Run Configuration}
There are only 3 run options to configure for {\cmerge}.

\paragraph{Verbosity:} 
There are 5 verbosity levels a user can configure as listed below:
\begin{itemize}
\item[] {\ttfamily kNone}
  \begin{itemize}
    \item No report from {\cmerge}
  \end{itemize}

\item[] {\ttfamily kPerEvent}
  \begin{itemize}
    \item Report made at the end of even processing
  \end{itemize}

\item[] {\ttfamily kPerIteration}
  \begin{itemize}
    \item Report made per merge iteration (see Sec.\ref{sec:fmwk:cmerge:process_flow}). This could be quite verbose.      
  \end{itemize}

\item[] {\ttfamily kPerMerging}
  \begin{itemize}
    \item Report made per {\cmalgo::Bool()} execution (see Sec.\ref{sec:fmwk:cmerge:process_flow}). This is even more verbose than {\ttfamily kPerIteration}.
  \end{itemize}

\end{itemize}
These are {\enum} type, {\ttfamily CMergeMSGLevel\_t}, defined under {\cmerge} class. 
The verbosity level can be set by the function:
\begin{lstlisting}
  void DebugMode(CMergeMSGLevel_t level)
\end{lstlisting}

\paragraph{Iterative Approach:}
As described in Sec.\ref{sec:fmwk:cmerge:process_flow}, {\cmerge} can be configured to attempt iterative approach for merging. This can be enabled by the following function call.
\begin{lstlisting}
  void MergeTillConverge(bool)
\end{lstlisting}
Providing {\ttfamily true} as an argument will enable this option. Disabled if {\ttfamily false} is provided.

\paragraph{Merge Priority:}
As described in Sec.\ref{sec:fmwk:cmerge:how}, ordering of cluster pairs for merging inspection matters to the result.
Options are also described in the same section. This can be set by the following function.
\begin{lstlisting}
  void SetMergePriority(CMergePriority_t level)
\end{lstlisting}

\subsection{Execute Merging}
To execute merging, simply call
\begin{lstlisting}
  void Process()
\end{lstlisting}

\subsection{Obtaining Result}
There are two kinds of results that may be useful for a user. First is a resulting (i.e. merged) set of clusters in terms of {\cpan}. This is useful for further processing as various cluster parameters are already computed as a part of {\cpan}. You can execute
\begin{lstlisting}
  const std::vector<cluster::ClusterParamsAlg>& GetClusters() const
\end{lstlisting}
to obtain such result set.

Second is a mapping of input cluster IDs (i.e. index) to the output (merged clusters). 
As described in Sec.\ref{sec:fmwk:base:cbk}, this can be easily obtained through {\cbkeeper} instance.
You can obtain the {\cbkeeper} instance held by {\cmerge} through the following function.
\begin{lstlisting}
  const CBookKeeper& GetBookKeeper() const
\end{lstlisting}


\section{{\cmatch}: Cluster Matching Tool}
\label{sec:fmwk:cmatch}
Tool not introduced yet in {\larsoft}. Will be documented when this is done.




% FAQ
\chapter{FAQ}
\label{chap:faq}

Despite its extreme simplicity let us call {\cmtool} a framework, or FMWK. 
In this chapter we discuss the building blocks of {\cmtool}.

\section{External Dependency}
\label{sec:fmwk:external}

The {\cmtool} development effort is originated from EM shower reconstruction for MicroBooNE experiment, and it is an extension of existing tool set. The following is the brief list of packages or {\CPP} classes on which {\cmtool} depends.
\begin{itemize}
\item[] {\cru}
    \begin{itemize}
    \item larreco/RecoAlg/ClusterRecoUtil in {\larsoft}
    \item UserDev/ClusterRecoUtil in {\larlight}
    \end{itemize}
\item[] {\geoutil}
    \begin{itemize}
    \item lardata/Utilities in {\larsoft}
    \item core/LArUtil in {\larlight}
    \end{itemize}
\item[] {\ttfamily PxUtil}
    \begin{itemize}
    \item lardata/Utilities in {\larsoft}
    \item core/LArUtil in {\larlight}
    \end{itemize}
\end{itemize}
These items are (or should be if not yet) described in a separate document.
The rest of this section briefly covers what they are without going into a technical detail.

\subsection{{\ttfamily ClusterRecoUtil}}
\label{sec:fmwk:external:cru}
Reconstruction parameters for EM shower 2D cluster involves a lot more variety of parameters than those held in a 2D cluster data product defined by a framework ({\larsoft} or {\larlight}). {\cru} is a package that defines code for both calculation of and storage for those parameters. In particular, {\cmtool} uses following {\CPP} classes from {\cru}:
\begin{itemize}
\item {\ttfamily ClusterParamsAlg ... (\cpan)}
  \begin{itemize}
    \item {\cpan} is designed as both parameter computation algorithm and storage class, and is used along this design principle in {\cmtool}. In particular, {\cpan} is created one per input 2D cluster.
  \end{itemize}
\item {\ttfamily cluster\_params}
  \begin{itemize}
    \item {\ttfamily cluster\_params} is a utility storage class for important parameters computed by \cpan. A const reference can be retrieved from {\cpan} instance through {\ttfamily GetParams()} function call. Note, however, one should keep {\cpan} instance rather than {\ttfamily cluster\_params} as a cluster parameter data set (see {\cpan} description above).
  \end{itemize}
\item {\ttfamily CRUException}
  \begin{itemize}
    \item This is a very simple exception class derived from {\ttfamily std::exception}. It is used instead of {\ttfamily cet::exception} to avoid framework dependency. Only additional feature to the base class is that the argument string to the constructor is sent to standard output streem by overriden {\ttfamily std::exception::what()} function.
  \end{itemize}
\end{itemize}

For details of all classes listed above, you should see Ref.\cite{ClusterRecoUtil}.

\subsection{{\ttfamily PxUtil}}
\label{sec:fmwk:external:pxutil}
Both {\cru} and {\cmtool} are designed to be independent of an analysis framework of user's choice (in particular {\larlight} and {\larsoft}). 
As mentioned in Sec.\ref{sec:fmwk:external:cru}, output of {\cpan} (i.e. cluster parameters) is stored with {\ttfamily cluster\_params} instance. 
{\pxutil}, on the other hand, is used to define input data struct to {\cpan} (and hence {\cmtool} and {\cru} in general). {\pxutil} defines three data objects:
\begin{itemize}
  \item {\ttfamily PxPoint}
    \begin{itemize}
    \item represents 2D point and contains wire [cm], time [cm], and plane ID (unsigned char)
    \end{itemize}
  \item {\ttfamily PxHit}
    \begin{itemize}
    \item inhertis from {\ttfamily PxPoint} with an additional attribute to hold charge [ADC], representing 2D hit
    \end{itemize}
  \item {\ttfamily PxLine}
    \begin{itemize}
    \item represents 2D line and contains start and end coordinates, (wire [cm], time [cm]), with a plane ID (unsigned char)
    \end{itemize}
\end{itemize}

It is a FMWK interface module's respnsibility to translate input data product into these format. In particular, this module must convert a vector of FMWK hit data product (= cluster) into a vector of {\pxhit}. This is discussed in FMWK interface chapter (see Ch.\ref{chap:fmwk_interface}).

\subsection{{\ttfamily GeometryUtilities}}
\label{sec:fmwk:external:geoutil}
{\CPP} class {\geoutil} is a utility class that helps to retrieve detector geometry related information such as converting 3D point coordinate into a projected 2D coordinate on a given plane. 
{\cmtool} uses {\geoutil} to interpret geometry related variables in full extent. 
For development of any non-existing function that treats detector geometry information, it is encouraged to develop in {\geoutil} (and certainly not in {\cmtool}). 
Please see Ref.\cite{GeometryUtilities} for more detailed documentation.


\section{FMWK Bulding Blocks}
\label{sec:fmwk:base}
In this section we discuss two core buliding block classes: (1) {\cbalgo} and (2) {\cbkeeper}. Though we discuss these basic {\CPP} classes first, it might be easier to understand these classes after you read about {\cmerge}, described in Sec.{\ref{sec:fmwk:cmerge}}, because there we discuss how {\cmerge} works with practical usage of {\cbalgo} and {\cbkeeper}.

\subsection{Book Keeping: {\cbkeeper}}
\label{sec:fmwk:base:cbk}
{\cbkeeper} inherits from {\ttfamily std::vector<unsigned short>}, and hence it is a vector of integers by itself.
The length of vector can be specified using a constructor argument.

This class is used to track how input clusters are merged/matched into an output. 
Here we focus on merging of input clusters for the simplicity sake.
The design assumes one to perform a merging attempt of 2 clusters at a time.
When one has many clusters to make such attempt on all possible pairs of clusters, it could be troublesome to perform book-keeping for which clusters are merged together.
{\cbkeeper} is designed such that a user can just log two to-be-merged clusters' ID at each attempt, and it holds a summary log of all merged cluster pairs.


Here are technical details.
The index of vector corresponds to the initial cluster ID numbers.
If you have 8 clusters to be subject for merging, {\cbkeeper} instance should be initialized as length 8.
Initially the values stored in the vector is same as their index numbers as shown in the top of Fig.\ref{sec:fmwk:base:cbk}.
One can re-set the state of {\cbkeeper} instance by calling the following function.
\begin{lstlisting}
        void Reset(unsigned short n=0)
\end{lstlisting}

\begin{figure}\begin{center}
\includegraphics[width=14cm]{./src/Pictures/CBookKeeperExample.pdf}
\caption{Example diagram showing how {\cbkeeper} keeps track of input and output cluster IDs. It shows {\cbkeeper} vector index (input cluster ID) and stored values (output cluster ID) for 8 input clusters (top), and how the stored values change as merging takes place. The middle shows the result of merging input cluster 1 and 2, and the bottom is for further merging 2 and 5.}
\end{center}\end{figure}
When one finds 2 clusters to be merged together, (s)he can call {\cbkeeper} attribute function:
\begin{lstlisting}
        void CBookKeeper::Merge(unsigned short, unsigned short)
\end{lstlisting}
with corresponding clusters' IDs (i.e. index numbers). 
Figure \ref{sec:fmwk:base:cbk} shows how the process works for calling {\ttfamily CBookKeeper::Merge()} function twice with arguments (1,2) and (2,5).
As a result, unique output cluster IDs are reduced to 6 because 3 clusters are combined together.

You can see {\cbkeeper} instance allows to retrieve which output cluster (stored value) an input cluster (vector index) ended up with in an intuitive way: simply use {\ttfamily std::vector::at()} method.
However, to create an output cluster, it is probably easier to ask the other way around: given output cluster ID, which input cluster(s) does it contain?
{\cbkeeper} has a simple function to answer this question:
\begin{lstlisting}
        std::vector<std::vector<unsigned short> > GetResult() const 
\end{lstlisting}
which returns a vector with length same as number of output (i.e. merged) clusters.
The value of this vector is {\ttfamily std::vector<unsigned short>} to contain possibly multiple input cluster indexes.
For instance, the return from {\cbkeeper} shown in the bottom of Fig.\ref{sec:fmwk:base:cbk} would be a vector of length 6, and the index1 contains a vector of input cluster ID 1, 2, and 5 as they have been merged into output cluster ID = 1.
Equivalent result can be obtained also by using
\begin{lstlisting}
        void PassResult(std::vector<std::vector<unsigned short> >&)
\end{lstlisting}
if that suits one's usage. 

If you are interested in checking which input cluster(s) included in a particular output (merged) cluster, you can use:
\begin{lstlisting}
        std::vector<unsigned short> GetMergedSet(unsigned short)
\end{lstlisting}
which takes an output cluster index as an argument and returns a vector that contains a list of input clusters that consist the output.

\begin{figure}[ht]\begin{center}
\includegraphics[width=8.3cm]{./src/Pictures/CBookKeeperProhibitExample2.pdf}
\includegraphics[width=8.3cm]{./src/Pictures/CBookKeeperProhibitExample1.pdf}
\caption{Example diagrams showing how {\cbkeeper} reacts to conflict {\ttfamily Merge()} and {\ttfamily ProhibitMerge()} function calls on the input cluster IDs. The left diagram shows {\crue} thrown due to calling {\ttfamily Merge()} on a cluster pair that is registered as merge-prohibit pair. The right diagram shows a result from calling {\ttfamily ProhibitMerge()} on a cluster pair that is already merged.}
\label{sec:fmwk:base:cbk_prohibit}
\end{center}\end{figure}

Finally, {\cmerge} can also prohibit merging of specific two cluster pair(s), and this is incoorporated by {\cbkeeper} as well.
This is, again, to support user to log merging in a mindless manner.
A user may specify specific 2 cluster pair should not be merged. 
Then (s)he can use the following function to register these 2 clusters for merge-prohibit list.
\begin{lstlisting}
        void ProhibitMerge(unsigned short, unsigned short)
\end{lstlisting}
The above function takes input cluster IDs (i.e. index numbers). 
Once 2 clusters are registered as merge-prohibit pair, attempting a function call {\ttfamily Merge()} will result in {\crue} thrown.
This is illustrated in the left of Fig.\ref{sec:fmwk:base:cbk_prohibit}.
Another possible case that results in {\crue} thrown would be calling {\ttfamily ProhibitMerge()} function on two input cluster IDs that are already merged. This is shown in the right hand side of Fig.\ref{sec:fmwk:base:cbk_prohibit}.

To avoid {\crue}, one should register all cluster pairs to prohibit-merge list before trying any merge attempt.
Upon attempting to call {\ttfamily Merge()}, then, one can use the function:
\begin{lstlisting}
        bool MergeAllowed(unsigned short, unsigned short)
\end{lstlisting}
to check if a cluster pair is registered to merge-prohibit list.
Calling this function prior to evaluation of merging (i.e. running some computation algorithm to decide whether to merge or not) may save some computation time and help the overall speed of the program (this is implemented in {\cmerge}).

\subsection{Algorithm Base: {\ttfamily CBoolAlgoBase}}
\label{sec:fmwk:base:cbalgo}
{\cbalgo} is a simple {\CPP} template class and also a base class for an algorithm used for cluster merging and matching by {\cmerge} and {\cmatch} application classes. In this and later sections, we use the term {\cmalgo} to refer to an algorithm class derived from {\cbalgo}.

\begin{figure}[ht]\begin{center}
\includegraphics[width=13cm]{./src/Pictures/CMergeStream.pdf}
\caption{The data processing stream of {\cmerge} framework. The purple allow shows the ordered process flow. It requires an analysis framework interface to convert framework dependent cluster data produdct into a vector of {\pxhit} and hand over to {\cmerge} application. Internally {\cmerge} calls various members of {\cbalgo} instances: one for merging and another for prohibiting. There is a configuration in {\cmerge} to perform iterative approach, which repeats internal loop over possible cluster pairs until no more newly merged cluster is found in the output. The merged cluster groups (i.e. output) can be obtained through {\cbkeeper} instance by an analysis framework interface to form an appropriate output data product.}
\label{sec:fmwk:base:cmerge_stream}
\end{center}\end{figure}

The idea is very simple: given two clusters, it is expected to return a boolean for decision making from the following function.
\begin{lstlisting}
        virtual bool Bool(const cluster::ClusterParamsAlg &, const cluster::ClusterParamsAlg &)
\end{lstlisting}
As you can see, ``clusters'' are handed over in terms of {\cpan} instance.
In {\cmerge}, the cluster merging application class, for instance, this is used for making merge decision as well as prohibit-merge decision (i.e. {\cmerge} holds two {\cbalgo} inherit class instance, one for merging and another for prohibiting).
An algorithm developpers should write his/her own {\CPP} algorithm class by inheriting from {\cbalgo}.
In the derived class, the {\ttfamily Bool()} function implementation should be over-riden. 
There is a script to generate a template (i.e. empty) {\cbalgo} inherit class and everyone is encouraged to use it for starting a new algorithm class. These are:
\begin{itemize}
\item For {\larsoft}
\begin{itemize}
        \item[]{\ttfamily> python \$MRB\_TOP/srcs/larreco/RecoAlg/CMTool/mac/gen\_cmalgo.py NAME}
\end{itemize}
\item For {\larlight}
\begin{itemize}
        \item[]{\ttfamily> python \$MAKE\_TOP\_DIR/UserDev/ClusterStudy/bin/gen\_cmalgo NAME}
\end{itemize}
\end{itemize}
where {\ttfamily NAME} should be replaced by the new algorithm's name. 
Running above command results in creating {\ttfamily CMAlgoNAME.h} and {\ttfamily CMAlgoNAME.cxx} in the appropriate directory.
Then you can proceed to build following the framework's guidance.

As mentioned, {\cbalgo} is operated by {\cmerge} and {\cmatch} applications.
For a given event, there is an internal loop of calling {\ttfamily Bool()} on all possible cluster pairs.
To provide flexibility in algorithm design, those applications also call other functions of {\cbalgo} in its processing.

The following is a list of such {\cbalgo} functions.
\begin{itemize}

\item[]{\ttfamily virtual void EventBegin(const std::vector<cluster::ClusterParamsAlg>\&)}
  \begin{itemize}
    \item Called at the beginning of an event, prior to looping over cluster pairs. The provided function argument is a vector of {\cpan} for all clusters subject for merging. This is where computation that requires the whole, initial cluster sets should be done.
  \end{itemize}

\item[]{\ttfamily virtual void IterationBegin(const std::vector<cluster::ClusterParamsAlg>\&)}
  \begin{itemize}
    \item Called at the beginning of an each loop over cluster pairs. Note {\cmerge} and {\cmatch} can have multiple iteration of looping over possible cluster pairs if configured to do so. For instance {\cmerge} can take iterate over multiple loops till no more to-be-merged  cluster pair is found (see Sec.\ref{sec:fmwk:cmerge} and Sec.\ref{sec:fmwk:cmatch} for details). The argument of this function is an output cluster sets, in terms of {\cpan} instance, from the previous loop. In case of the 1st loop, the provided cluster sets is identical to those provided in the argument of {\ttfamily EventBegin()} function.
  \end{itemize}

\item[]{\ttfamily virtual void IterationEnd()}
  \begin{itemize}
    \item Called at the end of each merge or match attempt loop over all cluster pairs, to clean up or summarize information.
  \end{itemize}

\item[]{\ttfamily virtual void EventEnd()}
  \begin{itemize}
    \item Called at the end of each event, after all iterations over cluster pairs are done.
  \end{itemize}

\end{itemize}

Further, there are two functions that are not yet implemented in {\cmerge} nor {\cmatch} but ideas exist.
Your input on how these should be implemented is welcome and very much appreciated.
\begin{itemize}

\item[]{\ttfamily virtual void Reset()}
  \begin{itemize}
    \item Meant to clear some member variables. But one can do that within {\ttfamily IterationEnd()} and/or {\ttfamily EventEnd()}, so it is not clear where this should be called. Possibly at the end of the whole event processing.
  \end{itemize}

\item[]{\ttfamily virtual void Report()}
  \begin{itemize}
    \item Meant to support some standardized {\ttfamily std::cout} and {\ttfamily std::cerr} report from an algorithm instance. {\cmerge} and {\cmatch} have ability to control such text output for different levels, and there is a hope this {\ttfamily Report()} can be also included there.
  \end{itemize}

\end{itemize}

\paragraph{Tips for writing {\cmalgo}}
First of all, you are encouraged to use {\ttfamily gen\_cmalgo} script as described earlier.
Another thing to note is to reduce a dependency on analysis framework.
This makes it easier to port your code between different frameworks.



\section{{\cmerge}: Cluster Merging Tool}
\label{sec:fmwk:cmerge}

{\cmerge} is a {\CPP} class that performs merging of 2D clusters, and is an application of {\cmalgo} algorithm classes that are inherit from {\cbalgo}, the algorithm base class. In a few sentences, {\cmerge} perform merging by looping over all possible 2D cluster pairs among the given set of clusters. For each pair, {\ttfamily Bool()} function of both merging and prohibit {\cmalgo} are called to inspect possible 2D cluster merging. It uses {\cbkeeper} to keep track of which pair of clusters are merged. For details of {\cbalgo} and {\cbkeeper}, see Sec.\ref{sec:fmwk:base:cbalgo} and Sec.\ref{sec:fmwk:base:cbk} respectively. The overall process stream diagram is shown in Fig.\ref{sec:fmwk:base:cmerge_stream} which we keep referring in this section.

\subsection{Why {\cmerge}?}
Why using {\cmerge}? Why not just writing a program that does everything from scratch?
Be my guest and write your code! That might inspire a better design or strategy toward merging. 
That being said, there are a few reasons to use {\cmerge}.

\paragraph{Task Outside Merging Decision:}
{\cmerge} The core part of merging is to compare 2 clusters and make a decision to merge or not. Outside this process, however, there is a non-negligible overburden of running such algorithm such as (1) keeping track of which cluster pairs are merged and (2) optimizing the loop process of cluster pair comparison. {\cmerge} does those tedious jobs while out-sourcing the decision making to an independent {\CPP} algorithm class that inherit from {\cbalgo}. 

\paragraph{Easy Development via Modulated Design:}
Because {\cmerge} outsource an independent algorithm class to perform merging of a cluster pair, it supports modulated code design. Accordingly each {\cmalgo} class becomes a pure algorithm class. This makes a code development very easy because a new algorithm can be written without re-writing the I/O or book-keeping part. Also it makes life easy to compare different algorithms: you can simple change which algorithm to be attached to {\cmerge} at run-time, and this avoids re-compilation or re-writing of code.

\paragraph{X-Framework Implementation}
{\cmerge} does not use data product type specific to an analysis framework (for instance {\larlight} or {\larsoft}), so it can be ported easily among different frameworks. That means {\cmalgo} developed by one person in one analysis framework can be shared among other frameworks (assuming an algorithm developer follows x-framework design pattern).\\

\subsection{How Does It Work?}
\label{sec:fmwk:cmerge:how}
The core part of decision making is simple: provided 2 clusters, decide whether they should be merged or not.
This is done through {\cmalgo::Bool()} function call:
\begin{lstlisting}
  virtual bool Bool( const cluster::ClusterParamsAlg &, const cluster::ClusterParamsAlg &)
\end{lstlisting}
If the return is {\ttfamily true}, two clusters are merged. Else not.
This is ilustrated in Fig.\ref{sec:fmwk:cmerge:merge_example}.

\begin{figure}[ht]\begin{center}
\includegraphics[width=13cm]{./src/Pictures/MergeExample.pdf}
\caption{Example of 2D clusters in a plane before and after merging attempt. Each oval represents a cluster with cluster 2D axis shown in a black solid arrow. The direction of an arrow represents the direction of a cluster development. Each cluster is labeled with an input cluster ID number while the oval color represents distinct output cluster group. Left: initial state with 4 input clusters. Right: example output after running a merge loop over cluster pairs. }
\label{sec:fmwk:cmerge:merge_example}
\end{center}\end{figure}
{\cmerge} takes care of creating a loop over all possible pairs of 2D clusters and calling the above function of {\cmalgo} for merging.
In the process, book-keeping of which pairs to be merged is recorded using {\cbkeeper}.

\begin{figure}[ht]\begin{center}
\includegraphics[width=13cm]{./src/Pictures/MergeProhibitExample.pdf}
\caption{Example of how prohibit algorithm works. Each oval and solid arrow represents a 2D cluster with its direction while the color represents distinct output (merged) cluster. Input cluster index (ID) is shown in the label with the cluster 2D angle. Right-top shows an output of running {\cmalgo} that merge clusters with angle difference within 40 degrees. Right-bottom shows an output of the same {\cmalgo} for merging but with another {\cmalgo} that prohibits merging of clusters with angle difference larger than 40 degrees.}
\label{sec:fmwk:cmerge:merge_prohibit_example}
\end{center}\end{figure}
Let us consider a case of two merged pairs that shares the same input cluster ID. As described in Sec.\ref{sec:fmwk:base:cbk}, {\cbkeeper} combines two pairs of 2D clusters in such case (see Fig.\ref{sec:fmwk:base:cbk}). This could be problematic sometimes. For instance, consider an example {\cmalgo} that combines two clusters if their start points are very close and the angle between two clusters are within 40 degrees. Further assume we have 3 input clusters that share the start point but have different angles at 20, 40, and -20 degrees. For convention these have input cluster ID 0, 1, and 2 respectively. This example algorithm would conclude pairs (0,1) and (0,2) but not for (1,2). But because merged pairs share cluster ID 0, it ends up all 3 clusters being merged. This is shown in the right-top ilustration of Fig.\ref{sec:fmwk:cmerge:merge_prohibit_example}.

This could be a problem, or may not be: this can be only answered by the algorithm author.
The way to avoid this in {\cmerge} is to implement another {\cmalgo} as {\it merge-prohibit}, or simple {\it prohibit} algorithm.
When prohibit algorithm is provided, {\cmerge} first run over all cluster pairs and execute {\ttfamily Bool()} function to find all cluster pairs that are prohibited to merge.
Then {\cmerge} run over the same set of cluster pairs except for those are prohibited to be merged.
The right-bottom ilustration of Fig.\ref{sec:fmwk:cmerge:merge_prohibit_example} shows how prohibit algorithm affects the result.

The prohibit rule is also inherited to merged cluster pairs. For instance, consider input clusters 0, 1 and 2. The pair (1,2) is prohibited to merge, and (0,1) is found to be merged. Then the merged cluster, (0,1) cannot be merged with 2. That means even if a merge algorithm finds (0,2) to be merged, it cannot be because the prohibit rule is inherited from cluster 1 when 0 and 1 aremerged. This feature brings up a question: what determines 0 to be merged with 1 rather than 2? If the merge algorithm is executed on the pair (0,2) first, then the result would have been different. So what determines a different outcome is an order of cluster pairs in the loop of merging inspection. There are different ordering options available in {\cmerge}. Currently available options are defined as {\enum} type in {\cmerge} class, {\ttfamily CMergePriority\_t} which include following values.
\begin{itemize}
\item[] {\ttfamily kIndex}
  \begin{itemize}
    \item From the smaller to larger input cluster ID, basically following input cluster set ordering
  \end{itemize}

\item[] {\ttfamily kPolyArea}
  \begin{itemize}
    \item From the clusters with smaller polygon area to the larger
  \end{itemize}

\item[] {\ttfamily kChargeSum}
  \begin{itemize}
    \item From the clusters with smaller charge sum to the larrger
  \end{itemize}

\item[] {\ttfamily kNHits}
  \begin{itemize}
    \item From the clusters with smaller number of hits to the larger
  \end{itemize}

\end{itemize}
The ordering parameters including all listed above are taken from {\ttfamily cluster\_params} member attributes.
If more useful parameter is found, one should make a request to make it available. 
It is very simple to implement in {\cmerge}.

In summary, {\cmerge} uses one {\cmalgo} instance for merging and another for merge-prohibit. A possible ambiguity of which cluster pairs to be merged in the presence of prohibited pairs is resolved by defining a priority for cluster pairs in the merging attempt. 

\subsection{Process Flow for Cluster Merging}
\label{sec:fmwk:cmerge:process_flow}
An overall process flow diagram is already shown in Fig.\ref{sec:fmwk:base:cmerge_stream}, so here we describe slightly more details in texts by breaking down into 5 blocks.
\begin{enumerate}
\item Event Preparation
  \begin{itemize}
    \item[] {\cmerge} receives an input cluster in terms of a set of {\pxhit} (see Sec.\ref{sec:fmwk:cmerge:input}). Then it creates {\cpan} instance for each cluster, and calls {\cmalgo::EventBegin()} for merging and prohibit algorithms.
  \end{itemize}

\item Iterative Merge Operation
  \begin{itemize}
    \item[] A loop over cluster pairs for prohibit (if provided) and merge {\cmalgo} can be configured to repeat until no newly merged cluster is found. This is shown in Fig.\ref{sec:fmwk:cmerge:iterative_stream}. If a user does not configure to choose iterative approach, iteration will not happen even if there is a newly merged cluster in the output.
  \end{itemize}

\item Prohibit {\cmalgo} Loop Over Cluster Pairs
  \begin{itemize}
    \item[] See Fig.\ref{sec:fmwk:cmerge:iterative_stream}. Only occurs if prohibit algorithm is provided by a user (optional). Records which cluster pair to be prohibited for merging using {\cbkeeper}.
  \end{itemize}

\item Prohibit {\cmalgo} Loop Over Cluster Pairs
  \begin{itemize}
    \item[] See Fig.\ref{sec:fmwk:cmerge:iterative_stream}. The algorithm's {\ttfamily Bool()} is not executed for a subject pair of clusters if (a) they are prohibited for merging or (b) already inspected in the previous process and not merged (i.e. result should not change). This is to save computation time.
  \end{itemize}

\item Merge Clusters
  \begin{itemize}
    \item[] Based on the outcome of merge/prohibit {\cmalgo} algorithms, actual merging is performed at this step. It creates a new set of {\cpan} for merged clusters. For clusters that are not merged, {\cpan} instance is copied from input to save {\cpan} computation time.
  \end{itemize}

\end{enumerate}

\begin{figure}[ht]\begin{center}
\includegraphics[width=13cm]{./src/Pictures/IterativeMergeApproach.pdf}
\caption{Illustration of how iterative merge approach is processed. Internally a loop over cluster pairs for prohibit and merge algorithm are repeated if (a) user configured to use an iterative approach and (b) newly merged cluster is found.}
\label{sec:fmwk:cmerge:iterative_stream}
\end{center}\end{figure}


\subsection{Setting an Input 2D Clusters}
\label{sec:fmwk:cmerge:input}
Input to {\cmerge} is a set of clusters where each cluster is represented as a vector of hits, or more precisely {\ttfamily std::vector<util::PxHit>}. This input can be set by the following function:
\begin{lstlisting}
  void SetClusters(const std::vector<std::vector<util::PxHit> > &)
\end{lstlisting}
where the argument is a vector of 2D clusters. When this function is called, {\cmerge} internally calls
\begin{lstlisting}
  void Reset()
\end{lstlisting}
which initialize previously stored input clusters, output clusters and {\cbkeeper} instance.
Then it creates {\cpan} instance per input 2D cluster using a set of {\pxhit}.
This completes setting (or initiating) the input 2D cluster sets to be used for internal processing.

\subsection{Setting Merge/Prohibit {\cmalgo} Instances}
{\cmalgo} instance is necessary for merging and optional for prohibit. Call the function,
\begin{lstlisting}
  void AddMergeAlg(CBoolAlgoBase* algo)
\end{lstlisting}
to set {\cmalgo} for merging, and 
\begin{lstlisting}
  void AddSeparateAlgo(CBoolAlgoBase* algo)
\end{lstlisting}
to set {\cmalgo} for merge-prohibit.

\subsection{Setting Run Configuration}
There are only 3 run options to configure for {\cmerge}.

\paragraph{Verbosity:} 
There are 5 verbosity levels a user can configure as listed below:
\begin{itemize}
\item[] {\ttfamily kNone}
  \begin{itemize}
    \item No report from {\cmerge}
  \end{itemize}

\item[] {\ttfamily kPerEvent}
  \begin{itemize}
    \item Report made at the end of even processing
  \end{itemize}

\item[] {\ttfamily kPerIteration}
  \begin{itemize}
    \item Report made per merge iteration (see Sec.\ref{sec:fmwk:cmerge:process_flow}). This could be quite verbose.      
  \end{itemize}

\item[] {\ttfamily kPerMerging}
  \begin{itemize}
    \item Report made per {\cmalgo::Bool()} execution (see Sec.\ref{sec:fmwk:cmerge:process_flow}). This is even more verbose than {\ttfamily kPerIteration}.
  \end{itemize}

\end{itemize}
These are {\enum} type, {\ttfamily CMergeMSGLevel\_t}, defined under {\cmerge} class. 
The verbosity level can be set by the function:
\begin{lstlisting}
  void DebugMode(CMergeMSGLevel_t level)
\end{lstlisting}

\paragraph{Iterative Approach:}
As described in Sec.\ref{sec:fmwk:cmerge:process_flow}, {\cmerge} can be configured to attempt iterative approach for merging. This can be enabled by the following function call.
\begin{lstlisting}
  void MergeTillConverge(bool)
\end{lstlisting}
Providing {\ttfamily true} as an argument will enable this option. Disabled if {\ttfamily false} is provided.

\paragraph{Merge Priority:}
As described in Sec.\ref{sec:fmwk:cmerge:how}, ordering of cluster pairs for merging inspection matters to the result.
Options are also described in the same section. This can be set by the following function.
\begin{lstlisting}
  void SetMergePriority(CMergePriority_t level)
\end{lstlisting}

\subsection{Execute Merging}
To execute merging, simply call
\begin{lstlisting}
  void Process()
\end{lstlisting}

\subsection{Obtaining Result}
There are two kinds of results that may be useful for a user. First is a resulting (i.e. merged) set of clusters in terms of {\cpan}. This is useful for further processing as various cluster parameters are already computed as a part of {\cpan}. You can execute
\begin{lstlisting}
  const std::vector<cluster::ClusterParamsAlg>& GetClusters() const
\end{lstlisting}
to obtain such result set.

Second is a mapping of input cluster IDs (i.e. index) to the output (merged clusters). 
As described in Sec.\ref{sec:fmwk:base:cbk}, this can be easily obtained through {\cbkeeper} instance.
You can obtain the {\cbkeeper} instance held by {\cmerge} through the following function.
\begin{lstlisting}
  const CBookKeeper& GetBookKeeper() const
\end{lstlisting}


\section{{\cmatch}: Cluster Matching Tool}
\label{sec:fmwk:cmatch}
Tool not introduced yet in {\larsoft}. Will be documented when this is done.




\chapter{Releases}
\label{chap:releases}

Despite its extreme simplicity let us call {\cmtool} a framework, or FMWK. 
In this chapter we discuss the building blocks of {\cmtool}.

\section{External Dependency}
\label{sec:fmwk:external}

The {\cmtool} development effort is originated from EM shower reconstruction for MicroBooNE experiment, and it is an extension of existing tool set. The following is the brief list of packages or {\CPP} classes on which {\cmtool} depends.
\begin{itemize}
\item[] {\cru}
    \begin{itemize}
    \item larreco/RecoAlg/ClusterRecoUtil in {\larsoft}
    \item UserDev/ClusterRecoUtil in {\larlight}
    \end{itemize}
\item[] {\geoutil}
    \begin{itemize}
    \item lardata/Utilities in {\larsoft}
    \item core/LArUtil in {\larlight}
    \end{itemize}
\item[] {\ttfamily PxUtil}
    \begin{itemize}
    \item lardata/Utilities in {\larsoft}
    \item core/LArUtil in {\larlight}
    \end{itemize}
\end{itemize}
These items are (or should be if not yet) described in a separate document.
The rest of this section briefly covers what they are without going into a technical detail.

\subsection{{\ttfamily ClusterRecoUtil}}
\label{sec:fmwk:external:cru}
Reconstruction parameters for EM shower 2D cluster involves a lot more variety of parameters than those held in a 2D cluster data product defined by a framework ({\larsoft} or {\larlight}). {\cru} is a package that defines code for both calculation of and storage for those parameters. In particular, {\cmtool} uses following {\CPP} classes from {\cru}:
\begin{itemize}
\item {\ttfamily ClusterParamsAlg ... (\cpan)}
  \begin{itemize}
    \item {\cpan} is designed as both parameter computation algorithm and storage class, and is used along this design principle in {\cmtool}. In particular, {\cpan} is created one per input 2D cluster.
  \end{itemize}
\item {\ttfamily cluster\_params}
  \begin{itemize}
    \item {\ttfamily cluster\_params} is a utility storage class for important parameters computed by \cpan. A const reference can be retrieved from {\cpan} instance through {\ttfamily GetParams()} function call. Note, however, one should keep {\cpan} instance rather than {\ttfamily cluster\_params} as a cluster parameter data set (see {\cpan} description above).
  \end{itemize}
\item {\ttfamily CRUException}
  \begin{itemize}
    \item This is a very simple exception class derived from {\ttfamily std::exception}. It is used instead of {\ttfamily cet::exception} to avoid framework dependency. Only additional feature to the base class is that the argument string to the constructor is sent to standard output streem by overriden {\ttfamily std::exception::what()} function.
  \end{itemize}
\end{itemize}

For details of all classes listed above, you should see Ref.\cite{ClusterRecoUtil}.

\subsection{{\ttfamily PxUtil}}
\label{sec:fmwk:external:pxutil}
Both {\cru} and {\cmtool} are designed to be independent of an analysis framework of user's choice (in particular {\larlight} and {\larsoft}). 
As mentioned in Sec.\ref{sec:fmwk:external:cru}, output of {\cpan} (i.e. cluster parameters) is stored with {\ttfamily cluster\_params} instance. 
{\pxutil}, on the other hand, is used to define input data struct to {\cpan} (and hence {\cmtool} and {\cru} in general). {\pxutil} defines three data objects:
\begin{itemize}
  \item {\ttfamily PxPoint}
    \begin{itemize}
    \item represents 2D point and contains wire [cm], time [cm], and plane ID (unsigned char)
    \end{itemize}
  \item {\ttfamily PxHit}
    \begin{itemize}
    \item inhertis from {\ttfamily PxPoint} with an additional attribute to hold charge [ADC], representing 2D hit
    \end{itemize}
  \item {\ttfamily PxLine}
    \begin{itemize}
    \item represents 2D line and contains start and end coordinates, (wire [cm], time [cm]), with a plane ID (unsigned char)
    \end{itemize}
\end{itemize}

It is a FMWK interface module's respnsibility to translate input data product into these format. In particular, this module must convert a vector of FMWK hit data product (= cluster) into a vector of {\pxhit}. This is discussed in FMWK interface chapter (see Ch.\ref{chap:fmwk_interface}).

\subsection{{\ttfamily GeometryUtilities}}
\label{sec:fmwk:external:geoutil}
{\CPP} class {\geoutil} is a utility class that helps to retrieve detector geometry related information such as converting 3D point coordinate into a projected 2D coordinate on a given plane. 
{\cmtool} uses {\geoutil} to interpret geometry related variables in full extent. 
For development of any non-existing function that treats detector geometry information, it is encouraged to develop in {\geoutil} (and certainly not in {\cmtool}). 
Please see Ref.\cite{GeometryUtilities} for more detailed documentation.


\section{FMWK Bulding Blocks}
\label{sec:fmwk:base}
In this section we discuss two core buliding block classes: (1) {\cbalgo} and (2) {\cbkeeper}. Though we discuss these basic {\CPP} classes first, it might be easier to understand these classes after you read about {\cmerge}, described in Sec.{\ref{sec:fmwk:cmerge}}, because there we discuss how {\cmerge} works with practical usage of {\cbalgo} and {\cbkeeper}.

\subsection{Book Keeping: {\cbkeeper}}
\label{sec:fmwk:base:cbk}
{\cbkeeper} inherits from {\ttfamily std::vector<unsigned short>}, and hence it is a vector of integers by itself.
The length of vector can be specified using a constructor argument.

This class is used to track how input clusters are merged/matched into an output. 
Here we focus on merging of input clusters for the simplicity sake.
The design assumes one to perform a merging attempt of 2 clusters at a time.
When one has many clusters to make such attempt on all possible pairs of clusters, it could be troublesome to perform book-keeping for which clusters are merged together.
{\cbkeeper} is designed such that a user can just log two to-be-merged clusters' ID at each attempt, and it holds a summary log of all merged cluster pairs.


Here are technical details.
The index of vector corresponds to the initial cluster ID numbers.
If you have 8 clusters to be subject for merging, {\cbkeeper} instance should be initialized as length 8.
Initially the values stored in the vector is same as their index numbers as shown in the top of Fig.\ref{sec:fmwk:base:cbk}.
One can re-set the state of {\cbkeeper} instance by calling the following function.
\begin{lstlisting}
        void Reset(unsigned short n=0)
\end{lstlisting}

\begin{figure}\begin{center}
\includegraphics[width=14cm]{./src/Pictures/CBookKeeperExample.pdf}
\caption{Example diagram showing how {\cbkeeper} keeps track of input and output cluster IDs. It shows {\cbkeeper} vector index (input cluster ID) and stored values (output cluster ID) for 8 input clusters (top), and how the stored values change as merging takes place. The middle shows the result of merging input cluster 1 and 2, and the bottom is for further merging 2 and 5.}
\end{center}\end{figure}
When one finds 2 clusters to be merged together, (s)he can call {\cbkeeper} attribute function:
\begin{lstlisting}
        void CBookKeeper::Merge(unsigned short, unsigned short)
\end{lstlisting}
with corresponding clusters' IDs (i.e. index numbers). 
Figure \ref{sec:fmwk:base:cbk} shows how the process works for calling {\ttfamily CBookKeeper::Merge()} function twice with arguments (1,2) and (2,5).
As a result, unique output cluster IDs are reduced to 6 because 3 clusters are combined together.

You can see {\cbkeeper} instance allows to retrieve which output cluster (stored value) an input cluster (vector index) ended up with in an intuitive way: simply use {\ttfamily std::vector::at()} method.
However, to create an output cluster, it is probably easier to ask the other way around: given output cluster ID, which input cluster(s) does it contain?
{\cbkeeper} has a simple function to answer this question:
\begin{lstlisting}
        std::vector<std::vector<unsigned short> > GetResult() const 
\end{lstlisting}
which returns a vector with length same as number of output (i.e. merged) clusters.
The value of this vector is {\ttfamily std::vector<unsigned short>} to contain possibly multiple input cluster indexes.
For instance, the return from {\cbkeeper} shown in the bottom of Fig.\ref{sec:fmwk:base:cbk} would be a vector of length 6, and the index1 contains a vector of input cluster ID 1, 2, and 5 as they have been merged into output cluster ID = 1.
Equivalent result can be obtained also by using
\begin{lstlisting}
        void PassResult(std::vector<std::vector<unsigned short> >&)
\end{lstlisting}
if that suits one's usage. 

If you are interested in checking which input cluster(s) included in a particular output (merged) cluster, you can use:
\begin{lstlisting}
        std::vector<unsigned short> GetMergedSet(unsigned short)
\end{lstlisting}
which takes an output cluster index as an argument and returns a vector that contains a list of input clusters that consist the output.

\begin{figure}[ht]\begin{center}
\includegraphics[width=8.3cm]{./src/Pictures/CBookKeeperProhibitExample2.pdf}
\includegraphics[width=8.3cm]{./src/Pictures/CBookKeeperProhibitExample1.pdf}
\caption{Example diagrams showing how {\cbkeeper} reacts to conflict {\ttfamily Merge()} and {\ttfamily ProhibitMerge()} function calls on the input cluster IDs. The left diagram shows {\crue} thrown due to calling {\ttfamily Merge()} on a cluster pair that is registered as merge-prohibit pair. The right diagram shows a result from calling {\ttfamily ProhibitMerge()} on a cluster pair that is already merged.}
\label{sec:fmwk:base:cbk_prohibit}
\end{center}\end{figure}

Finally, {\cmerge} can also prohibit merging of specific two cluster pair(s), and this is incoorporated by {\cbkeeper} as well.
This is, again, to support user to log merging in a mindless manner.
A user may specify specific 2 cluster pair should not be merged. 
Then (s)he can use the following function to register these 2 clusters for merge-prohibit list.
\begin{lstlisting}
        void ProhibitMerge(unsigned short, unsigned short)
\end{lstlisting}
The above function takes input cluster IDs (i.e. index numbers). 
Once 2 clusters are registered as merge-prohibit pair, attempting a function call {\ttfamily Merge()} will result in {\crue} thrown.
This is illustrated in the left of Fig.\ref{sec:fmwk:base:cbk_prohibit}.
Another possible case that results in {\crue} thrown would be calling {\ttfamily ProhibitMerge()} function on two input cluster IDs that are already merged. This is shown in the right hand side of Fig.\ref{sec:fmwk:base:cbk_prohibit}.

To avoid {\crue}, one should register all cluster pairs to prohibit-merge list before trying any merge attempt.
Upon attempting to call {\ttfamily Merge()}, then, one can use the function:
\begin{lstlisting}
        bool MergeAllowed(unsigned short, unsigned short)
\end{lstlisting}
to check if a cluster pair is registered to merge-prohibit list.
Calling this function prior to evaluation of merging (i.e. running some computation algorithm to decide whether to merge or not) may save some computation time and help the overall speed of the program (this is implemented in {\cmerge}).

\subsection{Algorithm Base: {\ttfamily CBoolAlgoBase}}
\label{sec:fmwk:base:cbalgo}
{\cbalgo} is a simple {\CPP} template class and also a base class for an algorithm used for cluster merging and matching by {\cmerge} and {\cmatch} application classes. In this and later sections, we use the term {\cmalgo} to refer to an algorithm class derived from {\cbalgo}.

\begin{figure}[ht]\begin{center}
\includegraphics[width=13cm]{./src/Pictures/CMergeStream.pdf}
\caption{The data processing stream of {\cmerge} framework. The purple allow shows the ordered process flow. It requires an analysis framework interface to convert framework dependent cluster data produdct into a vector of {\pxhit} and hand over to {\cmerge} application. Internally {\cmerge} calls various members of {\cbalgo} instances: one for merging and another for prohibiting. There is a configuration in {\cmerge} to perform iterative approach, which repeats internal loop over possible cluster pairs until no more newly merged cluster is found in the output. The merged cluster groups (i.e. output) can be obtained through {\cbkeeper} instance by an analysis framework interface to form an appropriate output data product.}
\label{sec:fmwk:base:cmerge_stream}
\end{center}\end{figure}

The idea is very simple: given two clusters, it is expected to return a boolean for decision making from the following function.
\begin{lstlisting}
        virtual bool Bool(const cluster::ClusterParamsAlg &, const cluster::ClusterParamsAlg &)
\end{lstlisting}
As you can see, ``clusters'' are handed over in terms of {\cpan} instance.
In {\cmerge}, the cluster merging application class, for instance, this is used for making merge decision as well as prohibit-merge decision (i.e. {\cmerge} holds two {\cbalgo} inherit class instance, one for merging and another for prohibiting).
An algorithm developpers should write his/her own {\CPP} algorithm class by inheriting from {\cbalgo}.
In the derived class, the {\ttfamily Bool()} function implementation should be over-riden. 
There is a script to generate a template (i.e. empty) {\cbalgo} inherit class and everyone is encouraged to use it for starting a new algorithm class. These are:
\begin{itemize}
\item For {\larsoft}
\begin{itemize}
        \item[]{\ttfamily> python \$MRB\_TOP/srcs/larreco/RecoAlg/CMTool/mac/gen\_cmalgo.py NAME}
\end{itemize}
\item For {\larlight}
\begin{itemize}
        \item[]{\ttfamily> python \$MAKE\_TOP\_DIR/UserDev/ClusterStudy/bin/gen\_cmalgo NAME}
\end{itemize}
\end{itemize}
where {\ttfamily NAME} should be replaced by the new algorithm's name. 
Running above command results in creating {\ttfamily CMAlgoNAME.h} and {\ttfamily CMAlgoNAME.cxx} in the appropriate directory.
Then you can proceed to build following the framework's guidance.

As mentioned, {\cbalgo} is operated by {\cmerge} and {\cmatch} applications.
For a given event, there is an internal loop of calling {\ttfamily Bool()} on all possible cluster pairs.
To provide flexibility in algorithm design, those applications also call other functions of {\cbalgo} in its processing.

The following is a list of such {\cbalgo} functions.
\begin{itemize}

\item[]{\ttfamily virtual void EventBegin(const std::vector<cluster::ClusterParamsAlg>\&)}
  \begin{itemize}
    \item Called at the beginning of an event, prior to looping over cluster pairs. The provided function argument is a vector of {\cpan} for all clusters subject for merging. This is where computation that requires the whole, initial cluster sets should be done.
  \end{itemize}

\item[]{\ttfamily virtual void IterationBegin(const std::vector<cluster::ClusterParamsAlg>\&)}
  \begin{itemize}
    \item Called at the beginning of an each loop over cluster pairs. Note {\cmerge} and {\cmatch} can have multiple iteration of looping over possible cluster pairs if configured to do so. For instance {\cmerge} can take iterate over multiple loops till no more to-be-merged  cluster pair is found (see Sec.\ref{sec:fmwk:cmerge} and Sec.\ref{sec:fmwk:cmatch} for details). The argument of this function is an output cluster sets, in terms of {\cpan} instance, from the previous loop. In case of the 1st loop, the provided cluster sets is identical to those provided in the argument of {\ttfamily EventBegin()} function.
  \end{itemize}

\item[]{\ttfamily virtual void IterationEnd()}
  \begin{itemize}
    \item Called at the end of each merge or match attempt loop over all cluster pairs, to clean up or summarize information.
  \end{itemize}

\item[]{\ttfamily virtual void EventEnd()}
  \begin{itemize}
    \item Called at the end of each event, after all iterations over cluster pairs are done.
  \end{itemize}

\end{itemize}

Further, there are two functions that are not yet implemented in {\cmerge} nor {\cmatch} but ideas exist.
Your input on how these should be implemented is welcome and very much appreciated.
\begin{itemize}

\item[]{\ttfamily virtual void Reset()}
  \begin{itemize}
    \item Meant to clear some member variables. But one can do that within {\ttfamily IterationEnd()} and/or {\ttfamily EventEnd()}, so it is not clear where this should be called. Possibly at the end of the whole event processing.
  \end{itemize}

\item[]{\ttfamily virtual void Report()}
  \begin{itemize}
    \item Meant to support some standardized {\ttfamily std::cout} and {\ttfamily std::cerr} report from an algorithm instance. {\cmerge} and {\cmatch} have ability to control such text output for different levels, and there is a hope this {\ttfamily Report()} can be also included there.
  \end{itemize}

\end{itemize}

\paragraph{Tips for writing {\cmalgo}}
First of all, you are encouraged to use {\ttfamily gen\_cmalgo} script as described earlier.
Another thing to note is to reduce a dependency on analysis framework.
This makes it easier to port your code between different frameworks.



\section{{\cmerge}: Cluster Merging Tool}
\label{sec:fmwk:cmerge}

{\cmerge} is a {\CPP} class that performs merging of 2D clusters, and is an application of {\cmalgo} algorithm classes that are inherit from {\cbalgo}, the algorithm base class. In a few sentences, {\cmerge} perform merging by looping over all possible 2D cluster pairs among the given set of clusters. For each pair, {\ttfamily Bool()} function of both merging and prohibit {\cmalgo} are called to inspect possible 2D cluster merging. It uses {\cbkeeper} to keep track of which pair of clusters are merged. For details of {\cbalgo} and {\cbkeeper}, see Sec.\ref{sec:fmwk:base:cbalgo} and Sec.\ref{sec:fmwk:base:cbk} respectively. The overall process stream diagram is shown in Fig.\ref{sec:fmwk:base:cmerge_stream} which we keep referring in this section.

\subsection{Why {\cmerge}?}
Why using {\cmerge}? Why not just writing a program that does everything from scratch?
Be my guest and write your code! That might inspire a better design or strategy toward merging. 
That being said, there are a few reasons to use {\cmerge}.

\paragraph{Task Outside Merging Decision:}
{\cmerge} The core part of merging is to compare 2 clusters and make a decision to merge or not. Outside this process, however, there is a non-negligible overburden of running such algorithm such as (1) keeping track of which cluster pairs are merged and (2) optimizing the loop process of cluster pair comparison. {\cmerge} does those tedious jobs while out-sourcing the decision making to an independent {\CPP} algorithm class that inherit from {\cbalgo}. 

\paragraph{Easy Development via Modulated Design:}
Because {\cmerge} outsource an independent algorithm class to perform merging of a cluster pair, it supports modulated code design. Accordingly each {\cmalgo} class becomes a pure algorithm class. This makes a code development very easy because a new algorithm can be written without re-writing the I/O or book-keeping part. Also it makes life easy to compare different algorithms: you can simple change which algorithm to be attached to {\cmerge} at run-time, and this avoids re-compilation or re-writing of code.

\paragraph{X-Framework Implementation}
{\cmerge} does not use data product type specific to an analysis framework (for instance {\larlight} or {\larsoft}), so it can be ported easily among different frameworks. That means {\cmalgo} developed by one person in one analysis framework can be shared among other frameworks (assuming an algorithm developer follows x-framework design pattern).\\

\subsection{How Does It Work?}
\label{sec:fmwk:cmerge:how}
The core part of decision making is simple: provided 2 clusters, decide whether they should be merged or not.
This is done through {\cmalgo::Bool()} function call:
\begin{lstlisting}
  virtual bool Bool( const cluster::ClusterParamsAlg &, const cluster::ClusterParamsAlg &)
\end{lstlisting}
If the return is {\ttfamily true}, two clusters are merged. Else not.
This is ilustrated in Fig.\ref{sec:fmwk:cmerge:merge_example}.

\begin{figure}[ht]\begin{center}
\includegraphics[width=13cm]{./src/Pictures/MergeExample.pdf}
\caption{Example of 2D clusters in a plane before and after merging attempt. Each oval represents a cluster with cluster 2D axis shown in a black solid arrow. The direction of an arrow represents the direction of a cluster development. Each cluster is labeled with an input cluster ID number while the oval color represents distinct output cluster group. Left: initial state with 4 input clusters. Right: example output after running a merge loop over cluster pairs. }
\label{sec:fmwk:cmerge:merge_example}
\end{center}\end{figure}
{\cmerge} takes care of creating a loop over all possible pairs of 2D clusters and calling the above function of {\cmalgo} for merging.
In the process, book-keeping of which pairs to be merged is recorded using {\cbkeeper}.

\begin{figure}[ht]\begin{center}
\includegraphics[width=13cm]{./src/Pictures/MergeProhibitExample.pdf}
\caption{Example of how prohibit algorithm works. Each oval and solid arrow represents a 2D cluster with its direction while the color represents distinct output (merged) cluster. Input cluster index (ID) is shown in the label with the cluster 2D angle. Right-top shows an output of running {\cmalgo} that merge clusters with angle difference within 40 degrees. Right-bottom shows an output of the same {\cmalgo} for merging but with another {\cmalgo} that prohibits merging of clusters with angle difference larger than 40 degrees.}
\label{sec:fmwk:cmerge:merge_prohibit_example}
\end{center}\end{figure}
Let us consider a case of two merged pairs that shares the same input cluster ID. As described in Sec.\ref{sec:fmwk:base:cbk}, {\cbkeeper} combines two pairs of 2D clusters in such case (see Fig.\ref{sec:fmwk:base:cbk}). This could be problematic sometimes. For instance, consider an example {\cmalgo} that combines two clusters if their start points are very close and the angle between two clusters are within 40 degrees. Further assume we have 3 input clusters that share the start point but have different angles at 20, 40, and -20 degrees. For convention these have input cluster ID 0, 1, and 2 respectively. This example algorithm would conclude pairs (0,1) and (0,2) but not for (1,2). But because merged pairs share cluster ID 0, it ends up all 3 clusters being merged. This is shown in the right-top ilustration of Fig.\ref{sec:fmwk:cmerge:merge_prohibit_example}.

This could be a problem, or may not be: this can be only answered by the algorithm author.
The way to avoid this in {\cmerge} is to implement another {\cmalgo} as {\it merge-prohibit}, or simple {\it prohibit} algorithm.
When prohibit algorithm is provided, {\cmerge} first run over all cluster pairs and execute {\ttfamily Bool()} function to find all cluster pairs that are prohibited to merge.
Then {\cmerge} run over the same set of cluster pairs except for those are prohibited to be merged.
The right-bottom ilustration of Fig.\ref{sec:fmwk:cmerge:merge_prohibit_example} shows how prohibit algorithm affects the result.

The prohibit rule is also inherited to merged cluster pairs. For instance, consider input clusters 0, 1 and 2. The pair (1,2) is prohibited to merge, and (0,1) is found to be merged. Then the merged cluster, (0,1) cannot be merged with 2. That means even if a merge algorithm finds (0,2) to be merged, it cannot be because the prohibit rule is inherited from cluster 1 when 0 and 1 aremerged. This feature brings up a question: what determines 0 to be merged with 1 rather than 2? If the merge algorithm is executed on the pair (0,2) first, then the result would have been different. So what determines a different outcome is an order of cluster pairs in the loop of merging inspection. There are different ordering options available in {\cmerge}. Currently available options are defined as {\enum} type in {\cmerge} class, {\ttfamily CMergePriority\_t} which include following values.
\begin{itemize}
\item[] {\ttfamily kIndex}
  \begin{itemize}
    \item From the smaller to larger input cluster ID, basically following input cluster set ordering
  \end{itemize}

\item[] {\ttfamily kPolyArea}
  \begin{itemize}
    \item From the clusters with smaller polygon area to the larger
  \end{itemize}

\item[] {\ttfamily kChargeSum}
  \begin{itemize}
    \item From the clusters with smaller charge sum to the larrger
  \end{itemize}

\item[] {\ttfamily kNHits}
  \begin{itemize}
    \item From the clusters with smaller number of hits to the larger
  \end{itemize}

\end{itemize}
The ordering parameters including all listed above are taken from {\ttfamily cluster\_params} member attributes.
If more useful parameter is found, one should make a request to make it available. 
It is very simple to implement in {\cmerge}.

In summary, {\cmerge} uses one {\cmalgo} instance for merging and another for merge-prohibit. A possible ambiguity of which cluster pairs to be merged in the presence of prohibited pairs is resolved by defining a priority for cluster pairs in the merging attempt. 

\subsection{Process Flow for Cluster Merging}
\label{sec:fmwk:cmerge:process_flow}
An overall process flow diagram is already shown in Fig.\ref{sec:fmwk:base:cmerge_stream}, so here we describe slightly more details in texts by breaking down into 5 blocks.
\begin{enumerate}
\item Event Preparation
  \begin{itemize}
    \item[] {\cmerge} receives an input cluster in terms of a set of {\pxhit} (see Sec.\ref{sec:fmwk:cmerge:input}). Then it creates {\cpan} instance for each cluster, and calls {\cmalgo::EventBegin()} for merging and prohibit algorithms.
  \end{itemize}

\item Iterative Merge Operation
  \begin{itemize}
    \item[] A loop over cluster pairs for prohibit (if provided) and merge {\cmalgo} can be configured to repeat until no newly merged cluster is found. This is shown in Fig.\ref{sec:fmwk:cmerge:iterative_stream}. If a user does not configure to choose iterative approach, iteration will not happen even if there is a newly merged cluster in the output.
  \end{itemize}

\item Prohibit {\cmalgo} Loop Over Cluster Pairs
  \begin{itemize}
    \item[] See Fig.\ref{sec:fmwk:cmerge:iterative_stream}. Only occurs if prohibit algorithm is provided by a user (optional). Records which cluster pair to be prohibited for merging using {\cbkeeper}.
  \end{itemize}

\item Prohibit {\cmalgo} Loop Over Cluster Pairs
  \begin{itemize}
    \item[] See Fig.\ref{sec:fmwk:cmerge:iterative_stream}. The algorithm's {\ttfamily Bool()} is not executed for a subject pair of clusters if (a) they are prohibited for merging or (b) already inspected in the previous process and not merged (i.e. result should not change). This is to save computation time.
  \end{itemize}

\item Merge Clusters
  \begin{itemize}
    \item[] Based on the outcome of merge/prohibit {\cmalgo} algorithms, actual merging is performed at this step. It creates a new set of {\cpan} for merged clusters. For clusters that are not merged, {\cpan} instance is copied from input to save {\cpan} computation time.
  \end{itemize}

\end{enumerate}

\begin{figure}[ht]\begin{center}
\includegraphics[width=13cm]{./src/Pictures/IterativeMergeApproach.pdf}
\caption{Illustration of how iterative merge approach is processed. Internally a loop over cluster pairs for prohibit and merge algorithm are repeated if (a) user configured to use an iterative approach and (b) newly merged cluster is found.}
\label{sec:fmwk:cmerge:iterative_stream}
\end{center}\end{figure}


\subsection{Setting an Input 2D Clusters}
\label{sec:fmwk:cmerge:input}
Input to {\cmerge} is a set of clusters where each cluster is represented as a vector of hits, or more precisely {\ttfamily std::vector<util::PxHit>}. This input can be set by the following function:
\begin{lstlisting}
  void SetClusters(const std::vector<std::vector<util::PxHit> > &)
\end{lstlisting}
where the argument is a vector of 2D clusters. When this function is called, {\cmerge} internally calls
\begin{lstlisting}
  void Reset()
\end{lstlisting}
which initialize previously stored input clusters, output clusters and {\cbkeeper} instance.
Then it creates {\cpan} instance per input 2D cluster using a set of {\pxhit}.
This completes setting (or initiating) the input 2D cluster sets to be used for internal processing.

\subsection{Setting Merge/Prohibit {\cmalgo} Instances}
{\cmalgo} instance is necessary for merging and optional for prohibit. Call the function,
\begin{lstlisting}
  void AddMergeAlg(CBoolAlgoBase* algo)
\end{lstlisting}
to set {\cmalgo} for merging, and 
\begin{lstlisting}
  void AddSeparateAlgo(CBoolAlgoBase* algo)
\end{lstlisting}
to set {\cmalgo} for merge-prohibit.

\subsection{Setting Run Configuration}
There are only 3 run options to configure for {\cmerge}.

\paragraph{Verbosity:} 
There are 5 verbosity levels a user can configure as listed below:
\begin{itemize}
\item[] {\ttfamily kNone}
  \begin{itemize}
    \item No report from {\cmerge}
  \end{itemize}

\item[] {\ttfamily kPerEvent}
  \begin{itemize}
    \item Report made at the end of even processing
  \end{itemize}

\item[] {\ttfamily kPerIteration}
  \begin{itemize}
    \item Report made per merge iteration (see Sec.\ref{sec:fmwk:cmerge:process_flow}). This could be quite verbose.      
  \end{itemize}

\item[] {\ttfamily kPerMerging}
  \begin{itemize}
    \item Report made per {\cmalgo::Bool()} execution (see Sec.\ref{sec:fmwk:cmerge:process_flow}). This is even more verbose than {\ttfamily kPerIteration}.
  \end{itemize}

\end{itemize}
These are {\enum} type, {\ttfamily CMergeMSGLevel\_t}, defined under {\cmerge} class. 
The verbosity level can be set by the function:
\begin{lstlisting}
  void DebugMode(CMergeMSGLevel_t level)
\end{lstlisting}

\paragraph{Iterative Approach:}
As described in Sec.\ref{sec:fmwk:cmerge:process_flow}, {\cmerge} can be configured to attempt iterative approach for merging. This can be enabled by the following function call.
\begin{lstlisting}
  void MergeTillConverge(bool)
\end{lstlisting}
Providing {\ttfamily true} as an argument will enable this option. Disabled if {\ttfamily false} is provided.

\paragraph{Merge Priority:}
As described in Sec.\ref{sec:fmwk:cmerge:how}, ordering of cluster pairs for merging inspection matters to the result.
Options are also described in the same section. This can be set by the following function.
\begin{lstlisting}
  void SetMergePriority(CMergePriority_t level)
\end{lstlisting}

\subsection{Execute Merging}
To execute merging, simply call
\begin{lstlisting}
  void Process()
\end{lstlisting}

\subsection{Obtaining Result}
There are two kinds of results that may be useful for a user. First is a resulting (i.e. merged) set of clusters in terms of {\cpan}. This is useful for further processing as various cluster parameters are already computed as a part of {\cpan}. You can execute
\begin{lstlisting}
  const std::vector<cluster::ClusterParamsAlg>& GetClusters() const
\end{lstlisting}
to obtain such result set.

Second is a mapping of input cluster IDs (i.e. index) to the output (merged clusters). 
As described in Sec.\ref{sec:fmwk:base:cbk}, this can be easily obtained through {\cbkeeper} instance.
You can obtain the {\cbkeeper} instance held by {\cmerge} through the following function.
\begin{lstlisting}
  const CBookKeeper& GetBookKeeper() const
\end{lstlisting}


\section{{\cmatch}: Cluster Matching Tool}
\label{sec:fmwk:cmatch}
Tool not introduced yet in {\larsoft}. Will be documented when this is done.




%
%%% Appendix
%
%\appendix
%\appendixpage

% Bibliography
\bibliographystyle{unsrt}

\bibliography{LArLight}


\end{document}
