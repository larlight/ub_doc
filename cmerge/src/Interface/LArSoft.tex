
This section describes the {\larsoft} interface, called {\ttfamily ClusterMergeHelper}, and an example data producer module, {\ttfamily SimplerClusterMerger}.

\subsection{{\ttfamily ClusterMergeHelper}}
This is a simple wrapper class for {\cmerge} that can be found under:
\begin{lstlisting}
  larreco/RecoAlg/ClusterMergeHelper.h
\end{lstlisting}
It holds {\cmerge} instance which a user can retrieve through
\begin{lstlisting}
  CMergeManager& GetManager()
\end{lstlisting}
and configure as (s)he wish. 

There are two important wrapper functions that can take {\larsoft} data product as input, convert into a collection of {\pxhit} and feed into {\cmerge} instance. First one takes a vector of clusters where each cluster is represented as {\ttfamily std::vector<art::Ptr<recob::Hit> >}.
\begin{lstlisting}
  void SetClusters(const std::vector<std::vector<art::Ptr> > >&)
\end{lstlisting}
The second one takes {\ttfamily art::Event} reference with the producer module name for an input {\ttfamily recob::Cluster} data product.
\begin{lstlisting}
  void SetClusters(const art::Event&, const std::string&)
\end{lstlisting}

Finally, there is a function that can produce output data product and store in {\ttfamily art::Event}:
\begin{lstlisting}
  void AppendResult( art::EDProducer&,
                       art::Event&,
                       std::vector<recob::Cluster>&,
                       art::Assns<recob::Cluster,recob::Hit>& )
\end{lstlisting}
where arguments are producer module's object reference, {\ttfamily art::Event} reference that should have been provided to the module, output data products' reference (i.e. vector of {\ttfamily recob::Cluster} and {\ttfamily art::Assns}).

\subsection{\ttfamily SimpleClusterMerger}
This is an {\ttfamily art::EDProducer} inherit class, a data product producer module in {\larsoft}, that uses {\ttfamily ClusterMergeHelper} to perform cluster merging with some existing merge and prohibit algorithms.
It can be found in here:
\begin{lstlisting}
  larreco/ClusterFinder/SimpleClusterMerger_module.cc
\end{lstlisting}
The code is commented fairly alright, so one should see the code for details of what is done.
Basically all operations done in the module are functions already described in this document by this point.
